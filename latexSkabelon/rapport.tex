\documentclass[a4paper]{report}

\usepackage[utf8]{inputenc}
\usepackage[danish]{babel}
\usepackage [T1]{fontenc}
\usepackage[margin=2.5cm]{geometry}
\usepackage{graphicx}
\usepackage{hyperref}
\usepackage{listings}

\title{Rapportskabelon}
\author{Jon Sporring }

\begin{document}
\maketitle

\section{Forord}
Forordet omhandler selve rapporten. Forordet skal give læseren et billede af rammerne for
rapporten. Forordet indeholder ofte:
\begin{itemize}
\item Hvem har skrevet rapporten, hvornår og i hvilken forbindelse?
\item Hvordan er rapporten blevet til? Er der f.eks. en
  projektkontrakt, er den et eksamensprojekt, er der en opgavestiller
  og i fald hvem?
\item Tak til vejledere, faglige hjælpere, og andre, som har bidraget
  til rapporten.
\end{itemize}
Forordet er et selvstændigt kapitel i rapporten og i en hvis forstand
ikke en del af selve rapporten men mere et forklæde til rapporten. For
korte rapporter er forordet ofte udeladt.

\section{Introduktion}
Dette afsnit skal give en indledning til rapportens emne på et
overordnet plan. Introduktion skal typisk skrives så den kan læses
sammen med konklusionen med kun et overfladisk kendskab til resten af
rapportens indhold. Indledningen indeholder  \dots

"`Indledningen skal ikke handle om rapporten, men om rapportens emne. Man kan sige, at den kridter banen op ved at præsentere emneområdet.

Hvilket overordnet (samfundsmæssigt) emne tager rapporten udgangspunkt i? (Behov for at spare på energien - beskæftigelsessituationen - fortætning af parcelhuskvarterer for at undgå stigende transportafstande i fremtiden …)
Hvilket specifikt problem vil man derfor tage fat i og kigge nærmere på? (Energirenovering af parcelhuse fra 60'erne - bedre sammenhæng mellem folkeskole og gymnasiale uddannelser for at sikre bedre ungdomsuddannelse - udstykning af store villaer til andelsboliger …)
Herefter er der lagt op til problemformuleringen og metodebeskrivelsen:


Her er indledningen fra den ikke-eksisterende rapport "Halvkommerciel frugtavl i private haver i Storkøbenhavn":"'

\section{Problemformulering}
Kan slås sammen med indledning. Beskriv problemet med egne ord. Afgrænsning

"`Problemformulering: Hvilket spørgsmål vil denne rapport helt præcist stille skarpt på, og hvilke delspørgsmål vil det måske være nødvendigt at besvare for at kunne besvare hovedspørgsmålet?
Metodebeskrivelse: Hvordan vil det blive gjort? (Opgaven bygger på den-og-den litteratur samt interviews med dem og dem. Først undersøger vi dét, og så undersøger vi dét. Derefter …)
Til sidst - enten som del af metodebeskrivelsen eller som et selvstændigt underafsnit - kan der komme en kort redegørelse for, hvor rapportens afgrænsning til tilgrænsende emner går, altså hvad den ikke kommer ind på:

Afgrænsning: Hvilke emner / delemner kommer rapporten ikke ind på? Hvorfor?
Det er meget almindeligt, at problemformulering og metoderedegørelse simpelthen er en del af indledningen, men så bør de normalt være markeret med underrubrikker (under-overskrifter) for overskuelighedens skyld. Hvis du er uddannelsessøgende, så tjek, om skolen eller institutionen (eller din vejleder) har særlige krav på det punkt."'

\section{Problem analyse og design}
Skal indeholde diskussion af alternative løsninger med begrundet
valg. De centrale datastrukturer skitseres f.eks. med vigtigeste
funktioner og deres typer.

\section{Programbeskrivelse}
Hvordan er er design omformuleret til et program. Fokuser på
ikke-oplagte, særlige indsigter og tricks, der er brugt ved
transformationen af designet til program tekst.

\section{Afprøvning og eksperimenter}

\section{Diskussion og konklusion}
opsummering, reflektion, mangler, hvad er lært, konklusion

\end{document}
