\documentclass[a4paper]{article}

\usepackage{cmap}
\usepackage[utf8x]{inputenc}
\usepackage{latexsym}
\usepackage[english]{babel}
\usepackage{graphicx}
\usepackage{hyperref}
\usepackage[all]{hypcap}
\usepackage{enumerate}
\usepackage[margin=2.5cm]{geometry}

\begin{document}
\title{Patterns, Tuples, and Lists}

\author{Torben Mogensen}
\date{\today}

\maketitle

\noindent
These notes are supplementary to chapters 3 and 4 in Hansen \&
Rischel.  We will focus on the \emph{functional} style of programming.

\section{Pattern Matching}

We have already seen a recursive version of a function for computing
the $n$'th Fibonacci version.  Here is a variant of this that is
defined only on non-negative numbers:

\begin{verbatim}
let rec fib n =
  if n = 0 then
    0
  elif n = 1 then
    1
  else
    fib (n - 1) + fib (n - 2)
\end{verbatim}

\noindent
We can test this for the first few integers by writing


\begin{verbatim}
for i = 0 to 10 do
  printfn "fib(%d) = %d" i (fib i)
\end{verbatim}

\noindent
which will yield

\begin{verbatim}
fib(0) = 0
fib(1) = 1
fib(2) = 1
fib(3) = 2
fib(4) = 3
fib(5) = 5
fib(6) = 8
fib(7) = 13
fib(8) = 21
fib(9) = 34
fib(10) = 55
\end{verbatim}

\noindent
We used an if-then-else expression for special-casing on \texttt{n=0},
\texttt{n=1}, and everything else.  An alternative to using
if-then-else is to use \emph{pattern matching}.  The \texttt{fib}
function can be written using pattern matching as

\begin{verbatim}
let rec fib n =
  match n with
  | 0 -> 0
  | 1 -> 1
  | m -> fib(m-1) + fib(m-2)
\end{verbatim}

\noindent
The match-with expression matches a value with any number of rules,
where each rule is composed of a vertical bar (\texttt{|}), a pattern,
an arrow (\texttt{->}) and an expression.  A match-with
expression first calculates the value of the expression between the
\texttt{match} and \texttt{with} keywords to a value $v$ and then
applies the \emph{first} rule that matches $v$.  A rule
\texttt{|~$p$~->~$e$} matches $v$ if the pattern $p$ matches $v$.  If
$p$ matches $v$, any variables in $p$ are \emph{locally} in $e$ bound
to the corresponding parts of $v$ and the expression $e$ is evaluated.

In the function above, \texttt{n} is evaluated to a value $v$.  If
$v=0$, the first rule matches, and 0 is returned.  If $v=1$, the first
rule does not match, but the second rule does, so 1 is returned.  If
$v$ is neither 0 nor 1, neither the first nor the second rule
matches.  The pattern in the third rule is a variable.  A variable
pattern always matches, so the third rule will match.  The variable
\texttt{m} is bound to the the value $v$, and the expression after the
arrow is evaluated.  For example, if $v=2$, the expression
\texttt{fib(m-1) + fib(m-2)} is evaluated with the binding
$\texttt{m}=2$, so the results of calling \texttt{fib(1)} and
\texttt{fib(0)} are added, giving the result 2, as expected.

Note that the order of rules is important.  If the third rule is moved
up before the two other rules, it will match all values, so the other
two rules are never used.

So far, we have seen two kinds of pattern:

\begin{description}

\item[Constant patterns] are identical to constant expressions.  We
  have seen only integer constant expressions, but floating-point
  constants, string constants, and so on are all also valid patterns.
  A constant pattern will match exactly the value that the constant
  represents.
\item[Variable patterns] are variable names. A variable pattern will
  match any value and in the process bind the variable to the value
  locally in the rule.  A special case is a \emph{wildcard pattern},
  which is written as an underscore (\verb|_|).  It also matches any
  value, but it does not bind any variables.  In the third rule of the
  example above, we do not really need \texttt{m} to hold the value,
  as it is already contained in \texttt{n}, so we could equivalently
  write the third rule as

\begin{verbatim}
  | _ -> fib(n-1) + fib(n-2)
\end{verbatim}

\end{description}


\noindent
Patterns can also be used when defining functions.  The function
definition

\begin{verbatim}
fun f 0 = 0
\end{verbatim}

\noindent
defines a function that given 0 returns 0 and is undefined on all
other values.  The compiler will give a warning saying that \texttt{f}
is not defined on all values, but it will generate runnable code.
Applying \texttt{f} to 0 will, as expected, return 0, but applying
\texttt{f} to any other integer will result in an error message.

Using pattern matching instead of if-then-else can often lead to more
readable programs, and the ability of patterns to bind variables to
components of a value, such as the elements of tuples or lists, which
we will see below.

\section{Tuples}

``Tuple'' is a generic name for pairs, triples, quadruples, etc.

A \emph{pair} is a combination of two values $v$ and $w$.  The pair is
written as \texttt{($v$,$w$)}, using a notation that should be familiar
from coordinate pairs in math.  $v$ and $w$ do not need to have the
same type, so \texttt{(2,true)} and \texttt{("pi",3.14159)} are both
valid tuples.

The type of a pair is written as the types of the components separated
by \texttt{*}.  So the two tuples above have types \texttt{int~*~bool}
and \texttt{string~*~float}, respectively.  The notation is similar to
the math notation for cartesian products, where, for example, the set
of pairs of two integers is written as $\mathsf{I\!\!N} \times
\mathsf{I\!\!N}$.  But since \texttt{×} is not a character in the
ASCII character set, \texttt{*} is used instead.

Building a pair is easy: You just enclose the components in
parentheses separated by a comma.  So the function

\begin{verbatim}
let makePair x y = (x,y)
\end{verbatim}

\noindent
has type \texttt{makePair : x:'a -> y:'b -> 'a * 'b}.  This states
that \texttt{makePair} takes two arguments \texttt{x:'a} and
\texttt{y:'b} and produces a pair of type \texttt{'a * 'b}.
\texttt{makePair} can make pairs of any two types, so it is
\emph{polymorphic}.  ``Polymorphic'' is derived from Greek and means
``of multiple forms''.  In programming, we use the term for functions
that have multiple types.  You can read the type of \texttt{makePair}
as ``For any to types $a$ and $b$, take a value $x$ of type $a$ and a
value $y$ of type $b$ and return a pair of type $a\texttt{*}b$''.

Getting the components of a pair is best done using pattern matching.
For example, the function


\begin{verbatim}
let flip (x,y) = (y,x)
\end{verbatim}

\noindent
has the type \texttt{flip : x:'a * y:'b -> 'b * 'a}.  It takes a pair
$(v,w)$ and returns the pair $(w,v)$.  Note that the pattern
\texttt{(x,y)} matches any value of type \texttt{'a~*~'b}, so there is
no warning about values not being matched.  The definition is
equivalent to


\begin{verbatim}
let flip pair =
  match pair with
  | (x,y) -> (y,x)
\end{verbatim}

\noindent
or, using pattern matching in a let-binding,


\begin{verbatim}
let flip pair =
  let (x,y) = pair
  (y,x)
\end{verbatim}

\noindent
but when written as either of the above, the type is shown as
\texttt{flip : 'a * 'b -> 'b * 'a}.  The two types, though they look
different, are the same, as naming components of types (such as
\texttt{x:a}) do not change the type.

Note that a pattern can not contain the same variable more than once,
so all of the following definitions are illegal:

\begin{verbatim}
let f x x = 1
let g (x, x) = 2
let h (x, (y, x)) = 3
\end{verbatim}

\noindent
Pairs can be components of other pairs, so \texttt{(3,(4,5))} is a
pair of type \texttt{int~*~(int~*~int)}.  The parentheses in the type
are significant: The types \texttt{int~*~(int~*~int)},
\texttt{(int~*~int)~*~int}, and\newline \texttt{int~*~int~*~int} are all
different.

This brings us to \emph{triples}.  Where pairs are composed of two
values, triples are composed of three values.  Examples of triples
(with types) are

\vspace{1ex}

\texttt{
\begin{tabular}{r@{ : }l}
(3,4,5) & int * int * int\\
(true,"abc",3.14) & bool * string * float\\
((1,2),(3,4),(5,6)) & (int * int) * (int * int) * (int * int)
\end{tabular}}

\noindent
Building triples and using patterns to decompose triples is done
exactly the same way as for pairs, except that there are three
components.  For example, a function that checks if a triple is a
Pythagorean triple can be written as

\begin{verbatim}
let pythagorean (a,b,c) = a*a + b*b = c*c
\end{verbatim}

\noindent
which has type \texttt{pythagorean : a:int * b:int * c:int -> bool}.

We can also write a function for testing pythagorean numbers by using
three separate arguments

\begin{verbatim}
let pythagorean' a b c = a*a + b*b = c*c
\end{verbatim}

\noindent
which has type \texttt{pythagorean' : a:int -> b:int -> c:int ->
  bool}.  If you use separate arguments, you can partially apply the
function, so

\begin{verbatim}
let p34 = pythagorean' 3 4
\end{verbatim}

\noindent
is a value \texttt{p34 : (int -> bool)}, which takes a single argument
$c$ and tests if $3*3+4*4 = c*c$.  On the other hand, you can apply
the \texttt{pythagorean} function to a value that is already a triple:


\begin{verbatim}
let triple = (3,4,5)
pythagorean triple
\end{verbatim}

\noindent
and if you want to return three values from a function, a triple is
the obvious way to do so.

Pairs and triples extend to tuples with four, five or more components
in the obvious way.

\section{Lists}

A list is a sequence of values that have the same type, but where the
length of the sequence is not specified in the type.  This contrasts
with tuple types, where the components can have different types, but
the number of components is specified in the type.  A list is written
as a sequence of elements separated by semicolons and enclosed in
square brackets.  For example, a list of four numbers can be written
as \texttt{[3;~1;~4;~1]} and has type \texttt{int~list}, indicating
that it is a list of integers.  The elements can be of any type, as
long as the types are consistent, so the following are all legal
lists:

\vspace{1ex}

\texttt{
\begin{tabular}{r@{ : }l}
  ['a'; 'e'; 'i'; 'o'; 'u'; 'y'] & char list \\{}
  [true] & bool list\\{}
  [(3, 4, 5); (5, 12, 13); (8, 15, 17)] & (int * int * int) list\\{}
  [[]; [1]; [1; 2]; [1; 2; 3]] & int list list \\{}
  [] & 'a list
\end{tabular}}

\vspace{1ex}

\noindent
Note that the last example -- the empty list -- has type
\texttt{'a~list} because it is an empty list of any type of element.
Note, also, the parentheses in the third example.  These are required
because the \texttt{list} type constructor binds more tightly than the
\texttt{*} type constructor.  The type \texttt{int~*~int~*~int~list}
is equivalent to \texttt{int~*~int~*~(int~list)} and would have
elements like \texttt{(1, 2, [5; 6])}.  Similarly,
\texttt{int~list~list} is equivalent to \texttt{(int~list)~list}.

Just as with strings, you can use dot-notation to get to elements or
sublists of lists:

\vspace{1ex}

\texttt{
\begin{tabular}{r@{ = }l}
  ['a'; 'e'; 'i'; 'o'; 'u'; 'y'].[2] & 'i' \\{}
  ['a'; 'e'; 'i'; 'o'; 'u'; 'y'].[2..4] & ['i'; 'o'; 'u'] \\{}
  ['a'; 'e'; 'i'; 'o'; 'u'; 'y'].[2..14] & \emph{error message}
\end{tabular}}

\vspace{1ex}

\noindent
In fact, a list of characters is very similar to a string, but they
are separate types.

You can concatenate two lists using the \texttt{@} operator.  Compare

\vspace{1ex}

\texttt{
\begin{tabular}{r@{ = }l}
  ['a'; 'e'; 'i'] @ ['o'; 'u'; 'y']  &  ['a'; 'e'; 'i'; 'o'; 'u'; 'y']
  : char list\\
  "aei" + "ouy" & "aeiouy" : string
\end{tabular}}

\vspace{1ex}

\noindent
The function \texttt{List.length} finds the length of a list.  Compare

\vspace{1ex}

\texttt{
\begin{tabular}{r@{ = }l}
  List.length  ['a'; 'e'; 'i'; 'o'; 'u'; 'y'] & 6 \\
  String.length "aeiouy" & 6
\end{tabular}}

\vspace{1ex}

\noindent
You can also pattern match on lists.  For example, we can write this
function that finds the sum of elements of integer lists of length up
to three:

\begin{verbatim}
let sum ns =
  match ns with
  | [] -> 0
  | [n1] -> n1
  | [n1; n2] -> n1 + n2
  | [n1; n2; n3] -> n1 + n2 + n3
\end{verbatim}

\noindent
This is easily (albeit verbosely) extended to list up to any fixed
length, but if we want to handle lists of arbitrary length, we use the
\texttt{::} constructor, also called the \emph{cons} operator.

The \texttt{::} constructor can be used as an infix operator both in
expressions (to build lists) and in patterns (to match against lists),
but unlike other infix operators (such as \texttt{+} or \texttt{@}) it
is not a function, so writing \texttt{(::)} will give an error
message.  Used in an expression, \texttt{::} takes on its left-hand
side a value $x$ of any type $a$ and on its right-hand side a list
$xs$ of type $a$~\texttt{list} and will produce a list of type
$a$~\texttt{list} by adding $x$ as an element in front of $xs$.
Examples:

\vspace{1ex}

\texttt{
\begin{tabular}{r@{ = }l}
  'a' :: ['e'; 'i'; 'o'; 'u'; 'y']  &  ['a'; 'e'; 'i'; 'o'; 'u'; 'y']\\
  1 :: [] & [1]\\
  1 :: 2 :: 3 :: [] = [1; 2; 3]
\end{tabular}}

\vspace{1ex}

\noindent
Note that \texttt{a :: b :: c} is equivalent to \texttt{a :: (b ::
  c)}, so \texttt{::} is \emph{right associative}: multiple
applications of \texttt{::} are implicitly grouped to the right.

The expression \texttt{x~::~xs} is equivalent to \texttt{[x]~@~xs}.
But where \texttt{@} can not be used in patterns, \texttt{::} can.
Typically, a function that operates on a list will be recursive and
use a match-with expressions with one rule for the empty list and one
rule for the non-empty list:

\begin{verbatim}
let rec listSum ns =
  match ns with
  | [] -> 0
  | n :: ns -> n + listSum ns
\end{verbatim}

\noindent
The first pattern \texttt{[]} matches the empty list, which has sum 0.
The second pattern matches a list with at least one element, and binds
\texttt{n} to the first element of the list (also called the
\emph{head} of the list) and \texttt{ns} to the rest of the list (also
called the \emph{tail} of the list).  Note that the variable name
\texttt{ns} is rebound locally within the second rule.  Application of
the \texttt{listSum} function to an argument can be illustrated by the
following reduction sequence:


\vspace{1ex}

\texttt{
\begin{tabular}{rl}
listSum [1;4;9;16] \\
$\leadsto$ & 1 + listSum [4;9;16]  \\
$\leadsto$ & 1 + 4 + listSum [9;16]   \\
$\leadsto$ & 1 + 4 + 9 + listSum [16]   \\
$\leadsto$ & 1 + 4 + 9 + 16 + listSum []  \\
$\leadsto$ & 1 + 4 + 9 + 16 + 0 \\
$\leadsto$ & 30
\end{tabular}}

\vspace{1ex}

\noindent
Note that a function equivalent to \texttt{listSum} is predefined in
F\# as \texttt{List.sum}, but where \texttt{listSum} only works on
lists of integers, \texttt{List.sum} works on lists of floats or other
number types as well.

Let us consider writing a function that tests if a list is sorted.  An
empty list is always sorted, and so is a list consisting of exactly
one element.  If a list has two or more elements, it is sorted if the
first element is less than or equal to the second element and the rest
of the list (starting from the second element) is also sorted.  We can
write this as


\begin{verbatim}
let rec isSorted ns =
  match ns with
  | [] -> true
  | [n] -> true
  | n1 :: n2 :: ns -> n1 <= n2 && isSorted (n2 :: ns)
\end{verbatim}

\noindent
This has the type \texttt{isSorted : ns:'a list -> bool when 'a :
  comparison}.  This is because the \texttt{<=} operator is
\emph{overloaded} and works on several (but not all) types.  The set
of types where \texttt{<=} is defined are categorised by the
pseudo-type \texttt{comparison}, so \texttt{'a : comparison} means
``any type \texttt{'a} where \texttt{<=} is defined''.  So we can use
\texttt{isSorted} on many, but not all, list types, for example:


\vspace{1ex}

\texttt{
\begin{tabular}{r@{ = }l}
  isSorted [2; 1; 3; -1] & false\\
  isSorted ["Cc"; "ab"; "c"] & true\\
  isSorted [1e-2; 1.0; 1e0] & true
\end{tabular}}

\vspace{1ex}

\noindent
We note that we rebuild the list from the second element onwards by
writing \texttt{n2~::~ns} in the recursive call.  This is rather
inefficient, so an alternative is to use \emph{as-pattterns}.  An
as-pattern is a pattern $p$ followed by the keyword \texttt{as} and a
variable $x$.  The as-pattern matches if $p$ matches but in addition
to binding the variables in $p$ to the corresponding components, it
binds $x$ to the entire value.  We can write \texttt{sorted} using an
as-pattern by

\begin{verbatim}
let rec isSorted ns =
  match ns with
  | [] -> true
  | [n] -> true
  | n1 :: (n2 :: _ as ns) -> n1 <= n2 && isSorted ns
\end{verbatim}

\noindent
The pattern \verb|n2 :: _ as ns| matches a non-empty list $v$ and
binds \texttt{ns} to $v$, \texttt{n2} to the head of $v$.  We don't
need the tail of $v$, so we use a wildcard pattern.

In a similar way, we can write a function that tests if all the lists
in a list of lists have the same length:

\begin{verbatim}
let rec sameLength xs =
  match xs with
  | [] -> true
  | [x] -> true
  | x1 :: (x2 :: _ as xs)-> List.length x1 = List.length x2 && sameLength xs
\end{verbatim}

\noindent
which has type \texttt{sameLength : xs:'a list list -> bool}.

While we by using the as-pattern have avoided rebuilding the list, we
compute the length of most lists twice.  To avoid this, we can use a
helper-function that compares the length of all lists in a list of
lists to the length of the first list:

\begin{verbatim}
let sameLength xs =
  match xs with
  | [] -> true
  | x1 :: xs ->
      let lx1 = List.length x1
      let rec sameLengthAs xs =
        match xs with
        | [] -> true
        | x1 :: xs -> List.length x1 = lx1 && sameLengthAs xs
      sameLengthAs xs
\end{verbatim}

\noindent
Note that both versions of \texttt{sameLength} compute the same
function, the only difference is efficiency, and that difference is
less than a factor of two.

As a slightly more complex example, let us consider a function that
finds the largest element of a list of values.  This function is not
defined on the empty list, as an empty list does not have a largest
element, so we write the function as

\begin{verbatim}
let rec largest ns =
  match ns with
  | [n] -> n
  | n :: ns -> max n (largest ns)
\end{verbatim}

\noindent
The F\# compiler will complain that the pattern matching is
incomplete, as it does not match the empty list, but it will compile
it to a function with type \texttt{largest : ns:'a list -> 'a when 'a
  : comparison}.  Note that the function is not restricted to lists of
integers, but can be applied to, for example, lists of floats or lists
of strings.  This is because the \texttt{max} function used in the
last line is overloaded and works on all types where comparison using
\texttt{<=} is defined, just as we saw for the function
\texttt{isSorted} earlier.

Applying \texttt{largest} to an empty list will yield a rather cryptic
error message:

\begin{verbatim}
error FS0030: Value restriction. The value 'it' has been inferred to have generic type
    val it : '_a when '_a : comparison    
Either define 'it' as a simple data term, make it a function with explicit arguments or,
if you do not intend for it to be generic, add a type annotation.
\end{verbatim}

\noindent
The reason for this error message is rather complicated, so we will
ignore it for now.  The interested reader can see Section~4.5 of
Hansen \& Rischel.

\section{Patterns Revisited}

We summarise the forms of patterns we have seen so far.  A pattern can
be:

\begin{itemize}
\item A constant in a type that is comparable be equality.  A constant
  pattern matches any value that tests for true with equality with the
  constant.  No variables are bound.
\item A variable.  This matches any value and binds the variable to
  that value.
\item An underscore (\verb|_|).  This \emph{wildcard pattern} matches
  anything, but does not bind anything.
\item A tuple pattern of the form \texttt{($p_1$,\ldots,\,$p_n$)}.
  This matches a tuple value \texttt{($v_1$,\ldots,\,$v_m$)} if $m=n$
  (which is verified by the type checker) and $p_i$ matches $v_i$ for
  all $i\in\{1,\ldots,\,n\}$.  Variables can be bound when matching
  each $p_i$ to $v_i$, and all bindings are in scope in the rule that
  uses the pattern.  No variable may be repeated in the pattern.
\item A list pattern of the form \texttt{[$p_1$;\ldots;\,$p_n$]}.
  This matches a list value \texttt{[$v_1$;\ldots;\,$v_m$]} if $m=n$
  and $p_i$ matches $v_i$ for all $i\in\{1,\ldots,\,n\}$.  The same
  rules for binding and repeated variables that apply to tuple
  patterns also apply to list patterns.
\item A constructor pattern.  We have seen the nullary constructor
  pattern \texttt{[]} and the infix constructor pattern \texttt{$p_1$
    :: $p_2$}.  Generally, a constructor pattern is either a nullary
  constructor (a constructor that does not take arguments) or a
  constructor applied to a single argument which may be a tuple or
  list pattern.  An infix constructor pattern is equivalent to a
  constructor applied to a pair pattern.  A constructor pattern
  matches if the value is constructed by the same constructor and the
  argument of that constructor (if any) match the argument pattern.
\item An as-pattern.  This is of the form \texttt{$p$ as $x$}, where
  $p$ is a pattern and $x$ is a variable.  It matches a value $v$ if
  $p$ matches $v$, and $x$ is bound to $v$ in addition to any bindings
  made when matching $p$ to $v$.  $x$ may not occur in $p$.
\item Operator precedence and parentheses can effect grouping of
  nested patterns as they affect nested expressions or types.
  Operators in patterns have the same precedence as in expressions.
\end{itemize}

\noindent
We have seen patterns used in the following contexts:

\begin{itemize}
\item In match-with expressions, where patterns are used in rules of
  the form \texttt{| $p$ -> $e$}.  A warning is given if the set of
  rules do not cover all possible values allowed by the type of the
  matched expression, and if at runtime a value is given that does not
  match any rule, an error is reported.

  A warning is also given if a rule can not be reached because all
  values matched by that rule are already matched by earlier rules.
  This can not cause runtime errors, but if you get this error, don't
  just delete the rule that can never me reached.  Instead, think of
  \emph{why} it can not be reached, and fix the problem with this
  understanding in mind.
\item In function definitions, where a pattern can be used instead of
  an argument variable. If the pattern does not cover all values
  allowed by the argument type, a warning is given, and if at runtime
  an argument is given that does not match the pattern, an error is
  reported.
\item In let-definitions, where a pattern can be used instead of a
  variable.  Again, warnings and errors are reported if the pattern
  does not cover all possible values.
\end{itemize}

\noindent
Additionally, rules in the style of match-with can be used in an
alternative notation for function definition.  The \texttt{fib}
function shown in Section~1 can alternatively be written as

\begin{verbatim}
let rec fib = function
  | 0 -> 0
  | 1 -> 1
  | m -> fib(m-1) + fib(m-2)
\end{verbatim}

\noindent
You can read the notation as ``\texttt{fib} is a function that maps 0
to 0, 1 to 1, and any other value $m$ to
$\texttt{fib}(m-1)+\texttt{fib}(m-2)$''.

Note that the argument to \texttt{fib} is not named, it is implicitly
given to the rules shown after the \texttt{function} keyword.

Similarly, \texttt{listSum} from Section~3 can be written as


\begin{verbatim}
let rec listSum = function
  | [] -> 0
  | n :: ns -> n + listSum ns
\end{verbatim}

\noindent
The book by Hansen \& Rischel uses the \texttt{function} notation by default.
\end{document}

