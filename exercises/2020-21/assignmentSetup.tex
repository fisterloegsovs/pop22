\usepackage{import}
\usepackage{etoolbox}

\newrobustcmd{\typeName}[1]{
  \expandafter\ifstrequal\expandafter{#1}{g}{gruppeopgave}{individuel opgave}
} % \expandafter\ifstrequal\expandafter allows for argument to be a command.

\newrobustcmd{\exerciseTitle}[5]{
\title{Programmering og Problemløsning\\Datalogisk Institut, Københavns Universitet\\Arbejdsseddel #1 - \typeName{#2}}
\author{#5}
\date{#3.\\Afleveringsfrist: #4.}
}

\newrobustcmd{\learningPurposeTxt}[1]{
Emnerne for denne arbejdsseddel er:
\listItemize{#1}
}

\DeclareListParser*{\iterateItemize}{,}
\newcommand{\listItemize}[1]{%
\begin{itemize}
  \iterateItemize{\item}{#1}
\end{itemize}
}

\newrobustcmd{\typeCommand}[1]{
}

\newrobustcmd{\learningConditionsTxt}[1]{
Opgaverne er opdelt i øve- og afleveringsopgaver. I denne periode skal I
arbejde %
\expandafter\ifstrequal\expandafter{#1}{g}{i grupper}{individuelt} %
 med jeres afleveringsopgaver. Regler for gruppe- og individuelle
 afleveringsopgaver er beskrevet i "`Noter, links, software m.m."'$\rightarrow$"`Generel information om opgaver"'.
}

\newcommand{\dome}[1]{\item #1}
\newcommand{\listItemizeInternal}[1]{%
\begin{itemize}
\forlistloop{\dome}{#1}
\end{itemize}
}

%%%%%%%%%%%%%%%%%%%%%5
% Jon style exercise database
\DeclareListParser*{\iterateImportEnumerate}{,}
\newcommand{\listImportEnumerate}[3]{%
  \begin{enumerate}[#1]
    \iterateImportEnumerate{\item \subimport*{#2}}{#3}
  \end{enumerate}
}

\newrobustcmd{\exercises}[4]{
  \ifstrempty{#3}{}{\subimport*{#2}{#3}}
  \listImportEnumerate{label=#1\arabic*,start=0}{#2}{#4}
}

\newrobustcmd{\continueExercises}[4]{
  \ifstrempty{#3}{}{\subimport*{#2}{#3}}
  \listImportEnumerate{label=#1\arabic*,resume}{#2}{#4}
}

%%%%%%%%%%%%%%%%%%%%%5

\DeclareListParser*{\iterateEnumerate}{,}
\newcommand{\listEnumerate}[2]{%
\begin{enumerate}[#1]
  \iterateEnumerate{\item }{#2}
\end{enumerate}
}

\newrobustcmd{\exercisesWHeader}[5]{
\section*{#1}
\ifstrempty{#4}{}{\subimport*{#3}{#4}}
\listImportEnumerate{label=#2\arabic*,start=0}{#3}{#5}
}
\newrobustcmd{\trainingExercises}[4]{\exercisesWHeader{Øveopgaver}{#1}{#2}{#3}{#4}}
\newrobustcmd{\handInExercises}[4]{\exercisesWHeader{Afleveringsopgaver}{#1}{#2}{#3}{#4}}

%%%%%%%%%%%%%%%%%%%%%5
% for probsoln packages
\newrobustcmd{\exerciseProblem}[2]{
  \begin{enumerate}[label=#1\arabic*,start=0]
    \renewcommand*{\do}[1]{\item \useproblem{##1}}
    \docsvlist{#2}
  \end{enumerate}
}
\newrobustcmd{\continueExerciseProblem}[2]{
  \begin{enumerate}[label=#1\arabic*,resume]
    \renewcommand*{\do}[1]{\item \useproblem{##1}}
    \docsvlist{#2}
  \end{enumerate}
}
%%%%%%%%%%%%%%%%%%%% 


\newcommand{\handinTextOne}{Afleveringen skal bestå af
\begin{itemize}\item upload af de udviklede programmer i øvelsesholdets studio
på Scratch' hjemmeside, \item en zip-fil og \item en rapport i
pdf-format.\end{itemize} Rapporten skal dokumentere jeres proces og
beskrive jeres programmer. Den skal skrives i \LaTeX, indeholde et passende antal figurer og henvise til jeres programmer (i øvelsesholdets studio på Scratch' hjemmeside). \LaTeX\ koden (\texttt{.tex}) inklusiv billeder skal
organiseres i en mappe kaldet \texttt{tex}, således at pdf-filen kan
laves ved kørsel af \texttt{pdflatex} kommandoen i mappen. Mappen skal
zippes (\texttt{.zip}) og afleveres sammen med den oversatte pdf fil
(\texttt{.pdf}) som 2 filer i Absalon.}

\newcommand{\handinTextOneOne}[1]{Afleveringen skal bestå
af \begin{itemize}\item en zip-fil\item en pdf-fil\end{itemize}
Zip-filen skal indeholde en \texttt{src} mappe, en \texttt{tex} mappe
og filen \texttt{README.txt}. Mappen \texttt{src} skal indeholde
fsharp koden, der skal være en fsharp tekstfil per fsharp-opgave, og
de skal navngives \texttt{#1} osv. De skal kunne oversættes med
fsharpc, og de oversatte filer skal kunne køres med mono. Mappen
\texttt{tex} skal indeholde \LaTeX\ koden. Filen \texttt{README.txt} skal ganske
kort beskrive, hvordan rapporten oversættes til pdf og koden
oversættes og køres.}

\newcommand{\handinTextTwoI}{Afleveringen skal bestå af
\begin{itemize}
\item - en zip-fil, navngivet \texttt{fornavnEfternavnHoldXX-2i.zip}
\item - en pdf-fil, navngivet \texttt{fornavnEfternavnHoldXX-2i.pdf}
\end{itemize}
\textbf{Eksempel:} Jon Sporring er på hold 1, så filerne hedder:
\begin{itemize}
\item \texttt{jonSporringHold01-2i.zip}
\item \texttt{jonSporringHold01-2i.pdf}
\end{itemize}
Zip-filen skal indeholde to mapper: \texttt{src/} og \texttt{tex/},
samt filen \texttt{README.txt}.  Mappen \texttt{src/} skal indeholde
en enkelt fil navngivet \texttt{2i1.fsx}, der indeholder fsharp-koden
til opgave 2i1.  \texttt{2i1.fsx} skal kunne oversættes med
\texttt{fsharpc}, og den oversatte fil skal kunne køres med
\texttt{mono}.  Mappen \texttt{tex/} skal indeholde \LaTeX koden.
Filen \texttt{README.txt} skal ganske kort beskrive, hvordan rapporten
oversættes til pdf og koden oversættes og køres.}

\newcommand{\handinTextTwo}[1]{Afleveringen skal bestå
af \begin{itemize}\item en zip-fil\end{itemize} Zip-filen skal
indeholde en \texttt{src} mappe og filen \texttt{README.txt}. Mappen
skal indeholde fsharp koden, der skal være en fsharp tekstfil per
fsharp-opgave, og de skal navngives \texttt{#1} osv. De skal kunne
oversættes med fsharpc, og de oversatte filer skal kunne køres med
mono. Funktioner skal dokumenteres ifølge dokumentationsstandarden, og
udover selve programteksten skal besvarelserne indtastes som
kommentarer i de fsx-filer, de hører til. Filen \texttt{README.txt}
skal ganske kort beskrive, hvordan koden køres.  }

\newcommand{\handinTextTwoOne}[1]{Afleveringen skal bestå af
\begin{itemize}
  \item en zip-fil, der hedder \texttt{#1\_<gruppe>.zip} (f.eks. \texttt{#1\_gruppe4.zip})
\end{itemize}
Zip-filen \texttt{#1\_<gruppe>.zip} skal indeholde en \texttt{src}
mappe og filen \texttt{README.txt}.  I \texttt{src} skal der ligge
præcis 5 filer: \texttt{vec2d.fsi, vec2d.fs, #11.fsx, #12.fsx,
#13.fsi} svarende til den udleverede fil og de 4 delopgaver.  De skal
kunne oversættes med fsharpc og de oversatte filer skal kunne køres
med mono. Funktioner skal dokumenteres ifølge dokumentationsstandarden
som minimum ved brug af \texttt{<summary>}, \texttt{<param>} og
\texttt{<returns>} XML-tagsne. Udover selve koden skal besvarelser
indtastes som kommentarer i de fsx-filer, de hører til. Filen
\texttt{README.txt} skal ganske kort beskrive, hvordan koden
oversættes og køres.}

\newcommand{\handinTextTwoTwo}[2]{Afleveringen skal bestå af
\begin{itemize}
  \item en zip-fil, der hedder \texttt{#1\_<navn>.zip}  (f.eks. \texttt{#1\_jon.zip})
\end{itemize}
Zip-filen \texttt{#1\_<navn>.zip} skal indeholde en \texttt{src} mappe
og filen \texttt{README.txt}.  I \texttt{src} skal der ligge følgende
og kun følgende filer: #2 svarende til hver af delopgaverne.  De skal
kunne oversættes med \lstinline[language=console]{fsharpc}, og de
oversatte filer skal kunne køres med mono.  Funktioner skal
dokumenteres ifølge dokumentationsstandarden som minimum ved brug af
\texttt{<summary>}, \texttt{<param>} og \texttt{<returns>}
XML-tagsne. Udover selve koden skal besvarelser indtastes som
kommentarer i de fsx-filer, de hører til. Filen \texttt{README.txt}
skal ganske kort beskrive, hvordan koden oversættes og køres, og
eventuelle Black-box, White-box og håndkøringsresultater når relevant.}

\newcommand{\handinTextTwoThree}[2]{Afleveringen skal bestå af
\begin{itemize}
  \item en zip-fil, der hedder \texttt{#1\_<navn>.zip}  (f.eks. \texttt{#1\_jon.zip})
\end{itemize}
Zip-filen \texttt{#1\_<navn>.zip} skal indeholde en \texttt{src} mappe
og filen \texttt{README.txt}.  I \texttt{src} skal der ligge følgende
og kun følgende filer: #2 svarende til hver af delopgaverne.  De skal
kunne oversættes med \lstinline[language=console]{fsharpc}, og de
oversatte filer skal kunne køres med mono.  Funktioner skal
dokumenteres ifølge dokumentationsstandarden som minimum ved brug af
\texttt{<summary>}, \texttt{<param>} og \texttt{<returns>}
XML-tagsne. Udover selve koden skal besvarelser indtastes som
kommentarer i de fsx-filer, de hører til. Filen README.txt skal
indeholde:\begin{itemize}
\item En kort beskrivelse af hvordan koden oversættes og køres.
\item En beskrivelse af resultaterne for hhv. Black- og White-box test og kommentere evt. fejlede test.
\item Besvarelse på opg. 4i3, inkl. håndkøringstabel.\end{itemize}}

%\usepackage{forest}
\newcommand{\handinTextOneThree}[2]{Afleveringen skal bestå af
\begin{itemize}
  \item en zip-fil, der hedder \texttt{#1\_<navn>.zip}  (f.eks. \texttt{#1\_jon.zip})
\end{itemize}
Zip-filen \texttt{#1\_<navn>.zip} skal indeholde en og kun en mappe
som hedder \texttt{#1<navn>}.  I den mappe skal der ligge en
\texttt{src} mappe og filen \texttt{README.txt}.  I \texttt{src} skal
der ligge følgende og kun følgende filer: #2 svarende til de relevante
delopgaver. De skal kunne oversættes med
\lstinline[language=console]{fsharpc}, og de oversatte filer skal
kunne køres med \lstinline[language=console]{mono}. Funktioner skal
dokumenteres ifølge dokumentationsstandarden som minimum ved brug af
\texttt{<summary>}, \texttt{<param>} og \texttt{<returns>}
XML-tagsne. Filen README.txt skal ganske kort beskrive, hvordan koden
oversættes og køres.}

\newcommand{\handinTextOneTwo}[2]{Afleveringen skal bestå af
\begin{itemize}
  \item en zip-fil, der hedder \texttt{#1\_<navn>.zip}  (f.eks. \texttt{#1\_jon.zip})
\end{itemize}
Zip-filen \texttt{#1\_<navn>.zip} skal indeholde en og kun en mappe
\texttt{#1\_<navn>}. I den mappe skal der ligge en \texttt{src} mappe
og filen \texttt{README.txt}.  I \texttt{src} skal der ligge følgende
og kun følgende filer: #2 svarende til de relevante delopgaver.
    % % Figur der viser hvad de studerende skal aflevere,
    % % aka. 6g0.fsx til 6g4.fsx samt README.txt i src mappen i zipfilen
    % \begin{figure}[ht]
    %   \centering
    % \begin{forest}
    %   for tree={
    %     font=\ttfamily,
    %     grow'=0,
    %     child anchor=west,
    %     parent anchor=south,
    %     anchor=west,
    %     calign=first,
    %     edge path={
    %       \noexpand\path [draw, \forestoption{edge}]
    %       (!u.south west) +(7.5pt,0) |- node[fill,inner sep=1.25pt] {} (.child anchor)\forestoption{edge label};
    %     },
    %     before typesetting nodes={
    %       if n=1
    %         {insert before={[,phantom]}}
    %         {}
    %     },
    %     fit=band,
    %     before computing xy={l=15pt},
    %   }
    %   [6g\_gruppe<NR>.zip
    %     [src
    %       [6g0.fsx]
    %       [6g1.fsx]
    %       [6g2.fsx]
    %       [6g3.fsx]
    %       [6g4.fsx]
    %       [README.txt]
    %     ]
    %   ]
    % \end{forest}
    %   \caption{Din mappestruktur skal se således ud i zip-filen}
    %   \label{fig:zip-eksempel}
    % \end{figure}
De skal kunne oversættes med \lstinline[language=console]{fsharpc}, og
de oversatte filer skal kunne køres med mono.  Funktioner skal
dokumenteres ifølge dokumentationsstandarden som minimum ved brug af
\texttt{<summary>}, \texttt{<param>} og \texttt{<returns>}
XML-tagsne. Udover selve koden skal besvarelser indtastes som
kommentarer i de fsx-filer, de hører til. Filen \texttt{README.txt}
skal ganske kort beskrive, hvordan koden oversættes og køres.}

\newcommand{\handinTextThree}[1]{Afleveringen skal bestå
af \begin{itemize}\item en zip-fil\item en pdf-fil\end{itemize}
Zip-filen skal indeholde en \texttt{src} mappe og filen
\texttt{README.txt}.  Mappen skal indeholde fsharp koden, der skal
være en fsharp tekstfil per fsharp-opgave, og de skal navngives
\texttt{#1} osv. De skal kunne oversættes med fsharpc og den
oversatte fil skal kunne køres med mono. Funktioner skal dokumenteres
ifølge dokumentationsstandarden. Filen \texttt{README.txt} skal ganske
kort beskrive, hvordan koden oversættes og køres. Pdf-filen skal
indeholde jeres rapporten oversat fra \LaTeX.}

\newcommand{\handinTextThreeOne}[2]{Afleveringen skal bestå af
\begin{itemize}
  \item en zip-fil, der hedder \texttt{#1\_<navn>.zip}
  (f.eks. \texttt{#1\_jon.zip})
  \item en pdf-fil , der hedder \texttt{#1\_<navn>.pdf}
  (f.eks. \texttt{#1\_jon.pdf})
\end{itemize}
Zip-filen \texttt{#1\_<navn>.zip} skal indeholde en og kun en mappe
\texttt{#1\_<navn>}. I den mappe skal der ligge en \texttt{src} mappe
og filen \texttt{README.txt}.  I \texttt{src} skal der ligge følgende
og kun følgende filer: #2 svarende til hver af delopgaverne.  De skal
kunne oversættes med \lstinline[language=console]{fsharpc}, og de
oversatte filer skal kunne køres med mono.  Funktioner skal
dokumenteres ifølge dokumentationsstandarden som minimum ved brug af
\texttt{<summary>}, \texttt{<param>} og \texttt{<returns>}
XML-tagsne. Filen \texttt{README.txt} skal ganske kort beskrive,
hvordan koden oversættes og køres. Pdf-filen skal indeholde jeres
rapport oversat fra \LaTeX. Husk at pdf-filen skal uploades ved siden af zip-filen på Absalon.}

% \handinTextThreeTwo{exNumber}{listOfHandInFiles}{reportType}
% reportType: 0 - none, comments inline; 1 - readme.txt; 2 - Files/LaTeX/opgave.pdf, 3 - Files/LaTeX/report.pdf
\newcommand{\handinTextThreeTwo}[3]{% 
  Afleveringen skal bestå af:
  \begin{itemize}
  \item en zip-fil, der hedder \texttt{#1\_<(gruppe)navn>.zip} (f.eks. \texttt{#1\_jon.zip})
    \expandafter\ifnumcomp{#3}{>}{1}{%
    \item en pdf-fil , der hedder \texttt{#1\_<(gruppe)navn>.pdf} (f.eks. \texttt{#1\_jon.pdf})%
    }{}%
%    \expandafter\ifstrequal{true}{\item en pdf-fil , der hedder \texttt{#1\_<(gruppe)navn>.pdf} (f.eks. \texttt{#1\_jon.pdf})}{} %
  \end{itemize}
  Zip-filen \texttt{#1\_<(gruppe)navn>.zip} skal indeholde en og kun en mappe \texttt{#1\_<(gruppe)navn>}. I den mappe skal der ligge en \texttt{src} mappe og filen \texttt{README.txt}.

  I \texttt{src} skal der ligge følgende og kun følgende filer:
  \begin{itemize}
    \item #2,
  \end{itemize}
  som beskrevet i opgaveteksten.  Programmerne skal kunne oversættes med \lstinline[language=console]{fsharpc}, og de oversatte filer skal kunne køres med mono.  Funktioner skal dokumenteres ifølge dokumentationsstandarden som minimum ved brug af \texttt{<summary>}, \texttt{<param>} og \texttt{<returns>} XML-tagsne. Filen \texttt{README.txt} skal ganske kort beskrive, hvordan koden oversættes og køres.
  
  \ifnumequal{#3}{0}{Evt.\ kommentarer i forbindelse med opgavebesvarelsen indsættes i kildekoden.}{} %
  \ifnumequal{#3}{1}{README.txt filen skal også inkludere et eller flere få eksempler på kørsler af hvert program, der illustrerer at og hvordan de virker.}{} %
  \ifnumequal{#3}{2}{Pdf-filen skal indeholde jeres rapport ifølge: \begin{itemize}\item\lstinline[language=console]{Absalon->Files->noter->LaTeX->opgave.pdf}\end{itemize}  guiden og oversat fra \LaTeX. Husk at pdf-filen skal uploades ved siden af zip-filen på Absalon.}{} %
  \ifnumequal{#3}{3}{Pdf-filen skal indeholde jeres rapport ifølge: \begin{itemize}\item\lstinline[language=console]{Absalon->Files->noter->LaTeX->rapport.pdf}\end{itemize} guiden og oversat fra \LaTeX. Husk at pdf-filen skal uploades ved siden af zip-filen på Absalon.}{} %
%  \expandafter\ifstrequal\expandafter{#3}{true}{Pdf-filen skal indeholde jeres rapport oversat fra \LaTeX. Husk at pdf-filen skal uploades ved siden af zip-filen på Absalon.}{README.txt filen skal også inkludere et eller nogle få eksempler på kørsler af hvert program, der illustrerer at og hvordan de virker.} %
}

\newcommand{\subAssignment}[1]{\textbf{#1}\\[2mm]}


\newcommand{\handinText}[3]{% 
  Afleveringen skal bestå af:
  \begin{itemize}
  \item en zip-fil, der hedder \texttt{#1\_<(gruppe)navn>.zip} (f.eks. \texttt{#1\_jon.zip})
    \expandafter\ifnumcomp{#3}{>}{1}{%
    \item en pdf-fil , der hedder \texttt{#1\_<(gruppe)navn>.pdf} (f.eks. \texttt{#1\_jon.pdf})%
    }{}%
%    \expandafter\ifstrequal{true}{\item en pdf-fil , der hedder \texttt{#1\_<(gruppe)navn>.pdf} (f.eks. \texttt{#1\_jon.pdf})}{} %
  \end{itemize}
  Zip-filen \texttt{#1\_<(gruppe)navn>.zip} skal indeholde en og kun en mappe \texttt{#1\_<(gruppe)navn>}. I den mappe skal der ligge en \texttt{src} mappe og filen \texttt{README.txt}.

  I \texttt{src} skal der ligge følgende og kun følgende filer:
  \begin{itemize}
    \item #2,
  \end{itemize}
  som beskrevet i opgaveteksten.  Programmerne skal kunne oversættes med \lstinline[language=console]{fsharpc}, og de oversatte filer skal kunne køres med mono.  Funktioner skal dokumenteres ifølge dokumentationsstandarden som minimum ved brug af \texttt{<summary>}, \texttt{<param>} og \texttt{<returns>} XML-tagsne. Filen \texttt{README.txt} skal ganske kort beskrive, hvordan koden oversættes og køres.
  
  \ifnumequal{#3}{0}{Evt.\ kommentarer i forbindelse med opgavebesvarelsen indsættes i kildekoden.}{} %
  \ifnumequal{#3}{1}{README.txt filen skal også inkludere et eller flere få eksempler på kørsler af hvert program, der illustrerer at og hvordan de virker.}{} %
  \ifnumequal{#3}{2}{Pdf-filen skal indeholde jeres rapport ifølge: \begin{itemize}\item\lstinline[language=console]{Absalon->Files->noter->LaTeX->opgave.pdf}\end{itemize}  guiden og oversat fra \LaTeX. Husk at pdf-filen skal uploades ved siden af zip-filen på Absalon.}{} %
  \ifnumequal{#3}{3}{Pdf-filen skal indeholde jeres rapport ifølge: \begin{itemize}\item\lstinline[language=console]{Absalon->Files->noter->LaTeX->rapport.pdf}\end{itemize} guiden og oversat fra \LaTeX. Husk at pdf-filen skal uploades ved siden af zip-filen på Absalon.}{} %
%  \expandafter\ifstrequal\expandafter{#3}{true}{Pdf-filen skal indeholde jeres rapport oversat fra \LaTeX. Husk at pdf-filen skal uploades ved siden af zip-filen på Absalon.}{README.txt filen skal også inkludere et eller nogle få eksempler på kørsler af hvert program, der illustrerer at og hvordan de virker.} %
}


