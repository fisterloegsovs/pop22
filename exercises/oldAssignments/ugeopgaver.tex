\documentclass[a4paper,12pt]{article}

\usepackage[top=2cm,left=25mm]{geometry}
\usepackage[T1]{fontenc}
\usepackage[utf8]{inputenc}
\usepackage[danish]{babel}
\usepackage{amsmath}
\usepackage{enumerate}
\usepackage{listings}
\usepackage{fancyvrb}
\usepackage{graphics,tikz}
\usepackage{hyperref}
\hypersetup{pdftitle={Introduktion til programmering, ugeseddel 1},
            pdfsubject={},
            pdfauthor={},
            pdfkeywords={fsx, rekursion, funktioner},
            pdfborder={0 0 0}}
\usepackage{advdate}
\usepackage{ifthen}
\usepackage{enumitem}
\newcommand{\fs}{\texttt{F\#} }
\newcommand{\fsi}{\texttt{F\# Interactive} }
\newcommand{\fsc}{\texttt{F\# Compiler} }
\newcommand{\opg}[4]{{#1}: {#2};\\\hspace*{1cm} {#3} \ifthenelse{\equal{#4}{}}{}{({#4})}}
\newcommand{\individuel}[3]{\opg{Individuel opgave}{#1}{#2}{#3}}
\newcommand{\gruppe}[3]{\opg{Gruppeopgave}{#1}{#2}{#3}}

\title{Programmering og Problemløsning: Ugeopgaver}
\author{Jon Sporring}
\begin{document}
\maketitle{}

\SetDate[01/09/2015] % Første tirsdag i blok 1, 2015
\begin{enumerate}
\item \AdvanceDate[7] \gruppe{\today}{
    \begin{enumerate}
  \item What can you make with 10
    blocks? a) Use pen and paper, and write a program. b) Simulate the
    computer and describe, what is shown/computed. c) Implement it in
    Scratch and compare. 
  \item Design a game with 2-5 moveable sprites,
    clickable content, hide-show, approximately 1 minutes
    game-time. 
  \item Write report on game in LaTeX: max 3 pages, must
    include the following sections: Introduction (Introduktion),
    Design and program description (Design- og programbeskrivelse),
    Test (afprøvning), Conclusion (Konklusion). Hand-in: game in
    class' group and report as latex and pdf.
  \end{enumerate}
}{Goal: Get started,
    introduction to Scratch and imperative programming (statements,
    variables, loops, boolean expressions, and conditions. minor
    emphasism on: threads, events, and messages), make a program of
    moderate complexity, introduction to the design process,
    introduction to peer review/feedback, make a report in
    latex. Monday: Get startet, upload a program to class
    project, start on 10 block program. Tuesday: Design a game (no
    computer), Friday-Monday: Implement game-test-improve. Tuesday:
    Write report}
\item \AdvanceDate[7] \individuel{\today}{
    \begin{enumerate}
    \item HR: 1.1, 2, 4, 5, 8, 
    \item HR: 2.1, 8, 9, 10, 13. 
    \item Write a report in LaTeX where each exercise is an individual
      subsection, including the program, the result when run, and a
      max 3 line description of the solution. Hand-in: One zip file
      including a single source file for each exercise, that is
      compilable with fsharpc. Naming convention must be,
      \begin{quote}
        \lstinline|<instructor's-initial>_<your-name>_<exericse-number>.fsk|
      \end{quote}
    \end{enumerate}
    }{Goal: Get started with
  fsharp/mono and particularly fsharpc. Introduction to functional programming, Use the automatic code
  correction system. Programming concepts: values/bindings, types, functions,
  recursions, 2-tuples, environment, numbers, booleans, \texttt{unit},
  precedence and associations, characters and strings, operators. }
\item \AdvanceDate[7] \individuel{\today}{HR: 3.1, 2, 4, 5, 6, 7.
    Extra: Skriv en funktion
    \begin{quote}
      \lstinline|solve2 : float * float * float -> float * float|,
    \end{quote}
    sådan at
    \lstinline|solve2 a b c| 
    giver de to løsninger for $x$ i ligningen $a x^2+b x+ c=0$,
    såfremt $b^2-4ac \geq 0$. Du behøver ikke at tage stilling til
    tilfældet $b^2-4 a c < 0.$. Vink: Kvadratrodsfunktionen hedder
     \lstinline|sqrt|
    . }{Goal: lean group work, programming concepts: tuples, records,
    local bindings, invariants, enumeration types, exceptions}
\item \AdvanceDate[7] \gruppe{\today}{HR: 4.1, 4, 9, 13, 17, 22,
    23}{Lists, recursion over lists, polymorphisms, value restriction}
\item \AdvanceDate[7] \individuel{\today}{HR: 5.1, 3, 7, 11}{Programming concepts: Lists, sets, and maps.} % 2 ugers opgave
\item \AdvanceDate[14] \gruppe{\today}{HR: 6.1, 2, 6}{Finite trees,
    tree traversal}
\item \AdvanceDate[7] \gruppe{\today}{På vej}{Unit test}
\item \AdvanceDate[28] \individuel{\today}{HR: 7.1, 4, 5, 7,
    9}{Modules, signature and implementation files, brief introduction
    to classes and objects in fsharp}  % mellemuge 
\item \AdvanceDate[7] \gruppe{\today}{HR: 8.1, 2, 3, 5 }{Imperative
    programming in Fsharp, mutable variables, arrays.}
\item \AdvanceDate[14] \gruppe{\today}{På vej}{Klasser, objekter, design}
\item \AdvanceDate[28] \gruppe{\today}{På vej}{Nedarvning og brugergrænseflader}  %
\item \AdvanceDate[7] \individuel{\today}{På vej}{Opsamlingsprojekt}  %
\end{enumerate}

\end{document}

%%% Local Variables:
%%% mode: latex
%%% TeX-master: t
%%% End:
