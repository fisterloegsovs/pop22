\documentclass[a4paper]{article}

\usepackage[utf8x]{inputenc}
\usepackage{latexsym}
\usepackage[danish]{babel}
\usepackage{graphicx}
\usepackage{color}
\usepackage{hyperref}
\usepackage[all]{hypcap}
\usepackage{enumerate}
\usepackage[margin=2.5cm]{geometry}

\begin{document}
\title{Programmering og Problemløsning\\
Datalogisk Institut, Københavns Universitet\\
Uge(r)seddel 6 -- gruppeopgave}

\author{Revideret 5/10 af Torben Mogensen}
\date{Deadline 20. oktober}

\maketitle

\noindent
I denne periode skal I arbejde grupper.  Formålet er at arbejde
med endelige træer.

Vi arbejder i opgaverne med en træstruktur til at beskrive geometriske
figurer med farver.  For at gøre det muligt at afprøve jeres opgaver,
udleveres et simpelt bibliotek \texttt{makeBMP.dll}, der kan lave
bitmapfiler.  I dette bibliotek findes kun en funktion:

\begin{verbatim}
makeBMP : string -> int -> int -> (int*int -> int*int*int) -> unit
\end{verbatim}

\noindent
Det første argument er navnet på den ønskede bitmapfil (uden
extension).  Det andet argument er bredden af billedet i antal pixel,
det tredje element er højden af billedet i antal pixel og det sidste
argument er en funktion, der afbilder koordinater i billedet til
farver.  En farve er en tripel af tre tal mellem 0 og 255 (begge
inklusive), der beskriver hhv.\ den røde, grønne og blå del af farven.

Koordinaterne starter med $(0,0)$ i nederste venstre hjørne og
$(w-1,h-1)$ i øverste højre hjørne, hvis bredde og højde er hhv.\ $w$
og $h$.  For eksempel vil en programfil \texttt{testBMP.fsx} med
indholdet

\begin{verbatim}
makeBMP.makeBMP "test" 256 256 (fun (x,y) -> (x,y,0))
\end{verbatim}

\noindent
kunne oversættes med kommandoen

\begin{verbatim}
fsharpc testBMP.fsx -r makeBMP.dll -o testBMP.exe
\end{verbatim}

\noindent
til et program, der kan køres med kommandoen \texttt{mono
  testBMP.exe} og lave en billedfil med navnet \texttt{test.bmp},
der indeholder følgende billede:

\begin{center}
\includegraphics{test.png}
\end{center}

\noindent
Bemærk, at alle programmer, der bruger \texttt{makeBMP} skal
oversættes med \texttt{-r makeBMP.dll} som en del af kommandoen.
Tilsvarende skal man bruge \texttt{fsharpi -r makeBMP.dll} for at
bruge \texttt{makeBMP} i det interaktive system.

\textbf{\textcolor{red}{NB!}} Nogle har haft problemer med at bruge
biblioteket \texttt{makeBMP.dll} ved først at oversætte med
\texttt{fsharpc -r makeBMP.dll} og siden \texttt{mono} på den
oversatte fil.  Hvis du får det problem, så prøv at køre
\texttt{fsharpi~-r~makeBMP.dll~testBMP.fsx} eller lignende for dine
egne programmer.  \textbf{\textcolor{red}{Update 7/10:}} En anden
løsning er at finde filen \texttt{FSharp.Core.dll} og kopiere den til
den folder, hvorfra Mono køres.  Så kan Mono finde dette bibliotek.

Bemærk endvidere, at \LaTeX\ ikke kan inkludere BMP-filer med
\texttt{includegraphics}.  Men man kan bruge diverse billedprogrammer
(IrfanView, osv.) til at konvertere BMP til PNG.

I Linux kan man konvertere billeder med kommandoen \texttt{convert}.
Hvis vi har billedfilen \texttt{test.bmp}, vil kommandoen
\texttt{convert~test.bmp~test.png} lave en tilsvarende billedfil
\texttt{test.png} i PNG-formatet (som \LaTeX\ \emph{kan} håndtere).

\vspace{2ex}

\noindent
Vi arbejder i opgaverne med følgende datastruktur:

\begin{verbatim}
type point = int * int // (x, y)
type colour = int * int * int  // (red, green, blue), 0..255

type figure = Circle of point * int * colour
              // center, radius^2, colour
            | Rectangle of point * point * colour
              // bottom-left, top-right, colour
            | Mix of figure * figure
              // combine figures with average colours at overlap
\end{verbatim}

\noindent
Bemærk, at vi angiver kvadratet på radius af cirklen i stedet for blot
 radius.

For eksempel kan man lave følgende funktion til at finde farven af en
 figur i et punkt (forudsat, at punktet ligger i figuren):

\begin{verbatim}
// finds colour of figure at point
let rec colourAt (x,y) = function
  | Circle ((cx,cy), r2, col) ->
      if (x-cx)*(x-cx)+(y-cy)*(y-cy) <= r2
      then Some col else None
  | Rectangle ((x0,y0), (x1,y1), col) ->
      if x0<=x && x <= x1 && y0 <= y && y <= y1
      then Some col else None
  | Mix (f1, f2) ->
      match (colourAt (x,y) f1, colourAt (x,y) f2) with
      | (None, c) -> c
      | (c, None) -> c
      | (Some (r1,g1,b1), Some (r2,g2,b2)) ->
           Some ((r1+r2)/2, (g1+g2)/2, (b1+b2)/2)
\end{verbatim}

\vspace{1ex}

\noindent
Opgaverne i denne uge er delt i øve- og afleveringsopgaver.

\subsubsection*{Øveopgaverne er:}

\begin{description}

\item[Ø6.1.] Lav en figur af typen \texttt{figure}, der består af en
  rød cirkel med centrum i (50,50) og radius $\sqrt{2000}$ samt en blå
  rektangel med hjørnerne (40,40) og (90,110).

\item[Ø6.2.] Brug \texttt{makeBMP} og \texttt{colourAt} til at lave en
  funktion

\texttt{makePicture : string -> figure -> int -> int -> unit}

\noindent
sådan at kaldet \texttt{makePicture \emph{filnavn figur b h}} laver en
billedfil ved navn \texttt{\emph{filnavn}.bmp} med et billede af
\texttt{\emph{figur}} med bredde \texttt{\emph{b}} og højde
\texttt{\emph{h}}.

På punkter, der ingen farve har (jvf.\ \texttt{colourAt}), skal farven
være grå (som defineres med RGB-værdien (128,128,128)).

Du kan bruge denne funktion til at afprøve dine opgaver.

\item[Ø6.3.] Lav med \texttt{makePicture} en billedfil med navnet
  \texttt{63.bmp} og størrelse $100\times150$ (bredde 100, højde 150),
  der viser figuren fra opgave Ø6.1.

\item[Ø6.4.] Lav en funktion \texttt{checkFigure : figure -> bool},
  der undersøger, om en figur er korrekt: At kvadratradiusen i cirkler
  er ikke.negativ, at nederste venstre hjørne i en rektangel faktisk
  er nedenunder og til venstre for det øverste højre hjørne (bredde og
  højde kan dog godt være 0), og at farvekompenterne ligger mellem 0
  og 255.

\item[Ø6.5.] Lav en funktion \texttt{move : figure -> int * int ->
  figure}, der givet en figur og en vektor flytter figuren langs
  vektoren.

\item[Ø6.6.] Lav en funktion \texttt{boundingBox : figure -> point *
  point}, der givet en figur finder hjørnerne (bund-venstre og
  top-højre) for den mindste akserette rektangel, der indeholder hele
  figuren.  \textbf{Vink:} Man kan runde et kommatal $x$ op til det
  nærmeste heltal med \texttt{int (System.Math.Ceiling $x$)}.

\item[Ø6.7.] Lav en funktion \texttt{scale : figure * real -> figure},
  der skalerer en figur med en skala angivet som et positivt kommatal.
  Alle værdier i den nye figur skal afrundes til nærmeste heltal.  NB!
  Husk at cirkler angiver kvadratet på radius!.  Hvis skalaen er $\leq
  0.0$, skal en passende fejlmeddelse gives.  \textbf{Vink:} Man kan
  runde et kommatal $x$ af til det nærmeste heltal med \texttt{int
    (System.Math.Round $x$)}.

\item[Ø6.8.] Lav en funktion \texttt{flipVertical : figure * int ->
  figure}, der givet en figur og et $x$, spejler figuren i den
  lodrette akse omkring $x$.  NB! Pas på hjørnerne af rektangler.

\end{description}

\subsubsection*{Afleveringsopgaven er:}

 Vi udvider typen \texttt{figure} til en ny type \texttt{figure'} med
 følgende definition:

\begin{verbatim}
type figure' = Circle of point * int * colour
               // center, radius^2, colour
             | Rectangle of point * point * colour
               // bottom-left, top-right, colour
             | Mix of figure' * figure'
               // combine figures with average colours at overlap
             | Ellipse of point * point * int * colour
               // Two focal points, great axis, colour
\end{verbatim}

\noindent
Som tilføjer ellipser givet ved to brændpunkter og en storakse.  Et
punkt er i en ellipse, hvis \emph{summen} af afstandende mellem
punktet og de to brændpunkter er mindre end eller lig med storaksen.
Se evt.\ \url{https://da.wikipedia.org/wiki/Ellipse_%28geometri%29}.


\begin{description}

\item[A6.1.] Lav en figur \texttt{twoEllipses : figure'}, som består
  af en blå ellipse med brændpunkter i (10,10) og (80,80) og storakse
  110 samt en gul ellipse med brændpunkter i (20,60) og (40,15) og
  storakse 70.

\item[A6.2.] Lav funktioner \texttt{colourAt'}, \texttt{makePicture'},
  \texttt{checkFigure'}, \texttt{move'} og \texttt{scale'} svarende
  til de tilsvarende funktioner for \texttt{figure}.

Bemærk, at i en korrekt ellipse skal storaksen være mindst lige
så stor som afstanden mellem brændpunkterne.

\end{description}

\noindent
Afleveringsopgaven skal afleveres som både \LaTeX, den genererede PDF,
samt en fsx fil med løsningen for hver delopgave, navngivet efter
opgaven (f.eks.\ \texttt{A6-1.fsx}), som kan oversættes med
fsharpc, og hvis resultat kan køres med mono.  Det hele samles i en
zip-fil med sædvanlig navnekonvention (se tidligere ugesedler).


\vspace{1ex}

\hfill God fornøjelse

\section*{Ugens nød 3}

Vi vil i udvalgte uger stille særligt udfordrende og sjove opgaver,
som interesserede kan løse.  Det er helt frivilligt at lave disse
opgaver, som vi kalder ``Ugens nød'', men der vil blive givet en
mindre præmie til den bedste løsning, der afleveres i Absalon.

Denne uges opgave omhandler billeder.  Opgaven går i sin enkelhed ud
på at bruge funktionen \texttt{makeBMP} til at lave det flotteste
billede på $512\times512$ pixels.
Instruktorer og undervisere vil fungere som dommere og udkåre det flotteste billede.

Den eneste begrænsning er, at programmet højest må være på 20 linjer
på hver maksimalt 80 tegn.  Blanke linjer og kommentarlinjer tæller
ikke med.

Du skal aflevere et billede, det program, der blev brugt til at lave
billedet, og en rapport, der beskriver, hvordan billedet er lavet.

Programmet skal kunne oversættes med kommandoen

\texttt{fsharpc -r makeBMP.dll \emph{filnavn}.fsx}

og kunne køre i mono, eller køre med kommandoen

\texttt{fsharpi -r makeBMP.dll \emph{filnavn}.fsx}



Dit program må ikke bruge andre imperative
features end \texttt{makeBMP}-funktionen.

\end{document}
