% Modificeret udgave af ugesedlen fra 2012
% Ændringer:
%  * Opdeling af tirsdagsopgaver i to så der er også er stof
%    til fredagsøvelserne
%  * Tilføj to halerekursionsopgaver til fredagsøvelserne
%  * Oversæt alle opgaver til at bruge engelske funktionsnavne for at
%  holde navngivningskonventionerne vi har brugt de foregående uger.
% * Ændring af valgfrie opgaver til dem der har mere tid

\documentclass[a4paper]{article}

\title{Introduktion til programmering, ugeseddel 6}
\author{Version 1.0}
\date{3.\ oktober 2014}

\usepackage{cmap}
\usepackage[top=2cm,left=25mm]{geometry}
\usepackage[T1]{fontenc}
\usepackage[utf8]{inputenc}
\usepackage[danish]{babel}
%\usepackage{microtype}
\hyphenpenalty=750

% \usepackage{amsmath}
% %\usepackage{lmodern}
% %\usepackage{mathptmx}
% %\usepackage{libertine}
% %\usepackage[scaled=0.90]{inconsolata}
% %\usepackage[scaled=0.83]{beramono}

\usepackage{enumerate}
% \usepackage{graphics,graphpap}
% \usepackage{listing}
% \usepackage{fancyvrb}
% \usepackage{graphics,tikz}
% \usepackage{listings}
% \usepackage{framed}
% \usepackage{upquote}

\setlength{\parskip}{2ex}
\setlength{\parindent}{0pt}
\setlength{\parfillskip}{30pt plus 1 fil}

\usepackage{sudoku}

\setlength\sudokusize{7cm}
% \renewcommand*\sudokuformat[1]{\sffamily #1}

\newenvironment{program}{\begin{flushleft}\ttfamily\begin{tabbing}}%
{\end{tabbing}\end{flushleft}}

\begin{document}

\maketitle

Uge 6 drejer sig om \emph{ind- og udl{\ae}sning} (p{\aa} engelsk \emph{input/output}, ofte forkortet I/O) af tekst
samt \emph{halerekursion}.  Data til og fra vores programmer er hidtil optr{\aa}dt direkte i afviklingsvinduet for Mosml;
vi skal nu se, hvordan man fra et program kan komme i forbindelse med computerens filsystem.
Begrebet ``halerekursion'' d{\ae}kker over en bestemt m{\aa}de at
definere funktioner p{\aa}, s{\aa}ledes at det rekursive kald til funktionen selv er det \emph{sidste}, som sker i funktionskroppen. Halerekursivt definerede funktioner
er ofte (men ikke altid) mere effektive end funktioner, som ikke er.

Ugens opgavetema drejer sig om funktioner til udfyldelse af sudoku-puslespil med anvendelse
af filer. Opgaverne tr{\ae}ner dels ind- og udl{\ae}sning,
 dels er de oplagte til repetition af h{\o}jereordensfunktioner og rekursion i almindelighed.

\section*{Plan for ugen}
\label{sec:pensum-og-plan}

\subsubsection*{Mandag}
Input/Output, interaktive programmer og bivirkninger (også kaldet sideeffekter).

\textit{Pensum:} HR: kap.\ 14 og kap.\ 15

\subsubsection*{Tirsdag}
Mere om Input/Output, halerekursion, \textit{records} og \textit{record}-typer.

\textit{Pensum:} HR: 3.4-3.6 og kap.\ 17 (samt evt. IP-2: 7.7)

\subsubsection*{Fredag}
Repetition af ugens pensum (v. Torben Mogensen)

\newpage
\section{Opgavetema}\label{sec:opgavetema}

Temaet for ugens opgaver er at programmere et lille sudoku-spil.

\emph{Sudoku} er et puslespil, som er blevet opfundet uafh{\ae}ngigt
flere gange; den tidligste ``{\ae}gte'' version af sudoku synes at kunne spores
tilbage til det franske dagblad \emph{Le Si{\`e}cle} i 1892. Den moderne udgave
har fundet vej til adskillige aviser p{\aa} verdensplan i l{\o}bet af de sidste 7 {\aa}r.

Vi betragter her kun den basale variant, som spilles p{\aa} en matrix af 81
sm{\aa} felter, arrangeret i 9 r{\ae}kker og 9 s{\o}jler.  Matricen er desuden
inddelt i 9 ``bokse'' eller ``regioner'', hver med 3 gange 3 felter.

Nogle af felterne er udfyldt p{\aa} forh{\aa}nd, og puslespillet g{\aa}r ud p{\aa}
at udfylde de resterende felter p{\aa} en s{\aa}dan m{\aa}de, at hver af de
9 r{\ae}kker, hver af de 9 s{\o}jler og hver af de 9 regioner kommer til at
indeholde en permutation af symbolerne fra et forelagt alfabet af st{\o}rrelse 9;
vi v{\ae}lger her (som man plejer at se det) alfabetet best{\aa}ende af cifrene fra 1 til 9.

Her er en lovlig starttilstand for et spil sudoku:

\begin{sudoku}
|5|3| | |7| | | | |.
|6| | |1|9|5| | | |.
| |9|8| | | | |6| |.
|8| | | |6| | | |3|.
|4| | |8| |3| | |1|.
|7| | | |2| | | |6|.
| |6| | | | |2|8| |.
| | | |4|1|9| | |5|.
| | | | |8| | |7|9|.
\end{sudoku}

Og f{\o}lgende er en vindende tilstand (en ``l{\o}sning'') af ovenst{\aa}ende:

\begin{sudoku}
|5|3|4|6|7|8|9|1|2|.
|6|7|2|1|9|5|3|4|8|.
|1|9|8|3|4|2|5|6|7|.
|8|5|9|7|6|1|4|2|3|.
|4|2|6|8|5|3|7|9|1|.
|7|1|3|9|2|4|8|5|6|.
|9|6|1|5|3|7|2|8|4|.
|2|8|7|4|1|9|6|3|5|.
|3|4|5|2|8|6|1|7|9|.
\end{sudoku}

\paragraph{Nummerering}

Lad os nummerere r{\ae}kkerne 0, 1, \ldots, 8 ovenfra og ned og s{\o}jlerne
0, 1, \ldots, 8 fra venstre mod h{\o}jre.  Ogs{\aa} regionerne vil vi
nummerere 0, 1, \ldots, 8, i ``normal l{\ae}seretning'' (for vestlige sprog).
Sammenh{\ae}ngen mellem r{\ae}kkenummer $r$,
s{\o}jlenummer $s$ og regionsnummer $q$ kunne s{\aa} udtrykkes i f{\o}lgende
formel:  Feltet i r{\ae}kke nummer $r$ og s{\o}jle nummer $s$
vil ligge i region nummer
\[q = \texttt{$r$ div 3 * 3 + $s$ div 3}\]

Omvendt vil region nummer $q$ best{\aa} af de felter,\newline
som b{\aa}de ligger i en af r{\ae}kkerne
\texttt{$q$ div 3 * 3}, \texttt{$q$ div 3 * 3 + 1}
eller \texttt{$q$ div 3 * 3 + 2}\newline
og en af s{\o}jlerne
\texttt{$q$ mod 3 * 3}, \texttt{$q$ mod 3 * 3 + 1} eller \texttt{$q$ mod 3 * 3 + 2}.

Vi vil lave et simpelt SML-program, som kan indl{\ae}se en starttilstand
for et sudokuspil, som kan pr{\ae}sentere den nuv{\ae}rende tilstand visuelt
for brugeren, og som tillader en bruger at udfylde felter, der ikke allerede er
fyldt ud.

\paragraph{Filformat}

En starttilstand er givet ved en fil, hvis indhold \emph{altid}
antages at se ud som f{\o}lger:

\begin{itemize}

\item Der er mindst 90 tegn i filen (der kan v{\ae}re flere, men vi er kun interesseret i de 90 f{\o}rste)

\item De f{\o}rste 90 tegn i filen er delt op i 9 grupper,
hver best{\aa}ende af 10 tegn:
F{\o}rst 9 tegn, som er et blandt
\texttt{\#"1"}, \ldots, \texttt{\#"9"}, \texttt{\#"*"},
og til sidst tegnet \texttt{\#"$\backslash$n"}.

\end{itemize}

F.eks.\ er nedenst{\aa}ende (NB! Bliv ej bange, hvis det er sv{\ae}rt at l{\ae}se)
indholdet i en fil, som indeholder starttilstanden for ovenst{\aa}ende sudoku:

\begin{center}
{\footnotesize
\begin{verbatim}
53**7****\n6**195***\n*98****6*\n8***6***3\n4**8*3**1\n7***2***6\n*6****28*\n***419**5\n****8**79\n
\end{verbatim}
}
\end{center}

Hvis vi fortolker tegnet \texttt{\#"$\backslash$n"} som ``ny linje'',
bliver ovenst{\aa}ende lettere at l{\ae}se:

\begin{center}
\begin{verbatim}
53**7****
6**195***
*98****6*
8***6***3
4**8*3**1
7***2***6
*6****28*
***419**5
****8**79
\end{verbatim}
\end{center}

I hele ugesedlen vil vi skrive indholdet af filer som ovenfor,
idet tegnet \texttt{\#"$\backslash$n"} vil blive fortolket som ``ny linje''.

\paragraph{Repr{\ae}sentation i SML}

Principielt burde et sudoku-spil repr{\ae}senteres ved en passende datatype i ML.
For at holde jeres opgavebesvarelser relativt begr{\ae}nsede i deres syntaks,
vil vi dog i stedet her bruge ``r{\aa}'' lister til repr{\ae}sentationen.

En given tilstand af et sudoku-spil vil blive repr{\ae}senteret
som en liste af ni lister, som alle indeholder ni tegn---alts{\aa}
som en speciel v{\ae}rdi af type \texttt{char list list}.

\newpage
Det gennemg{\aa}ende eksempel ovenfor vil for eksempel
blive repr{\ae}senteret som f{\o}lgende liste:
\begin{center}
{\small
\begin{verbatim}
[[#"5", #"3", #"*", #"*", #"7", #"*", #"*", #"*", #"*"],
 [#"6", #"*", #"*", #"1", #"9", #"5", #"*", #"*", #"*"],
 [#"*", #"9", #"8", #"*", #"*", #"*", #"*", #"6", #"*"],
 [#"8", #"*", #"*", #"*", #"6", #"*", #"*", #"*", #"3"],
 [#"4", #"*", #"*", #"8", #"*", #"3", #"*", #"*", #"1"],
 [#"7", #"*", #"*", #"*", #"2", #"*", #"*", #"*", #"6"],
 [#"*", #"6", #"*", #"*", #"*", #"*", #"2", #"8", #"*"],
 [#"*", #"*", #"*", #"4", #"1", #"9", #"*", #"*", #"5"],
 [#"*", #"*", #"*", #"*", #"8", #"*", #"*", #"7", #"9"]]
\end{verbatim}
}
\end{center}

\paragraph{{\AE}ndring af et sudoku-spils tilstand}

Vi kan opfatte et felt p{\aa} et sudokubr{\ae}t som givet ved to koordinater:
\emph{r{\ae}kken} $r$ og \emph{s{\o}jlen} $s$, som vi kan skrive som et koordinatpar
$(r,s)$. (Vi vedtog ovenfor at t{\ae}lle r{\ae}kker og s{\o}jler fra {\o}verste venstre hj{\o}rne
og at bruge numrene $0, 1,\ldots, 8$.)

Hvis vi for eksempel {\o}nsker at skrive et $7$-tal i
det tomme felt p{\aa} koordinat $(r,s) = (1,2)$ i vores eksempelsudoku,
f{\aa}r vi alts{\aa} nedenst{\aa}ende (hvor vi har {\it kursiveret} $7$-tallet):

\begin{sudoku}
|5|3| | |7| | | | |.
|6| |{\it 7}|1|9|5| | | |.
| |9|8| | | | |6| |.
|8| | | |6| | | |3|.
|4| | |8| |3| | |1|.
|7| | | |2| | | |6|.
| |6| | | | |2|8| |.
| | | |4|1|9| | |5|.
| | | | |8| | |7|9|.
\end{sudoku}


Bem{\ae}rk, at i ML-repr{\ae}sentationen af sudoku-tilstanden som en liste af lister svarer ovenst{\aa}ende inds{\ae}ttelse til,
at man i liste nummer 1 p{\aa} plads nummer 2 (med 0-baseret nummerering)
erstatter tegnet
\texttt{\#"*"} med tegnet \texttt{\#"7"}, s{\aa}ledes at vi f{\aa}r:

\begin{center}
{\small
\begin{verbatim}
[[#"5", #"3", #"*", #"*", #"7", #"*", #"*", #"*", #"*"],
 [#"6", #"*", #"7", #"1", #"9", #"5", #"*", #"*", #"*"],
 [#"*", #"9", #"8", #"*", #"*", #"*", #"*", #"6", #"*"],
 [#"8", #"*", #"*", #"*", #"6", #"*", #"*", #"*", #"3"],
 [#"4", #"*", #"*", #"8", #"*", #"3", #"*", #"*", #"1"],
 [#"7", #"*", #"*", #"*", #"2", #"*", #"*", #"*", #"6"],
 [#"*", #"6", #"*", #"*", #"*", #"*", #"2", #"8", #"*"],
 [#"*", #"*", #"*", #"4", #"1", #"9", #"*", #"*", #"5"],
 [#"*", #"*", #"*", #"*", #"8", #"*", #"*", #"7", #"9"]]
\end{verbatim}
}
\end{center}

S{\aa}: \emph{At skrive et tal p{\aa} koordinat} $(r,s)$ \emph{i sudokuen svarer til at erstatte et tegn i liste} $r$, \emph{plads} $s$ \emph{med et andet tegn}.

\subsection{Filers adresser}\label{sec:filadresser}

Som bekendt er en computers kataloger (ogs{\aa} kaldet ``mapper'', ''foldere'' eller ``directories'')
og filer ordnet i en tr{\ae}struktur, og n{\aa}r man arbejder med maskinen, refererer
man til mapper og filer ud fra et bestemt punkt (``arbejdskataloget'') i denne struktur.
Adressen p{\aa} (referencen til) et katalog eller en fil kaldes ogs{\aa} for
katalogets eller filens \emph{sti}.

Stier, der ikke indledes med en skr{\aa}streg,
opfattes \emph{relativt til arbejdskataloget} og kaldes ogs{\aa} for
\emph{relative stier}.  Refererer man for eksempel til \texttt{navn1},
betyder det et katalog eller en fil direkte under ar\-bejds\-ka\-ta\-log\-et,
og \texttt{navn1/navn2} betyder filen eller kataloget \texttt{navn2}
indeholdt i kataloget \texttt{navn1} indeholdt i arbejdskataloget.
(Under DOS og Windows skal skr{\aa}stregen vende den anden vej,
men n{\aa}r Moscow ML fortolker en tekst som en sti, s{\o}rges
der for, at skilletegnet bliver det rigtige!)

Stier kan ogs{\aa} v{\ae}re \emph{absolutte}, hvilket vil sige,
at de indledes med en skr{\aa}streg og bev{\ae}ger sig ned gennem
katalogsystemet fra katalogtr{\ae}ets \emph{rod}.  Moscow ML's fortolker kunne
for eksempel have den absolutte sti \texttt{/Users/nina/mosml/bin/mosml},
og ens katalog med kursusmateriale til IP kunne have den absolutte sti
\texttt{/Users/nina/Documents/studium/ip/}. Under Windows skal stier startes med et drevbogstav og et kolon, eksempelvis \texttt{c:}.

\paragraph{I ordrer til operativsystemet}
kan man bruge to nyttige ``forkortelser''
under arbejde med stier:  Et enkelt punktum (\texttt{.}) betyder det aktuelle
katalog, og to punktummer (\texttt{..}) angiver kataloget et niveau
\emph{opad} i katalogtr{\ae}et.  I forts{\ae}ttelse af det ovenn{\ae}vnte
eksempel ville man alts{\aa} fra sit katalog med IP-kursusmateriale kunne
aktivere Moscow ML med ordren \texttt{../../../mosml/bin/mosml}

\paragraph{Fra Moscow ML}
er skrivem{\aa}den med punktummerne desv{\ae}rre ikke til r{\aa}dighed,
s{\aa} n{\aa}r man fra et program skal referere til en fil, \emph{m{\aa}
det normalt ske via filens absolutte sti!}  (Indledes stien ikke med
en skr{\aa}streg, s{\aa}dan at man alts{\aa} bruger en \emph{relativ} sti,
vil henvisningen blive fortolket \emph{relativt til det katalog,
som Moscow ML-fortolkeren selv ligger i!})

Oplysningerne her skulle v{\ae}re tilstr{\ae}kkelige til l{\o}sning
af de opgaver, man m{\o}der p{\aa} kurset, men der kan v{\ae}re
grund til at g{\o}re opm{\ae}rsom p{\aa}, at biblioteket \texttt{OS}
indeholder to strukturer \texttt{OS.FileSys} og \texttt{OS.Path}
med en lang r{\ae}kke hj{\ae}lpefunktioner til h{\aa}ndtering af
kataloger og stier.  (I Mosml er disse strukturer ogs{\aa}
tilg{\ae}ngelige som de selvst{\ae}ndige biblioteker \texttt{FileSys}
og \texttt{Path}.)

I {\o}vrigt henvises til hele standardbiblioteket for SML, det s{\aa}kaldte
\emph{Standard ML Basis Library} med adressen
\texttt{http://mosml.org/mosmllib/}

\newpage
\section{Mandagsopgaver}

\begin{enumerate}[{6}M1]

\item{Benyt \texttt{stdOut} til at udskrive teksten
"Denne tekst indeholder ikke et linjeskift" p{\aa} sk{\ae}rmen.

Gentag ovenst{\aa}ende, men s{\o}rg denne gang for, at der kommer et linjeskift
efter teksten.}

\item{Definer en funktion \texttt{lssuffix :\ string -> string}, som
f{\o}jer et linjeskift til en tekst. Kaldet
\texttt{lssuffix "Simonsen"} skal for eksempel
returnere teksten \texttt{"Simonsen$\backslash$n"}.

Definer dern{\ae}st en funktion \texttt{lssuffixout :\ string -> unit},
som f{\o}jer et linjeskift til en tekst og dern{\ae}st skriver teksten
ud p{\aa} sk{\ae}rmen.

Hvorfor er returtypen for \texttt{lssuffixout} den s{\ae}rlige type
\texttt{unit}?}

\item{Indtast i et SML-afviklingsvindue f{\o}lgende linjer:
\begin{program}
val outstrm = TextIO.openOut "udgangsfil.txt";\\
TextIO.output (outstrm, "Dette er en pr{\o}ve!$\backslash$n");\\
TextIO.flushOut outstrm;
\end{program}
og {\aa}bn derefter den dannede fil i et passende tekstbehandlingsprogram
(for eksempel GEdit, Notesblok, TextEdit eller Emacs) for at kontrollere, at teksten
faktisk er blevet skrevet i den.  (Se afsnit~\ref{sec:filadresser} ovenfor om
filers adresser.)

Afh{\ae}ngigt af, hvordan tegntabellerne er sat op p{\aa} maskinen, kan det v{\ae}re
n{\o}dvendigt med s{\ae}rbehandling af de danske bogstaver, for eksempel med
\begin{program}
TextIO.output (outstrm, "Dette er en pr$\backslash$248ve!$\backslash$n");
\end{program}
Pr{\o}v at skrive noget mere ud til filen, og bem{\ae}rk, hvordan tekst f{\o}rst
kommer til syne i tekst\-be\-hand\-lings\-sy\-ste\-mets vindue efter \texttt{TextIO.flushOut}.
M{\aa}ske skal tekst\-vin\-du\-et genindl{\ae}ses (eller lukkes og
gen{\aa}bnes).  Husk til sidst at skrive
\begin{program}
TextIO.closeOut outstrm;
\end{program}}

\item{Opret p{\aa} din maskine med et passende tekstbehandlingsprogram
en fil med navnet \texttt{indgangsfil.txt}. Tast 3--5 linjers tekst ind, og gem filen
(som r{\aa} eller ``flad'' tekst, eng. \emph{plaintext}).

Opret nu en indgangsstr{\o}m til filen med
\begin{program}
val instrm = TextIO.openIn "indgangsfil.txt";
\end{program}
(Se afsnit~\ref{sec:filadresser} ovenfor om filers adresser.)

Fors{\o}g nu at l{\ae}se f{\o}rste linje i filen med indgangsstr{\o}mmen.

Forts{\ae}t med kald til \texttt{TextIO.inputLine}, til alle linjer i filen er l{\ae}st.
Husk til sidst at skrive
\begin{program}
TextIO.closeIn instrm;
\end{program}}
\end{enumerate}

\newpage
\section{Tirsdagsopgaver}

Du forventes at have forberedt l{\o}sninger til nedenst{\aa}ende til f{\o}rste {\o}velsestime om tirsdagen.

Brug gerne h{\o}jereordensfunktioner, hvis opportunt og koncist.

\begin{enumerate}[{6}T1]

\item{Definer en funktion \texttt{listLines :\ string -> string list},
som danner listen best{\aa}ende af hver linje i en fil (uden linjernes afsluttende linjeskifttegn).
Hvis \texttt{filnavn} er en v{\ae}rdi af type \texttt{string}, som repr{\ae}senterer
et filnavn, da skal kaldet \texttt{listLines filnavn} alts{\aa} returnere
en liste af de linjer, som findes i \texttt{filnavn}.

Hvis \texttt{filnavn} for eksempel har f{\o}lgende indhold:

\begin{program}
Der er en Trolddom p{\aa} din L{\ae}be\\
Der er en Afgrund i dit Blik\\
Der er i Lyden af din Stemme\\
En Dr{\o}ms {\AE}teriske Musik
\end{program}

Da skal kaldet \texttt{listLines filnavn} returnere v{\ae}rdien

\begin{program}
[\="Der er en Trolddom p{\aa} din L{\ae}be", "Der er en Afgrund i dit Blik",\+\\
"Der er i Lyden af din Stemme", "En Dr{\o}ms {\AE}teriske Musik"]
\end{program}

\emph{Pas p{\aa}!} Det sidste tegn i filen beh{\o}ver ikke at v{\ae}re \texttt{\#"$\backslash$n"}
(filen kan stoppe brat ``midt'' p{\aa} en linje).}

\item{Definer en funktion \texttt{listChars :\ string -> char list list},
som for hver linje i filen danner listen af tegn, som findes i hver
linje. Kaldet \texttt{listChars filnavn}
(hvor \texttt{filnavn} er givet som ovenfor)
skal for eksempel returnere f{\o}lgende:
{\footnotesize
\begin{program}
[[\#"D", \#"e", \#"r", \#" ", \#"e", \#"r", \#" ", \#"e", \#"n", \ldots, \#"L", \#"{\ae}", \#"b", \#"e" ],\\
{} [\#"D", \#"e", \#"r", \#" ", \ldots,  \#"B", \#"l", \#"i", \#"k"],\\
\ \ldots\\
{} [\#"E", \#"n", \#" ", \#"D", \ldots,  \#"s", \#"i", \#"k"]]
\end{program}}
}

\item Skriv et program, som tillader brugeren at indtaste et tegn ad gangen
(vha.\ \texttt{stdIn}). Hvis brugeren trykker p{\aa} tasten \emph{Q} (i ML-notation:
indtaster tegnet \texttt{\#"q"}) (efterfulgt af linjeskift),
standser programmet.
Hvis brugeren trykker p{\aa} en hvilken som helst anden grafisk tast (efterfulgt af linjeskift),
beder programmet om et nyt tegn
(ved at skrive en passende besked til \texttt{stdOut}).

Vink I: Hvis der ikke skrives noget ud p{\aa} sk{\ae}rmen, har I m{\aa}ske glemt
at t{\o}mme en buffer.

Vink II: Husk, at semikolonnet \texttt{;} tillader at udf{\o}re en handling (med bivirkning)
efter en anden.

Vink III: Hvis programmet skal sp{\o}rge om noget mere end en gang, kunne man m{\aa}ske bruge den
mest fundamentale teknik p{\aa} dette kursus \ldots
\end{enumerate}

\section{Fredagsopgaver}

Du forventes at have forberedt l{\o}sninger til nedenst{\aa}ende til f{\o}rste {\o}velsestime om fredagen.

\begin{enumerate}[{6}F1]
\item HR 17.3
\item HR 17.2 (løs evt. HR 1.4 på normal vis før du går i gang)

\end{enumerate}


\section{Gruppeaflevering}
Bem{\ae}rk, at de fire f{\o}rste opgaver er obligatoriske.
Opgaven afleveres i Absalon.  Der afleveres en fil pr.\ gruppe, men
den skal angive alle deltageres fulde navne i kommentarlinjer øverst i
filen. Filens navn skal være af formen
\texttt{6G-\textit{initialer}.sml}, hvor initialer er erstattet af
gruppemedlemmernes initialer. Hvis f.eks.\ Bill~Gates, Linus~Torvalds,
Steve~Jobs og Gabe~Logan~Newell afleverer en opgave sammen, skal filen
hedde \texttt{6G-BG-LT-SJ-GLN.sml}. Brug gruppeafleveringsfunktionen i
Absalon.

Gruppeopgaven giver op til 2 point, som tæller til de 20 point, der
kræves for eksamensdeltagelse.  Genaflevering kan hæve pointtallet fra
første aflevering med højest 1 point, så sørg for at gøre jeres bedste
allerede i første aflevering.

Gruppeopgaverne g{\aa}r ud p{\aa} at programmere et lille system, som kan indl{\ae}se
en sudoku-starttilstand og derefter gentagent bede brugeren om at
indtaste koordinater til at opdatere spillets tilstand.

\emph{L{\ae}s beskrivelsen af ugens opgavetema, f{\o}r i g{\aa}r i gang}.

Brug gerne halerekursive funktioner og h{\o}jereordensfunktioner, hvor I finder det anvendeligt.


\begin{enumerate}[{6}G1]

\item{Definer en funktion \texttt{readSudoku :\ string -> char list list},
s{\aa}ledes at hvis \texttt{filnavn} er navnet p{\aa} en fil, som indeholder
en sudoku-tilstand, da returnerer et kald \texttt{readSudoku filnavn}
den liste af lister af tegn, som repr{\ae}senterer sudoku-tilstanden
(l{\ae}s beskrivelsen af ugens opgavetema igen, hvis dette virker forvirrende).

Vink: Tirsdags{\o}velserne indeholder essentielt en l{\o}sning af denne opgave.}

\item{Definer en funktion \texttt{showSudoku :\ char list list -> unit},
som skriver en sudoku-tilstand ud p{\aa} sk{\ae}rmen.

Vink: En tilstand er en liste af lister. Skriv hver liste ud, en ad gangen. Husk linjeskift.}

\item{Definer en funktion
\texttt{modifySudoku :\ char list list -> int * int * char -> char list list},
s{\aa}\-le\-des at hvis \texttt{r} og \texttt{s}
begge er v{\ae}rdier (i intervallet mellem \texttt0 og \texttt8) af typen \texttt{int},
og \texttt{xs} repr{\ae}senterer en sudoku-tilstand,
da returnerer kaldet \texttt{modifySudoku xs (r,s,c)} en ny sudokutilstand
\texttt{ys}, som er resultatet af at erstatte tegnet i r{\ae}kke \texttt{r},
plads \texttt{s} i \texttt{xs} med tegnet \texttt{c}.

I eksemplet fra afsnittet ``{\AE}ndring af sudokuspillets tilstand'' ovenfor
skal man alts{\aa} for eksempel udf{\o}re kaldet \texttt{modifySudoku xs (1,2,\#"7")}.

% TODO: skal de bare ignorer invalidt input? alstå r/s udenfor [0,8] ?
%       tror mange gerne vil rejse undtagelser

\emph{Husk: Funktionen skal have den angivne type!}}

\item{(Denne opgave kan tjene som inspiration for, eller v{\ae}re direkte brugbar i,
ugens individuelle opgaver).

Definer en funktion \texttt{regionList :\ char list list -> int -> char list}, s{\aa}
kaldet\newline
\texttt{regionList xs n} returnerer listen af tegn, som forekommer
i den \texttt{n}'te region af \texttt{xs}, hvor \texttt{xs} repr{\ae}senterer
en sudokutilstand. I m{\aa} selv om, hvilken r{\ae}kkef{\o}lge tegnene angives i.

% TODO: skal de bare ignorer invalidt input? alstå n udenfor [0,8] ?
%       tror mange gerne vil rejse undtagelser

Hvis \texttt{xs} er listen af lister af tegn, som repr{\ae}senterer starttilstanden i det gennemg{\aa}ende eksempel, da kan kaldet \texttt{regionList xs 3}
returnere listen \texttt{[\#"8", \#"*", \#"*", \#"4", \#"*", \#"*", \#"7", \#"*", \#"*"]}.}


\item{\label{opg-dialog}%
(Valgfri --- \textbf{men st{\ae}rkt anbefalet} --- ekstraopgave) Skriv et program, som tillader en bruger
i dialog med sk{\ae}rmen at angive et filnavn, som indeholder en sudoku-starttilstand, og som dern{\ae}st bliver ved at sp{\o}rge brugeren, hvilke koordinater, som der
skal skrives p{\aa}.

Brugeren skal kunne indtaste koordinater og {\ae}ndringer i formatet

\texttt{r s c}

(afsluttet med linjeskift).

N{\aa}r en {\ae}ndring er foretaget, skal funktionen skrive den ny tilstand
ud p{\aa} sk{\ae}rmen, s{\aa} brugeren kan se, hvad den nuv{\ae}rende tilstand er.

Funktionen skal dern{\ae}st tillade nye {\ae}ndringer.

Der er ingen krav om, at programmet skal kunne afg{\o}re, om spillet er f{\ae}rdigt, eller skal forhindre brugeren i at overskrive allerede skrevne v{\ae}rdier.

Vink: L{\ae}s eventuelt HR, kap. 16.}

\item{(Valgfri ekstraopgave) Udvid programmet fra opgave~6G\ref{opg-dialog},
s{\aa} det ikke er muligt at overskrive talv{\ae}rdier fra spillets \emph{start}tilstand
(men til geng{\ae}ld at overskrive tidligere v{\ae}rdier, som brugeren selv har angivet).}

\end{enumerate}



\section{Individuel aflevering}
Besvarelsen afleveres i Absalon som en fil med navnet
\texttt{6I-navn.sml}, hvor \emph{navn} er erstatttet med dit
navn. Hvis du hedder Anders A. And, skal filnavnet f.eks. v{\ae}re
\texttt{6I-Anders-A-And.sml}. Skriv ogs{\aa} dit fulde navn som en
kommentar i starten af filen.

Den individuelle opgave drejer sig om at implementere checks for, om
et sudoku-spil er vundet.

\emph{L{\ae}s ugens opgavetema igen, hvis du er i tvivl om, hvad det vil sige, at et spil er vundet}.

Brug gerne halerekursive funktioner og h{\o}jereordensfunktioner, hvor du finder det anvendeligt.

\begin{enumerate}[{6I}1]

\item{Definer en funktion \texttt{sublist :\ ''a list -> ''a list -> bool},
s{\aa}ledes at et kald\newline
\texttt{sublist xs ys} returnerer \texttt{true}
netop hvis alle elementer i listen \texttt{xs} ogs{\aa} er elementer i listen \texttt{ys} (du kan i denne opgave antage, at \texttt{xs} ikke indeholder dubletter).

F.eks.\ skal \texttt{sublist [\#"1", \#"2"] [\#"1", \#"3"]} returnere
\texttt{false}, mens kaldet
\texttt{sublist [\#"1", \#"2"] [\#"2", \#"1"]} skal returnere \texttt{true}.

Vink: HR, afsnit 5.7. Hvis du har \texttt{member} til r{\aa}dighed,
skal du blot for hvert element i \texttt{xs} checke, at det ogs{\aa} er med i
\texttt{ys}.}

% TODO : eksempel på lister med forskellig længe? Eller skal de regner det ud
%        selv?

\item{Definer en funktion \texttt{column :\ int -> 'a list list -> 'a list},
s{\aa}ledes at et kald \texttt{column n xs} returnerer listen
af de elementer, som (ved 0-baseret nummerering)
st{\aa}r p{\aa} plads \texttt{n} i lister fra \texttt{xs}.

F.eks.\ skal kaldet

\texttt{column 1 [[\#"3", \#"6"], [\#"6", \#"2"], [\#"5", \#"9"]]}

returnere listen

\texttt{[\#"6", \#"2", \#"9"]}.

Hvis en af listerne i \texttt{xs} har f{\ae}rre end \texttt{n+1} elementer, skal der rejses en passende undtagelse.

Vink I: Definer f{\o}rst (eller find i HR \ldots) en funktion,
som returnerer det \texttt{n}'te element i en liste.

Vink II: Hvad sker der, hvis man tager funktionen fra foreg{\aa}ende vink og kombinerer den passende
med \texttt{map}?}

\item{Forklar \emph{kort og pr{\ae}cist}, hvordan \texttt{sublist} kan bruges til at checke,
om tallene $1$--$9$ alle er blevet brugt netop en gang i hver r{\ae}kke i (ML-repr{\ae}sentationen af) et sudoku-spil

Forklar dern{\ae}st, hvordan \texttt{sublist} og \texttt{column} kan bruges til at checke, om tallene
$1$--$9$ alle er blevet brugt netop en gang i hver s{\o}jle i (ML-repr{\ae}sentationen af) et sudoku-spil.}

\item{\label{opg:rowcolcheck}
Definer en funktion \texttt{rowcolCheck :\ char list list -> bool}, som checker,
om tallene $1$--$9$ alle er blevet brugt netop en gang i hver r{\ae}kke og hver s{\o}jle i (ML-repr{\ae}sentationen af) et sudoku-spil (funktionen skal returnere \texttt{true}, hvis dette er tilf{\ae}ldet).}

\item{Definer til slut en funktion \texttt{checkSudoku :\ char list list -> bool},
som checker, om (ML-re\-pr{\ae}\-sen\-ta\-ti\-o\-nen af) en sudukotilstand repr{\ae}senterer en vindende tilstand.

Vink I: Tre kriterier skal v{\ae}re opfyldt: Et for r{\ae}kker, et for s{\o}jler, og et for regioner. Du har i opgave~6i\ref{opg:rowcolcheck}
l{\o}st problemet for r{\ae}kker og s{\o}jler og mangler s{\aa}ledes kun at l{\o}se problemet for regioner. M{\aa}ske kan en funktion fra ugens gruppeopgaver benyttes?}

\item{ (Valgfri ekstraopgave) Indbyg checket fra foreg{\aa}ende opgave i det interaktive sudoku-spil
fra de valgfri gruppeopgaver. Stop spillet, n{\aa}r det er vundet (med en passende meddelelse til brugeren).}

\end{enumerate}






\section{Valgfri ekstraopgaver}

Hvis man har tid, kan nedenst{\aa}ende opgaver l{\o}ses; instruktorernes hj{\ae}lp
til disse opgaver vil v{\ae}re begr{\ae}nset, da studerende, som endnu ikke har l{\o}st de obligatoriske opgaver, har fortrinsret.

\begin{itemize}
\item HR: 15.1 eller (mere krævende) 15.3
\item HR: 14.4 (bemærk at der mangler et \texttt{'a} argument i typerne for \texttt{fileFoldr} og \texttt{fileFoldl})
\end{itemize}

%Desuden de valgfri opgaver fra gruppeopgaverne og de individuelle opgaver.

Til slut: Skriv en sudoku\emph{l{\o}ser}, som givet en starttilstand automatisk finder frem til en
l{\o}sning uden menneskelig indblanding. Teknikker fra kombinatorisk s{\o}gning
(se f.eks. 8-dronninge-problemet) kan ret let overf{\o}res.

\section{Ugens n{\o}d}
Ugens n{\o}d er en opgave for s{\ae}rligt interesserede studerende,
som enten har mere tid til r{\aa}dighed end de 20 timer, som der
forventes brugt p{\aa} kurset, eller som har s{\ae}rlige
foruds{\ae}tninger. Det forventes \emph{ikke}, at man som studerende
kan lave ugens n{\o}d, medmindre man l{\ae}gger en god portion
ekstraarbejde.

Afleveringsfristen er den samme som for den individuelle opgave. Ved
fredagsforelæsningen i uge 7 (dvs. ugen efter efterårsferien) kåres
den bedste løsning, og vinderen får ud over h{\ae}deren og
forel{\ae}serens respekt en lille præmie.

\subsection*{Dovne tvedelende tr{\ae}er}

N{\aa}r f{\ae}nomener skal repr{\ae}senteres i en computer,
kan man pakke data mere eller mindre t{\ae}t.  Som regel f{\aa}s en
hurtig tilgangstid, hvis man er villig til at afs{\ae}tte meget lager,
mens lagerforbruget tilsvarende kan mindskes, hvis der m{\aa} bruges
l{\ae}ngere beregningstid p{\aa} at finde data frem.

I yderste konsekvens kan man benytte en s{\aa}kaldt ``doven'' repr{\ae}sentation,
hvor ingen data er lagret direkte, men f{\o}rst beregnes, n{\aa}r der er
behov for dem.  En s{\aa}dan repr{\ae}sentation g{\o}r det til geng{\ae}ld
muligt at h{\aa}ndtere datastrukturer, der i princippet er uendeligt store:
Forud for hver anvendelse regner man sig frem til de dele, som skal bruges
(og kun endelige dele skal bruges ad gangen).

Afsnit 9.11 i l{\ae}rebogen af Hansen \& Rischel omtaler ganske kort,
hvordan SML, der ellers benytter ``ivrig'' evaluering, kan simulere
doven evaluering ved at pakke et udtryk \texttt{e} ind som funktion\newline
\texttt{(fn \_ => e)}.  Emnet for ugens n{\o}d er dovne tvedelende tr{\ae}er
(\emph{lazy binary trees}) med f{\o}lgende definition:
\begin{program}
datatype 'elt infbtree = Node of (unit -> 'elt infbtree) * 'elt * (unit -> 'elt infbtree)
\end{program}

Som eksempel p{\aa} en v{\ae}rdi af type \texttt{int infbtree}
opstilles de positive hele tal i et uendeligt tr{\ae},
idet hver knude $n$ har venstre og h{\o}jre undertr{\ae} med rod
henholdsvis $2n$ og $2n+1$:
\begin{program}
fun posinttree n = Node (fn \_ => posinttree (2 * n),n,fn \_ => posinttree (2 * n + 1))\\
val posints = posinttree 1
\end{program}
Figur~\ref{fig:posints} illustrerer \texttt{posints}.
\begin{figure}[htb]\centering
\begin{picture}(150,80)(0,-10)
\put(70,60){\makebox(10,10){1}}
\put(75,60){\line(-4,-1){40}}\put(75,60){\line(4,-1){40}}
\put(30,40){\makebox(10,10){2}}
\put(35,40){\line(-2,-1){20}}\put(35,40){\line(2,-1){20}}
\put(10,20){\makebox(10,10){4}}
\put(15,20){\line(-1,-1){10}}\put(15,20){\line(1,-1){10}}
\put(0,0){\makebox(10,10){8}}\put(20,0){\makebox(10,10){9}}
\put(50,20){\makebox(10,10){5}}
\put(55,20){\line(-1,-1){10}}\put(55,20){\line(1,-1){10}}
\put(40,0){\makebox(10,10){10}}\put(60,0){\makebox(10,10){11}}
\put(110,40){\makebox(10,10){3}}
\put(115,40){\line(-2,-1){20}}\put(115,40){\line(2,-1){20}}
\put(90,20){\makebox(10,10){6}}
\put(95,20){\line(-1,-1){10}}\put(95,20){\line(1,-1){10}}
\put(80,0){\makebox(10,10){12}}\put(100,0){\makebox(10,10){\ldots}}
\put(130,20){\makebox(10,10){7}}
\put(135,20){\line(-1,-1){10}}\put(135,20){\line(1,-1){10}}
\put(120,0){\makebox(10,10){\ldots}}\put(140,0){\makebox(10,10){\ldots}}
\put(5,0){\line(-1,-2){5}}\put(5,0){\line(1,-2){5}}
\put(25,0){\line(-1,-2){5}}\put(25,0){\line(1,-2){5}}
\put(45,0){\line(-1,-2){5}}\put(45,0){\line(1,-2){5}}
\put(65,0){\line(-1,-2){5}}\put(65,0){\line(1,-2){5}}
\put(85,0){\line(-1,-2){5}}\put(85,0){\line(1,-2){5}}
\end{picture}
\caption{\label{fig:posints}Det uendelige tr{\ae} \texttt{posints}.}
\end{figure}

Et kald af formen \texttt{bredde $k$ $t$} vil ``h{\o}ste''
de f{\o}rste $k$ elementer af det uendelige
tvedelende tr{\ae} $t$ ``bredde f{\o}rst'':
\begin{program}
local fu\=n \=bredde' 0 \_ = []\+\\
| bredde' k (Node (left,cargo,right) ::\ ts)\+\\
= cargo ::\ bredde' (k - 1) (ts @ [left (), right ()])\-\-\\
in fun bredde k t = bredde' k [t] end;\\
bredde 10 posints;\\
\emph{val it = [1, 2, 3, 4, 5, 6, 7, 8, 9, 10] :\ int list}
\end{program}

\subsection*{Opgave}

Selve n{\o}dden g{\aa}r ud p{\aa} at konstruere en funktion
\texttt{filtinord :\ ('elt -> bool) -> 'elt infbtree -> 'elt list}
og en v{\ae}rdi \texttt{sternBrocot :\ qnum infbtree} som n{\ae}rmere
beskrevet i det f{\o}lgende.

Et funktionskald af formen \texttt{filtinord $p$ $t$} skal
returnere en liste med knuderne fra $t$ i ``inorder'',
men dog s{\aa}dan, at hvis en knude ikke opfylder $p$,
skal hverken den eller nogen af dens underliggende knuder medtages
i listen.  (For at kaldet kan afsluttes, m{\aa} man foruds{\ae}tte,
at der kun bliver endeligt mange knuder tilbage i listen.)

For eksempel skal der g{\ae}lde:
\begin{program}
filtinord (fn n => n mod 4 > 0 andalso n <= 20) posints;\\
\emph{val it = [2, 10, 5, 11, 1, 6, 13, 3, 14, 7, 15] :\ int list}
\end{program}

Figur~\ref{fig:sternBrocot} viser det s{\aa}kaldte Stern-Brocot-tr{\ae}
(efter Moritz Stern og Achille Brocot),
som er et uendeligt tvedelende tr{\ae} med alle de positive rationale tal.
\begin{figure}[htb]\centering
\begin{picture}(150,140)(0,-10)
\put(0,110){\makebox(10,20){$\frac01$}}\put(140,110){\makebox(10,20){$\frac10$}}
\put(70,90){\makebox(10,20){$\frac11$}}
\put(75,90){\line(-4,-1){40}}\put(75,90){\line(4,-1){40}}
\put(30,60){\makebox(10,20){$\frac12$}}
\put(35,60){\line(-2,-1){20}}\put(35,60){\line(2,-1){20}}
\put(10,30){\makebox(10,20){$\frac13$}}
\put(15,30){\line(-1,-1){10}}\put(15,30){\line(1,-1){10}}
\put(0,0){\makebox(10,20){$\frac14$}}\put(20,0){\makebox(10,20){$\frac25$}}
\put(50,30){\makebox(10,20){$\frac23$}}
\put(55,30){\line(-1,-1){10}}\put(55,30){\line(1,-1){10}}
\put(40,0){\makebox(10,20){$\frac35$}}\put(60,0){\makebox(10,20){$\frac34$}}
\put(110,60){\makebox(10,20){$\frac21$}}
\put(115,60){\line(-2,-1){20}}\put(115,60){\line(2,-1){20}}
\put(90,30){\makebox(10,20){$\frac32$}}
\put(95,30){\line(-1,-1){10}}\put(95,30){\line(1,-1){10}}
\put(80,0){\makebox(10,20){$\frac43$}}\put(100,0){\makebox(10,20){\ldots}}
\put(130,30){\makebox(10,20){$\frac31$}}
\put(135,30){\line(-1,-1){10}}\put(135,30){\line(1,-1){10}}
\put(120,0){\makebox(10,20){\ldots}}\put(140,0){\makebox(10,20){\ldots}}
\put(5,0){\line(-1,-2){5}}\put(5,0){\line(1,-2){5}}
\put(25,0){\line(-1,-2){5}}\put(25,0){\line(1,-2){5}}
\put(45,0){\line(-1,-2){5}}\put(45,0){\line(1,-2){5}}
\put(65,0){\line(-1,-2){5}}\put(65,0){\line(1,-2){5}}
\put(85,0){\line(-1,-2){5}}\put(85,0){\line(1,-2){5}}
\end{picture}
\caption{\label{fig:sternBrocot}Stern-Brocot-tr{\ae}et med alle positive
rationale tal.}
\end{figure}

Tr{\ae}et konstrueres p{\aa} f{\o}lgende m{\aa}de:  Indholdet af en given
plads findes ved, at man ops{\o}ger den n{\ae}rmest forudg{\aa}ende br{\o}k
$\frac{a}b$ til venstre for pladsen og den n{\ae}rmest forudg{\aa}ende br{\o}k
$\frac{c}d$ til h{\o}jre for pladsen.  P{\aa} pladsen skrives i s{\aa} fald
$\frac{a+c}{b+d}$.

Betragt for eksempel de to pladser lige under $\frac23$:
De forudg{\aa}ende br{\o}ker for den venstre efterf{\o}lger er $\frac12$
og $\frac23$; den venstre efterf{\o}lger skal derfor v{\ae}re
$\frac{1+2}{2+3}=\frac35$.  De forudg{\aa}ende br{\o}kder for den
h{\o}jre efterf{\o}lger er $\frac23$ og $\frac11$, s{\aa} p{\aa}
den plads skal der st{\aa} $\frac{2+1}{3+1}=\frac34$.

Man kan vise (men at g{\o}re dette indg{\aa}r ikke i den stillede opgave), at
\begin{itemize}
\item Br{\o}ker konstrueret p{\aa} den beskrevne m{\aa}de bliver altid
uforkortelige.
\item Alle positive rationale tal kommer netop \'en gang med i tr{\ae}et.
\item Tallene vil blive placeret efter st{\o}rrelse (i forhold til
``inorder'').  Tr{\ae}et bliver med andre ord et ``s{\o}getr{\ae}''
(j{\ae}vnf{\o}r Example 8.1 i l{\ae}rebogen af Hansen \& Rischel).
\end{itemize}

V{\ae}lg en repr{\ae}sentation \texttt{qnum} af de rationale tal
(j{\ae}vnf{\o}r l{\ae}rebogens afsnit 4.2, 4.3 og 8.7.1), og konstruer
Stern-Brocot-tr{\ae}et som en v{\ae}rdi \texttt{sternBrocot :\ qnum infbtree}.

\subsubsection*{Farey-r{\ae}kken}

Farey-r{\ae}kken $\mathcal{F}_n$ af orden $n$ er de
uforkortelige br{\o}ker mellem 0 og 1, hvis n{\ae}vner h{\o}jst er $n$,
ordnet efter st{\o}rrelse.
For eksempel g{\ae}lder
\[\mathcal{F}_7 = \frac01,\frac17,\frac16,\frac15,\frac14,\frac27,\frac13,\frac25,\frac37,
\frac12,\frac47,\frac35,\frac23,\frac57,\frac34,\frac45,\frac56,\frac67,\frac11\]

Antag, at \texttt{enum :\ qnum -> int} og \texttt{denom :\ qnum -> int}
i den repr{\ae}sentation af rationale tal,
man har valgt, finder henholdsvis tallets t{\ae}ller og n{\ae}vner,
og at \texttt{nul :\ qnum} repr{\ae}senterer $\frac01$.
Da kan man som afpr{\o}vning af sine
konstruktioner tjekke, at
\begin{program}
nul ::\ filtinord (fn q => enum q <= denom q andalso denom q <= $n$) sternBrocot
\end{program}
beregner Farey-r{\ae}kken af orden $n$.
(Der er ogs{\aa} andre (og nemmere) m{\aa}der at beregne $\mathcal{F}_n$ p{\aa}.)

\end{document}
