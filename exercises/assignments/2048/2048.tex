Make a library consisting of a signature and an implementation file. The library must contain the following functions
\begin{lstlisting}
// convert a 2048-value v to a canvas color
fromValue: v: value -> Canvas.color
// give the 2048-value which is the next in order from c
nextColor: c: value -> value
// return the list of pieces on a row r on board s
filter: r: int -> s: state -> state
// tilt all pieces on the board s to the left
shiftLeft: s: state -> state
// flip the board s such that all pieces position change as (i,j) -> (N-1-i,j)
fliplr: s: state -> state
// transpose the pieces on the board s such all piece positiosn change as (i,j) -> (j,i)
transpose: s: state -> state
// find the list of empty positions on the board s
empty: s: state -> pos list
// randomly place a new piece of color c on an empty position on the board s
addRandom: c: value -> s: state -> state option
\end{lstlisting}
With these functions and Canvas it is possible to program the game in a few lines. Add the following to your library:
\begin{enumerate}
\item Write a canvas draw function
  \begin{quote}
    \lstinline{draw: w: int -> h: int -> s: state -> canvas}
  \end{quote}
  which makes a new canvas and draws the board in s.
\item Write a canvas react function
  \begin{quote}
    \lstinline{react: s: state -> k: key -> state option}
  \end{quote}
  which titles the board base according to the arrow-key, the user presses. Note that tilt left is given by the \lstinline{shiftLeft} function. Tilt right can be accomplished by \lstinline{fliplr >> shiftLeft >> fliplr}, and tilt up and down can likewise be accomplished with the additional use of \lstinline{transpose}.
\end{enumerate}
Finally, make an application program, which calls \lstinline{runApp "2048" 600 600 draw react board}.

All above mentioned functions are to be documented using the XML-standard, and simple test examples are to be made for each function showing that it likely works.
