Make a library consisting of a signature and an implementation file. The library must contain the following functions
\begin{lstlisting}
// convert a 2048-value v to a canvas color E.g., 
// > fromValue Green;;
// val it: color = { r = 0uy
//   g = 255uy
//   b = 0uy
//   a = 255uy }
fromValue: v: value -> Canvas.color

// give the 2048-value which is the next in order from c, e.g.,
// > nextColor Blue;;
// val it: value = Yellow
// > nextColor Black;;
// val it: value = Black
nextColor: c: value -> value

// return the list of pieces on a column k on board s, e.g.,
// > filter 0 [(Blue, (1, 0)); (Red, (0, 0))];;
// val it: state = [(Blue, (1, 0)); (Red, (0, 0))]
// > filter 1 [(Blue, (1, 0)); (Red, (0, 0))];;
// val it: state = []
filter: k: int -> s: state -> state

// tilt all pieces on the board s to the left (towards zero on
// the first coordinate), e.g.,
// > shiftUp [(Blue, (1, 0)); (Red, (2, 0)); (Black, (1,1))];;
// val it: state = [(Blue, (0, 0)); (Red, (1, 0)); (Black, (0, 1))]
shiftUp: s: state -> state

// flip the board s such that all pieces position change as
// (i,j) -> (2-i,j), e.g.
// > flipUD [(Blue, (1, 0)); (Red, (2, 0))];;                 
// val it: state = [(Blue, (1, 0)); (Red, (0, 0))]
flipUD: s: state -> state

// transpose the pieces on the board s such all piece positiosn
// change as (i,j) -> (j,i), e.g.,
// > transpose [(Blue, (1, 0)); (Red, (2, 0))];;
// val it: state = [(Blue, (0, 1)); (Red, (0, 2))]
transpose: s: state -> state

// find the list of empty positions on the board s, e.g., 
// > empty [(Blue, (1, 0)); (Red, (2, 0))];;    
// val it: pos list = [(0, 0); (0, 1); (0, 2); (1, 1); (1, 2); (2, 1); (2, 2)]
empty: s: state -> pos list

// randomly place a new piece of color c on an empty position on
// the board s, e.g.,
// > addRandom Red [(Blue, (1, 0)); (Red, (2, 0))];;
// val it: state option = Some [(Red, (0, 2)); (Blue, (1, 0)); (Red, (2, 0))]
addRandom: c: value -> s: state -> state option
\end{lstlisting}
With these functions and Canvas it is possible to program the game in a few lines. Add the following to your library:
\begin{enumerate}
\item Write a canvas draw function
  \begin{quote}
    \lstinline{draw: w: int -> h: int -> s: state -> canvas}
  \end{quote}
  which makes a new canvas and draws the board in s.
\item Write a canvas react function
  \begin{quote}
    \lstinline{react: s: state -> k: key -> state option}
  \end{quote}
  which titles the board base according to the arrow-key, the user presses. Note that tilt left is given by the \lstinline{shiftLeft} function. Tilt right can be accomplished by \lstinline{fliplr >> shiftLeft >> fliplr}, and tilt up and down can likewise be accomplished with the additional use of \lstinline{transpose}.
\end{enumerate}
Finally, make an application program, which calls \lstinline{runApp "2048" 600 600 draw react board}.

All above mentioned functions are to be documented using the XML-standard, and simple test examples are to be made for each function showing that it likely works.
