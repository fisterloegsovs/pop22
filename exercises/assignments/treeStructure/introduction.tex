I de følgende opgaver skal vi arbejde med en træstruktur til at beskrive geometriske
figurer med farver.  For at gøre det muligt at afprøve jeres opgaver
skal I gøre brug af det udleverede bibliotek \texttt{img\_util.dll}, der
blandt andet kan omdanne såkaldte canvas-objekter til png-filer.  Biblioteket er
beskrevet i forelæsningerne (i uge 7) og koden for biblioteket ligger
sammen med forelæsningsplancherne for uge 6 og er også tilgængeligt via github på \url{https://github.com/diku-dk/img-util-fs}.

Her bruger vi funktionerne til at tegne på et canvas samt til at gemme
canvas-objektet som en png-fil:\footnote{Bemærk at interfacet ikke
definerer de konkrete repræsentationstyper for
typerne \lstinline{color} og \lstinline{canvas}. Disse typer er
holdt \emph{abstrakte}, hvilket vil sige at deres repræsentationer
ikke kan ses af brugeren af modulet.}

\begin{lstlisting}[numbers=none,frame=none,mathescape]
// colors
type color
val fromRgb : int * int * int -> color

// canvas
type canvas
val mk       : int -> int -> canvas
val setPixel : color -> int * int -> canvas -> unit

// save a canvas as a png file
val toPngFile : string -> canvas -> unit
\end{lstlisting}

Funktionen \lstinline{toPngFile} tager som det første argument navnet
på den ønskede png-fil (husk extension).  Det andet argument er
canvas-objektet som ønskes konverteret og gemt. Et canvas-objekt kan
konstrueres med funktionen \lstinline{ImgUtil.mk}, der tager som
argumenter vidden og højden af billedet i antal pixels, samt funktionen
\lstinline{ImgUtil.setPixel}, der kan bruges til at opdatere canvas-objektet
før det eksporteres til en png-fil. Funktionen \lstinline{ImgUtil.setPixel}
tager tre argumenter. Det første argument repræsenterer en farve og
det andet argument repræsenterer et punkt i canvas-objektet (dvs.\@ i
billedet). Det tredie argument repræsenterer det canvas-objekt, der
skal opdateres.  En farve kan nu konstrueres med funktionen
\lstinline{ImgUtil.fromRgb} der tager en triple af tre tal mellem 0 og
255 (begge inklusive), der beskriver hhv.\@ den røde, grønne og blå del
af farven.

Koordinatsystemet har nulpunkt $(0,0)$ i øverste venstre hjørne og,
såfremt vidden og højden af koordinatsystemet er henholdsvis $w$ og
$h$, optræder punktet $(w-1,h-1)$ i nederste højre hjørne.  Antag for
eksempel at programfilen \texttt{testPNG.fsx} indeholder følgende F\#
kode:

\begin{lstlisting}[numbers=none,frame=none,mathescape]
let C = ImgUtil.mk 256 256
do ImgUtil.setPixel (ImgUtil.fromRgb (255,0,0)) (10,10) C
do ImgUtil.toPngFile "test.png" C
\end{lstlisting}

\noindent
Det er nu muligt at generere en png-fil med navn \texttt{test.png} ved
at køre følgende kommando:

\vspace{-4mm}
\begin{verbatim}
  fsharpi -r img_util.dll testPNG.fsx
\end{verbatim}
\vspace{-4mm}

\noindent
Den genererede billedfil \texttt{test.png} vil indeholde
et hvidt billede med et pixel af rød farve i punktet (10,10).

\noindent
Bemærk, at alle programmer, der bruger \texttt{ImgUtil} skal køres
eller oversættes med \texttt{-r img\_util.dll} som en del af
kommandoen.
%
Bemærk endvidere, at \LaTeX\ kan inkludere png-filer med
kommandoen \texttt{includegraphics}.

\vspace{2ex}

\noindent
I det følgende vil vi repræsentere geometriske figurer med følgende
datastruktur:

\begin{lstlisting}[numbers=none,frame=none,mathescape]
type point = int * int // a point (x, y) in the plane
type color = ImgUtil.color

type figure =
  | Circle of point * int * color
     // defined by center, radius, and color
  | Rectangle of point * point * color
     // defined by corners top-left, bottom-right, and color
  | Mix of figure * figure
     // combine figures with mixed color at overlap
\end{lstlisting}

\noindent
Man kan, for eksempel, lave følgende funktion til at finde farven af en
figur i et punkt.  Hvis punktet ikke ligger i figuren, returneres
\texttt{None}, og hvis punktet ligger i figuren, returneres
\texttt{Some $c$}, hvor $c$ er farven.

\begin{lstlisting}[numbers=none,frame=none,mathescape]
// finds color of figure at point
let rec colorAt (x,y) figure =
  match figure with
  | Circle ((cx,cy), r, col) ->
      if (x-cx)*(x-cx)+(y-cy)*(y-cy) <= r*r
        // uses Pythagoras' equation to determine
        // distance to center
      then Some col else None
  | Rectangle ((x0,y0), (x1,y1), col) ->
     if x0<=x && x <= x1 && y0 <= y && y <= y1
        // within corners
     then Some col else None
  | Mix (f1, f2) ->
      match (colorAt (x,y) f1, colorAt (x,y) f2) with
      | (None, c) -> c  // no overlap
      | (c, None) -> c  // no overlap
      | (Some c1, Some c2) ->
      let (a1,r1,g1,b1) = ImgUtil.fromColor c1
      let (a2,r2,g2,b2) = ImgUtil.fromColor c2
      in Some(ImgUtil.fromArgb((a1+a2)/2, (r1+r2)/2,   // calculate
                               (g1+g2)/2, (b1+b2)/2))  // average color
\end{lstlisting}

\noindent
Bemærk, at punkter på cirklens omkreds og rektanglens kanter er med i
figuren.  Farver blandes ved at lægge dem sammen og dele med to, altså
finde gennemsnitsfarven.
