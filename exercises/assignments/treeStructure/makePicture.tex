Brug \texttt{ImgUtil}-funktionerne og \texttt{colourAt} til at lave en
  funktion

  \vspace{-4mm}
\begin{verbatim}
makePicture : string -> figure -> int -> int
           -> unit
\end{verbatim}
  \vspace{-4mm}
  
  \noindent
  sådan at kaldet \texttt{makePicture \emph{filnavn figur b h}} laver en
  billedfil ved navn \texttt{\emph{filnavn}.png} med et billede af
  \texttt{\emph{figur}} med bredde \texttt{\emph{b}} og højde
  \texttt{\emph{h}}.
  
  På punkter, der ingen farve har (jvf.\ \texttt{colourAt}), skal farven
  være grå (som defineres med RGB-værdien (128,128,128)).
  
  Du kan bruge denne funktion til at afprøve dine opgaver.
