\textbf{Modulet \texttt{Solve}}.

I denne delopgave skal der skrives en funktion \lstinline{solve}, som foretager en udtømmende søgning efter en række træk som vil efterlade brættet i en konstellation med kun en pind, placeret i midten af brættet. Funktionen kan passende have følgende type:
\begin{lstlisting}
  type state = Board.b * Board.mv list
  val solve : state -> Board.mv -> state option
\end{lstlisting}

Her består en tilstand af ``den nuværende brætkonstellation'' samt en
liste af de træk der leder frem til denne konstellation (med det
seneste træk forekommende først i listen). Funktionen tager yderligere
et træk som argument. Ved at det gøres muligt at iterere gennem alle
mulige træk (for hver konstellation), fra det første
træk \lstinline{((0,0),Right)} til det sidste
træk \lstinline{((6,6),Up)}, kan vi sikre at alle træk
prøves. Funktionen skal benytte sig af rekursion til at foretage den
(muligvis) udtømmende søgning. Givet en vilkårlig brætkonstellation
samt en kandidat til et træk kan funktionen \lstinline{solve}
undersøge om trækket vil efterlade brættet i en ny konstellation eller
om trækket ikke er gyldigt. Afhængigt af udfaldet kan det enten
undersøges (ved eventuelt rekursivt at kalde \lstinline{solve}) om den nye
brætkonstellation har (eller er) en løsning eller om vi har bedre held med det
næste træk i trækordningen (hvis et sådan træk findes).

For at implementere funktionen er det nyttigt først at implementere nogle hjælpefunktioner:

\begin{enumerate}
\item Skriv en funktion \lstinline{nextdir} af type \lstinline{dir -> dir option}, som ``roterer'' en retningsværdi således at \lstinline{Up} bliver til \lstinline{Some Right}, \lstinline{Right} bliver til \lstinline{Some Down}, \lstinline{Down} bliver til \lstinline{Some Left} og \lstinline{Left} bliver til \lstinline{None}.

\item Skriv en funktion \lstinline{nextpos} af type \lstinline{pos -> pos option}, som returnerer den næste position på et $7 \times 7$ hullers bræt (row-major). Et kald \lstinline{nextpos(2,5)} skal returnere værdien \lstinline{Some(2,6)} og et kald \lstinline{nextpos(1,6)} skal returnere værdien \lstinline{Some(2,0)}.

\item Skriv en funktion \lstinline{nextmv} af type \lstinline{mv -> mv option}, som passende benytter sig af de to ovenfor specificerede funktioner. Funktionen skal give mulighed for at iterere gennem alle mulige flytninger, startende med flytningen \lstinline{((0,0),Up)}. Bemærk at funktionen skal operere uden hensyn til en konkret brætkonstellation og at funktionen ikke skal tage højde for de præcise forekomster af huller i brættet (flytningerne kan senere filtreres blandt andet ved brug af funktionen \lstinline{valid}).
%
Således skal et kald \lstinline{nextmv((1,2),Down)} returnere værdien \lstinline{Some((1,2),Left)}, et kald \lstinline{nextmv((1,6),Left)} skal returnere værdien \lstinline{Some((2,0),Up)}. Endelig skal kaldet \lstinline{nextmv((6,6),Left)} returnere værdien \lstinline{None}.
\end{enumerate}

I rapporten skal I vise koden for jeres implementation af den rekursive funktion \lstinline{solve} og argumentere for at den finder en løsning til brætspillet, såfremt en sådan findes. Skriv også kode til at udskrive de fundne træk og vis i rapporten at jeres implementation finder en løsning til spillet i form af en liste af træk.

Rapporten skal også indeholde en beskrivelse af implementationens begrænsninger samt en reflektion over hvordan implementationen kan generaliseres til at finde løsninger til andre brætspecifikationer.
