\textbf{Modulet \texttt{Board}}.

\begin{minipage}{0.65\linewidth}
Hvert hul i brættet er identificeret ved en position $(r,c)$, hvor $r$ er rækken for hullet (se billedet til højre) og $c$ er kolonnen hullet optræder i. Således er positionen for den tomme plads i midten $(3,3)$.

I det følgende skal vi benytte os af 64-bit heltal til at indeholde en komplet brætkonstellation (vi gør kun brug af de 49 mindstbetydende bit).
%Denne designbeslutning gør det muligt for os, ved brug af meget lidt plads, at repræsentere mange tusinder forskellige brætkonstellationer samtidig.
\end{minipage}\hfill
\begin{minipage}[b]{0.3\linewidth}
    \[
      \begin{array}{r|c|c|c|c|c|c|c}
        & 0 & 1 & 2 & 3 & 4 & 5 & 6 \\ \hline
        0 & & & \bullet & \bullet & \bullet & & \\ \hline
        1 & & & \bullet & \bullet & \bullet & & \\ \hline
        2 & \bullet & \bullet & \bullet & \bullet & \bullet & \bullet & \bullet \\ \hline
        3 & \bullet & \bullet & \bullet &  & \bullet & \bullet & \bullet \\ \hline
        4 & \bullet & \bullet & \bullet & \bullet & \bullet & \bullet & \bullet \\ \hline
        5 & & & \bullet & \bullet & \bullet & & \\ \hline
        6 & & & \bullet & \bullet & \bullet & &
      \end{array}
    \]
\end{minipage}

I F\# kan et bræt således repræsenteres ved brug af
typen \lstinline{uint64}, der repræsenterer (unsigned) 64-bit heltal:
\begin{lstlisting}
  type b = uint64
\end{lstlisting}

I den første del af opgaven ønskes der implementeret en række funktioner til at operere på brætkonstellationer. Funktionerne ønskes implementeret i et modul \lstinline{Board}, som vil kunne bruges både af en rigtig spiller til at spille spillet og af et modul der har til hensigt at finde en løsning til spillet.

Modulet \lstinline{Board} skal indeholde følgende typer og funktioner:
\begin{lstlisting}
  type b                              // board type
  type pos = int * int                // position type
  type dir = Up | Down | Left | Right // move direction
  type mv = pos * dir                 // move

  val init     : unit -> b            // initial board
  val valid    : pos -> bool          // is the position valid?
  val peg      : b -> pos -> bool     // true if pos valid and
                                      //  hole contains a peg
  val mv       : b -> mv -> b option  // returns new board
  val pegcount : b -> int             // number of pegs
  val print    : b -> string          // string representation
\end{lstlisting}

Her følger nogle gode råd til hvordan ovenstående modul implementeres:
\begin{itemize}

\item Start med at implementere to hjælpefunktioner \lstinline{seti} og \lstinline{geti} til henholdsvis at sætte en givet bit i en \lstinline{uint64}-værdi samt at undersøge om en givet bit er sat (hertil skal I benytte et udvalg af bit-operationer, inklusiv \lstinline{|||}, \lstinline{&&&}, \lstinline{~~~}, \lstinline{>>>} og \lstinline{<<<}).

\item Implementér en hjælpefunktion \lstinline{posi} til at omdanne en position (row-column pair) til et bit-index i brætrepræsentationen.

\item Funktionen \lstinline{valid} skal returnere \lstinline{false} hvis positionen ikke repræsenterer en hul-position i et tomt bræt.
%% \item Implementér funktionen \lstinline{peg} til at undersøge om en
%% pind er sat i et bræthul samt funktionen
%% \item Implementér derefter funktionen \lstinline{print} således at I kan undersøge om de to ovenstående funktioner virker korrekt.
\item Implementér en funktion \lstinline{neighbor} af type \lstinline{pos -> dir -> pos option}, som, givet en valid position og en retning, returnerer en valid naboposition, hvis en sådan findes i den specificerede retning, eller værdien \lstinline{None}. Et kald \lstinline{neighbor(1,4)Right} skal returnere værdien \lstinline{None} og et kald \lstinline{neighbor(2,4)Right} skal returnere værdien \lstinline{Some(2,5)}.
\item Funktionen \lstinline{mv} kan nu implementeres ved brug af funktionerne \lstinline{peg}, \lstinline{neighbor}, \lstinline{seti} og \lstinline{posi}. Funktionen skal, givet en brætkonstellation \lstinline{b} og et træk \lstinline{(p,d)} returnere værdien \lstinline{Some b'}, hvis (1) \lstinline{p} er en position indeholdende en pind ifølge brætkonstellationen \lstinline{b}, (3) \lstinline{(p,d)} er et lovligt træk og (3) brættet \lstinline{b'} er den konstellation, der fremkommer ved trækket \lstinline{(p,d)}. Ellers skal funktionen returnere værdien \lstinline{None}.
\item For at implementere funktionen \lstinline{print} kan der benyttes to nestede rekursive funktioner (eller to nestede for-løkker), som hver itererer over henholdsvis rækkerne og kolonnerne på brættet.
\end{itemize}

I rapporten skal I beskrive jeres designovervejelser og demonstrere at
jeres implementation fungerer som forventet (skriv unit-tests for de
implementerede funktioner).
