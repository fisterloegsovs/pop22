Denne opgave omhandler undtagelser (exceptions), option typer og Stirlings formel. Stirlings formel er en approximation til fakultetsfunktionen via $$\ln n! \simeq n \ln n - n.$$

I denne opgave skal der arbejdes mod at få computeren til at finde en
løsning til den engelske version af Peg Solitaire. En løsning vil
bestå i at computeren udskriver de træk der skal flyttes. Opgaven er
delt i tre dele. I den første delopgave arbejdes der mod at
implementere et modul til at repræsentere en brætkonstellation samt
operationer til at foretage flytninger og derved danne nye
brætkonstellationer. I den anden delopgave skal der arbejdes mod at
gøre det muligt for en spiller at spille spillet ved brug
af \texttt{fsharpi} således at computeren tillader at der foretages træk
hvorefter den nye brætkonstellation udskrives. I den tredie delopgave skal
der skrives en algoritme, som returnerer en liste af træk, der
efterlader en enkelt pind i midten af brættet.

\subsection*{Del 1: Modulet \lstinline{Board}}
\begin{minipage}{.65\textwidth}
Hvert hul i brættet er identificeret ved en position $(r,c)$, hvor $r$
er rækken for hullet (se billedet til højre) og $c$ er kolonnen hullet
optræder i. Således er positionen for den tomme plads i midten
$(3,3)$.

I det følgende skal vi benytte os af 64-bit heltal til at indeholde en
komplet brætkonstellation (vi gør kun brug af de 49 mindstbetydende
bit). Denne designbeslutning gør det muligt for os, ved brug af meget
lidt plads, at repræsentere mange tusinder forskellige
brætkonstellationer samtidig.
\end{minipage}
\begin{minipage}{.3\textwidth}
\[
\begin{array}{r|c|c|c|c|c|c|c}
  & 0 & 1 & 2 & 3 & 4 & 5 & 6 \\ \hline
0 & & & \bullet & \bullet & \bullet & & \\ \hline
1 & & & \bullet & \bullet & \bullet & & \\ \hline
2 & \bullet & \bullet & \bullet & \bullet & \bullet & \bullet & \bullet \\ \hline
3 & \bullet & \bullet & \bullet &  & \bullet & \bullet & \bullet \\ \hline
4 & \bullet & \bullet & \bullet & \bullet & \bullet & \bullet & \bullet \\ \hline
5 & & & \bullet & \bullet & \bullet & & \\ \hline
6 & & & \bullet & \bullet & \bullet & &
\end{array}
\]
\end{minipage}

I F\# kan et bræt således repræsenteres ved brug af typen \lstinline{uint64}:
\begin{lstlisting}
  type b = uint64
\end{lstlisting}

I den første del af opgaven ønskes der implementeret en række
funktioner til at operere på brætkonstellationer. Funktionerne ønskes
implementeret i et modul \lstinline{Board}, som vil kunne bruges både
af en rigtig spiller til at spille spillet og af et modul der har til
hensigt at finde en løsning til spillet.

Modulet \lstinline{Board} skal indeholde følgende typer og funktioner:
\begin{lstlisting}
  type b                               // board type
  type pos = int*int                   // position type
  type dir = Up | Down | Left | Right  // move direction
  type mv = pos * dir                  // move

  val init     : unit -> b             // initial board
  val valid    : pos -> bool           // is the position valid?
  val peg      : b -> pos -> bool      // true if pos valid and
                                       //  hole contains a peg
  val mv       : b -> mv -> b option   // returns new board
  val pegcount : b -> int              // number of pegs
  val pr       : b -> string           // string representation
\end{lstlisting}

Her følger nogle gode råd til hvordan ovenstående modul implementeres:
\begin{itemize}

\item Start med at implementere to hjælpefunktioner \lstinline{seti} og \lstinline{geti}
til henholdsvis at sætte en givet bit i en \lstinline{uint64}-værdi
samt at undersøge om en givet bit er sat (hertil skal I benytte et
udvalg af bit-operationer,
inklusiv \lstinline{|||}, \lstinline{&&&}, \lstinline{~~~}, \lstinline{>>>}
og \lstinline{<<<}).

\item Implementér en hjælpefunktion \lstinline{posi} til at omdanne en
position (row-column pair) til et bit-index i brætrepræsentationen.

\item Funktionen \lstinline{valid} skal returnere \lstinline{false} hvis positionen ikke repræsenterer en hul-position i et tomt bræt.
%% \item Implementér funktionen \lstinline{peg} til at undersøge om en
%% pind er sat i et bræthul samt funktionen
%% \item Implementér derefter funktionen \lstinline{pr} således at I kan undersøge om de to ovenstående funktioner virker korrekt.
\item Implementér en funktion \lstinline{neighbor} af type \lstinline{pos -> dir -> pos option}, som, givet en valid position og en retning, returnerer en valid naboposition, hvis en sådan findes i den specificerede retning, eller værdien \lstinline{None}. Således skal et kald \lstinline{neighbor(2,5)Right} returnere værdien \lstinline{None} og et kald \lstinline{neighbor(3,5)Right} skal returnere værdien \lstinline{Some(3,6)}.
\item Funktionen \lstinline{mv} kan nu implementeres ved brug af funktionerne \lstinline{peg}, \lstinline{neighbor}, \lstinline{seti} og \lstinline{posi}.
\end{itemize}

I rapporten skal I beskrive jeres designovervejelser og demonstrere at
jeres implementation fungerer som forventet (skriv unit-tests for de
implementerede funktioner).

\subsection*{Del 2: Modulet \lstinline{Game}}

Implementér et modul \lstinline{Game} med følgende funktionalitet:
\begin{lstlisting}
  val start : unit -> unit
  val mv    : Board.mv -> unit
\end{lstlisting}

Modulet skal indeholde en reference til en ``nuværende''
brætkonstellation. Funktionen \lstinline{start} skal sætte den
nuværende brætkonstellation til det initielle bræt og udskrive
brættet. Funktionen \lstinline{mv} skal foretage en flytning (hvis det
er lovligt) og udskrive den nye brætkonstellation.

I rapporten skal I demonstrere brugen af modulet \lstinline{Game}
i \texttt{fsharpi}.

\subsection*{Del 3: Modulet \lstinline{Solve}}

I denne delopgave skal der skrives en funktion \lstinline{solve}, som
foretager en udtømmende søgning efter en række træk som vil efterlade
brættet i en konstellation med kun en pind, placeret i midten af
brættet. Funktionen kan passende have følgende type:
\begin{lstlisting}
  type state = Board.b * Board.mv list
  val solve : state -> Board.mv -> state option
\end{lstlisting}

Her består en tilstand af ``den nuværende brætkonstellation'' samt en
liste af de træk der leder frem til denne konstellation (med det seneste træk forekommende først i listen).

For at implementere funktionen er det nyttigt først at implementere nogle
hjælpefunktioner:

\begin{enumerate}
\item Skriv en funktion \lstinline{nextdir} af type \lstinline{dir -> dir option},
som ``roterer'' en retningsværdi således at \lstinline{Up}
bliver til \lstinline{Some Right}, \lstinline{Right} bliver
til \lstinline{Some Down}, \lstinline{Down} bliver til \lstinline{Some Left}
og \lstinline{Left} bliver til \lstinline{None}.

\item Skriv en funktion \lstinline{nextpos} af type \lstinline{pos -> pos option},
som returnerer den næste position på et $7 \times 7$ hullers
bræt (row-major). Således skal et kald \lstinline{nextpos(2,5)}
returnere værdien \lstinline{Some(2,6)} og et
kald \lstinline{nextpos(1,6)} skal returnere
værdien \lstinline{Some(2,0)}.

\item Skriv en funktion \lstinline{nextmv} af type \lstinline{mv -> mv option}, som passende benytter sig af de to ovenfor specificerede funktioner. Funktionen skal give mulighed for at iterere gennem alle mulige flytninger, startende med flytningen \lstinline{((0,0),Up)}. Bemærk at funktionen skal operere uden hensyn til en konkret brætkonstellation og at funktionen ikke skal tage højde for de præcise forekomster af huller i brættet (flytningerne kan senere filtreres blandt andet ved brug af funktionen \lstinline{valid}).
%
Således skal et kald \lstinline{nextmv((1,2),Down)} returnere værdien \lstinline{Some((1,2),Left)}, et kald \lstinline{nextmv((1,6),Left)} skal returnere værdien \lstinline{Some((2,0),Up)} og kaldet \lstinline{nextmv((6,6),Left)} skal returnere værdien \lstinline{None}.
\end{enumerate}

I rapporten skal I vise koden for jeres implementation af den
rekursive funktion \lstinline{solve} og argumentere for at den finder
en løsning til brætspillet, såfremt en sådan findes. Skriv også kode
til at udskrive de fundne træk og vis i rapporten at jeres
implementation finder en løsning til spillet i form af en liste af
træk.

Rapporten skal også indeholde en beskrivelse af implementationens
begrænsninger samt en reflektion over hvordan implementationen kan
generaliseres til at finde løsninger til andre brætspecifikationer.
