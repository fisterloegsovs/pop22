In a not-so-distant future drones will be used for delivery of groceries. Imagine that the drone-traffic has become intense in your area and that you have been asked to decide if drones collide. Assume that all drones fly at the same altitude and that drones fly with different speeds measured in meters/minute and in different directions. If 2 drones are less than 5 meters from each other, then they collide. When a drone reaches its destination, then it lands and can no longer collide with any drone. Create an implementation file \lstinline{simulate.fs}, and add to it a \lstinline{Drone} class with properties and methods:
\begin{itemize}
\item The constructor must take start-position, -destination, and -speed.
\item \lstinline{position} (property): returns the drone's position in $(x, y)$ coordinates.
\item \lstinline{speed} (property): returns the drone's present speed in meters/minute.
\item \lstinline{destination} (property): returns the drone's present destination in $(x, y)$ coordinates. If the drone is not flying, then its present position and its destination are the same.
\item \lstinline{fly} (method): Set the drone's new position after 1 minutes flight.
\item \lstinline{isFinished} (method): Returns true or false depending on whether the drone has reached its destination or not.
\end{itemize}

Extend your implementaiton file with a class \lstinline{Airspace}, which contains the drones and as a minimum has the properties and methods:
\begin{itemize}
\item \lstinline{drones} (property): The collection of drones instances.
\item \lstinline{droneDist} (method): The distance between two given drones.
\item \lstinline{flyDrones} (method): Advance the position of all flying drones in the collection by 1 minute.
\item \lstinline{addDrone} (method): Add a new drone to the collection of drones.
\item \lstinline{willCollide} (method): Given a time interval, determine which drones will collide.
\end{itemize}

Write a white-box test class \lstinline{testSimulate.fsx} that tests both the above classes.
