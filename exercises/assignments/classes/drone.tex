In a not-so-distant future drones will be used for delivery of
groceries. Imagine that the drone traffic has become intense in your
area and that you have been asked to decide if drones collide.

We will make the simplifying assumptions that all drones
fly at the same altitude, that drones fly with different speeds
measured in centimetre/second and in different directions, and that
drones fly with constant speed (no acceleration). If two drones are
less than 5 meters from each other, then they collide. When a drone
reaches its destination, then it lands and can no longer collide with
any other drone. Create an implementation file
\lstinline{simulate.fs}, and add to it a \lstinline{Drone} class with
the following properties and methods:

\begin{itemize}
\item \lstinline{Position} (property): returns the drone's current
  position in $(x, y)$ coordinates.

\item \lstinline{Speed} (property): returns the drone's present speed
  in centimetre/second.

\item \lstinline{Destination} (property): returns the drone's present
  destination in $(x, y)$ coordinates. If the drone is not flying,
  its present position and its destination are the same.

\item \lstinline{Fly} (method): Set the drone's new position after one
  second flight.

\item \lstinline{AtDestination} (method): Returns \lstinline{true} or
  \lstinline{false} depending on whether the drone has reached its
  destination or not.
\end{itemize}

The constructor must take the start position, the destination, and the speed as arguments.
All positions and speeds are integers.

Extend your implementation file with a class \lstinline{Airspace} that
contains the drones and as a minimum has the following
properties and methods:
\begin{itemize}
\item \lstinline{Drones} (property): The collection of drones instances.
\item \lstinline{DroneDist} (method): The distance between two given drones.
\item \lstinline{FlyDrones} (method): Advance the position of all
  flying drones in the collection by one second.
\item \lstinline{AddDrone} (method): Add a new drone to the collection
  of drones.
\item \lstinline{WillCollide} (method): Given a time interval (number
  of minutes), determine which drones will collide. After two (or
  more) drones collide they are assumed to fall to the ground and are
  no longer considered. The method should return a list of pairs of
  drones that collided in the time interval.

  In the unfortunate event that three drones $A$, $B$ and $C$ are
  destined to collide at the same time, the list should contain the
  pairs $(A,B)$, $(A,C)$ and $(B,C)$. In this case, you may choose
  between two interpretations: either the three drones collide
  simultaneously or one of them gets a lucky break and dodges the crash
  (in which case it shouldn't be in the list of collisions). Clearly
  document the choice you take.
\end{itemize}

Write a black-box test class \lstinline{testSimulate.fsx} that tests
both the \lstinline{Drone} class and the \lstinline{Airspace} class.

Notice that the required methods and properties are \emph{minimum
  requirements}; feel free to add methods and properties if you need
  them.
