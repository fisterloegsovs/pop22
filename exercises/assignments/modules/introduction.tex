A color is often represented as a tripple (red, green, blue), where
each entry is called a color-channel, and each channel is typically an
integer between and including 0 and 255:
\begin{align}
  c &= (r, g,b)
\end{align}
Colors can be added, by adding their channels,
\begin{align}
  c_1 + c_2 &= \big(\text{trunc}(r_1+r_2), \text{trunc}(g_1+g_2), \text{trunc}(b_1+b_2)\big),
  \\c_i &= (r_i, g_i,b_i)
  \\\text{trunc}(v) &=
  \begin{cases}
    0, &v < 0
    \\255, &v > 255
    \\v, &\text{ellers}
  \end{cases}
\end{align}
and colors can be scaled by a factor by multiplying each channel with
that same factor,
\begin{align}
  a c &= \big(\text{trunc}(a r), \text{trunc}(a g), \text{trunc}(a b)\big)
\end{align}
Colors where the channels have identical values, $v=r=g=b$, are grays,
and colors are converted to grays as the average,
\begin{align}
  v = \text{gray}(c) &= \frac{r+g+b}{3}
\end{align}
set in each channel of the corresponding gray tripple.