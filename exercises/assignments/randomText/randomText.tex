\begin{enumerate}
\item The script \lstinline[language=console]{readFile.fsx} reads the
  content of the text file
  \lstinline[language=console]{readFile.fsx}. Convert this script into
  a function which reads the content of any text file and has the
  following type:
  \begin{quote}
    \mbox{\lstinline!readText : filename:string -> string!}
  \end{quote}
\item Write a program that converts a string, such that all letters
  are converted to lower case, and removes all characters except
  a\ldots z. It should have the following type:
  \begin{quote}
    \mbox{\lstinline!convertText : src:string -> string!}
  \end{quote}
\item Write a program that counts occurrences of each lower-case letter
  of the English alphabet in a string and returns a list. The first
  element of the list should be the count of 'a's, second the count of
  'b's etc. The function must have the type:
  \begin{quote}
    \mbox{\lstinline!histogram : src:string -> int list!}
  \end{quote}
\item The script \lstinline[language=console]{sampleAssignment.fsx}
  contains the function
  \begin{quote}
    \mbox{\lstinline!randomString : hist:int list -> len:int -> string!}
  \end{quote}
  which generates a string of a given length, and contains
  random characters distributed according to a given histogram. Modify
  the code to use your histogram function. Further, write a program,
  which reads the text
  \lstinline[language=console]{littleClausAndBigClaus.txt} using
  \lstinline{readText}, converts it using \lstinline{convertText}, and
  calculates its histogram and generates a new random string using
  \lstinline{histogram} and \lstinline{randomString}.  Test the
  quality of your code by comparing the histograms of the two texts.
\item Write a program that counts occurrences of each pairs of
  lower-case letter of the English alphabet in a string and returns a
  list of lists (a table). The first list should be the count of 'a'
  followed by 'a's, 'b's, etc., second list should be the count of 'b'
  followed by 'a's, 'b's, etc. etc.
  \begin{quote}
    \mbox{\lstinline!cooccurrence : src:string -> int list list!}
  \end{quote}
\item Write a program that generates a random string of length
  \lstinline!len!, whose character pairs are distributed according to
  a user specified cooccurrence histogram \lstinline!cooc!.  The
  function must have the type:
  \begin{quote}
    \mbox{\lstinline!fstOrderMarkovModel : cooc:int list list -> len:int -> string!}
  \end{quote}
  Test your function by generating a random string, whose character
  pairs are distributed as the converted characters in H.C.\
  Andersen's fairy tale, ``Little Claus and Big Claus'', calculate the
  cooccurrence histogram for the random string, and compare this with
  the original cooccurrence histogram.
\end{enumerate}
You must also write a short report, which 
\begin{itemize}
\item is no larger than 5 pages;
\item contains a brief discussion on how your implementation works,
  and if there are any possible alternative implementations, and in
  case, why you chose the one, you did;
\item includes output that demonstrates that your program works as intented.
\end{itemize}
The report is to be handed in as a pdf document together with the
single F\# source code as an fsx file.
