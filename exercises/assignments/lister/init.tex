In the following, you are to work with different ways to create a list:
\begin{enumerate}
\item Make an empty list, and bind it with the name \lstinline{lst}.
\item Create a second list \lstinline{lst2}, with \lstinline{lst} and the cons operator \keyword{::}, which contains the single element \lstinline{"F#"}. Consider whether the types of the old and new list are the same.
\item Create a third list \lstinline{lst3} which consists of 3 identical elements \lstinline{"Hello"}, and which is created with \lstinline{List.init} and the anonymous function \lstinline{fun i -> "Hello"}.
\item Create a fourth list \lstinline{lst4} which is a concatenation of \lstinline{lst2} and \lstinline{lst3} using \lexeme{@}.
\item Create a fifth list \lstinline{lst5} as \lstinline{[1; 2; 3]} using \lstinline{List.init}
\item Write a recursive function \lstinline{oneToN : n:int -> int list} which uses the concatenation operator, \lexeme{@}, and returns the list of integers \lstinline{[1; 2; ...; n]}. Consider whether it would be easy to create this list using the \lexeme{::} operator.
\item Write a recursive function \lstinline{oneToNRev : n:int -> int list} which uses the cons operator, \lexeme{::}, and returns the list of integers \lstinline{[n; ...; 2; 1]}.  Consider whether it would be easy to create this list using the \lexeme{@} operator.
\end{enumerate}
