Betragt følgende sum af heltal,
  \begin{equation}
    \sum_{i=1}^n i.
  \end{equation}
  Man kan ved induktion vise, at $\sum_{i=1}^n i = \frac{n(n+1)}{2},\, n\geq 0$. Opgaven har følgende delafleveringer:
  \begin{enumerate}
  \item \label{sum} Skriv en funktion
    \begin{quote}
      \mbox{\lstinline!sum : n:int -> int!}
    \end{quote}
    som ud over tælleværdien benytter en lokal variabel \lstinline!s! og en \lstinline!while! løkke til at udregne summen $1 + 2 + \dots + n$.
    % \item Lav en funktion
    %   \begin{quote}
    %     \mbox{\lstinline!recSum : n:int -> int!}
    %   \end{quote}
    %   som benytter rekursion og uden brug af variable til at udregne summen $1 + 2 + \dots + n$. Hint: $\sum_{i=1}^n i = n + \sum_{i=1}^{n-1} i$.
  \item Lav en funktion
    \begin{quote}
      \mbox{\lstinline!simpleSum : n:int -> int!}
    \end{quote}
    som i stedet benytter formlen $\frac{n(n+1)}{2}$.
%  \item Lav et program, som skriver en tabel ud på skærmen med 4 kolonner: \lstinline!n!, \lstinline!sum n!, \mbox{\lstinline!recSum n!} og \mbox{\lstinline!simpleSum n!}, og verificer at de 3 funktioner kommer til samme resultat.
  \item Skriv et program, som beder brugeren indtaste et tal \lstinline!n!, læser det fra tastaturet, og derefter udskriver resultatet af \lstinline!sum n! og \lstinline!simpleSum n!.
  \item Lav et program, som skriver en tabel ud på skærmen med 3 kolonner: \lstinline!n!, \lstinline!sum n! og \mbox{\lstinline!simpleSum n!}, og et passende antal rækker. Verificer ved hjælp af tabellen at de 2 funktioner beregner til samme resultat.
  \item Hvad er det største $n$ de 2 versioner kan beregne \lstinline{sum} funktionen korrekt for? Hvordan kan programmet modificeres, så funktionen kan beregnes for større værdier af $n$?
  \end{enumerate}
