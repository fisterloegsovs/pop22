Som en variant af Opgave~\ref{multiplicationTable}, skal der arbejdes med funktionen
  \begin{quote}
    \mbox{\lstinline!mulTable : n:int -> string!}
  \end{quote}
  som tager 1 argument og returnerer en streng indeholdende de første $1\leq n\leq 10$ linjer i multiplikationstabellen inklusiv ny-linje tegn, således at hele tabellen kan udskrives med et enkelt \lstinline!printf "%s"! statement. F.eks.\ skal kald til \mbox{\lstinline!mulTable 3!} returnere
  \begin{codeNOutput}[label=mulTab]{: Eksempel på brug og output fra \lstinline!mulTab!.}
\begin{lstlisting}
printf "%s" (mulTab 3);;
       1   2   3   4   5   6   7   8   9  10
   1   1   2   3   4   5   6   7   8   9  10
   2   2   4   6   8  10  12  14  16  18  20
   3   3   6   9  12  15  18  21  24  27  30
\end{lstlisting}
\end{codeNOutput}
hvor alle indgange i tabellen har samme bredde. Opgaven har følgende del\-afleveringer:
  \begin{enumerate}
  \item Lav
    \begin{quote}
      \mbox{\lstinline!mulTable : n:int -> string!}
    \end{quote}
så den som lokal værdibinding benytter en og kun en streng, der indholder tabellen for $n=10$, og benyt streng-indicering til at udtrække dele af tabellen for $n<10$.  Afprøv \mbox{\lstinline!mulTable n!} for $n= 1, 2, 3, 10$.
  \item Lav
    \begin{quote}
      \mbox{\lstinline!loopMulTable : n:int -> string!}
    \end{quote}
så den benytter en lokal streng-variabel, som bliver opbygget dynamisk vha.\ 2 \lstinline!for! løkker og \lstinline!sprintf!.  Afprøv \mbox{\lstinline!loopMulTable n!} for $n= 1, 2, 3, 10$.
%   \item Lav
%     \begin{quote}
%       \mbox{\lstinline!recMulTable : n:int -> string!}
%     \end{quote}
% som benytter rekursion og uden brug af variable opbygger strengen. Afprøv \mbox{\lstinline!recMulTable n!} for $n= 1, 2, 3, 10$.
  % \item Lav et program, som benytter sammenligningsoperatoren for strenge \lstinline!=!, og som skriver en tabel ud på skærmen med 3 kolonner: \lstinline!n!, og resultatet af sammenligningen af \mbox{\lstinline!mulTable n!} med hhv.\ \mbox{\lstinline!loopMulTable n!} og \mbox{\lstinline!recMulTable n!} som \lstinline!true! eller \lstinline!false!.
  % \item Forklar forskellen mellem at benytte \lstinline!printf "%s"! og \lstinline!printf "%A"! til at printe resultatet af \lstinline!mulTab!.
  \item Lav et program, som benytter sammenligningsoperatoren for strenge \lexeme{=}, og som skriver en tabel ud på skærmen med 2 kolonner: \lstinline!n!, og resultatet af sammenligningen af \mbox{\lstinline!mulTable n!} med \mbox{\lstinline!loopMulTable n!} som \lstinline!true! eller \lstinline!false!.
  \item Forklar forskellen mellem at benytte \lstinline!printf "%s"! og \lstinline!printf "%A"! til at printe resultatet af \lstinline!mulTab!.
  \end{enumerate}
