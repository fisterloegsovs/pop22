Consider multiplication tables of the form,
\begin{center}
  \begin{tabular}[c]{r|rrrrrrrrrr}
    & 1& 2& 3& 4& 5& 6& 7& 8& 9& 10\\\hline
   1 & 1& 2& 3& 4& 5& 6& 7& 8& 9& 10\\
   2 & 2& 4& 6& 8& 10& 12& 14& 16& 19& 20\\
   3 & 3& 6& 9& 12& 15& 18& 21& 24& 27& 30\\
    $\hdots$&&&&&&&&&&  
  \end{tabular}
\end{center}
where the elements of the top row and left column are multiplied and
the result is written at their intersection.

In this assignment, you are to work with a function
\begin{quote}
  \mbox{\lstinline!mulTable : n:int -> string!}
\end{quote}
which takes 1 argument and returns a string containing the first
$1\leq n\leq 10$ lines in the multiplication table including
\lstinline{<newline>} characters. The resulting string must be
printable with a single\lstinline!printf "%s"!%
statement. For example, the call \mbox{\lstinline!mulTable 3!} must
return.
  \begin{codeNOutput}[label=mulTab]{: An example of the output from \lstinline!mulTable!.}
\begin{lstlisting}
printf "%s" (mulTable 3);;
       1   2   3   4   5   6   7   8   9  10
   1   1   2   3   4   5   6   7   8   9  10
   2   2   4   6   8  10  12  14  16  18  20
   3   3   6   9  12  15  18  21  24  27  30
\end{lstlisting}
\end{codeNOutput}
All entries must be padded with spaces such that the rows and columns
are right-aligned. Consider the following sub-assignments:
\begin{enumerate}
\item Make
  \begin{quote}
    \mbox{\lstinline!mulTable : n:int -> string!}
  \end{quote}
  such that it has one and only one value binding to a string, which
  is the resulting string for $n=10$, and use indexing to return the
  relevant tabel for $n\leq 10$.  Test \mbox{\lstinline!mulTable n!}
  for $n= 1, 2, 3, 10$.  The function should return the empty string
  for values $n < 1$ and $n>10$.
\item Make
  \begin{quote}
    \mbox{\lstinline!loopMulTable : n:int -> string!}
  \end{quote}
  such that it uses a local string variable, which is built
  dynamically using 2 nested \lstinline!for!-loops and the \lstinline!sprintf!-function.  Test \mbox{\lstinline!loopMulTable n!} for $n= 1, 2, 3, 10$.
%   \item Lav
%     \begin{quote}
%       \mbox{\lstinline!recMulTable : n:int -> string!}
%     \end{quote}
% som benytter rekursion og uden brug af variable opbygger strengen. Afprøv \mbox{\lstinline!recMulTable n!} for $n= 1, 2, 3, 10$.
  % \item Lav et program, som benytter sammenligningsoperatoren for strenge \lstinline!=!, og som skriver en tabel ud på skærmen med 3 kolonner: \lstinline!n!, og resultatet af sammenligningen af \mbox{\lstinline!mulTable n!} med hhv.\ \mbox{\lstinline!loopMulTable n!} og \mbox{\lstinline!recMulTable n!} som \lstinline!true! eller \lstinline!false!.
  % \item Forklar forskellen mellem at benytte \lstinline!printf "%s"! og \lstinline!printf "%A"! til at printe resultatet af \lstinline!mulTab!.
  \item Make a program, which uses the comparison operator for
    strings, \lexeme{=}, and write a table to the screen with 2
    columns: \lstinline!n!, and the result of comparing the output of
    \mbox{\lstinline!mulTable n!} with \mbox{\lstinline!loopMulTable
      n!} as \lstinline!true! or \lstinline!false!, depending on
    whether the output is identical or not.
  \item Use \lstinline!printf "%s"!%
    and \lstinline!printf "%A"!%
    to print the result of \lstinline!mulTable!, and explain the difference.
  \end{enumerate}
