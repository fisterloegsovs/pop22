Enter the following program in a text file, compile, and execute the program:
  \begin{codeNOutput}[label=linear]{: Value bindings.}
\begin{lstlisting}
let a = 3
let b = 4
let x = 5
printfn "%A * %A + %A = %A" a x b (a * x + b)
\end{lstlisting}
\end{codeNOutput}
Explain why the the parenthesis in the call to \lstinline!printfn! is
necessary. Add a line, which calculates the expression $ax+b$ and
binds the result to the name \lstinline!y!. Modify the call to
\lstinline!printfn!, such that it uses this new name. Is it still
necessary to use parentheses?
