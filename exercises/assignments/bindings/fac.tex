Consider the factorial-function,
\begin{equation}
  n! = \prod_{i=1}^n i = 1\cdot 2\cdot \ldots \cdot n
  \label{eq:factorial}
\end{equation}
\begin{enumerate}
\item Write a function
  \begin{quote}
    \mbox{\lstinline!fac : n:int -> int!}
  \end{quote}
  which uses a \lstinline!while!-loop, a counter variable, and a local
  variable to calculate the factorial-function as \eqref{eq:factorial}.
  % \item Lav en variant
  %   \begin{quote}
  %     \mbox{\lstinline!recFac : n:int -> int!}
    %   \end{quote}
    %   som benytter rekursion og ingen variable til at beregne fakultetsfunktionen.
    % \item Afprøv begge funktioner ved at lave et program, som laver en tabel med 3 kolonner \lstinline!n!, \mbox{\lstinline!fac n!} og \mbox{\lstinline!recFac n!}, og sikr dig at de 2 funktioner regner rigtigt.
    % \item Hvad er det største $n$, som disse funktioner kan beregne fakultetsfunktionen for, og hvad er begrænsningen?
  \item Write a program, which asks the user to enter the number
    \lstinline!n! using the keyboard, and which writes the result of \lstinline!fac n!.
  \item Make a new version, 
    \begin{quote}
      \mbox{\lstinline!fac64 : n:int -> int64!}
    \end{quote}
    which uses \lstinline{int64} instead of \lstinline{int} to
    calculate the factorial-function. What are the largest values $n$,
    for which \lstinline{fac} and \lstinline{fac64} respectively can
    calculate the factorial-function for?
  \end{enumerate}
