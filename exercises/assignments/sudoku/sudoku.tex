I skal programmere et Sudoku spil og skrive en rapport. Afleveringen skal bestå af en pdf indeholdende rapporten, et katalog med et eller flere fsharp programmer som kan oversættes med Monos fsharpc kommando og derefter køres i mono, og en tekstfil der angiver sekvensen af oversættelseskommandoer nødvendigt for at oversætte jeres program(mer). Kataloget skal zippes og uploades som en enkelt fil. Kravene til programmeringsdelen er:
\begin{enumerate}
\item Programmet skal kunne læse en (start-)tilstand fra en fil.
\item Brugeren skal kunne indtaste filnavnet for (start-)tilstanden
\item Brugeren skal kunne indtaste triplen $(r,s,v)$, og hvis feltet er tomt og indtastningen overholder spillets regler, skal matrixen opdateres, og ellers skal der udskrives en fejlmeddelelse på skærmen
\item Programmet skal kunne skrive matricens tilstand på skærmen (på en overskuelig måde)
\item Programmet skal kunne foreslå lovlige tripler $(r,s,v)$.
\item Programmet skal kunne afgøre, om spillet er slut.
\item Brugeren skal have mulighed for at afslutte spillet og gemme tilstanden i en fil.
\item Programmet skal kommenteres ved brug af fsharp kommentarstandarden
\item Programmet skal struktureres ved brug af et eller flere moduler, som I selv har skrevet
\item Programmet skal unit-testes
\end{enumerate}
Kravene til rapporten er:
\begin{enumerate}[resume]
\item Rapporten skal skrives i \LaTeX.
\item I skal bruge \texttt{rapport.tex} skabelonen
\item Rapporten skal som minimum i hoveddelen indeholde afsnittene Introduktion, Problemformulering, Problemanalyse og design, Programbeskrivelse, Afprøvning, og Diskussion og Konklusion. Som bilag skal I vedlægge afsnittene Brugervejledning og Programtekst.
\item Alle gruppemedlemmer skal give feedback på et af hovedafsnittene i en anden gruppes rapport. Hvis og hvilke dele I gav feedback og hvem der gav feedback på jeres rapport skal skrives i Forordet i rapporten.
\item Rapporten må maximalt være på 20 sider alt inklusivt.
\end{enumerate}
Bemærk, at Sudoku eksemplerne i denne tekst er sat med \LaTeX-pakken \texttt{sudoku}.
