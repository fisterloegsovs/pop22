\begin{enumerate}
  \item Please read and understand the data structure Trie that has been handed out in \texttt{spellCheck.fsx} and understand how a word tree is created in this Trie and how the operations are performed. 
  \item We need to create an \texttt{autoComplete} functionality based on a lookup of a specific word in the trie. This function \texttt{lookup} should have the signature \texttt{let lookup (prefix: string) (trie: Trie<char>) : Trie<char> Option}. It should return an option of the (sub)trie found by the lookup. \texttt{autoComplete} will use \texttt{lookup} to check if the word beginning with the \texttt{prefix} exists, after which it should return a sequence of strings of words which \texttt{prefix} in the trie. The signature of \texttt{autoComplete} is \texttt{let autoComplete (prefix: string) (trie: Trie<char>) : string seq}. 
  \item Implement \texttt{spellCheck} with the signature \texttt{let spellCheck (word: string) (trie: Trie<char>) : bool} with the functionality to check if a given word is contained in the trie. 
  \item Implement \texttt{genText} with the signature \texttt{let genText (len: int) (trie: Trie<char>) : string} which generates a string text of length \texttt{len} of random words that are generated by the function \texttt{randWord} with the signature \texttt{let randWord (trie: Trie<char>) : char list}. \texttt{randWord} should retrieve a random word from the given trie. 
  \item Test your implementation with the given tests and, potentially, add extra tests. 
\end{enumerate}
