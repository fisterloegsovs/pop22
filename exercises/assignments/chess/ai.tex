Extend \lstinline{Player} with an artificial intelligence (AI), which simulate all possible series of moves at least $n \geq 0$ turns ahead or until a King is strikken. Each series should be given a fitness, and the AI should pick the move, which is the beginning of a series with the largest fitness. If there are several moves which have series with same fitness, then the AI should pick randomly among them. The fitness number must be calculated as the sum of the fitness of each move. A move, which does not strike any pieces gets value 0, if an opponent's rook is strikken, then the move has value 3. If the opponent strikes the player's rook, then the value of the move is -3. The king has in the same manner value $\pm100$. As an example, consider the series of 2 moves starting from \Cref{fig:init}, and it is black's turn to move.
\begin{figure}
  \centering
  \newgame
  \loadgame{\subfix{kingsGame}}
  \subfigure[Initial position\label{fig:init}]{\showboard}
  \loadgame{\subfix{kingsGame2}}
  \subfigure[b5 c5\label{fig:move1}]{\showboard}
  \loadgame{\subfix{kingsGame3}}
  \subfigure[b2 b4\label{fig:move2}]{\showboard}
%  \mainline{1.b2 b5}
%  \mainline{1.e4 e5 2. Nf3 Nc6 3.Bb5}
  \caption{Starting at the left and moving white rook to b4.}
  \label{fig:chessKingsGame}
\end{figure}
The illustrated series is \lstinline{["b5 b6"; "b2 b4"]}, the fitness of the corresponding moves are \lstinline{[0;-3]}, and the fitness of the series is \lstinline{-3}. Another series among all possible is \lstinline{["b5 b6"; "b2 c2"]}, which has fitness \lstinline{0}. Thus, of the moves considered,  \lstinline{"b5 b6"} has the maximum fitness of \lstinline{0} and is the top candidate for a move by the AI. Note that a rook has at maximum 14 possible squares to move to, and a king 8, so for a game where each player has a rook and a king each, then the number of series looking $n$ turns ahead is $\mathcal{O}(22^n)$.
