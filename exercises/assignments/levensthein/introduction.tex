Du skal i de følgende to opgaver arbejde med en funktion til at
  bestemme den såkaldte \emph{Levensthein-distance} mellem to strenge
  $a$ og $b$. Distancen er defineret som det mindste antal
  editeringer, på karakter-niveau, det er nødvendigt at foretage på
  strengen $a$ før den resulterende streng er identisk med strengen
  $b$. Som editeringer forstås (1) sletninger af karakterer, (2)
  indsættelser af karakterer, og (3) substitution af
  karakterer.

  Varianter af Levensthein-distancen mellem to strenge kan
  således benyttes til at identificere om studerende selv har løst
  deres indleverede opgaver eller om der potentielt set er tale om
  plagiatkode ;)

  Matematisk set kan Levensthein-distancen $\id{leven}(a,b)$, mellem
  to karakterstrenge $a$ og $b$, defineres som
  $\id{lev}_{a,b}(|a|,|b|)$, hvor $|a|$ og $|b|$ henviser til
  længderne af henholdsvis $a$ og $b$, og hvor funktionen $\id{lev}$ er
  defineret som følger:\footnote{See
    \url{https://en.wikipedia.org/wiki/Levenshtein_distance}.}

  \[
  \id{lev}_{a,b}(i,j) = \left \{ \begin{array}{ll} \mathrm{max}(i,j) & \mathrm{if}~\mathrm{min}(i,j) = 0, \\
    \mathrm{min} \left \{ \begin{array}{l}\id{lev}_{a,b}(i-1,j)+1 \\
                                          \id{lev}_{a,b}(i,j-1)+1 \\
                                          \id{lev}_{a,b}(i-1,j-1)+1_{(a_i\not=b_j)} \end{array} \right . & \mathrm{otherwise}. \end{array} \right .
  \]
  hvor $1_{(a_i\not=b_j)}$ henviser til \emph{indikatorfunktionen}, som er $1$ når $a_i\not=b_j$ og $0$ ellers.

