\textbf{Spillet}

Til brug for implementation af spillet skal vi benytte følgende type
til at repræsentere en spillers situation under et spil:
\begin{lstlisting}
  type player = deck
\end{lstlisting}
En spillers situation i et spil er således repræsenteret ved en stak af kort.

I første omgang ønskes implementet to funktioner til henholdsvis at
trække et kort fra en spillers hånd (såfremt hånden ikke er tom) samt
indsætte en stak kort i bunden af en spillers kortstak:
\begin{lstlisting}
  val getCard  : player -> (card * player) option
  val addCards : player -> deck -> player
\end{lstlisting}
Når der tilføjes en kortstak i bunden af en spillers hånd skal
kortstakken først blandes. Bemærk at \lstinline{getCard} trækker kort
fra toppen af kortstakken og \lstinline{addCards} tilføjer kort i
bunden af kortstakken.

Der ønskes nu implementeret en rekursiv funktion \lstinline{game} der
simulerer at to spillere spiller spillet krig mod hinanden:
\begin{lstlisting}
  val game : card list -> player -> player -> int
\end{lstlisting}
Funktionen \lstinline{game} tager some det første argument en liste af
kort, der repræsenterer de kort der ligger på bordet (til at starte
med er denne liste tom). Derudover tager funktionen repræsentationerne
af de to spillere som argumenter. Kroppen af funktionen kan nu
passende trække kort fra spillernes hænder og undersøge om den ene
spiller vinder runden eller om der er krig. Funktionen skal enten
foretage et rekursivt kald eller umiddelbart returnere et heltal, som
har til hensigt at identificere hvilken af spillerne der har vundet
(der skal returneres 1 hvis spiller 1 har vundet, 2 hvis spiller 2 har
vundet og 0 hvis det er uafgjort.)

\textbf{Hint:} Kroppen af funktionen skal også indeholde kode der tager
sig af at håndtere det tilfælde at der trækkes to ens kort (der er
krig). I det tilfælde forsøges der igen med træk af kort fra
spillernes hænder, hvorefter der fortsættes rekursivt med de trukne
kort lagt på bordet.

I rapporten skal I beskrive jeres designovervejelser og demonstrere at
jeres funktioner fungerer som forventet (skriv tests for de
implementerede funktioner). Husk at vise jeres implementation af
funktionen \lstinline{game} i rapporten. Bemærk at det kan være
vanskeligt at teste kode der benytter sig af
funktionen \lstinline{rand}. For at adressere dette problem kan I
eventuelt (til brug for tests) erstatte funktionen \lstinline{shuffle}
med en mere deterministisk (dvs. funktionel) funktion. Til testformål
vil det også være oplagt at køre koden på kortstakke med væsentlig
færre end de sædvanlige 52 kort (f.eks. 4 og 6 kort).
