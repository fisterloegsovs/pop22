\textbf{Kortspilsoperationer}.

I denne første delopgave skal der implementeres en række generelle
funktioner på kortspil. I det følgende skal vi antage at kuløren på et
kort er uden betydning. Vi kan derfor repræsentere et kort som et
heltal mellem 2 og 14 (et Es repræsenteres som værdien 14). Et spil
kort repræsenteres som en liste af kort:
\begin{lstlisting}
  type card = int
  type deck = card list
\end{lstlisting}

Implementér en funktion \lstinline{deal} med
type \lstinline{deck->deck*deck}, som deler en kortstak i to lige
store stakke ved skiftevis at give et kort til hver stak. I tilfælde
af at der er et ulige antal kort i argumentstakken skal den første
stak i resultatparret prioriteres. Kaldet \lstinline{deal [4;9;2;5;3]}
skal således returnere parret \lstinline{([3;2;4],[5;9])}.

Følgende funktion kan benyttes til at generere tilfældige heltal i F\#:
\begin{lstlisting}
  let rand : int -> int = let rnd = System.Random()
                          in fun n -> rnd.Next(0,n)
\end{lstlisting}
Funktionen \lstinline{rand} tager et heltal $n > 0$ og returnerer et tal
i intervallet $[0;n[$. Bemærk at funktionen ikke er funktionel i
matematisk forstand (den returnerer ikke nødvendigvis den samme værdi
hver gang den kaldes med det samme argument.)

Implementér nu, ved brug af \lstinline{rand}, en
funktion \lstinline{shuffle} med type \lstinline{deck->deck}, der
givet en kortstak returnerer en blandet version af
kortstakken. Her følger en mulig strategi for at blande en kortstak
$D$ af længde $n$:
\begin{enumerate}
\item Konstruér en liste $L$ af $n$ tilfældige tal.
\item Konstruér en liste af par bestående af parvis elementer fra $D$ og $L$.
\item Sortér listen af par ved brug af en ordningsrelation på de tilfældige tal i parrene.
\item Udtræk kortene fra den sorterede liste.
\end{enumerate}
Ovenstående strategi kan implementeres i F\# ved brug
af \lstinline{List.map2}, \lstinline{List.sortWith}, samt
funktionen \lstinline{compare} anvendt på
heltal.\footnote{Funktionen \lstinline{compare:int->int->int} tager to
argumenter \lstinline{a} og \lstinline{b} og returnerer 1
hvis \lstinline{a>b}, -1 hvis \lstinline{a<b} og 0
hvis \lstinline{a=b}. Funktionen \lstinline{List.sortWith} har
type \lstinline{('a->'a->int)->'a list->'a list} og tillader således
at tage en ordningsrelation som argument.}

Der findes andre (f.eks. rekursive) strategier hvormed en kortstak
kan blandes. I er velkomne til at designe og skrive jeres egen
blandingsfunktion.

Implementer også en funktion \lstinline{newdeck} med
type \lstinline{unit->deck}, der returnerer en blandet kortstak
bestående af 52 kort (4 serier af 13 kort med pålydende værdier fra 2
til 14).

I rapporten skal I beskrive jeres designovervejelser og demonstrere at
jeres funktioner fungerer som forventet (skriv tests for de
implementerede funktioner).
