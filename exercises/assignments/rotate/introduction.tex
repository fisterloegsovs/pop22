In this assignment, you will be working with a puzzle called Rotate. The puzzle consists of a square chess-like board with $n\times n, n \in \{2, 3, 4, 5\}$ fields. Each field has a unique id-number, which we will call the field's position, and, for a particular configuration of the board, each field is associated with a unique letter from the alphabet 'a', 'b', \dots. For example, with $n=4$, here is a possible board configuration:
\begin{center}
  \begin{tabular}{cccc}
     h &o &l &k
     \\b &i &g &e
     \\f &m &c &a
     \\j &n &d &p
  \end{tabular}
\end{center}
Moreover, the positions of the fields are laid out as follows:
\begin{center}
  \begin{tabular}{cccc}
    1  &2  &3  &4
    \\ 5  &6  &7  &8
    \\ 9 &10 &11 &12
    \\13 &14 &15 &16
  \end{tabular}
\end{center}
The puzzle is solved by rotating the letters in small $2\times 2$ subsquares clockwise until the board reaches the \emph{solved} configuration:
\begin{center}
  \begin{tabular}{cccc}
    a &b &c &d
    \\ e &f &g &h
    \\ i &j &k &l
    \\ m &n &o &p
  \end{tabular}
\end{center}
A rotation is specified by the position of its top-left corner, and all but the right-most column and the bottom-most row are valid inputs to the rotation operation. Let $p_1, p_2, p_3, p_4 \rightarrow q_1, q_2, q_3, q_4$ denote a rotation from $p_*$ to $q_*$, where $p_1$ is the top-left corner. Then, a rotation of subsquare 1 results in $1, 2, 5, 6\rightarrow 5, 1, 6, 2$, or equivalently,
\begin{center}
  \begin{tabular}{cccc}
     h &o &l &k
     \\b &i &g &e
     \\f &m &c &a
     \\j &n &d &p
  \end{tabular}
  $\rightarrow$
  \begin{tabular}{cccc}
     b &h &l &k
     \\i &o &g &e
     \\f &m &c &a
     \\j &n &d &p
  \end{tabular}
\end{center}
The overall task of this assignment is to build a program that generates rotate-puzzles and that allows you iteratively to enter a sequence of positions until the puzzle is solved.

Here is a list of detailed requirements:
\begin{itemize}
\item If your program includes loops, the loops must be programmed using recursion.
\item Your program must use lists and not arrays.
\item Your program is not allowed to use mutable values (or variables).
\item Your solution must be parameterized by $n$, the size of the board.
\item You must represent your board as a list of letters. Thus, for $n=4$, the board for a solved puzzle must be identical to the list \lstinline{['a' .. 'p']}
\item Your program must consist of the following files
\begin{quote}
\lstinline{game.fsx}, \lstinline{rotate.fsi}, \lstinline{rotate.fs}, \lstinline{whiteboxtest.fsx}, and \lstinline{blackboxtest.fsx}.
\end{quote}
The files \lstinline{rotate.fsi} and \lstinline{rotate.fs} must constitute the interface and the implementation of a library with your main types, functions, and values; the file \lstinline{game.fsx} must be a maximally 10-line program that defines the value $n$ and starts the game; and \lstinline{whiteboxtest.fsx} and \lstinline{blackboxtest.fsx} must contain your tests for the library.
\end{itemize}
As part of this assignment, you are to write a maximally 10-page report following the template \lstinline{rapport.tex}.

Notice that calls to \lstinline{System.Random ()} returns a random number generator object. This object has a method \lstinline{Next : n:int -> int}, which draws a random non-negative integer less than \lstinline{n}. For example, the code
%\begin{codeNOutput}{: Generating random integers.}
\begin{lstlisting}
let rnd = System.Random ()
for i = 1 to 3 do
  printfn "%d" (rnd.Next 10)
\end{lstlisting}
%\end{codeNOutput}
prints 3 random non-negative integers less than 10.
