In this assignment, you are to work with a puzzle called Rotate. The puzzle consists of a square board with $n\times n, n \in \{2, 3, 4, 5\}$ fields similarly to a chess-board. Each field has a unique id-number, which we will call the field's position, and on each field is one unique letter from the alphabet 'a', 'b', \dots. For example, when $n=4$, then the board could look like,
\begin{center}
  \begin{tabular}{cccc}
     h &o &l &k
     \\b &i &g &e
     \\f &m &c &a
     \\j &n &d &p
  \end{tabular}
\end{center}
and the position of the respective fields are
\begin{center}
  \begin{tabular}{cccc}
    1  &2  &3  &4
    \\ 5  &6  &7  &8
    \\ 9 &10 &11 &12
    \\13 &14 &15 &16
  \end{tabular}
\end{center}
The puzzle is solved by rotating the letters in small $2\times 2$ subsquares clockwise until the board reaches the state
\begin{center}
  \begin{tabular}{cccc}
    a &b &c &d
    \\ e &f &g &h
    \\ i &j &k &l
    \\ m &n &o &p
  \end{tabular}
\end{center}
A rotation is specified by the position of its top-left corner, and all but the right-most column and the bottom-most row are valid inputs to the rotation operation. Let $p_1, p_2, p_3, p_4 \rightarrow q_1, q_2, q_3, q_4$ denote a rotation from $p_*$ to $q_*$, where $p_1$ is the top-left corner, then for example, specifying a rotation of subsquare 1 results in $1, 2, 5, 6\rightarrow 5, 1, 6, 2$, or equivalently,
\begin{center}
  \begin{tabular}{cccc}
     h &o &l &k
     \\b &i &g &e
     \\f &m &c &a
     \\j &n &d &p
  \end{tabular}
  $\rightarrow$
  \begin{tabular}{cccc}
     b &h &l &k
     \\i &o &g &e
     \\f &m &c &a
     \\j &n &d &p
  \end{tabular}
\end{center}
The overall task of this assignment is to build a program, that will generate rotate-puzzles and allow you to iteratively enter a sequence of positions untill the puzzle is solved. Detailed requirements are:
\begin{itemize}
\item If your program includes loops, then the loops must be programmed using recursion
\item Your program must use lists and not arrays.
\item Your program must not use mutable values (variables).
\item Your solution must be parameterized by $n$, the size of the board.
\item you must represent your board as a list of letters, e.g., for $n=4$ the board for a solved puzzle must be \lstinline{['a' .. 'p']}
\item Your program must consist of the following files
\begin{quote}
\lstinline{game.fsx}, \lstinline{rotate.fsi}, \lstinline{rotate.fs}, \lstinline{whiteboxtest.fsx}, and \lstinline{blackboxtest.fsx}.
\end{quote}
The files \lstinline{rotate.fsi}, \lstinline{rotate.fs} must be the interface and implementation of a library with your main types, functions, and values; the file \lstinline{game.fsx} must be a maximally 10 line program, which defines the value $n$ and starts the game; and \lstinline{whiteboxtest.fsx} and \lstinline{blackboxtest.fsx} must contain your test for the library.
\end{itemize}
As part of this assignment, you are to write a maximally 10-page report following \lstinline{rapport.tex} template.

Note that calls to \lstinline{System.Random ()} returns a random number generator which has a function \lstinline{Next : n:int -> int} which draws a random non-negative integer less than \lstinline{n}. For example,
%\begin{codeNOutput}{: Generating random integers.}
\begin{lstlisting}
let rnd = System.Random ()
for i = 1 to 3 do             
  printfn "%d" (rnd.Next 10)
\end{lstlisting}
%\end{codeNOutput}
prints 3 random non-negative integers less than 10.
