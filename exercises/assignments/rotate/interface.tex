Write the interface file for the library \lstinline{rotate} with user defined types \lstinline{Board} which is a list of characters and \lstinline{Position} which is an integer and with the following functions,
\begin{quote}
  \lstinline{create : n:uint -> Board}
  \\\lstinline{board2Str : b:Board -> string}
  \\\lstinline{validPosition : b:Board -> p:Position -> bool}
  \\\lstinline{rotate : b:Board -> p:Position -> Board}
  \\\lstinline{scramble : b:Board -> m:uint -> Board}
  \\\lstinline{solved : b:Board -> bool}
\end{quote}
The function \lstinline{create} must take an integer $n$ and return a $n\times n$ board in its solved state.

The function \lstinline{board2Str} must take a board and return a string, containing the board formatted such that it can be printed with the \lstinline{printfn "%s"} command and formatting string.

The function \lstinline{validPosition} must take a board and rotation position and return true or false depending on whether the position is a valid rotation position or not.

The function \lstinline{rotate} must take a board and a position and return another board, which is identical to the original but where a local $2\times 2$ rotation has been performed at the indicated position. If an invalid position is given, then the function must return an empty board.

The function \lstinline{scramble} must take a board and an unsigned int \lstinline{m} and return another board, where all the elements of the original board have been rotated by \lstinline{m} random set of legal rotations using \lstinline{rotate}.

The function \lstinline{solved} must take a board and return true or false depending on whether the puzzle has been solved or not.

The interface must include documentation following the documentation standard.
