Write the interface file for the library \lstinline{rotate}. The interface should specify two user-defined types, named \lstinline{Board} and \lstinline{Position}, which are defined to be a list of characters and an integer, respectively. The interface should also specify the following functions:
\begin{quote}
  \lstinline{create : n:uint -> Board}
  \\\lstinline{board2Str : b:Board -> string}
  \\\lstinline{validRotation : b:Board -> p:Position -> bool}
  \\\lstinline{rotate : b:Board -> p:Position -> Board}
  \\\lstinline{scramble : b:Board -> m:uint -> Board}
  \\\lstinline{solved : b:Board -> bool}
\end{quote}
The function \lstinline{create} must take an integer $n$ as argument and return an $n\times n$ board in its solved state.

The function \lstinline{board2Str} must take a board as argument and return a string containing the board formatted such that it can be printed with the \lstinline{printfn "%s"} command and formatting string.

The function \lstinline{validRotation} must take a board and a rotation position as arguments and return true or false depending on whether the position is a valid rotation position or not.

The function \lstinline{rotate} must take a board and a rotation position as argument and return another board, which is identical to the original but where a local $2\times 2$ rotation has been performed at the indicated position. If an invalid position is given, the function must return the board that was passed as the first argument.

The function \lstinline{scramble} must take a board and an unsigned int \lstinline{m} as arguments and return another board, where all the elements of the original board have been rotated by \lstinline{m} random legal rotations using \lstinline{rotate}.

The function \lstinline{solved} must take a board as argument and return true or false depending on whether the board is in the solved configuration.

The interface must include documentation following the documentation standard.
