First extend the implementation-file \lstinline{readNWrite.fs} with a function,
\begin{quote}
  \mbox{\lstinline!tac : filenames:string list -> string option!}
\end{quote}
which takes a list of files, reads their content with \lstinline{readFile} (Exercise~\ref{cat:readFile}), reverses the order of each file in a line-by-line manner (i.e.\ the opposite of \lstinline{cat} on a line-by-line basis) and concatenates the result.  If any of the files do not exist, then the function should return \lstinline{None}.

Then write a program, \lstinline[language=console]{tac}, which takes a list of filenames as command-line arguments, calls the \lstinline{tac} function with this list and prints the resulting string to the screen. The program must return 0 or 1 depending on whether the operation was successful or not.
