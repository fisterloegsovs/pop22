First extend the implementation-file \lstinline{readNWrite.fs} with a function,
\begin{quote}
  \mbox{\lstinline!tac : filenames:string list -> string!}
\end{quote}
which takes a list of files, reads their content with
\lstinline{readFile} (Exercise~\ref{cat:readFile}), concatenates them,
and returns the result as a string in reverse order line-by-line
(i.e.\ the opposite of \lstinline{cat} on a line-by-line basis).  If any of the
files do not exist, then the function should return \lstinline{None}.

Then write a program, \lstinline[language=console]{tac}, which takes a
list of filenames as command-line arguments, calls the \lstinline{tac}
function with this list and prints the resulting string to the screen
using \lstinline{printFile} (Exercise~\ref{cat:printFile}). The
program must return 0 or 1 depending on whether the operation was
successful or not.
