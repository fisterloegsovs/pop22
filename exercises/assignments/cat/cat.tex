First extend the library \lstinline{readNWrite.fs} with a function,
\begin{quote}
  \mbox{\lstinline!cat : filenames:string list -> string option!}
\end{quote}
which takes a list of filenames. The function should use
\lstinline{readFile} (Exercise~\ref{cat:readFile}) to read the
contents of the files. The contents of the files should be merged into
a single \lstinline{string option}, which the function returns. If any of the
files do not exist, then the function should return \lstinline{None}.

Then write an application, \lstinline[language=console]{cat}, which takes a
list of filenames as command-line arguments, calls the \lstinline{cat}
function with this list and prints the resulting string to the screen. The
program must return 1 in case of an error and 0 otherwise.