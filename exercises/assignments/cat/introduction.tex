The program \lstinline[language=console]{cat} is a UNIX-program, which concatenates (i.e.\ joins) files. The program exists on both Linux and macOS. When passing two text files to \lstinline[language=console]{cat}, e.g. \lstinline[language=console]{a.txt} and \lstinline[language=console]{b.txt}, then the program prints the contents of file \lstinline[language=console]{a.txt} followed by the contents of \lstinline[language=console]{b.txt} to the screen. UNIX also has an inverse version of \texttt{cat}, called \texttt{tac}, which prints the files in reverse order and reverses their content line-by-line. For example, if the file \lstinline[language=console]{a.txt} contains the characters \lstinline[language=console]{aaa\nbbb\n} and the file \lstinline[language=console]{b.txt} contains the characters \lstinline[language=console]{ccc\nddd\n} with \lstinline[language=console]{\n} being the newline character, then
\begin{quote}
\lstinline[language=console]{cat a.txt b.txt}
\end{quote}
will output \lstinline[language=console]{aaa\nbbb\nccc\nddd\n} to the screen. In contrast,
\begin{quote}
\lstinline[language=console]{tac a.txt b.txt}
\end{quote}
will output \lstinline[language=console]{ddd\nccc\nbbb\naaa\n} to the screen.

In the following assignments you are to write a (functional) implementation of \lstinline[language=console]{cat} and \lstinline[language=console]{tac} in F\#.
