The Collatz conjecture is a famous unsolved problems in mathematics\footnote{\url{https://en.wikipedia.org/wiki/Collatz_conjecture}}. The conjecture states that for any integer larger than 0, recursively applying,

\[
  f(n) = 
  \begin{cases}
    \frac{n}{2} \qquad \mathrm{if} ~ n \% 2 = 0 \\
    3n+1 \quad \mathrm{if} ~ n \% 2 = 1
  \end{cases}
\]

will always lead the value 1. For example, starting with 19, we get the sequence, \lstinline{[19; 58; 29; 88; 44; 22; 11; 34; 17; 52; 26; 13; 40; 20; 10; 5; 16; 8; 4; 2; 1]}.

\begin{enumerate}
  \item Implement a non-recursive function \lstinline{collatzStep : n:int -> int} that gives the next number in the collatz sequence. Examples:
  \begin{quote}
    \lstinline{collatzStep 1 = 1} \\
    \lstinline{collatzStep 2 = 1} \\
    \lstinline{collatzStep 3 = 10} \\
    \lstinline{collatzStep 9 = 28} \\    
  \end{quote}
  The function should use pattern matching. 
  \item Implement a recursive function that counts the number of steps in the collatz sequence starting at $n$:
  \begin{quote}
  \lstinline{collatzStepsHelper : count:int -> n:int -> int}:
\end{quote}
The function should use pattern matching, recursion and the previous function \lstinline{collatzStep} to compute each intermediate step of a sequence. 
  \item Implement a non-recursive function \lstinline{collatzSteps : n:int -> int} that simply calls \lstinline{collatzStepsHelper} with an initial \lstinline{count} of 0. 
\end{enumerate}
