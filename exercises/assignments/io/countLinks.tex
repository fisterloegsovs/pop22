In the html-standard links are given by the \lstinline!<a></a>! tags. For example, a link to Google's homepage is written as \lstinline[language=console]!<a href="https://google.com">Press to go to Google</a>!.

Make a program
  \begin{quote}
    \mbox{\lstinline!countLinks : url:string -> int!}
  \end{quote}
which reads the page given in \lstinline!url! and counts how many links that page has to other pages. You should count by counting the number of \lstinline!<a! substrings.

Notice, most internet pages requires a valid certificate before they will allow your program to access it. By default, Mono has no certificates installed. One way to install useful certificates is to use \lstinline[language=console]{mozroots}, which is part of the Mono package. On Linux/MacOS you do the following from the console:
  \begin{quote}
    \lstinline[language=console]{mozroots --import --sync}
  \end{quote}
On Windows you type the following (on one line)
  \begin{quote}
    \lstinline[language=console]{mono "C:\Program Files (x86)\Mono\lib\mono\4.5\mozroots.exe" --import --sync}
  \end{quote}
Note that your installation of \lstinline[language=console]{mozroots} may be in a different path, and you may have to adapt the above path to your installation. After running the above, your program should be able to read most pages without being rejected.
