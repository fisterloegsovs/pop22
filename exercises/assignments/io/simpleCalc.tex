Make a calculator program
\begin{quote}
  \mbox{\lstinline!simpleCalc : unit -> unit!}
\end{quote}
which starts an infinite dialogue with the user. The user must be able to enter
simple expressions of positive numbers. Each expression must consist
of a value, one of the binary operators \lstinline!+!, \lstinline!-!,
\lstinline!*!, \lstinline!/!, and a value. When the user presses
\lstinline[language=console]!<enter>!, the the expression is evaluated
and the result is written as \lstinline[language=console]{ans = <result>} with the correct result entered. The input-values can either
be a positive integer or the string ``ans'', and the string ``ans''
should be the result of the previous evaluated expression or 0, in
case this is the first expression typed. As an example, a dialogue
could be as follows:
\begin{quote}
  \mbox{\lstinline!$ simpleCalc!}\\ %$
  \mbox{\lstinline!>3+5!}\\
  \mbox{\lstinline!ans=8!}\\
  \mbox{\lstinline!>ans/2!}\\
  \mbox{\lstinline!ans=4!}\\
\end{quote}
Here we used the character \lstinline{>} to indicate, that the program
is ready to accept input.

If the input is invalid or the evaluation results in an error, then
the program should give an error message, and the input should be
ignored.

