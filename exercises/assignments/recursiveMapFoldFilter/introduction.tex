The functions \lstinline!map!, \lstinline!fold!, and \lstinline!filter! are very powerful functions for processing lists, and they are often essential parts of high-performing parallel programs. In this assigment, you are to work with implementing some of these yourself.

In the file \lstinline[language=console]{recursiveMapFoldFilter.fsx} there is a fully functioning program, which must be compiled and executed from the console. It takes 2 arguments: a string and a postive integer $n$. The string can be either \lstinline[language=console]!map!, \lstinline[language=console]!fold!, or \lstinline[language=console]!filter!. The output is a random list of length $n$ consisting of positive integers less than 10 and a processed list. For \lstinline[language=console]!map!, the random elements have been multiplied by 2, for \lstinline[language=console]!fold!, the random elements have also been multiplied by 2 but their order have been reversed, and for \lstinline[language=console]!filter!, only those elements larger than 4 have been included.