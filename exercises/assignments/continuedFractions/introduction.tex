In this assignment, you will work with simple continued fractions\footnote{\url{https://en.wikipedia.org/wiki/Continued_fraction}}, henceforth just called continued fractions. Continued fractions are lists of integers which represent real numbers. The list is finite for rational numbers and infinite for irrational numbers.

\paragraph{Continued fractions to decimal numbers}
A continued fraction is written as $x = [q_0; q_1, q_2, \ldots]$ and the corresponding decimal number is found by the following recursive algorithm:
\begin{equation}
  x = q_0 + \frac{1}{q_1 + \frac{1}{q_2 + \dots}}.
\end{equation}
The series of fractions continues as long as there are elements in the continued fraction.

For example, $[3;4, 12, 4] = 3.245$, since:
\begin{align}
  x &= 3 + \frac{1}{4 + \frac{1}{12 + \frac{1}{4}}}
  \\&= 3 + \frac{1}{4 + \frac{1}{12.25}}
  \\&= 3 + \frac{1}{4.081632653}
  \\&= 3.245.
\end{align}
Note that all but the first digit must be larger than 0, e.g., $[1;0]$ is an illigal number, and that every rational number has exactly 2 representations $[q_0; q_1,\ldots,q_n] = [q_0; q_1,\ldots,(q_n-1),1]$ where the first is called the canoncial representation. E.g., $[2; 3] = [2; 2, 1]$, since
\begin{align}
  2 + \frac{1}{3} = 2 + \frac{1}{2 + \frac{1}{1}}.
\end{align}


\paragraph{Decimal numbers to continued fractions}
For a given number $x$ on decimal form, its continued fraction $[q_0; q_1, q_2, \ldots]$ can be found using the following algorithm:

Let $x_0 = x$ and $i \geq 0$, and calculate
\begin{align}
q_i &= \lfloor x_i \rfloor\\
r_i &= x_i - q_i\\
x_{i+1} &= 1/r_i\\
\end{align}
recursively until $r_i = 0$. The continued fraction is then the sequences of $q_i$.

For example, if $x=3.245$ then
\begin{center}
  \begin{tabular}{|l|l|l|l|l|}
    \hline
    $i$ & $x_i$ & $q_i = \lfloor x_i \rfloor$ & $r_i = x_i - q_i$ & $x_{i+1}=1/r_i$\\
    \hline
    0 & 3.245 & 3 & 0.245 & 4.081632653\ldots\\
    1 & 4.081632653\ldots & 4 & 0.081632653 & 12.25\\
    2 & 12.25 & 12 & 0.25 & 4\\
    3 & 4 & 4 & 0 & -\\
    \hline
  \end{tabular}
\end{center}
and hence, the continued fraction is in the third column as $3.245 = [3; 4, 12, 4]$.

% \paragraph{Heltalsbrøk til kædebrøk}
% Kædebrøken for en heltals brøk $t/n$ udregnes ved
% følgende algoritme: Lad $r_{-2} = t$ og $r_{-1}=n$ og $i\geq-2$, udregn
% \begin{align}
%   r_i &= r_{i-2}\text{ \% }r_{i-1}\quad\text{(rest ved heltalsdivision)},\label{rest}
%   \\q_i &= r_{i-2}\text{ div }r_{i-1}\quad\text{(heltalsdivision)},\label{div}
% \end{align}
% indtil $r_{i-1}=0$. Så vil $[q_0; q_1,\ldots,q_j]$ vil være $t/n$ som
% kædebrøk. F.eks.\ for brøken $t/n = 649/200$ beregnes:
% \begin{center}
%   \begin{tabular}{|r|r|r|r|r|}
%     \hline
%     $i$ & $r_{i-2}$ & $r_{i-1}$ & $r_i = r_{i-2}\text{ \% }r_{i-1} $ & $q_i = r_{i-2} \text{ div } r_{i-1}$\\
%     \hline
%     0 & 649 & 200 & 49 & 3 \\
%     1 & 200 & 49 & 4 & 4\\
%     2 & 49 & 4 & 1  & 12\\
%     3 & 4 & 1 & 0  & 4\\
%     4 & 1 & 0 & -  & -\\
%     \hline
%   \end{tabular}
% \end{center}
% Kædebrøken aflæses som højre søjle: $649/200 = [3; 4, 12, 4]$.

% Kædebrøker af heltalsbrøker $t/n$ er særligt effektive at udregne
% vha.\ Euclids algoritme for største fællesnævner. Algoritmen i
% Opgave~\ref{gcd} regner rekursivt på relationen mellem heltalsdivision
% og rest: Hvis $a = t \text{ div } n$ er heltalsdivision mellem $t$ og
% $n$, og $b = t \text{ \% } n$ er resten efter heltalsdivision, så er
% $t = a n + b$. For \eqref{rest}--\eqref{div} skal der altså gælde, at
% $r_{i-2} = q_i r_{i-1} + r_i$. Algoritmen i Opgave~\ref{gcd} regner
% udelukkende på $r_i$ som transformationen
% $(r_{i-2}, r_{i-1}) \rightarrow (r_{i-1}, r_i) = (r_{i-1},
% r_{i-2}\text{ \% }r_{-1})$ indtil $(r_{i-2}, r_{i-1}) =
% (r_{i-2},0)$. Hvis man tilføjer beregning af $q_i$ i rekursionen, har
% man samtidigt beregnet brøken som kædebrøk.


% \paragraph{Kædebrøk til Heltalsbrøk(er)}
% En kædebrøk kan approximeres som en heltalsbrøk
% $\frac{t_i}{n_i}, i \geq 0$ ved følgende algorime. Lad
% $t_{-2} = n_{-1} = 0$ og $t_{-1} = n_{-2} =1$, udregn
% \begin{align}
%   t_i &= q_it_{i-1}+t_{i-2},
%   \\n_i &= q_in_{i-1}+n_{i-2},
% \end{align}
% indtil $i$ er passende stor, eller der ikke er flere cifre
% $q_i$.  F.eks.\ for kædebrøken $[3; 4, 12, 4]$ beregnes,
% \begin{align}
%   \frac{t_0}{n_0} &= \frac{q_0t_{-1} + t_{-2}}{q_0n_{-1}+n_{-2}} = \frac{3\cdot 1+0}{3\cdot 0 + 1} = \frac{3}{1} = 3,
%   \\\frac{t_1}{n_1} &= \frac{q_1t_0 + t_{-1}}{q_1n_0+n_{-1}} = \frac{4\cdot 3 + 1}{4\cdot 1+0} = \frac{13}{4} = 3.25,
%   \\\frac{t_2}{n_2} &= \frac{q_2t_1 + t_{0}}{q_2n_1+n_{0}} = \frac{12\cdot 13 + 3}{12\cdot 4 + 1} = \frac{159}{49} = 3.244897959\ldots,
%   \\\frac{t_3}{n_3} &= \frac{q_3t_2 + t_{1}}{q_3n_2+n_{1}} = \frac{4\cdot 159 + 13}{4\cdot 49+4} = \frac{649}{200} = 3.245.
% \end{align}
% Altså kan kædebrøkken $[3;4, 12, 4]$ approximeres som heltalsbrøkkerne
% $3/1$, $13/4$, $159/49$ og $649/200$ med gradvist stigende nøjagtighed.
