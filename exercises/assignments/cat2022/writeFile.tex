\subsection*{Writing bytes to file}

Implement a function \texttt{writeFile: byte[] string -> int} with the following specification. 
\begin{itemize}
\item Precondition: None.
\item Postcondition: All bytes in the byte array are written to the file with the given name. If the file does not exist, it is created. If it exists, it is overwritten with the contents of the byte array and the exit status is 0.
  If an error occurs, the string \\ \verb|cat: Could not open or create file | \textit{filename} \verb|.\n| is written to stderr. The exit status is 1.
\end{itemize}

The function should obtain a \texttt{FileStream} and use \texttt{writeBytes} to write the bytes to the filestream. 
You should use \texttt{File.Open} with an appropriate \texttt{FileMode}. 
You might need to call the \texttt{Flush()} method on your \texttt{FileStream} after writing bytes to it. 
