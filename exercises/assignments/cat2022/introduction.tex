\section*{File concatenation}

The \texttt{cat}-utility from 
Unix\footnote{Unix is the predecessor operating system for MacOS, Linux and most server operating system in current practical use.} 
is a program that concatenates files.  
This exercise is about building a \texttt{cat}-like program in F\# in a file called \texttt{cat.fs} that contains the following functions, plus additional definitions as you see fit to solve this exercise.

We recommend that you create a new dotnet project using \texttt{dotnet new console -lang "F\#" -o cat}, then create the files \texttt{Cat.fsi} and \texttt{Cat.fs} and add them to \texttt{cat.fsproj}.


You can then use the following \textit{skeletons} for the three files:

\begin{itemize}
\item \textbf{Cat.fsi} \\
  \begin{lstlisting}{fsharp}
module Cat
open System.IO

val readBytes: FileStream -> byte[]
val readFile: string -> byte[]
val readFiles: string list -> byte[] option list
val writeBytes: byte[] -> FileStream -> unit
val writeFile: byte[] -> string -> int
val cat: string[] -> int
  \end{lstlisting}
\item \textbf{Cat.fs} \\
  \begin{lstlisting}{fsharp}
module Cat
open System.IO

let readBytes (fs:FileStream) : byte[] =
    [||] // Replace this with a proper implementation

let readFile (filename:string) : byte[] =
    [||] // Replace this with a proper implementation

let readFiles (filenames : string list) : byte[] option list =
    [] // Replace this with a proper implementation

let writeBytes (bytes : byte[]) (fs:FileStream) =
    () // Replace this with a proper implementation

let writeFile (bytes: byte[]) (filename:string) =
    0 // Replace this with a proper implementation

let cat (filenames : string[]) =
    0 // Replace this with a proper implementation
  \end{lstlisting}
\item \textbf{Program.fsx} \\
  \begin{lstlisting}{fsharp}
open Cat

[<EntryPoint>]
let main (args : string[]) =
    // args is a string array
    // containing the command-line arguments
    printfn "%A" args 
    0 // The exit code, 0 means "all is good"
  \end{lstlisting}
\end{itemize}
