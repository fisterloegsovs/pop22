\subsection*{Concatenating file contents and writing to files}

Implement a function \texttt{cat: (filenames:string[]) -> int} that outputs to the last \texttt{filename} in \texttt{filenames} the concatenation of the contents of all files in the input array except for the last, in the sequence they occur. Its specification is as follows.
\begin{itemize}
\item Precondition: None (any array of strings is acceptable and must be handled).
\item Postcondition: 
\begin{itemize}
\item If all of the input files exist and are readable, the output written to the last \texttt{filename} contains their concatenated contents in the order given in the input array. Nothing is written to stderr and the exit status (result of the function) is $0$.
\item If one or more of the files does not exist or is not readable, then
  nothing is written to the last \texttt{filename}. The exit status is $k$ where $k$ is the minimum of 255 and the number of nonexistent/unreadable files. For each string $s$ that is a nonexistent/unreadable file, the string \\ \verb|cat: The file | $s$ \verb|does not exist or is not readable.\n| \\ is written to stderr.
\item If \texttt{filenames} contain a single element, that file is either created or overwritten with nothing. The concatenation of ``nothing'' is the empty string.
  \item If \texttt{filenames} is empty, the string \verb|cat: no input files\n| is written to stderr. The exit status is 0. 
\end{itemize}
\end{itemize} 

\subsection*{Putting it all together}

In \texttt{Program.fs} call \texttt{cat} with the command line arguments.

\texttt{dotnet run file1.txt file2.txt file3.txt} should result in \texttt{file3.txt} being either created or overwritten, and should contain the concatenated contents of \texttt{file1.txt} and \texttt{file2.txt}.

