Consider a simulation of a natural habitat as two groups of animals interact. One group is the prey, a population of animals that are the food source for the other population of animals, the predators. Both groups have a fixed birthrate. The prey usually procreate faster than the predators, allowing for a growing prey population. But as the population of prey increases, the habitat can support a higher number of predators. This leads to an increasing predator population, and, after some time, a decreasing prey population. Around that time, the predator population grows so large as to reach a critical point, where the number of prey can no longer support the present predator population, and the predator population begins to wane. As the predator population declines, the prey population recovers, and the two populations continue this interaction of growth and decay.

An actual example of studying predator-prey relationships is the one between wolves and moose on Isle Royale in Lake Superior (http://www.isleroyalewolf.org/). Its population of wolves and moose are isolated on the island.
