\subsection*{Looking up, inserting and deleting elements}

Provide implementations, using explicit recursion, of functions
\begin{lstlisting}{fsharp}
val item : int -> 'a catlist -> 'a
val insert : int -> 'a -> 'a catlist -> 'a catlist
val delete : int -> 'a catlist -> 'a catlist
\end{lstlisting}
where \verb|item i xs| returns the \verb|i+1|-th element in \verb|xs| under the assumption (precondition) that \verb|0 <= i < length xs|;
\verb|insert i v xs| inserts \verb|v| after the \verb|i|-the element in \verb|xs|, under the assumption that \verb|0 <= i <= length xs|;
and \verb|delete i xs| deletes the \verb|i+1|-th element in \verb|xs| under the assumption that \verb|0 <= i < length xs|.

You may use the function \verb|length : 'a catlist -> int| in your definitions. This makes your implementation slow, but is okay since it can subsequently be implemented in constant time by data augmentation. Length-augmented catenable lists or sometimes called \emph{ropes}.

Using an inefficient, but correct implementation, as in this exercise, is a valuable intermediate step in the systematic design of efficient data structures.  
