\begin{defproblem}{install}
  \begin{onlyproblem}
    Install Scratch on your machine.
  \end{onlyproblem}
\end{defproblem}

\begin{defproblem}{helloWorld}
  \begin{onlyproblem}
    Make your own ``hello world'' Scratch-program. The program must make
    default sprite say ``Hello World'' when you press the green flag.
  \end{onlyproblem}
\end{defproblem}

\begin{defproblem}{moveIt}
  \begin{onlyproblem}
    Make a Scratch program with a sprite of your own choosing, which
    moves on the screen using the 'glide' and the 'forever' loop.
  \end{onlyproblem}
\end{defproblem}

\begin{defproblem}{screenShots}
  \begin{onlyproblem}
    Take one or more screenshots of your Scratch-program while it runs.
  \end{onlyproblem}
\end{defproblem}

\begin{defproblem}{countDown}
  \begin{onlyproblem}
    Make a Scratch-program, which counts down from 10 to 1. You must use a
    variable and a repeat loop.
  \end{onlyproblem}
\end{defproblem}

\begin{defproblem}{countDownWhenPressed}
  \begin{onlyproblem}
    Make a Scratch-program, which counts down from 10 to 1. The countdown must
    first start, when you press the mouse.
  \end{onlyproblem}
\end{defproblem}

\begin{defproblem}{countDoubles}
  \begin{onlyproblem}
    Make a Scratch-program, which counts up every even number from 0 to 20.
  \end{onlyproblem}
\end{defproblem}

\begin{defproblem}{commandLine}
  \begin{onlyproblem}[fragile]
    Start the command line (or terminal on MacOS), select and use \texttt{cd} to move the filepointer to a suitable place for your work. Create a directory from the command line. Use a text-editor to create a \LaTeX\ document using the class \texttt{article}. The preamble must define the title ``Hello world'', your name as the author, and today's date as the date. The main part of the document must use \verb\maketitle ~to produce the title and the text ``Hello again''. Convert the \LaTeX\ to \texttt{pdf} from the command line.
  \end{onlyproblem}
\end{defproblem}

\begin{defproblem}{report}
  \begin{onlyproblem}[fragile]%
    Write a short report in \LaTeX\ with Emacs and translate the
    \lstinline[language=console]{tex}-file to a
    \lstinline[language=console]{pdf}-file using the command line. The
    report should as minimum contain:
     \begin{itemize}
     \item A title produced using \verb\maketitle, 
     \item A section with a section title using \verb\section,
     \item One or more figures of screenshots from your program and
       by using the \verb figure ~environment, and it
       must include a caption text using \verb\caption.
     \item A reference to the figure using the \verb\label--\verb\ref\ pair.
     \item The Danish letters 'æ', 'ø', and 'å'.
    \end{itemize}
  \end{onlyproblem}
\end{defproblem}
