We can represent 2-dimensional vectors as tuples, as seen in the below code snippet:
  \begin{quote}
    \mbox{\lstinline!let myVector = (1,2)!} \\
    \mbox{\lstinline!let zeroVector = (0,0)!}
  \end{quote}




\begin{enumerate}
  \item Write a function:
  \begin{quote}
    \mbox{\lstinline!createIntVec : x:int -> y:int -> int * int!}
  \end{quote}
  that given two integers $x$ and $y$ returns a vector represented as a tuple, \lstinline{(x, y)}.
\item Write a function:
  \begin{quote}
    \mbox{\lstinline!createFloatVec : x:float -> y:float -> float * float!}
  \end{quote}
  that given two floats $x$ and $y$ returns a vector of floats.
\item Write a function:
  \begin{quote}
    \mbox{\lstinline!createFloatVecFromInts : x:int -> y:int -> float * float!}
  \end{quote}
  that given two integers $x$ and $y$ returns a vector of floats.
\item Write a function:
  \begin{quote}
    \mbox{\lstinline!intToFloatVec : int * int -> float * float!}
  \end{quote}
  that given a vector of two integers, returns a vector of floats.
\item Write a function:
  \begin{quote}
    \mbox{\lstinline!floatToIntVec : float * float -> int * int!}
  \end{quote}
  that given a vector of two floats, returns a vector of ints.

\item Write a function:
  \begin{quote}
    \mbox{\lstinline!addIntVecs : int * int -> int * int -> int * int!}
  \end{quote}
  that given two vectors of ints, adds them with vector addition and returns the resulting vector.

\item Write a function:
  \begin{quote}
    \mbox{\lstinline!subIntVecs : int * int -> int * int -> int * int!}
  \end{quote}
  that given two vectors of ints, subtracts them with vector subtraction and returns the resulting vector.

\item Write a function:
  \begin{quote}
    \mbox{\lstinline!scaleIntVec : int * int -> scalar:int -> int * int!}
  \end{quote}
  that given a vector of ints and a scalar, scales the vector and returns the resulting vector.
  
  
  
\end{enumerate}