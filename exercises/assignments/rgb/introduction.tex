En farve repræsenteres ofte som en triple (rød, grøn, blå), hvor hver indgang kaldes en farvekanal eller blot en kanal, og hver kanal er typisk et heltal mellem 0 og 255:
\begin{align}
  c &= (r, g,b)
\end{align}
Farver kan lægges sammen ved at addere deres kanaler,
\begin{align}
  c_1 + c_2 &= \big(\text{trunc}(r_1+r_2), \text{trunc}(g_1+g_2), \text{trunc}(b_1+b_2)\big),
  \\c_i &= (r_i, g_i,b_i)
  \\\text{trunc}(v) &=
  \begin{cases}
    0, &v < 0
    \\255, &v > 255
    \\v, &\text{ellers}
  \end{cases}
\end{align}
og farver kan skaleres ved at gange hver kanal med samme konstant.
\begin{align}
  a c &= \big(\text{trunc}(a r), \text{trunc}(a g), \text{trunc}(a b)\big)
\end{align}
Farver, hvor alle kanaler har samme værdi, $v=r=g=b$, kaldes gråtoner, og man kan konvertere en farve til gråtone ved at udregne gråtoneværdien som gennemsnittet af de 3 kanaler,
\begin{align}
  v = \text{gray}(c) &= \frac{r+g+b}{3}
\end{align}
