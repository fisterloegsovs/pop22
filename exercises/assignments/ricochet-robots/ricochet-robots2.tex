\textbf{Interaktion}

Implementer klassen \lstinline{Board}:

\begin{lstlisting}
  type Board =
    class
      new : ...
      member AddRobot : robot:Robot -> unit
      member AddElement : element:BoardElement -> unit
      member Elements : BoardElement list
      member Robots : Robot list
      member Move: Robot -> Direction -> unit
    end
\end{lstlisting}

Metoden \lstinline{AddElement} bruges til at sætte en spilleplade
op. Typisk inden spillet går i gang. Property \lstinline{Elements}
bruges til at få en liste af alle spilelementer (inklusiv robotter),
og \lstinline{Robots} bruges til at få en liste af alle robotter. Et
\lstinline{Board} har altid et \lstinline{BoardFrame} spilelement.

Metoden \lstinline{Move} bruges til at tage et træk med en robot. En
robot flyttes ved at der fortages et antal skridt med robotten i en
givet retning. Inden hvert skridt løbes gennem alle spilelementer (undtagen
robotten selv), og metoden \lstinline{Interact} kaldes for hvert
element. Hvis alle spilelementer returnere \lstinline{Ignore} kan
robotten flyttes eet felt i den givne retning, og robotten forsøges at
flyttes endnu et felt. Hvis et spilelement returnerer \lstinline{Stop pos},
stoppes robottens flytning i felt \lstinline{pos} (som ikke
nødvendigvis er robottens nuværende position). Hvis et spilelement
returnerer \lstinline{Continue dir pos} fortsætter robotten fra felt
\lstinline{pos} med retning \lstinline{dir} (bemærk at ingen af de
obligatoriske spilelementer bruger \lstinline{Continue}).


Implementer klassen \lstinline{Game}:

\begin{lstlisting}
  type Game =
    class
      new : Board -> Game
      member Play: unit -> int
    end
\end{lstlisting}

Metoden \lstinline{Play} bruges til at starte spillet, og tager sig af
interaktionen med brugeren via terminalen, returværdien er hvor mange
træk der blev brugt. Spillet
foregår på følgende vis:
\begin{enumerate}
 \item Vis hvordan pladen ser ud, hvor mange træk der er brugt
  indtilvidere, navnene på robotterne, samt evt anden information som
  I finder relevant.
\item Lad brugeren vælge en robot (fx ved at skrive navnet på
  robotten), herefter kan robotten flyttes rundt ved brug af
  pile-tasterne indtil at der tastes enter.

\item Når en robots træk er slut, får alle spilelementer mulighed for at
  afgøre om et spil er slut. Hvis spillet er slut, vis et
  afslutningsskærmbillede og stop spillet.
\end{enumerate}

Klasserne \lstinline{Board} og \lstinline{Game}, samt de andre klasser fra de andre
delopgaver, skal være i filen \texttt{robots.fs} i modulet
\lstinline{Robots}. Lav derudover en fil \texttt{robots-game.fsx}, der
som minimum laver en ny split og kalder \lstinline{Play}.


\textbf{Hints:}
\begin{itemize}
\item Det er en vigtig pointe at \lstinline{Board} holder styr på hvor
  de forskellige spilelementer er og ikke tager sig af at rendere
  spilelementer, men blot skaber et \lstinline{BoardDisplay}, som de
  forskellige spilelementer kan rendere sig selv på.
\item Brug \lstinline{System.Console.Clear()} at fjerne alt fra
  terminalen inden verden vises.
\item Brug \lstinline{Console.ReadKey(true)} til at hente tryk på
  piletasterne fra brugeren
\item Hvis \lstinline{key} er resultatet fra
  \lstinline{Console.ReadKey} så er \lstinline{key.Key} lig med\\
  \lstinline{System.ConsoleKey.UpArrow}, hvis brugeren trykkede på
  op-pilen.
\end{itemize}





%%% Local Variables:
%%% mode: latex
%%% TeX-master: "main"
%%% End:
