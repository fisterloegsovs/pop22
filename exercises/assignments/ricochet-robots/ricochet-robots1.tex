\textbf{Spilelementer}

Vi bruger den abstrakte klassen \lstinline{BoardElement} til at
repræsentere spilelementer og klassen \lstinline{Robot} til
repræsentere robotter. Tag udgangspunkt i følgende
erklæringer:

\begin{lstlisting}
type Direction = North | South | East | West
type Action =
  | Stop of Position
  | Continue of Direction * Position
  | Ignore

[<AbstractClass>]
type BoardElement() =
    abstract member RenderOn : BoardDisplay -> unit
    abstract member Interact : Robot -> Direction -> Action
    default __.Interact _ _ = Ignore
    abstract member GameOver : Robot list -> bool
    default __.GameOver _ = false

and Robot(row:int, col:int, name:string) =
    inherit BoardElement()
    member this.Position = ...
    override this.Interact other dir = ...
    override this.RenderOn display = ...
    member val Name = ...
    member robot.Step dir = ...
\end{lstlisting}

Hvor typen \lstinline{Direction} bruges til at angive en retning i
forhold til spillepladen. Typen \lstinline{BoardElement} bruges til at
repræsentere spilelementer som ikke er robotter, denne klasse har tre
abstrakte metoder: \lstinline{RenderOn} til at rendere et element på
et \lstinline{BoardDisplay}; \lstinline{GameOver} som bruges til at
afgøre om et spil er gennemført, baseret på robotternes positioner;
\lstinline{Interact} som bruges at afgøre et spilelements indflydelse
på en robot, inden den flyttes. Typen \lstinline{Action} bruges til at
specificere hvilken indflydelse et spilelement har på en robots
flytning.

Klassen \lstinline{Robot} bruges til at repræsentere
robotter. Robotter er også spilelementer, da de kan påvirke andre
robotter. Det vil sige, at klassen skal implementere metoderne fra
\lstinline{BoardElement}. Derudover har en robot to properties:
\lstinline{Position} et par, $(r,c)$, der angiver hvor
på pladen robotten er og \lstinline{Name} der angiver robottens
navn. Metoden \lstinline{Interact} bruges til at stoppe en anden robot
der forsøges at flytte ind i robotten. Fx, hvis \lstinline{other}
(argumenter til \lstinline{Interact}, den anden robot) står i
felt~$(1,4)$ og er på vej vest, og robotten (\lstinline{this}) står i
felt~$(1,3)$, så skal \lstinline{Interact} returnere \lstinline{Stop(1,4)}.

\TODO{Beskriv Name og Step}

Implementér følgende fire konkrete klasser, der alle nedarver fra
\lstinline{BoardElement}, til at repræsentere indre og ydre vægge samt målfelt:

\begin{itemize}
\item \lstinline{Goal} der skal have en konstruktor der tager to
  heltals argumenter, $r$ og $c$, som angiver et målfelt. Metoden
  \lstinline{GameOver} skal returnere \lstinline{true}, hvis en robot
  er stoppet ovenpå målfeltet.
\item \lstinline{BoardFrame} bruges til at repræsentere rammen for
  en spilleplade (de ydre vægge). Klassen skal have en konstruktor der tager to
  heltals argumenter, $r$ og $c$, som angiver størrelsen på pladen.
\item \lstinline{VerticalWall} bruges til at repræsentere en indre
  vertikal væg. Klassen skal have en konstruktor der tager tre
  heltals argumenter, $r$, $c$ og $n$, som angiver startfelter for
  væggen, $(r,c)$, samt længden af væggen, $n$. Hvis $n$ er positiv
  løber væggen fra nord til syd, ellers løber den fra syd mod
  nord. Væggen er på østsiden af startfeltet.
\item \lstinline{HorizontalWall} bruges til at repræsentere en indre
  horisontal væg. Klassen skal have en konstruktor der tager tre
  heltals argumenter, $r$, $c$ og $n$, som angiver startfelter for
  væggen, $(r,c)$, samt længden af væggen, $n$. Hvis $n$ er positiv
  løber væggen fra vest til øst, ellers løber den fra øst mod
  vest. Væggen er på sydsiden af startfeltet.
\end{itemize}

Hvis I får behov for det må I gerne tilføje tilstand (data og
properties) samt flere metoder til alle klasser. Men de skal kunne
bruges med de angivende minimums specifikationer.




%%% Local Variables:
%%% mode: latex
%%% TeX-master: "main"
%%% End:
