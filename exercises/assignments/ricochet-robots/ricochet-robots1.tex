\textbf{Spilelementer}




Vi bruger klassen \lstinline{BoardElement} til at repræsentere et
spilelement og \lstinline{BoardPiece} til et spilelement som fylder
netop et felt og kan flyttes. Tag udgangspunkt i
følgende signaturer:

\begin{lstlisting}
type Direction = North | South | East | West
type Action =
  | Stop of Position
  | Continue of Direction * Position
  | Ignore
type BoardElement =
  class
    new : unit -> BoardElement
    abstract member Interact : BoardPiece -> Direction -> Action
    abstract member RenderOn : BoardDisplay -> unit
    override Interact : BoardPiece -> Direction -> Action
  end
and BoardPiece =
  class
    inherit BoardElement
    new : x:int * y:int * s:string -> BoardPiece
    member Position : int * int
  end
\end{lstlisting}

Hvis I får behov for det må I gerne tilføje tilstand (data og
properties), samt metoder.

Til at repræsentere robotter, indre og ydre vægge og målfelt bruger vi klasserne:

\begin{lstlisting}
type Robot =
  class
    inherit BoardPiece
    new : x:int * y:int * name:string -> Robot
    override Interact : other:BoardPiece -> dir:Direction -> Action
    member Move : dir:Direction -> unit
    member Name : string
  end
type Goal =
  class
    inherit BoardPiece
    new : x:int * y:int -> Goal
    member GameOver : robots:Robot list -> bool
  end
type VerticalWall =
  class
    inherit BoardElement
    new : x:int * y:int * length:int -> VerticalWall
    override Interact : robot:BoardPiece -> dir:Direction -> Action
    override RenderOn : canvas:BoardDisplay -> unit
  end
type HorizontalWall =
  class
    inherit BoardElement
    new : x:int * y:int * length:int -> HorizontalWall
    override Interact : robot:BoardPiece -> dir:Direction -> Action
    override RenderOn : canvas:BoardDisplay -> unit
  end
type BoardFrame =
  class
    inherit BoardElement
    new : rows:int * cols:int -> BoardFrame
    override Interact : robot:BoardPiece -> dir:Direction -> Action
    override RenderOn : canvas:BoardDisplay -> unit
  end
\end{lstlisting}

% En spiller starter med ti hit points. En
% spiller er død hvis de har mindre end nul hit points. En spiller
% har et maksimum hit points de kan helbredes op til (I betemmer
% hvor mange, husk at dokumentere det i rapporten).

% Metoderne \lstinline{Damage} og \lstinline{Heal} bruge til at gøre
% skade på og, henholdsvis, helbrede spilleren med et antal hit points.

% Til at repræsentere genstande og skabninger, som spilleren kan
% interagere med, bruger vi den abstrakte klasse \lstinline{Item}:

% \begin{lstlisting}
%   type Item =
%     class
%       inherit Entity
%       abstract member InteractWith : Player -> bool
%       member FullyOccupy : bool
%     end
% \end{lstlisting}

% Den måde en spiller interagerer med et \lstinline{Item} på, er ved at
% gå ind i \lstinline{Item} (det kommer vi tilbage til i næste
% delopgave). Til dette skal vi bruge \lstinline{FullyOccupy} til at
% sige om \lstinline{Item} fylder feltet helt ud eller om spilleren kan
% stå i samme felt som genstanden. Metoden \lstinline{InteractWith}
% bruges dels til at genstanden kan have effekter på spilleren, og dels
% så siger retur-værdien om genstanden stadigvæk skal være i verden
% (\lstinline{true}) efter interaktionen, eller om den skal fjernes
% (\lstinline{false}) fra verden.

% Implementér følgende fem konkrete klasser der nedarver fra \lstinline{Item}:

% \begin{itemize}
% \item \lstinline{Wall} der fylder et helt felt, men ellers ikke har effekter på
%   spilleren.
% \item \lstinline{Water} der ikke fylder feltet helt ud, og helbreder med to
%   hit points.
% \item \lstinline{Fire} der ikke fylder feltet helt ud, og giver ét hit point i
%   skade ved hver interaktion med spilleren. Når spilleren har
%   interagereret fem gange med ilden går den ud.
% \item \lstinline{FleshEatingPlant} der fylder feltet helt ud, og giver fem hit point i
%   skade ved hver interaktion med spilleren.
% \item \lstinline{Exit} vejen ud af dungeon!
% \end{itemize}



%%% Local Variables:
%%% mode: latex
%%% TeX-master: "main"
%%% End:
