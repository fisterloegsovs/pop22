\label{sierpinskikeys.ov} I stedet for at benytte
  funktionen \lstinline{ImgUtil.runSimpleApp} er det nu meningen at du skal benytte
  funktionen \lstinline{ImgUtil.runApp}, som giver mulighed for at din løsning
  kan styres ved brug af tastaturet. Funktionen \lstinline{ImgUtil}
  har følgende type:
\begin{footnotesize}
\begin{lstlisting}[numbers=none,frame=none,mathescape]
val runApp : string -> int -> int
          -> (int -> int -> 's -> canvas)
          -> ('s -> Key -> 's option)
          -> 's -> unit
\end{lstlisting}
\end{footnotesize}
De tre første argumenter til \lstinline{runApp} er vinduets titel (en
streng) samt vinduets initielle vidde og højde. Funktionen
\lstinline{runApp} er parametrisk over en brugerdefineret type af tilstande (\lstinline{'s}). Antag at funktionen kaldes som følger:
\begin{footnotesize}
\begin{lstlisting}[numbers=none,frame=none,mathescape]
do runApp title width height draw react init
\end{lstlisting}
\end{footnotesize}
Dette kald vil starte en GUI applikation med titlen \lstinline{title},
vidden \lstinline{width} og højden \lstinline{height}. Funktionen
\lstinline{draw}, som brugeren giver som 4.\@ argument kaldes initielt
når applikationen starter og hver gang vinduets størrelse justeres
eller ved at funktionen \lstinline{react} er blevet kaldt efter en
tast er trykket ned på tastaturet. Funktionen \lstinline{draw} modtager
også (udover værdier for den aktuelle vidde og højde) en værdi for den
brugerdefinerede tilstand, som initielt er sat til værdien
\lstinline{init}. Funktionen skal returnere et canvas, som for
eksempel kan konstrueres med funktionen \lstinline{ImgUtil.mk} og
ændres med andre funktioner i \lstinline{ImgUtil}
(f.eks. \lstinline{setPixel}).

Funktionen \lstinline{react}, som brugeren giver som 5.\@ argument
kaldes hver gang brugeren trykker på en tast. Funktionen tager som
argument:
\begin{itemize}
\item en værdi svarende til den nuværende tilstand for
applikationen, og
\item et argument der kan benyttes til at afgøre hvilken
tast der blev trykket på.\footnote{Hvis \lstinline{k} har typen
  \lstinline{Gdk.Key} kan betingelsen
  \lstinline{k = Gdk.Key.d} benyttes til at
  afgøre om det var tasten ``d'' der blev trykket på. Desværre er det ikke muligt med den nuværende version af \lstinline{ImgUtil} at reagere på tryk på piletasterne.}
\end{itemize}
Funktionen kan
nu (eventuelt) ændre på dens tilstand ved at returnere en ændret værdi
for tilstanden.

Tilpas applikationen således at dybden af fraktalen kan styres ved
brug af piletasterne, repræsenteret ved værdierne
\lstinline{Gdk.Key.u} og \lstinline{Gdk.Key.d}.
