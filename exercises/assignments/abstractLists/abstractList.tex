A linked list is an abstract datastructure, handily built in to \texttt{F\#}.

In this exercise you will create two implementations of a linked list, one only supporting \texttt{int} and a generic linked list that can be used with any datatype, just like the built-in \texttt{F\#} lists.

\begin{enumerate}
\item Write off the following signature in \texttt{intLinkedList.fsi} and write a corresponding implementation in \texttt{intLinkedList.fs}
  \begin{lstlisting}
module IntLinkedList
type intLinkedList = Nil | Cons of int * intLinkedList
val head : intLinkedList -> int 
val tail : intLinkedList -> intLinkedList
val isEmpty : intLinkedList -> bool
val length : intLinkedList -> int
val add : int -> intLinkedList -> intLinkedList
\end{lstlisting}

Write a small test program to ensure your implementation works as expected.
You can take inspiration from the following listing:
\begin{lstlisting}
open IntLinkedList
let emptyList = Nil
let l1 = Cons (1, Nil)
let l2 = Cons (1, Cons (2, Nil))
let l3 = add 3 l2
isEmpty emptyList |> printfn "Empty list is empty: %A"
isEmpty l1 |> not |> printfn "Non-empty list is not empty: %A"
head l1 = 1 |> printfn "head gives the first element: %A"
tail l1 = Nil |> printfn "tail gives the rest of the list: %A"
length l3 = 3 |> printfn "l3 has length 3: %A"  
\end{lstlisting}

\item
  In the previous sub-exercise the linked list is restricted to the type \texttt{int}.
  In this sub-exercise you should construct a generic linked list module.
  \item Write off and finish the following signature in \texttt{linkedList.fsi} and write a corresponding implementation in \texttt{linkedList.fs}. 
    \begin{lstlisting}
module LinkedList
type LinkedList<'a> = Nil | Cons of 'a * intLinkedList<'a>
val head : LinkedList<'a> -> 'a
val tail : ?? // Fill in yourself
val isEmpty : ?? // Fill in yourself
val length :  ?? // Fill in yourself
val add :  ?? // Fill in yourself
\end{lstlisting}

Write a small test-program showing you can construct linked lists of all types.
You should be able to reuse \texttt{all} of your test from the previous sub-exercise and thus create linked lists of \lstinline{LinkedList<int>}, as well as the following:

\begin{lstlisting}
open LinkedList
let emptyList = Nil
let l1Float = add 2.0 emptyList |> add 3.14 // A float list
let l1String = add "Linked lists are cool!" emptyList
let l2String = add "What is cool?" l1String 
// A list of int lists
let intLstLst = add l1 emptyList |> add l2 |> add emptyList
// a list of string lists
let strLstLst = add l1String emptyList |> add l2String
\end{lstlisting}


\item Implement \texttt{fold} for your linked list module, similar to \texttt{List.fold}.
\item Implement \texttt{foldBack} for your linked list module, similar to \texttt{List.foldBack}.  
\item Implement \texttt{map} for your linked list module, similar to \texttt{List.map}.

\end{enumerate}
