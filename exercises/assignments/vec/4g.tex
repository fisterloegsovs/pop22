\item Skriv et bibliotek \filename{vec2d.fs}, som implementerer følgende signatur fil:
\fsSignature{vec2d}{vec2d}{A signature file.}{}
\item Skriv en White-box afprøvning af biblioteket.
\item Punkter på en cirkel med radius 1 kan beregnes som $(\cos \theta, \sin \theta), \;\theta\in [0,2\pi)$. Betragt det lukkede polygon, som består af $n>1$ punkter på en cirkel, hvor $\theta_i = \frac{2\pi i}{n},\; i = 0..(n-1)$. 
  \begin{enumerate}
  \item Skriv et program med en funktion,
  \begin{quote}
    \lstinline{polyLen : n:int -> float}
  \end{quote}
som benytter ovenstående bibliotek til at udregne længden af polygonet. Længden udregnes som summen af længden af vektorerne mellem nabopunkter. Programmet skal desuden udskrive en tabel af længder for et stigende antal værdier $n$, og resultaterne skal sammenlignes med omkredsen af cirklen med radius $1$. 
\item Udform en hypotese ud fra tabellen for længden af polygonet når $n\rightarrow\infty$.
  \end{enumerate}
\item Biblioteket \filename{vec2d} tager udgangspunkt i en repræsentation af vektorer som par (2-tupler). Lav et udkast til en signaturfil for en variant af biblioteket, som ungår tupler helt. Diskut\'{e}r eventuelle udfordringer og større ændringer, som varianten ville kræve både for implementationen og programmet.
