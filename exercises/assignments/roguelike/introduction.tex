
Denne opgave går ud på at lave et såkaldt retro-style \emph{roguelike}
spil. Et roguelike går ud på at spilleren skal udforske en verden og løse
nogle opgaver, ofte er denne verden et underjordisk fantasy
\emph{dungeon} befolket af monstre som skal nedkæmpes, og gåder der
skal løses.

I denne opgave skal der arbejdes med at lave et objekt-orienteret
design, som gør det nemt at udvide spillet med nye skabninger og
spil-mekanismer.

Opgaven er delt i fire dele. I den første delopgave skal der arbejdes
med at implementere en \emph{canvas} i terminalen til at vise vores
verden. Anden delopgave går ud på at lave et klasse-hierarki til at
repræsentere skabninger og genstande i verden. Endelig skal der i den
tredie delopgave arbejdes mod at sætte de forskellige dele sammen til
et samlet spil. Fjerde del indeholde en række forslag til udvidelser,
hvoraf I skal implementere mindst to.

I det følgende er der kun givet minimums-krav til hvilke metoder og
properties I skal implementere på jeres klasser. I må gerne lave
ekstra metoder eller hjælpe-funktioner, hvis I synes det kan hjælpe
til at skrive et mere elegant og forståeligt program.

\textbf{Rapport}

Ud over jeres programkode skal I også aflevere en rapport (skrevet i
\LaTeX). I rapporten skal I beskrive implementeringen af jeres
klasser, det vil sige hvilken skjult tilstand (interne variable og
lignende), som jeres metoder arbejder på.

Ligeledes skal rapporten indeholde et UML diagram over klasserne i
jeres løsning.


%%% Local Variables:
%%% mode: latex
%%% TeX-master: "main"
%%% End:
