\textbf{Terminal Brugergrænseflade}

For at vise spillets verden implementer vi en klasse
\lstinline{Canvas}, som er et gitter af felter. Hvor feltet
 der er i øverste ventre hjørne har position $(0,0)$.

Hvert felt har en \lstinline{char}, og så \emph{kan}
feltet have en forgrundsfarve, og det \emph{kan} have en baggrundsfarve.

Implementér en klassen \lstinline{Canvas} som har følgende signatur:

\begin{lstlisting}
  type Color = System.ConsoleColor
  type Canvas =
    class
      new : rows:int * cols:int -> Canvas
      member MaxX : int
      member MaxY : int
      member Set : x:int * y:int * cont:char * fg:Color * bg:Color -> unit
      member Show : unit -> unit
    end
\end{lstlisting}

Det vil sige:
\begin{itemize}
\item En konstruktør der tager antal rækker og koloner som argumenter.
\item To properties der angiver det maksimale række-indeks, \lstinline{MaxX}, og
  kolonne-indeks, \lstinline{MaxY}.
\item en metode \lstinline{Set} til at sætte indhold og farver på et felt.
\item en metode \lstinline{Show} til at vise en canvas i terminalen.
\end{itemize}

I rapporten skal I beskrive jeres designovervejelser, samt redegøre for
hvilken skjult tilstand en canvas har.

\textbf{Hints}: Brug følgende funktionalitet fra standard-biblioteket:

\begin{itemize}
\item \lstinline{System.Console.ForegroundColor <- System.ConsoleColor.White}
  til at sætte forgrundsfarven til hvid.
\item \lstinline{System.Console.BackgroundColor <- System.ConsoleColor.Blue} til
  at sætte baggrundsfarven til blå.
\item \lstinline{System.Console.ResetColor()} til at sætte farverne i terminalen
  tilbage til normal.
\end{itemize}


%%% Local Variables:
%%% mode: latex
%%% TeX-master: "main"
%%% End:
