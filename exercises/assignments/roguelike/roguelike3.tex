\textbf{Udvidelser}\\
Lav mindst 2 udvidelser til spillet og beskriv dem i jeres
rapport. Følgende er nogle forslag til udvidelser, men I må gerne selv lade
fantasien råde.

I er ligeledes velkommen til at udvide storyline.

\begin{itemize}
\item Teleport, lav en teleport der flytter spilleren fra et sted i
  verden til et (evt tilfældigt) andet sted i verdenen.
\item Udvid \lstinline{Item} så de kan påvirke verdenen. Fx, så kunne
  \lstinline{FleshEatingPlant} sætte en stikling (en ny
  \lstinline{FleshEatingPlant}) i et ledigt felt ved
  siden af den, hver tredje tur den ikke interagerer med spilleren.
\item Monstre der kan bevæge sig rundt i verden, fx tilfældigt hvis de
  er langt fra spilleren, men går mod spilleren hvis de er tæt på.
\item Udvid \lstinline{Player} med et \emph{inventory}, så man kan samle ting op i
  verden og flytte rundt på dem. Det kan fx bruges til at spilleren
  skal finde en nøgle for at komme gennem en dør.
\item Udvid \lstinline{Canvas} til at kunne vise emoji. Det kan gøres ved at
  hvert felt kan indeholde en \lstinline{string} frem for kun en \lstinline{char}, og så
  skal I være opmærksomme på at emoji ofte fylder det samme som to
  almindelige tegn.
\item Skriv en level-generator (stor udvidelse!)
\item Gør det muligt at indlæse et level fra en tekstfil
\item Udvid Player-klassen med hhv. \lstinline{Hunger} og \lstinline{Thirst}.

  Tilføj, fx, \lstinline{Food} og \lstinline{WaterBottle} som
  \lstinline{Item}s. For hvert træk bliver \lstinline{Player} mere
  sulten og tørstig. \lstinline{Player} dør, hvis \lstinline{Player}
  løber tør for enten mad eller vand.

\item Giv \lstinline{Player} en bue/magi/et sværd og gør det muligt at slås med monstre.
\item Tilføj krukker der kan ødelægges. Krukkerne indeholder måske
  guld, som \lstinline{Player} kan samle op.
\end{itemize}

%%% Local Variables:
%%% mode: latex
%%% TeX-master: "main"
%%% End:
