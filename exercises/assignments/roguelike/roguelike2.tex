\textbf{Verden}

Implementer klassen \lstinline{World}:

\begin{lstlisting}
  type World =
    class
      new : ...
      member AddItem : item:Item -> unit
      member Play : unit -> unit
    end
\end{lstlisting}

Metoden \lstinline{AddItem} bruges til at befolke verden med ting som spilleren
kan interagere med. Typisk inden spillet går i gang.

Metoden \lstinline{Play} bruges til at starte spillet, og tager sig af
interaktionen med brugeren via terminalen. Spillet er tur-baseret og
foregår på følgende vis:
\begin{enumerate}
\item Vis hvordan verden ser ud, samt om der er eventuelt er sket
  noget for spilleren
\item Hent brugerens træk som gives ved brug af pile-tasterne.
\item Afgør hvilke \lstinline{Item}s som brugeren eventuelt interagerer med,
  samt hvad det betyder for hvad spillerens position og helbred er.
\item Hvis spilleren er død eller hvis spilleren har fundet
  \lstinline{Exit} vis et afslutningsskærmbillede og stop
  spillet, ellers start forfra.
\end{enumerate}

Klassen \lstinline{World} samt de andre klasser fra de andre
delopgaver skal være i filen \texttt{rougelike.fs}. Lav derudover en
fil \texttt{roguelike-game.fsx}, der som minimum laver en ny verden og
kalder \lstinline{Play}.


\textbf{Basal Storyline}

Den mest basale udgave af spillet: Spilleren starter et sted i et
underjordisk dungeon, og skal finde udgangen. Når spilleren finder
udgangen skal de have mindst fem \emph{hit points} for at kunne tvinge
døren op og undslippe dungeon.

Det er op til jer hvordan dungeon skal se ud, hvor spilleren starter,
samt hvor mange genstande og skabninger der er i verden.


\textbf{Hints:}
\begin{itemize}
\item Det er en vigtig pointe at \lstinline{World} ikke tager sig af
  at rendere spilleren og \lstinline{Item}s i verden, men blot skaber
  en canvas, som de forskellige \lstinline{Entry} kan rendere sig selv
  på.
\item Brug \lstinline{System.Console.Clear()} at fjerne alt fra terminalen inden
  verden vises.
\item Brug \lstinline{Console.ReadKey(true)} til at hente et træk fra brugeren
\item Hvis \lstinline{key} er resultatet fra \lstinline{Console.ReadKey} så er \lstinline{key.Key}
  lig med \lstinline{System.ConsoleKey.UpArrow}, hvis brugere trykkede på
  op-pilen.
\end{itemize}





%%% Local Variables:
%%% mode: latex
%%% TeX-master: "main"
%%% End:
