\textbf{Genstande og Skabninger}

Vi bruger klassen \lstinline{Entity} til at repræsentere genstande og skabninger
i vores verden. Disse skal kunne renderes på en canvas. Tag udgangspunkt i
følgende erklæring:

\begin{lstlisting}
  type Entity() =
    abstract member RenderOn : Canvas -> unit
    default this.RenderOn canvas = ()
\end{lstlisting}

Hvis I får behov for det må I gerne tilføje tilstand (data og
properties), metoder og en anden default implementering af
\lstinline{RenderOn} til \lstinline{Entity}.

Til at repræsentere spilleren bruger vi klassen \lstinline{Player}:

\begin{lstlisting}
  type Player =
    class
      inherit Entity
      new : ...
      member Damage : dmg:int -> unit
      member Heal : h:int -> unit
      member MoveTo : x:int * y:int -> unit
      member HitPoints : int
      member IsDead : bool
    end
\end{lstlisting}

En spiller er død hvis de har mindre end nul hit points. En spiller
har et maksimum hit points de kan helbredes op til.

Til at repræsentere genstande og skabninger, som spilleren kan
interagere med, bruger vi den abstrakte klasse \lstinline{Item}
\kfl{Find på bedre navn}:

\begin{lstlisting}
  type Item =
    class
      inherit Entity
      abstract member InteractWith : Player -> unit
      member FullyOccupy : bool
    end
\end{lstlisting}

Den måde en spiller interagerer med et \lstinline{Item} på, er ved at
gå ind i \lstinline{Item} (det kommer vi tilbage til i næste
delopgave). Til dette skal vi bruge \lstinline{FullyOccupy} til at
sige om \lstinline{Item} fylder feltet helt ud eller om spilleren kan
stå i samme felt som genstanden. Metoden \lstinline{InteractWith}
bruges dels til at genstanden kan have effekter på spilleren, og dels
så siger retur-værdien om genstanden stadigvæk skal være i verden
(\lstinline{true}) efter interaktionen, eller om den skal fjernes
(\lstinline{false}) fra verden.

Implementér følgende fem konkrete klasser der nedarver fra \lstinline{Item}:

\begin{itemize}
\item \lstinline{Wall} der fylder et helt felt, men ellers ikke har effekter på
  spilleren.
\item \lstinline{Water} der ikke fylder feltet helt ud, og helbreder med to
  hit points.
\item \lstinline{Fire} der ikke fylder feltet helt ud, og giver ét hit point i
  skade ved hver interaktion med spilleren. Når spilleren har
  interagereret fem gange med ilden går den ud.
\item \lstinline{FleshEatingPlant} der fylder feltet helt ud, og giver fem hit point i
  skade ved hver interaktion med spilleren.
\item \lstinline{Exit} vejen ud af dungeon!
\end{itemize}



%%% Local Variables:
%%% mode: latex
%%% TeX-master: "main"
%%% End:
