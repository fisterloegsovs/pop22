\documentclass[a4paper]{article}

%%%%%%%%%%%%%%%%%%%%%%%%%%%%%%%%%%
% Package for making LaTeX properly handle utf8 characters set and danish language rules
\usepackage{cmap}
\usepackage[utf8]{inputenc}
\usepackage[danish]{babel}

%%%%%%%%%%%%%%%%%%%%%%%%%%%%%%%%%%
% Package for changing to a nicer font 
\usepackage [T1]{fontenc}

%%%%%%%%%%%%%%%%%%%%%%%%%%%%%%%%%%
% Package for conctroling the text area
\usepackage[margin=2.5cm]{geometry}

%%%%%%%%%%%%%%%%%%%%%%%%%%%%%%%%%%
% Package for inserting clickable hyperlinks in pdf versions as produced by pdflatex
\usepackage{hyperref}

%%%%%%%%%%%%%%%%%%%%%%%%%%%%%%%%%%
% Package for including figures. TeX and thus LaTeX was developped before the existence of directory file-structures, but the graphicspath let's you add directories, that the \includegraphics will search.
\usepackage{graphicx}
\graphicspath{{figures/}{anotherFigureDirectory/}}

%%%%%%%%%%%%%%%%%%%%%%%%%%%%%%%%%%
% Package for typesetting programs. Listings does not support fsharp, but a little modification goes a long way
\usepackage{listings}
\usepackage{color}

\definecolor{bluekeywords}{rgb}{0.13,0.13,1}
\definecolor{greencomments}{rgb}{0,0.5,0}
\definecolor{turqusnumbers}{rgb}{0.17,0.57,0.69}
\definecolor{redstrings}{rgb}{0.5,0,0}

\lstdefinelanguage{FSharp}
                {morekeywords={let, new, match, with, rec, open, module, namespace, type, of, member, and, for, in, do, begin, end, fun, function, try, mutable, if, then, else},
    keywordstyle=\color{bluekeywords},
    sensitive=false,
    morecomment=[l][\color{greencomments}]{///},
    morecomment=[l][\color{greencomments}]{//},
    morecomment=[s][\color{greencomments}]{{(*}{*)}},
    morestring=[b]",
    stringstyle=\color{redstrings}
    }
%%%%%%%%%%%%%%%%%%%%%%%%%%%%%%%%%%
% Package for extended math settings, e.g. \eqref
\usepackage{amsmath}

%%%%%%%%%%%%%%%%%%%%%%%%%%%%%%%%%%
% These will be the title and author, as included when \maketitle is called.
\title{Assignment 6}
\author{Jon Sporring}

\begin{document}
\maketitle % Insert title etc.

% sections are the outer logical structure when using documentclass-article, while chapter would be the outer structure, when using report documentclass
\section{Preface}
This document is a response to Assignment~6 on the course Introduction to Functional Programming at UESTC, 2019/7/8-12.

\section{Introduction}
Assignment 6 includes the code \texttt{recursiveMapFoldFilter.fsx},
which uses \texttt{List.fold} and \texttt{List.filter}. My task is to
make my own implementations of these two functions and to verify that they work.

\texttt{List.fold} is defined as
\begin{quote}
  \texttt{List.fold : f:('a->'b->'a) -> acc:'a -> lst:'b list -> 'a}
\end{quote}
and for $acc=a$ and $lst=[e_1;e_2;\dots;e_n]$ computes $f \dots (f (f a e_1) e_2)\dot e_n$. For example, a call could be \texttt{List.fold (fun acc elm -> acc+elm) 0 [1;2]} which sum 0 with all the elements in the list and give 3.

\texttt{List.filter} is defined as
\begin{quote}
  \texttt{List.filter : p:('a->bool) -> lst:'a list -> 'a list}
\end{quote}
and returns the list of elements in \texttt{lst} for which \texttt{p} returns \texttt{true}. For example, a call could be \texttt{List.filter (fun elm -> elm\%2=0) [0..5] }, which returns \texttt{[0; 2; 4]}, i.e., all the even numbers from the list.

\section{Problem analysis and design}
The course requires us to use the functional programming paradigm and F\#, so I must use recursion and no mutable values, for-loops, nor while-loops.

\texttt{List.fold} takes a function, an initial accumulator, and a list. I have considered the following case:
\begin{description}
\item[The list is empty:] I have tested \texttt{List.fold}, and it returns the initial accumulator.
\item[The list is non-empty:] Here, \texttt{List.fold} acts as explained in the introduction.
\item[Error cases:] If arguments are missing, then F\# gives a syntax error.
\end{description}

\texttt{List.filter} takes a function and a list. I have considered the following case:
\begin{description}
\item[The list is empty:] I have tested \texttt{List.filter}. If the type of the empty filter can be determined by the function, then it returns an empty list. Otherwise, it gives an error.
\item[The list is non-empty:] Here, \texttt{List.filter} acts as explained in the introduction.
\item[Error cases:] If arguments are missing, then F\# gives a syntax error.
\end{description}

\section{Program description}
The resulting code pieces are so short that we have chosen to include them here in their entirety.

I have used recursion and match-with to implement fold. There are 2 cases: empty and non-empty. For the empty, list we return \texttt{acc}, and for the non-empty, we assume that \texttt{myFold} works, so I split the list into its first element and the rest and call \texttt{myFold} with a new accumulator and the rest. This is shown in Figure~\ref{fig:fold}.
\begin{figure}
  \lstset{language=FSharp}
\begin{lstlisting}
// fold f [a1; ...; an] = f ... (f (f acc a1) a2) ...) an
let rec myFold (f: 'b -> 'a -> 'b) (acc: 'b) (lst: 'a list) : 'b =
  // List.fold f acc lst
  match lst with
    | []     -> acc
    | a::rst -> myFold f (f acc a) rst
\end{lstlisting}
  \caption{The implemented code for fold.}
  \label{fig:fold}
\end{figure}

I have also used recursion and match-with to implement filter. It is
implemented similarly to fold. For an empty list, an empty list is
returned. This will give similar errors as \texttt{List.filter}. For a
non-empty list, I assume that \texttt{myFilter} works, and hence, if \texttt{p} is true for the first element, I prepend it to the result of \texttt{myFilter} on the rest. Otherwise, I just return the result of \texttt{myFilter} on the rest. I have found no way to avoid having two calls to \texttt{myFilter p rst}. The code is shown in Figure~\ref{fig:filter}.
\begin{figure}
  \lstset{language=FSharp}
\begin{lstlisting}
// filter f [a1; ...; an] = [ai; aj; ak ...], where f ai = true etc.,
let rec myFilter (p: 'a -> bool) (lst: 'a list) : 'a list =
  // List.filter p lst
  match lst with
    | a::rst -> if p a then a :: (myFilter p rst) else myFilter p rst
    | []     -> []
\end{lstlisting}
  \caption{The implemented code for filter.}
  \label{fig:filter}
\end{figure}

\section{Testing and experiements}
The code \texttt{recursiveMapFoldFilter.fsx} includes testing of my code, so I have run the modified program using a long list, which statistically should cover many although not all possible versions. The result is shown in Figure~\ref{fig:test}
\begin{figure}
  \lstset{language=bash}
\begin{lstlisting}
$ mono recursiveMapFoldFilter.exe fold 10
Random list is: [9; 5; 9; 6; 7; 6; 1; 7; 2; 9]
Result is CORRECT  : [18; 4; 14; 2; 12; 14; 12; 18; 10; 18]
$ mono recursiveMapFoldFilter.exe filter 10
Random list is: [9; 4; 6; 3; 1; 4; 4; 4; 5; 6]
Result is CORRECT  : [9; 6; 5; 6]
\end{lstlisting}
  \caption{Output from the console of test runs.}
  \label{fig:test}
\end{figure}
In the figure, the results of running my code is identical to the \texttt{List} implementations, so I conclude that my code is working correctly.

\section{Discussion and/or Conclusion}
I have worked with Assignment~6 and implemented my own versions of fold and filter. I have used recursion and match-statements, and I have tested the code using the supplied test function. My conclusion is that my code is running correctly.
\end{document}

