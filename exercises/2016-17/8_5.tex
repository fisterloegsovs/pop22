\documentclass[a4paper]{article}

\usepackage{cmap}
\usepackage[utf8x]{inputenc}
\usepackage{latexsym}
\usepackage[danish]{babel}
\usepackage{graphicx}
\usepackage{graphpap}
\usepackage{color}
\usepackage{hyperref}
\usepackage[all]{hypcap}
\usepackage{enumerate}
\usepackage[margin=2.5cm]{geometry}

\begin{document}
\title{Programmering og Problemløsning\\
Datalogisk Institut, Københavns Universitet\\
Øvelsesopgaver til uge 11 og 12}

\author{Torben Mogensen \& Jon Sporring}
\date{\today}

\maketitle

\noindent
I uge 11 og 12 (28/11 -- 9/12) er der ikke afleveringsopgaver (udover
8g, som afleveres 30/11).  Der er dog stadig øvelser i nyt stof, og
denne ``ugeseddel'', der er uden for nummerrækken, beskriver disse.

\subsubsection*{Øveopgaverne er:}

\begin{enumerate}[ø1.]
\item Denne opgave handler om at arbejde med vinduer i WinForms
  \begin{enumerate}
  \item Lav et program, som åbner et vindue og tegner en
    firkant. Hjørnerne skal være 25\% af vinduets størrelse fra
    vinduets hjørner, således at hvis billedet tegnbare areal har
    størrelse (bredde, højde), så har øverste venstre hjørne position
    (bredde/4, højde/4)
  \item Lav et tegneprogram, som når man klikker på et punkt i et
    vindue, så bliver punktet gemt i en liste, og der bliver tegnet en
    lille firkant med centrum i punktet.
  \item Udvid oventstående tegneprogram, således at man kan slette
    punkter igen. Man skal kunne slette ved at trykke tæt på et
    eksisterende punkt, og der skal indføres tastetryk, således at hvis
    man trykker 't', så vil programmet tegne firkanter næste gang der
    trykkes med musen og hvis man trykker 's', så vil man slette næste gang.
  \end{enumerate}

  \item Denne opgave handler om opbygning af moduler
    \begin{enumerate}
    \item Lav en fil \texttt{Figure.fs}, der indeholder definitionerne af typer og funktioner, I brugte til løsning af øvelsesopgaverne om figurer på ugeseddel 6, mere specifikt typerne \texttt{point}, \texttt{colour} og \texttt{figure} og funktionerne \texttt{colourAt}, \texttt{makePicture}, \texttt{checkFigure}, \texttt{move} og \texttt{boundingBox}.  Tilføj linjen \texttt{module Figure} som første linje i denne fil.
    \begin{enumerate}
    \item Kør denne fil ved at skrive \texttt{fsharpi -r makeBMP.dll Figure.fs}.  Bemærk, hvad fsharpi fortæller om de definerede navne.

    \item Lav fra denne kørsel af fsharpi en definition af en variabel med navn \texttt{o63}, som indeholder figuren fra øveopgave 6.3.  Bemærk, at du skal bruge modulnavnet \texttt{Figure} som præfix, når du bruger konstruktorerne \texttt{Circle}, \texttt{Rectangle} og \texttt{Mix}.

    \item Åben modulet, og kald \texttt{makePicture} (nu uden præfix) til at lave en 100×150 pixel stor billedfil for figuren \texttt{o63}.
    \end{enumerate}

  \item Lav en fil \texttt{Figure.fsi}, der indeholder en signatur til filen \texttt{Figure.fs}.

    \begin{enumerate}
    \item Oversæt modulet med kommandoen \texttt{fsharpc -r makeBMP.dll -a Figure.fsi Figure.fs}

    \item Kør \texttt{fsharpi -r Figure.dll}, gentag definitionen af \texttt{o63} og lav et funktionskald, der finder bounding box til denne.
    \end{enumerate}

  \item Udvid modulet fra forrige opgave, så kan vise billedet i et vindue: Kopier billedet ind i et bitmap:
\begin{verbatim}
let bitMap = new System.Drawing.Bitmap (width, height)
\end{verbatim}
  og vis resultatet med en picturebox control:
\begin{verbatim}
let pictureBox = new System.Windows.Forms.PictureBox ()
\end{verbatim}

  \end{enumerate}
\item Denne opgave handler om at definere overloadede operatorer.

  \begin{enumerate}
  \item  Udvid \texttt{Figure.fsi} og \texttt{Figure.fs} med overloadede
  definitioner af infixoperatorerne + og * (jvf.\ HR afsnit 7.3 og
  7.4) med typerne

  \texttt{( + ) : figure * figure -> figure}

  og

  \texttt{( * ) : (int * int) * figure -> figure}

  hvor \texttt{+} anvender konstruktøren \texttt{Mix} på sine argumenter
  og \texttt{*} anvender funktionen \texttt{move} på sine argumenter.

\item Genoversæt modulet og lav igen opgave ø2(b), men hvor du bruger \texttt{+} i
stedet for \texttt{Mix} til at konstruere figuren.  Bemærk, at du ikke
skal præfixe \texttt{+} med modulnavnet.

\item Lav en ny figur, hvor du bruger både \texttt{+} og \texttt{*}.

  \end{enumerate}
\end{enumerate}


\end{document}
