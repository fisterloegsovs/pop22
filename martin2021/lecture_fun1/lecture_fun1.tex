\documentclass[rgb]{beamer}

\usepackage[english]{babel}
\usepackage[utf8]{inputenc}
\usepackage{xcolor}
\usepackage{listings}
\usepackage{adjustbox}
\usepackage{amsmath}
\usepackage{multirow}
\usepackage[linewidth=1pt]{mdframed}

% Graphics
\usepackage{graphicx}

\usepackage{tikz}
\usetikzlibrary{calc,shapes.multipart,chains,arrows}

% Font
\usepackage{paratype}
\setbeamerfont{frametitle}{family=\bf}

% Beamer theme settings
\usecolortheme{seagull}
\setbeamertemplate{itemize item}{\raisebox{0.8mm}{\rule{1.8mm}{1.2mm}}}
\usenavigationsymbolstemplate{} % no navigation buttons

\usepackage{listings}

% Define Language
\lstdefinelanguage{fsharp}
{
  % list of keywords
  morekeywords={
    and,
    do,
    else,
    exception,
    for,
    fun,
    function,
    if,
    in,
    let,
    match,
    module,
    mutable,
    open,
    of,
    rec,
    then,
    try,
    type,
    unsafe,
    use,
    val,
    when,
    while,
    with,
  },
  sensitive=true, % keywords are not case-sensitive
  morecomment=[l]{//}, % l is for line comment
%  otherkeywords={>,<,=,<=,>=,!,*,/,-,+,|,&,||,&&,==,=>},
  morestring=[b]" % defines that strings are enclosed in double quotes
}

% Define Colors
\usepackage{color}
\definecolor{eclipseBlue}{RGB}{42,0.0,255}
\definecolor{eclipseGreen}{RGB}{63,127,95}
\definecolor{eclipsePurple}{RGB}{127,0,85}

\newcommand{\fop}[1]{\mbox{\ttfamily\color{eclipseBlue}#1}}
\newcommand{\fw}[1]{\mbox{\ttfamily\bfseries\color{eclipsePurple}#1}}

% Set Language
\lstset{
  language={fsharp},
  basicstyle=\ttfamily, % Global Code Style
  captionpos=b, % Position of the Caption (t for top, b for bottom)
  extendedchars=true, % Allows 256 instead of 128 ASCII characters
  tabsize=2, % number of spaces indented when discovering a tab
  columns=fixed, % make all characters equal width
  keepspaces=true, % does not ignore spaces to fit width, convert tabs to spaces
  showstringspaces=false, % lets spaces in strings appear as real spaces
  breaklines=true, % wrap lines if they don't fit
  frame=trbl, % draw a frame at the top, right, left and bottom of the listing
  frameround=tttt, % make the frame round at all four corners
  framesep=4pt, % quarter circle size of the round corners
  numbers=left, % show line numbers at the left
  numberstyle=\small\ttfamily, % style of the line numbers
  commentstyle=\slshape\bfseries\color{eclipseGreen}, % style of comments
  keywordstyle=\bfseries\color{eclipsePurple}, % style of keywords
  stringstyle=\color{eclipseBlue}, % style of strings
  emph=[1] {
    false,
    true,
    Set,
    Map,
    List,
    ImgUtil,
    Pegs,
    String,
    Array,
    Array2D
  },
  emphstyle=[1]{\color{eclipseBlue}},
  moredelim=**[is][\color{red}]{@@}{@@}
}

\newcommand{\theyear}{2020}
\newcommand{\sem}[1]{[\![#1]\!]}
\newcommand{\seme}[1]{\sem{#1}\varepsilon}
\newcommand{\semzero}[1]{\sem{#1}_0}

\newcommand{\emptymap}{\{\}}
\newcommand{\fracc}[2]{\begin{eqnarray} \frac{\begin{array}{c} #1
    \end{array}}{\begin{array}{c} #2 \end{array}} \end{eqnarray}}
\newcommand{\sembox}[1]{\hfill \normalfont \mbox{\fbox{\(#1\)}}}
\newcommand{\sempart}[2]{\subsubsection*{\rm\em #1 \sembox{#2}}}
\newcommand{\axiom}[1]{\begin{eqnarray} \begin{array}{c} #1 \end{array} \end{eqnarray}}
\newcommand{\fraccn}[2]{\refstepcounter{equation}\mbox{$\frac{\begin{array}{c} #1 \end{array}}{\begin{array}{c} #2 \end{array}}$}~(\arabic{equation})}
\newcommand{\fraccc}[2]{\mbox{$\frac{\begin{array}{c} #1 \end{array}}{\begin{array}{c} #2 \end{array}}$}}
\newcommand{\onepart}[1]{\noindent\hfill#1\hfill~\vspace{2mm}}
\newcommand{\twopart}[2]{\noindent\hfill#1\hfill#2\hfill~\vspace{2mm}}
\newcommand{\threepart}[3]{\noindent\hfill#1\hfill#2\hfill#3\hfill~\vspace{2mm}}
%\newcommand{\axiomm}[1]{\refstepcounter{equation}\mbox{$\begin{array}{c} #1 \end{array}$}~(\arabic{equation})}
\newcommand{\axiomm}[1]{$\begin{array}{c} #1 \end{array}$}
%\newcommand{\ar}[1]{\stackrel{#1}{\longrightarrow}}
\newcommand{\vd}{\vdash}
\newcommand{\Ran}{{\rm Ran}}
\newcommand{\Dom}{{\rm Dom}}
\newcommand{\kw}[1]{\texttt{#1}}
\newcommand{\id}[1]{\mbox{\it{#1}}}
\newcommand{\rarr}{\rightarrow}
\newcommand{\eval}{\rarr}
\newcommand{\evals}{\leadsto}
\newcommand{\larr}{\leftarrow}

\newcommand{\head}[1]{\vspace{3mm} \textbf{\normalsize #1}}
\newcommand{\headsp}[1]{\head{#1}\vspace{1ex}}
\newcommand{\size}{\ensuremath{\mathrm{size}}}
\renewcommand{\log}{\ensuremath{\mathrm{log}}}

\newcommand{\setallthemecolors}[1]{%
\setbeamercolor*{palette primary}{use=structure,fg=white,bg=#1}%
\setbeamercolor*{palette secondary}{use=structure,fg=white,bg=#1}%
\setbeamercolor*{palette tertiary}{use=structure,fg=white,bg=#1}}

\definecolor{black}{RGB}{0,0,0}
\definecolor{maroon}{RGB}{128,0,0}
\definecolor{olive}{RGB}{128,128,0}
\definecolor{green}{RGB}{0,128,0}
\definecolor{purple}{RGB}{128,0,128}
\definecolor{teal}{RGB}{0,128,128}
\definecolor{darkteal}{RGB}{0,92,92}
\definecolor{navy}{RGB}{0,0,128}
\definecolor{gray}{RGB}{128,128,128}
\definecolor{darkgray}{RGB}{60,60,60}
\definecolor{darkred}{RGB}{139,0,0}

%palette

% #173F5F (dark blue)
\definecolor{darkblue}{RGB}{23,63,95}
% #20639B (blue)
\definecolor{blue}{RGB}{32,99,155}
% #3CAEA3 (green)
\definecolor{magenta}{RGB}{60,174,163}
% #F6D55C (yellow)
\definecolor{yellow}{RGB}{246,213,92}
% #ED553B (red)
\definecolor{red}{RGB}{237,85,59}


\usecolortheme{whale}
\useoutertheme{infolines}
\useinnertheme{rectangles}

\newcommand{\popsettitle}[2]{%
\setallthemecolors{#1}%
\newcommand{\popemne}{#2}%
\title{Programmering og Problemløsning}%
\subtitle{#2}%
\author{Martin Elsman}%
\date{}%
\institute[DIKU]{Datalogisk Institut, Københavns Universitet (DIKU)}}

\newcommand{\popmaketitleframe}{%
  \frame{\titlepage%
   \vspace{-15mm}%
   \par\noindent\rule{\textwidth}{0.4pt}%

   \vspace{4mm}%
   \tableofcontents%
   \vspace{-4mm}%
   \par\noindent\rule{\textwidth}{0.4pt}%
  }%
  \section*{\popemne}%
}


\popsettitle{darkred}{Højereordens funktioner (Del 1)}  % see ../util.tex for colors

\begin{document}

\popmaketitleframe

\renewcommand{\sp}{\vspace{1ex}}
\newcommand{\shead}[1]{\vspace{1ex}\head{#1}\vspace{1ex}}

%%%%%%%%%%%%%%%%%%%%%%%%%%%%%%%%%%%%%%%%%%%%%%%%
%\subsection{Introduktion}
%%%%%%%%%%%%%%%%%%%%%%%%%%%%%%%%%%%%%%%%%%%%%%%%

\subsection{Introduktion og motivation}
\begin{frame}[fragile]
\begin{footnotesize}

  \head{Højereordens Funktioner}

  \vspace{1ex}

  \begin{minipage}{.75\textwidth}
  \begin{enumerate}
  \item \textbf{Brug af højereordens funktioner:}

    \begin{itemize}
    \item Funktionsbegrebet genbesøgt.

      (closures)
    \item Funktioner der tager funktioner som argument.

      (listeoperationer, generisk sortering, ...)

    \item Funktioner der returnerer funktioner.

      (funktionssammensætning, currying, uncurryring)

    \item Funktioner som/i datastrukturer.

      (afbildninger, pull arrays, funktionelle billeder, ...)

    \end{itemize}

  \item \textbf{Funktionelle Billeder.}

    Vi skal se hvordan vi ved at forstå et billede som en \textbf{funktion fra
    punkter i planen til farver} kan definere (og tegne) en lang række interessante billeder.

  \end{enumerate}
  \end{minipage}~~~\begin{minipage}{.2\textwidth}
  \includegraphics[width=1.2\textwidth]{../images/wav.png}
  \end{minipage}

\end{footnotesize}
\end{frame}

\subsection{Funktionsbegrebet og closures}

\begin{frame}[fragile]
\begin{footnotesize}

  \head{Closures}
  \vspace{1ex}

%  Vi har tidligere arbejdet med at definere forskellige konkrete
%  funktioner uden dog helt præcist at have defineret hvordan en
%  funktion er repræsenteret på køretid.

  På køretid er en funktion repræsenteret ved en såkaldt
  \emph{closure} der, abstrakt set, indeholder tre dele:
  \begin{enumerate}
  \item En definition af de \emph{formelle parametre} til funktionen (læs: variabelnavne).
  \item En \emph{omgivelse} der indeholder værdier for de variabler
    der ikke er formelle parametre til funktionen.
  \item Kode for \emph{kroppen} af funktionen.
  \end{enumerate}

  \shead{Eksempel F\# funktion defineret med \lstinline{let}:}

  \begin{lstlisting}[numbers=none,frame=none,mathescape]
    let a = 5+3
    let f x = x + a
  \end{lstlisting}

  På køretid er funktionen \lstinline{f} repræsenteret som

 $$\mathtt{f} ~~~\mapsto ~~~ (~~\mathtt{x}~~, ~~\{\mathtt{a} \mapsto 8\}~~, ~~\mathtt{x + a}~~)$$

  \vspace{-2mm}
  \head{Bemærk:}
  \begin{itemize}
  \item Med denne repræsentation kan funktionen benyttes også fra
    steder i programmet hvor \lstinline{a} ikke er kendt (f.eks. i et
    eksternt bibliotek).
    \end{itemize}
\end{footnotesize}
\end{frame}

\subsection{Anonyme funktioner}

\begin{frame}[fragile]
\begin{footnotesize}

  \head{Anonyme Funktioner}
  \vspace{1ex}

  Den samme kode kunne skrives ved brug af en \emph{anonym} funktion:
  \begin{lstlisting}[numbers=none,frame=none,mathescape]
    let a = 5+3
    let f = (fun x -> x + a)
  \end{lstlisting}

  \head{Repræsentation:}

 $$\mathtt{f} ~~~\mapsto ~~~ (~~\mathtt{x}~~, ~~\{\mathtt{a} \mapsto 8\}~~, ~~\mathtt{x + a}~~)$$

  \head{Bemærk:}
  \begin{itemize}
  \item Der er ingen forskel på repræsentationen af de to definitioner af \lstinline{f}.
  \item Anonyme funktioner benyttes ofte når en funktion umiddelbart
    skal gives til en anden funktion som argument, når vi umiddelbart
    skal gemme en funktion i en datastruktur, eller når vi umiddelbart
    skal returnere funktionen fra en anden funktion.
  \end{itemize}

\end{footnotesize}
\end{frame}

\subsection{Funktioner der tager funktionsargumenter}

\begin{frame}[fragile]
\begin{footnotesize}

  \head{Funktioner der tager funktionsargumenter}
  \vspace{1ex}

  Vi har tidligere set eksempler på funktioner der tager funktioner
  som parametre.

  \head{Eksempel: \lstinline{List} modulet}

\begin{lstlisting}[numbers=none,frame=none,mathescape]
  val map    : ('a -> 'b) -> 'a list -> 'b list
  val fold   : ('b -> 'a -> 'b) -> 'b -> 'a list -> 'b
  val filter : ('a -> bool) -> 'a list -> 'a list
\end{lstlisting}

Det kan ofte være hensigtsmæssigt selv at konstruere funktioner der er
parametiske over andre funktioner.

\sp

Et oplagt eksempel er sortering:

\begin{itemize}
\item Sammenligningsbaserede sorteringsrutiner afhænger normalt blot
  af en funktion, der definerer en total ordning på elementerne i
  mængden. Funktionen kunne passende være en ``mindre-end'' funktion ($<$)
  med type \lstinline{'a->'a->bool}:
  \begin{enumerate}
  \item $\forall x. ~x \not < x$ ~~~~(irreflexivity)
  \item $\forall x,y. ~x < y \Rightarrow y \not < x$ ~~~~(asymmetry)
  \item $\forall x,y,z. ~x < y \wedge y < z \Rightarrow x < z$ ~~~~(transitivity)
  \item $\forall x,y. ~x < y \vee y < x \vee y = x$ ~~~~(totality)
  \end{enumerate}
\end{itemize}

\end{footnotesize}
\end{frame}

\subsection{Generisk sortering}

\begin{frame}[fragile]
\begin{footnotesize}

  \head{Generisk Selection Sort}

  \vspace{2ex}

\newcommand{\ltbox}{\fbox{\texttt{lt}}}
\begin{lstlisting}[numbers=none,frame=none,mathescape]
let rec sel $\ltbox$ m ys =      // invariant: $\forall \kw{y} \in \kw{ys} \, .\,  \kw{m < y}$
  function [] -> (m,ys)
         | x::xs -> if $\ltbox$ x m then
                      sel $\ltbox$ x (m::ys) xs
                    else sel $\ltbox$ m (x::ys) xs

let rec ssort ($\ltbox$:'a->'a->bool) : 'a list -> 'a list =
  function [] -> []
         | x::xs ->
           let (m,xs) = sel $\ltbox$ x [] xs
           in m :: ssort $\ltbox$ xs
\end{lstlisting}

\shead{Brug af funktionen \lstinline{ssort}:}
\begin{lstlisting}[numbers=none,frame=none,mathescape]
> ssort (>) ["Dog"; "Apple"; "Horse"; "Monkey"];;
val it : string list = ["Monkey"; "Horse"; "Dog"; "Apple"]
\end{lstlisting}

\end{footnotesize}
\end{frame}

\subsection*{Konklusion}
\begin{frame}[fragile]
  \headsp{Konklusion}

  \vspace{3mm}
  \tableofcontents
\end{frame}

\end{document}
