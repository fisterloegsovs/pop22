\documentclass[rgb]{beamer}

\usepackage[english]{babel}
\usepackage[utf8]{inputenc}
\usepackage{xcolor}
\usepackage{listings}
\usepackage{adjustbox}
\usepackage{amsmath}
\usepackage{multirow}
\usepackage[linewidth=1pt]{mdframed}

% Graphics
\usepackage{graphicx}

\usepackage{tikz}
\usetikzlibrary{calc,shapes.multipart,chains,arrows}

% Font
\usepackage{paratype}
\setbeamerfont{frametitle}{family=\bf}

% Beamer theme settings
\usecolortheme{seagull}
\setbeamertemplate{itemize item}{\raisebox{0.8mm}{\rule{1.8mm}{1.2mm}}}
\usenavigationsymbolstemplate{} % no navigation buttons

\usepackage{listings}

% Define Language
\lstdefinelanguage{fsharp}
{
  % list of keywords
  morekeywords={
    and,
    do,
    else,
    exception,
    for,
    fun,
    function,
    if,
    in,
    let,
    match,
    module,
    mutable,
    open,
    of,
    rec,
    then,
    try,
    type,
    unsafe,
    use,
    val,
    when,
    while,
    with,
  },
  sensitive=true, % keywords are not case-sensitive
  morecomment=[l]{//}, % l is for line comment
%  otherkeywords={>,<,=,<=,>=,!,*,/,-,+,|,&,||,&&,==,=>},
  morestring=[b]" % defines that strings are enclosed in double quotes
}

% Define Colors
\usepackage{color}
\definecolor{eclipseBlue}{RGB}{42,0.0,255}
\definecolor{eclipseGreen}{RGB}{63,127,95}
\definecolor{eclipsePurple}{RGB}{127,0,85}

\newcommand{\fop}[1]{\mbox{\ttfamily\color{eclipseBlue}#1}}
\newcommand{\fw}[1]{\mbox{\ttfamily\bfseries\color{eclipsePurple}#1}}

% Set Language
\lstset{
  language={fsharp},
  basicstyle=\ttfamily, % Global Code Style
  captionpos=b, % Position of the Caption (t for top, b for bottom)
  extendedchars=true, % Allows 256 instead of 128 ASCII characters
  tabsize=2, % number of spaces indented when discovering a tab
  columns=fixed, % make all characters equal width
  keepspaces=true, % does not ignore spaces to fit width, convert tabs to spaces
  showstringspaces=false, % lets spaces in strings appear as real spaces
  breaklines=true, % wrap lines if they don't fit
  frame=trbl, % draw a frame at the top, right, left and bottom of the listing
  frameround=tttt, % make the frame round at all four corners
  framesep=4pt, % quarter circle size of the round corners
  numbers=left, % show line numbers at the left
  numberstyle=\small\ttfamily, % style of the line numbers
  commentstyle=\slshape\bfseries\color{eclipseGreen}, % style of comments
  keywordstyle=\bfseries\color{eclipsePurple}, % style of keywords
  stringstyle=\color{eclipseBlue}, % style of strings
  emph=[1] {
    false,
    true,
    Set,
    Map,
    List,
    ImgUtil,
    Pegs,
    String,
    Array,
    Array2D
  },
  emphstyle=[1]{\color{eclipseBlue}},
  moredelim=**[is][\color{red}]{@@}{@@}
}

\newcommand{\theyear}{2020}
\newcommand{\sem}[1]{[\![#1]\!]}
\newcommand{\seme}[1]{\sem{#1}\varepsilon}
\newcommand{\semzero}[1]{\sem{#1}_0}

\newcommand{\emptymap}{\{\}}
\newcommand{\fracc}[2]{\begin{eqnarray} \frac{\begin{array}{c} #1
    \end{array}}{\begin{array}{c} #2 \end{array}} \end{eqnarray}}
\newcommand{\sembox}[1]{\hfill \normalfont \mbox{\fbox{\(#1\)}}}
\newcommand{\sempart}[2]{\subsubsection*{\rm\em #1 \sembox{#2}}}
\newcommand{\axiom}[1]{\begin{eqnarray} \begin{array}{c} #1 \end{array} \end{eqnarray}}
\newcommand{\fraccn}[2]{\refstepcounter{equation}\mbox{$\frac{\begin{array}{c} #1 \end{array}}{\begin{array}{c} #2 \end{array}}$}~(\arabic{equation})}
\newcommand{\fraccc}[2]{\mbox{$\frac{\begin{array}{c} #1 \end{array}}{\begin{array}{c} #2 \end{array}}$}}
\newcommand{\onepart}[1]{\noindent\hfill#1\hfill~\vspace{2mm}}
\newcommand{\twopart}[2]{\noindent\hfill#1\hfill#2\hfill~\vspace{2mm}}
\newcommand{\threepart}[3]{\noindent\hfill#1\hfill#2\hfill#3\hfill~\vspace{2mm}}
%\newcommand{\axiomm}[1]{\refstepcounter{equation}\mbox{$\begin{array}{c} #1 \end{array}$}~(\arabic{equation})}
\newcommand{\axiomm}[1]{$\begin{array}{c} #1 \end{array}$}
%\newcommand{\ar}[1]{\stackrel{#1}{\longrightarrow}}
\newcommand{\vd}{\vdash}
\newcommand{\Ran}{{\rm Ran}}
\newcommand{\Dom}{{\rm Dom}}
\newcommand{\kw}[1]{\texttt{#1}}
\newcommand{\id}[1]{\mbox{\it{#1}}}
\newcommand{\rarr}{\rightarrow}
\newcommand{\eval}{\rarr}
\newcommand{\evals}{\leadsto}
\newcommand{\larr}{\leftarrow}

\newcommand{\head}[1]{\vspace{3mm} \textbf{\normalsize #1}}
\newcommand{\headsp}[1]{\head{#1}\vspace{1ex}}
\newcommand{\size}{\ensuremath{\mathrm{size}}}
\renewcommand{\log}{\ensuremath{\mathrm{log}}}

\newcommand{\setallthemecolors}[1]{%
\setbeamercolor*{palette primary}{use=structure,fg=white,bg=#1}%
\setbeamercolor*{palette secondary}{use=structure,fg=white,bg=#1}%
\setbeamercolor*{palette tertiary}{use=structure,fg=white,bg=#1}}

\definecolor{black}{RGB}{0,0,0}
\definecolor{maroon}{RGB}{128,0,0}
\definecolor{olive}{RGB}{128,128,0}
\definecolor{green}{RGB}{0,128,0}
\definecolor{purple}{RGB}{128,0,128}
\definecolor{teal}{RGB}{0,128,128}
\definecolor{darkteal}{RGB}{0,92,92}
\definecolor{navy}{RGB}{0,0,128}
\definecolor{gray}{RGB}{128,128,128}
\definecolor{darkgray}{RGB}{60,60,60}
\definecolor{darkred}{RGB}{139,0,0}

%palette

% #173F5F (dark blue)
\definecolor{darkblue}{RGB}{23,63,95}
% #20639B (blue)
\definecolor{blue}{RGB}{32,99,155}
% #3CAEA3 (green)
\definecolor{magenta}{RGB}{60,174,163}
% #F6D55C (yellow)
\definecolor{yellow}{RGB}{246,213,92}
% #ED553B (red)
\definecolor{red}{RGB}{237,85,59}


\usecolortheme{whale}
\useoutertheme{infolines}
\useinnertheme{rectangles}

\newcommand{\popsettitle}[2]{%
\setallthemecolors{#1}%
\newcommand{\popemne}{#2}%
\title{Programmering og Problemløsning}%
\subtitle{#2}%
\author{Martin Elsman}%
\date{}%
\institute[DIKU]{Datalogisk Institut, Københavns Universitet (DIKU)}}

\newcommand{\popmaketitleframe}{%
  \frame{\titlepage%
   \vspace{-15mm}%
   \par\noindent\rule{\textwidth}{0.4pt}%

   \vspace{4mm}%
   \tableofcontents%
   \vspace{-4mm}%
   \par\noindent\rule{\textwidth}{0.4pt}%
  }%
  \section*{\popemne}%
}


\popsettitle{purple}{Typer og Mønstergenkendelse (Del 1)}  % see ../util.tex for colors

\begin{document}

\popmaketitleframe

%%%%%%%%%%%%%%%%%%%%%%%%%%%%%%%%%%%%%%%%%%%%%%%%
%\subsection{Introduktion}
%%%%%%%%%%%%%%%%%%%%%%%%%%%%%%%%%%%%%%%%%%%%%%%%

\begin{frame}[fragile]
\begin{footnotesize}

  \head{Typer}

  \begin{quote}
    Typer kan forstås som mængder af værdier.
  \end{quote}

  \begin{itemize}
    \item Typebegrebet giver os således et sprog for at klassificere værdier.

    \item Vi kan for eksempel tale om at en funktion returnerer et heltal (type \lstinline{int}).

    \item F\# oversætteren kan ``type-tjekke'' vores programmer for at
  sikre os mod en lang række fejl når programmet køres!
  \end{itemize}

  \headsp{Nogle emner vi vil dække:}
  \begin{itemize}
    \item Hvordan kan type-begrebet udvides til at klassificere flere
      slags værdier, såsom funktioner (og funktioner der returnerer
      funktioner).
    \item Hvordan kan vi skrive genbrugelige \emph{type-generiske}
      funktioner, der kan køre på data af forskellig type.
    \item Vi vil senere se hvordan type-begrebet kan udvides til at
      kunne beskrive træer, grafer og andre strukturelle
      datastrukturer.
  \end{itemize}
\end{footnotesize}
\end{frame}

\newcommand{\Z}{\mathbb{Z}}
\newcommand{\N}{\mathbb{N}}
\newcommand{\R}{\mathbb{R}}

\subsection{Typer}
\begin{frame}[fragile]
\begin{footnotesize}

  \head{Typer kan forstås som mængder af værdier}

  \vspace{1ex}

  \textbf{Eksempler:}

  \vspace{1ex}

  \begin{tabular}{lcl}
    \texttt{int} & $\approx$ & $\Z = \{\ldots,-2,-1,0,1,2,\ldots\}$ \\[1ex]
    \texttt{float} & $\approx$ & $\R$ \\[1ex]
    \texttt{int * float} & $\approx$ & $\Z \times \R = \{\ldots,(3,1.2),\ldots\}$ \\
    & & (sæt af alle par med elementer fra $\Z$ og $\R$) \\[1ex]
    \texttt{int list} & $\approx$ & $\{[],[1],[2],\ldots,[1;-2],\ldots\}$ \\[1ex]
    \texttt{int -> float} & $\approx$ & $\Z \rightarrow \R$ \\ & & (sæt af alle afbildninger fra $\Z$ til $\R$) \\[1ex]
    \texttt{bool} & = & \{\texttt{true},\texttt{false}\} \\[1ex]
    \texttt{unit} & = & \{()\}
  \end{tabular}
\end{footnotesize}

\head{Spørgsmål:}
\begin{enumerate}
\item Hvorfor $\approx$ for de første fire tilfælde?
\item Hvorfor $\approx$ for funktioner?
\end{enumerate}
\end{frame}

\subsection{Typeforkortelser}
\begin{frame}[fragile]
\begin{footnotesize}

  \head{Typeforkortelser}
  \vspace{1ex}

  Det er nemt i F\# at give et navn til en type for derved at gøre kode lettere læselig.
\begin{lstlisting}[numbers=none,frame=none,mathescape]
  type department = string
  type costs = (department * float) list
  let total (costs:costs) : float =
    List.fold (fun acc (_,f) -> acc+f) 0.0 costs
\end{lstlisting}

  \vspace{1ex}

  \head{Bemærk:}
  \begin{itemize}
  \item Typen \lstinline{department} er blot et synonym for typen \lstinline{string}.
  \item Funktionen \lstinline{total} kan derfor benyttes på alle
    værdier af typen \lstinline{(string*float) list}.
  \end{itemize}

\end{footnotesize}
\end{frame}

\begin{frame}[fragile]
\begin{footnotesize}

  \head{Type-generiske type-forkortelser}
  \vspace{1ex}

  Type-forkortelser kan være generiske således at det er muligt at
  skrive generisk kode der henviser til en type-forkortelse:

  \vspace{1ex}

\begin{lstlisting}[numbers=none,frame=none,mathescape]
  // association lists mapping strings to values of type 'a
  type 'a alist = (string * 'a) list
  let add (m:'a alist) (s:string) (v:'a) : 'a alist =
     (s,v)::m
  let rec look (m:'a alist) (s:string) : 'a option =
    match m with
       [] -> None
     | h::t -> if fst h = s then Some(snd h)
               else look t s
  let empty () : 'a alist = []
\end{lstlisting}

  \head{Bemærk:}
  \begin{itemize}
  \item Vi kan benytte den tomme liste \lstinline{[]} til at repræsentere den tomme associationsliste.
  \item Vi skal senere se hvorledes vi med moduler kan sikre at typen
    \lstinline{'a alist} bliver ``fuldt abstrakt'' således at kun de
    nævnte funktioner kan benyttes til at opererere på de konstruerede
    associationslister.
  \end{itemize}

\end{footnotesize}
\end{frame}

\subsection{Record typer (generaliserede produkter)}

\begin{frame}[fragile]
\begin{footnotesize}

  \headsp{Record-typer (generaliserede produkter)}

  Records i F\# giver mulighed for at navngive elementer i et tuple.

  \vspace{1ex}

  Syntaksen for at definere en record-type er enkel:

\begin{lstlisting}[numbers=none,frame=none,mathescape]
  type person = {first:string; last:string; age:int}

  let xs = [{first="Lene"; last="Andersen"; age=56};
            {last="Hansen"; first="Jens"; age=39}]
  let name (p:person) : string = p.first + " " + p.last
  let incr_age (p:person) : person = {p with age=p.age+1}
  let ys = List.map incr_age xs
\end{lstlisting}

  \vspace{1ex}

  \head{Bemærk:}
  \begin{itemize}
  \item Ved konstruktion af en record er felt-rækkefølgen ubetydelig.
  \item Elementer i en record kan udtrækkes ved brug af \textbf{dot-notationen} (\lstinline{p.first}).
  \item En \textbf{ny} record kan konstrueres (med et opdateret
    element) ved brug af \lstinline{with}-konstruktionen.
  \end{itemize}

\end{footnotesize}
\end{frame}

\subsection*{Konklusion}
\begin{frame}[fragile]
  \headsp{Konklusion}

  \vspace{3mm}
  \tableofcontents
\end{frame}


\end{document}
