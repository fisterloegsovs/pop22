\documentclass[rgb]{beamer}

\usepackage[english]{babel}
\usepackage[utf8]{inputenc}
\usepackage{xcolor}
\usepackage{listings}
\usepackage{adjustbox}
\usepackage{amsmath}
\usepackage{multirow}
\usepackage[linewidth=1pt]{mdframed}

% Graphics
\usepackage{graphicx}

\usepackage{tikz}
\usetikzlibrary{calc,shapes.multipart,chains,arrows}

% Font
\usepackage{paratype}
\setbeamerfont{frametitle}{family=\bf}

% Beamer theme settings
\usecolortheme{seagull}
\setbeamertemplate{itemize item}{\raisebox{0.8mm}{\rule{1.8mm}{1.2mm}}}
\usenavigationsymbolstemplate{} % no navigation buttons

\usepackage{listings}

% Define Language
\lstdefinelanguage{fsharp}
{
  % list of keywords
  morekeywords={
    and,
    do,
    else,
    exception,
    for,
    fun,
    function,
    if,
    in,
    let,
    match,
    module,
    mutable,
    open,
    of,
    rec,
    then,
    try,
    type,
    unsafe,
    use,
    val,
    when,
    while,
    with,
  },
  sensitive=true, % keywords are not case-sensitive
  morecomment=[l]{//}, % l is for line comment
%  otherkeywords={>,<,=,<=,>=,!,*,/,-,+,|,&,||,&&,==,=>},
  morestring=[b]" % defines that strings are enclosed in double quotes
}

% Define Colors
\usepackage{color}
\definecolor{eclipseBlue}{RGB}{42,0.0,255}
\definecolor{eclipseGreen}{RGB}{63,127,95}
\definecolor{eclipsePurple}{RGB}{127,0,85}

\newcommand{\fop}[1]{\mbox{\ttfamily\color{eclipseBlue}#1}}
\newcommand{\fw}[1]{\mbox{\ttfamily\bfseries\color{eclipsePurple}#1}}

% Set Language
\lstset{
  language={fsharp},
  basicstyle=\ttfamily, % Global Code Style
  captionpos=b, % Position of the Caption (t for top, b for bottom)
  extendedchars=true, % Allows 256 instead of 128 ASCII characters
  tabsize=2, % number of spaces indented when discovering a tab
  columns=fixed, % make all characters equal width
  keepspaces=true, % does not ignore spaces to fit width, convert tabs to spaces
  showstringspaces=false, % lets spaces in strings appear as real spaces
  breaklines=true, % wrap lines if they don't fit
  frame=trbl, % draw a frame at the top, right, left and bottom of the listing
  frameround=tttt, % make the frame round at all four corners
  framesep=4pt, % quarter circle size of the round corners
  numbers=left, % show line numbers at the left
  numberstyle=\small\ttfamily, % style of the line numbers
  commentstyle=\slshape\bfseries\color{eclipseGreen}, % style of comments
  keywordstyle=\bfseries\color{eclipsePurple}, % style of keywords
  stringstyle=\color{eclipseBlue}, % style of strings
  emph=[1] {
    false,
    true,
    Set,
    Map,
    List,
    ImgUtil,
    Pegs,
    String,
    Array,
    Array2D
  },
  emphstyle=[1]{\color{eclipseBlue}},
  moredelim=**[is][\color{red}]{@@}{@@}
}

\newcommand{\theyear}{2020}
\newcommand{\sem}[1]{[\![#1]\!]}
\newcommand{\seme}[1]{\sem{#1}\varepsilon}
\newcommand{\semzero}[1]{\sem{#1}_0}

\newcommand{\emptymap}{\{\}}
\newcommand{\fracc}[2]{\begin{eqnarray} \frac{\begin{array}{c} #1
    \end{array}}{\begin{array}{c} #2 \end{array}} \end{eqnarray}}
\newcommand{\sembox}[1]{\hfill \normalfont \mbox{\fbox{\(#1\)}}}
\newcommand{\sempart}[2]{\subsubsection*{\rm\em #1 \sembox{#2}}}
\newcommand{\axiom}[1]{\begin{eqnarray} \begin{array}{c} #1 \end{array} \end{eqnarray}}
\newcommand{\fraccn}[2]{\refstepcounter{equation}\mbox{$\frac{\begin{array}{c} #1 \end{array}}{\begin{array}{c} #2 \end{array}}$}~(\arabic{equation})}
\newcommand{\fraccc}[2]{\mbox{$\frac{\begin{array}{c} #1 \end{array}}{\begin{array}{c} #2 \end{array}}$}}
\newcommand{\onepart}[1]{\noindent\hfill#1\hfill~\vspace{2mm}}
\newcommand{\twopart}[2]{\noindent\hfill#1\hfill#2\hfill~\vspace{2mm}}
\newcommand{\threepart}[3]{\noindent\hfill#1\hfill#2\hfill#3\hfill~\vspace{2mm}}
%\newcommand{\axiomm}[1]{\refstepcounter{equation}\mbox{$\begin{array}{c} #1 \end{array}$}~(\arabic{equation})}
\newcommand{\axiomm}[1]{$\begin{array}{c} #1 \end{array}$}
%\newcommand{\ar}[1]{\stackrel{#1}{\longrightarrow}}
\newcommand{\vd}{\vdash}
\newcommand{\Ran}{{\rm Ran}}
\newcommand{\Dom}{{\rm Dom}}
\newcommand{\kw}[1]{\texttt{#1}}
\newcommand{\id}[1]{\mbox{\it{#1}}}
\newcommand{\rarr}{\rightarrow}
\newcommand{\eval}{\rarr}
\newcommand{\evals}{\leadsto}
\newcommand{\larr}{\leftarrow}

\newcommand{\head}[1]{\vspace{3mm} \textbf{\normalsize #1}}
\newcommand{\headsp}[1]{\head{#1}\vspace{1ex}}
\newcommand{\size}{\ensuremath{\mathrm{size}}}
\renewcommand{\log}{\ensuremath{\mathrm{log}}}

\newcommand{\setallthemecolors}[1]{%
\setbeamercolor*{palette primary}{use=structure,fg=white,bg=#1}%
\setbeamercolor*{palette secondary}{use=structure,fg=white,bg=#1}%
\setbeamercolor*{palette tertiary}{use=structure,fg=white,bg=#1}}

\definecolor{black}{RGB}{0,0,0}
\definecolor{maroon}{RGB}{128,0,0}
\definecolor{olive}{RGB}{128,128,0}
\definecolor{green}{RGB}{0,128,0}
\definecolor{purple}{RGB}{128,0,128}
\definecolor{teal}{RGB}{0,128,128}
\definecolor{darkteal}{RGB}{0,92,92}
\definecolor{navy}{RGB}{0,0,128}
\definecolor{gray}{RGB}{128,128,128}
\definecolor{darkgray}{RGB}{60,60,60}
\definecolor{darkred}{RGB}{139,0,0}

%palette

% #173F5F (dark blue)
\definecolor{darkblue}{RGB}{23,63,95}
% #20639B (blue)
\definecolor{blue}{RGB}{32,99,155}
% #3CAEA3 (green)
\definecolor{magenta}{RGB}{60,174,163}
% #F6D55C (yellow)
\definecolor{yellow}{RGB}{246,213,92}
% #ED553B (red)
\definecolor{red}{RGB}{237,85,59}


\usecolortheme{whale}
\useoutertheme{infolines}
\useinnertheme{rectangles}

\newcommand{\popsettitle}[2]{%
\setallthemecolors{#1}%
\newcommand{\popemne}{#2}%
\title{Programmering og Problemløsning}%
\subtitle{#2}%
\author{Martin Elsman}%
\date{}%
\institute[DIKU]{Datalogisk Institut, Københavns Universitet (DIKU)}}

\newcommand{\popmaketitleframe}{%
  \frame{\titlepage%
   \vspace{-15mm}%
   \par\noindent\rule{\textwidth}{0.4pt}%

   \vspace{4mm}%
   \tableofcontents%
   \vspace{-4mm}%
   \par\noindent\rule{\textwidth}{0.4pt}%
  }%
  \section*{\popemne}%
}


\popsettitle{teal}{Træstrukturer (Del 4)}  % see ../util.tex for colors

\renewcommand{\log}{\ensuremath{\mathrm{log}}}

\begin{document}

\popmaketitleframe

\newcommand{\mytree}[1]{
    \begin{tikzpicture}[domain=0:80,scale=#1]
    %\draw[very thin,color=gray] (0,0) grid (80,80);
    \draw[->] (17.5,25) -- (12.5,15);
    \draw[->] (22.5,25) -- (27.5,15);
    \draw[->] (27.5,45) -- (22.5,35);
    \draw[->] (32.5,45) -- (37.5,35);
    \draw[->] (67.5,45) -- (62.5,35);
%    \draw[->] (72.5,45) -- (77.5,35);
    \draw[->] (45,65) -- (35,55);
    \draw[->] (55,65) -- (65,55);
    \node at (50,70) {Times};
    \node at (30,50) {Plus};
    \node at (70,50) {Sin};
    \node at (20,30) {Divide};
    \node at (40,30) {X};
    \node at (60,30) {X};
%    \node at (80,30) {2};
    \node at (10,10) {Const 3};
    \node at (30,10) {X};
    \end{tikzpicture}
}

\newcommand{\mytreetwo}[1]{
    \begin{tikzpicture}[domain=0:80,scale=#1]
    %\draw[very thin,color=gray] (0,0) grid (80,80);
    \draw[->] (20,25) -- (20,15);
    \draw[->] (40,25) -- (40,15);
    \draw[->] (57.5,25) -- (52.5,15);
    \draw[->] (62.5,25) -- (67.5,15);
%    \draw[->] (22.5,25) -- (27.5,15);
    \draw[->] (27.5,45) -- (22.5,35);
    \draw[->] (32.5,45) -- (37.5,35);
    \draw[->] (67.5,45) -- (62.5,35);
    \draw[->] (72.5,45) -- (77.5,35);
    \draw[->] (45,65) -- (35,55);
    \draw[->] (55,65) -- (65,55);
    \node at (50,70) {Plus};
    \node at (30,50) {Divide};
    \node at (70,50) {Power};
    \node at (20,30) {Sin};
    \node at (40,30) {Cos};
    \node at (60,30) {Plus};
    \node at (80,30) {2};
    \node at (20,10) {x};
    \node at (40,10) {x};
    \node at (52.5,10) {x};
    \node at (67.5,10) {1};
    \end{tikzpicture}
}

\newcommand{\mypostordertree}[1]{
    \begin{tikzpicture}[domain=0:80,scale=#1]
    %\draw[very thin,color=gray] (0,0) grid (80,80);
    \draw[->] (17.5,25) -- (12.5,15);
    \draw[->] (22.5,25) -- (27.5,15);
    \draw[->] (27.5,45) -- (22.5,35);
    \draw[->] (32.5,45) -- (37.5,35);
    \draw[->] (67.5,45) -- (62.5,35);
    \draw[->] (72.5,45) -- (77.5,35);
    \draw[->] (45,65) -- (35,55);
    \draw[->] (55,65) -- (65,55);
    \node at (50,70) {1};
    \node at (30,50) {2};
    \node at (70,50) {3};
    \node at (20,30) {4};
    \node at (40,30) {5};
    \node at (60,30) {6};
    \node at (80,30) {7};
    \node at (10,10) {8};
    \node at (30,10) {9};
    \end{tikzpicture}
}

%%%%%%%%%%%%%%%%%%%%%%%%%%%%%%%%%%%%%%%%%%%%%%%%
%\subsection{Introduktion}
%%%%%%%%%%%%%%%%%%%%%%%%%%%%%%%%%%%%%%%%%%%%%%%%

\subsection{Recap}

\begin{frame}[fragile]
\begin{footnotesize}

  \begin{minipage}{.7\textwidth}
  \headsp{Recap}

  \begin{itemize}
  \item Rekursive sum-typer
  \item Trægennemløb (depth-first, breath-first, ...)
  \item \textbf{Postordens gennemløb}:  8 9 4 5 2 6 7 3 1 \\
        (knude efter børn)
  \end{itemize}
  \end{minipage}
  \begin{minipage}{.25\textwidth}
    \mypostordertree{0.04}
  \end{minipage}

\begin{lstlisting}[numbers=none,frame=none,mathescape]
  type 'a t = L | T of 'a t * 'a * 'a t
  let E e = T(L,e,L)
  let mytree = T(T(T(E 8,4,E 9),2,E 5),1,T(E 6,3,E 7))
\end{lstlisting}


  \vspace{2mm}
  \headsp{Udvalgte spørgsmål}

  \begin{enumerate}
%  \item Hvordan deles kode i flere filer?

  \item Kan man lave pattern matching på værdier af komplekse typer,
    som f.eks. \lstinline{(int * string) option}?

  \item Kan man benytte pattern matching til at udtrække det sidste
    element i en liste?
  \end{enumerate}
\end{footnotesize}
\end{frame}

\subsection{Udtrykstræer}

\begin{frame}[fragile]
\begin{footnotesize}

  \head{Udtrykstræer og Symbolsk Differentiering}

  \vspace{1ex}

  \begin{minipage}{.6\textwidth}
  \begin{enumerate}
  \item \textbf{Udtrykstræer.}

    Sum-type definition (funktioner af en variabel)

  \item \textbf{Evaluering af udtryk.}

  \item \textbf{Simpel pretty-printing.}

    Pretty bad...

  \item \textbf{Symbolsk differentiering.}

    Vi implementerer gymnasiereglerne for differentiering...

  \item \textbf{Generering af \LaTeX{} kode.}

    Vi prøver at undgå unødige parenteser...

  \item \textbf{Simplificering.}

    Vi implementerer forskellige regneregler til simplificering...

  \end{enumerate}
  \end{minipage}\begin{minipage}{.25\textwidth}
    \mytree{.065}
{\large
    $$~~~~~\left(\frac{3}{x} + x\right) \cdot \sin x$$}
    \end{minipage}

\end{footnotesize}
\end{frame}

\begin{frame}[fragile]
\begin{footnotesize}

  \head{Regneudtryk i \'{e}n variabel}
  \vspace{1ex}

  Et udtryk kan (rekursivt) være på en af følgende former:

  \vspace{1ex}
  \begin{enumerate}
\item Variablen $x$.
\item En konstant $c \in \textsf{I\!R}$.
\item To udtryk adskilt af $+$.
\item To udtryk adskilt af $-$.
\item To udtryk adskilt af $\cdot$ (gange).
\item To udtryk adskilt af en brøkstreg.
\item Et udtryk opløftet til en potens $p$.
\item Funktionen $\sin$ anvendt på et udtryk.
\item Funktionen $\cos$ anvendt på et udtryk.
\item Funktionen $\log$ anvendt på et udtryk.
\item Konstanten $e$ opløftet til en potens angivet med et udtryk.
\item Et udtryk omgivet af parenteser.
\end{enumerate}

\vspace{1ex}
\textbf{Eksempel:} {\large $\frac{\sin\,x}{\cos\,x}+(x+1)^2$}

\end{footnotesize}
\end{frame}

\begin{frame}[fragile]
\begin{footnotesize}

  \head{Sum-Type til Udtrykstræer}
  \vspace{1ex}

\begin{lstlisting}[numbers=none,frame=none,mathescape]
type expr = X                       // The variable x
          | Const of float          // Constants
          | Plus of expr * expr     // Addition
          | Minus of expr * expr
          | Times of expr * expr
          | Divide of expr * expr
          | Power of expr * float   // $a^d$, e.g., $(x+2)^{2.3}$
          | Sin of expr
          | Cos of expr
          | Log of expr
          | Exp of expr             // $e^a$, e.g., $e^{(2+x)}$
\end{lstlisting}

\head{Bemærk}
\begin{itemize}
\item Vi understøtter kun funktioner af \'{e}n variabel.
\item Parenteser er implicitte; træet indkoder parenteser direkte.
\item Dvs: Parsning (at konvertere en streng til et udtrykstræ) er en
  separat problemstilling vi vil se på en anden gang...
\end{itemize}

\end{footnotesize}
\end{frame}

\subsection{Evaluering af Udtryk}

\begin{frame}[fragile]
\begin{footnotesize}

  \head{Evaluering af Udtryk (fortolkning)}
  \vspace{1ex}

  Et udtrykstræ repræsenterer kroppen på en funktion $f(x)$.

\begin{lstlisting}[numbers=none,frame=none,mathescape]
let rec eval e x =    // assume a value for the variable X
  match e with
    | X -> x
    | Const c -> c
    | Plus (e1, e2) -> eval e1 x + eval e2 x
    | Minus (e1, e2) -> eval e1 x - eval e2 x
    | Times (e1, e2) -> eval e1 x * eval e2 x
    | Divide (e1, e2) -> eval e1 x / eval e2 x
    | Power (e, p) -> (eval e x) ** p
    | Sin e -> sin (eval e x)
    | Cos e -> cos (eval e x)
    | Log e -> log (eval e x)
    | Exp e -> exp (eval e x)

let ee = Plus (Divide (Sin X, Cos X),
               Power (Plus (X, Const 1.0), 2.0))

let v = eval ee 3.0   // evaluates to 15.85745346
\end{lstlisting}

\vspace{-5cm}\hspace{8.3cm}
\mytreetwo{0.05}
\vspace{5cm}

\end{footnotesize}
\end{frame}

\subsection{Simpel Pretty-Printing}

\begin{frame}[fragile]
\begin{footnotesize}

  \head{Simpel Pretty-Printing (pretty bad)}

  \vspace{2ex}

\begin{lstlisting}[numbers=none,frame=none,mathescape]
let par s = "(" + s + ")"
let rec pp e : string =
  match e with
    | X -> "x"
    | Const c -> sprintf "%g" c
    | Plus (e1, e2) -> par(pp e1 + "+" + pp e2)
    | Minus (e1, e2) -> par(pp e1 + "-" + pp e2)
    | Times (e1, e2) -> par(pp e1 + "*" + pp e2)
    | Divide (e1, e2) -> par(pp e1 + "/" + pp e2)
    | Power (e, p) -> par(pp e + "^" + sprintf "%g" p)
    | Sin e -> "sin " + pp e
    | Cos e -> "cos " + pp e
    | Log e -> "log " + pp e
    | Exp e -> "exp " + pp e
\end{lstlisting}

\head{Hvad er der galt ved denne pretty-printer?}
\begin{enumerate}
\item \underline{\hspace{8cm}}
\item \underline{\hspace{8cm}}
\end{enumerate}

\end{footnotesize}
\end{frame}

\subsection{Symbolsk Differentiering}
\begin{frame}[fragile]
\begin{footnotesize}
  \head{Differentieringsregler}
\end{footnotesize}

\[\def\arraystretch{1.1}
\begin{array}{c@{\quad\quad}c}
h(x) & h'(x) \\\hline
x & 1 \\
c & 0 \\
f(x) + g(x) & f'(x) + g'(x) \\
f(x) - g(x) & f'(x) - g'(x) \\
f(x) \cdot g(x) & f'(x)\cdot g(x) + f(x)\cdot g'(x) \\[1ex]
\frac{f(x)}{g(x)} & \frac{f'(x)\cdot g(x) - f(x)\cdot g'(x)}{(g(x))^2}
\\[1ex]
x^n & n\cdot x^{(n-1)} \\
\sin~x & \cos~x \\
\cos~x & - \sin~x \\
\log~x &  1 / x \\
e^x & e^x \\
f(g(x)) & f'(g(x))\cdot g'(x)\\
\end{array}\]

\end{frame}

\begin{frame}[fragile]
\begin{footnotesize}
  \head{Symbolsk Differentiering i F\#}

\begin{lstlisting}[numbers=none,frame=none,mathescape]
let rec ddx e =
  match e with
  | X -> Const 1.0
  | Const c -> Const 0.0
  | Plus (e1, e2) -> Plus(ddx e1, ddx e2)
  | Minus (e1, e2) -> Minus (ddx e1, ddx e2)
  | Times (e1, e2) -> Plus (Times (ddx e1, e2),
                            Times (e1, ddx e2))
  | Divide (e1, e2) ->
      Divide(Minus (Times (ddx e1, e2), Times (e1, ddx e2)),
             Power (e2, 2.0))
  | Power (e, p) -> Times (Times (Const p, Power (e, p-1.0)),
                           ddx e)
  | Sin e -> Times (Cos e, ddx e)
  | Cos e -> Times (Minus (Const 0.0, Sin e), ddx e)
  | Log e -> Times (Divide (Const 1.0, e), ddx e)
  | Exp e -> Times (Exp e, ddx e)
\end{lstlisting}
\end{footnotesize}
\end{frame}

\begin{frame}[fragile]
\begin{footnotesize}
  \head{Symbolsk Differentiering i F\#}

\begin{lstlisting}[numbers=none,frame=none,mathescape]
> ddx ee;;
val it : expr =
  Plus
    (Divide
       (Minus
          (Times (Times (Cos X,Const 1.0),Cos X),
           Times (Sin X,Times (Minus (Const 0.0,Sin X),
                               Const 1.0))),
        Power (Cos X,2.0)),
     Times
       (Times (Const 2.0,Power (Plus (X,Const 1.0),1.0)),
        Plus (Const 1.0,Const 0.0)))

> pp (ddx ee);;
val it : string =
   "(((((cos x*1)*cos x)-(sin x*((0-sin x)*1)))/  ...
    (cos x^2))+((2*((x+1)^1))*(1+0)))"
\end{lstlisting}
\end{footnotesize}
\end{frame}

\subsection{Generering af \LaTeX{} kode}

\begin{frame}[fragile]
\begin{footnotesize}
  \headsp{Generering af \LaTeX{} kode}

  Vi kan generere \LaTeX{} kode i stedet for tekst.
  \vspace{1ex}

  Strategien er at lade vores pretty-printer returnere \textbf{et par} af
  to værdier:
  \begin{enumerate}
  \item En streng der repræsenterer den genererede \LaTeX{} kode.
  \item Et tal (præcedens) der siger hvor stærkt det underliggende udtryk binder.
  \end{enumerate}

  \head{Eksempler}

  \begin{itemize}
    \item \lstinline{pp (Plus(X, Const 1.0))} $\leadsto$ (\lstinline{"x+1.0"}, 3)
    \item \lstinline{pp (Times(Const 2.0, Const 3.0))} $\leadsto$ (\lstinline{"2.0\\cdot 3.0"}, 4)
  \end{itemize}

  Betragt udtrykket \lstinline{Times(X,Plus(X,Const 1.0))}:

  \vspace{1ex}

  Når vi skal pretty-printe det højre argument til \lstinline{Times}
  skal vi sætte paranteser rundt om udtrykket
  \lstinline{"x+1.0"}, da præcedens-tallet 3 er mindre end (eller lig med) præcedens
  for konstruktøren \lstinline{Times}:

    \begin{itemize}
    \item \lstinline{pp (Times(X, Plus(X, Const 1.0)))} $\leadsto$ (\lstinline{"x\\cdot (x+1.0)"}, 4)
  \end{itemize}

\end{footnotesize}
\end{frame}

\begin{frame}[fragile]
\begin{footnotesize}

  \headsp{Generering af \LaTeX{} kode---del 0}

  \vspace{1ex}

  Nogle hjælpfunktioner samt præcedens-konstanter:
  \vspace{1ex}

\begin{lstlisting}[numbers=none,frame=none,mathescape]
let p_plus   = 3     // precedence
let p_times  = 4
let p_divide = 0     // division with horizontal bar
let p_unop   = 5
let p_max    = 1000

// [par p s] adds parentheses around s if the relative
// precedence p (p_parent - p_child) is positive or 0
let par p s =
  if p < 0 then s else "(" + s + ")"
\end{lstlisting}

\vspace{1ex}
\headsp{Igen:}
\begin{itemize}
  \item Ingen parenteser er nødvendige omkring et barn hvis dets udtryk
    binder stærkere end forældre-udtrykket.
\end{itemize}

\end{footnotesize}
\end{frame}

\begin{frame}[fragile]
\begin{footnotesize}

  \head{Generering af \LaTeX{} kode---del 1}

\begin{lstlisting}[numbers=none,frame=none,mathescape]
let toLaTeX e =
  let rec pp e : string * int =
    match e with
    | X -> ("x", p_max)
    | Const c -> (sprintf "%g" c, p_max)
    | Plus (e1, e2) -> pp_binop ("+",p_plus) e1 e2
    | Minus (e1, e2) -> pp_binop ("-",p_plus) e1 e2
    | Times (e1, e2) -> pp_binop ("\\cdot ",p_times) e1 e2
    | Divide (e1, e2) ->
      let (s1,s2,_) = pp_bin p_divide e1 e2
      in ("\\frac{" + s1 + "}{" + s2 + "}", p_max)
    | Power (e, k) -> let (s1,s2,p) = pp_bin p_unop e (Const k)
                      in (s1 + "^{" + s2 + "}", p)
    | Sin e -> pp_unop "sin " e
    | Cos e -> pp_unop "cos " e
    | Log e -> pp_unop "log " e
    | Exp e -> let (s,a) = pp e
               in ("e^{" + par (p_unop-a) s + "}", p_unop)
  and ...
\end{lstlisting}

\end{footnotesize}
\end{frame}

\begin{frame}[fragile]
\begin{footnotesize}

  \head{Generering af \LaTeX{} kode---del 2}

\begin{lstlisting}[numbers=none,frame=none,mathescape]
  ...
  and pp_bin p e1 e2 =
    let (s1,p1) = pp e1
    let (s2,p2) = pp e2
    in (par (p-p1) s1, par (p-p2) s2, p)
  and pp_binop (op,p) e1 e2 =
    let (s1,s2,p) = pp_bin p e1 e2
    in (s1 + op + s2, p)
  and pp_unop op e =
    let (s,p) = pp e
    in ("\\"+op + par (p_unop-p) s, p_unop)
  in fst(pp e)
\end{lstlisting}

\textbf{Eksempel:} \lstinline{toLaTeX (ddx ee)} $\leadsto$
\begin{lstlisting}[numbers=none,frame=none,mathescape]
  "\frac{(\cos x\cdot 1)\cdot \cos x-\sin x\cdot ((0-\sin x)...
   \cdot 1)}{(\cos x)^{2}}+(2\cdot (x+1)^{1})\cdot (1+0)"
\end{lstlisting}

\[
\frac{(\cos x\cdot 1)\cdot \cos x-\sin x\cdot ((0-\sin x)
   \cdot 1)}{(\cos x)^{2}}+(2\cdot (x+1)^{1})\cdot (1+0)
\]

\end{footnotesize}
\end{frame}

\subsection{Simplificering af Symbolske Udtryk}

\begin{frame}[fragile]
\begin{footnotesize}

  \head{Simplificering af Symbolske Udtryk}

  \vspace{2ex}

  Der er tilsyneladende en række konstruktioner der kan simplificeres...

\begin{lstlisting}[numbers=none,frame=none,mathescape]
ddx ee $\leadsto$
  Plus
    (Divide
       (Minus
          (Times ($\color{blue}\texttt{Times (Cos X,Const 1.0)}$,Cos X),
           Times (Sin X,Times (Minus (Const 0.0,Sin X),
                               Const 1.0))),
        Power (Cos X,2.0)),
     Times
       (Times (Const 2.0,Power (Plus (X,Const 1.0),1.0)),
        $\color{blue}\texttt{Plus (Const 1.0,Const 0.0)}$))
\end{lstlisting}

\vspace{1ex}
\headsp{Eksempler:}
\begin{itemize}
  \item \lstinline{Times (Cos X,Const 1.0)} $\Longrightarrow$ \lstinline{Cos X}
  \item \lstinline{Plus (Const 1.0,Const 0.0)} $\Longrightarrow$ \lstinline{Const 1.0}
\end{itemize}
\end{footnotesize}
\end{frame}

\begin{frame}[fragile]
\begin{footnotesize}

  \head{En Simplificeringsfunktion i F\# --- del 1}

  \vspace{1ex}

\begin{lstlisting}[numbers=none,frame=none,mathescape]
let rec simplify e =
  match e with
  | Plus (e1, Const 0.0) -> simplify e1
  | Plus (Const 0.0, e2) -> simplify e2
  | Plus (Const a, Const b) -> Const (a + b)
  | Plus (e1, e2) -> Plus (simplify e1, simplify e2)
  | Minus (e1, Const 0.0) -> simplify e1
  | Minus (Const a, Const b) -> Const (a - b)
  | Minus (e1, Minus(Const 0.0, e2)) ->
     Plus (simplify e1, simplify e2)
  | Minus (e1, e2) -> Minus (simplify e1, simplify e2)
  | Divide (Const 0.0, e2) -> Const 0.0
  | Divide (e1, Const 1.0) -> simplify e1
  | Divide (e1, e2) -> Divide (simplify e1, simplify e2)
  | Power (e, 1.0) -> simplify e
  | Power (e, c) -> Power (simplify e,c)
  ...
\end{lstlisting}

\end{footnotesize}
\end{frame}

\begin{frame}[fragile]
\begin{footnotesize}

  \head{En Simplificeringsfunktion i F\# --- del 2}

  \vspace{1ex}

\begin{lstlisting}[numbers=none,frame=none,mathescape]
let rec simplify e =
  match e with
  ...
  | Times (e1, Const 0.0) -> Const 0.0
  | Times (Const 0.0, e2) -> Const 0.0
  | Times (e1, Const 1.0) -> simplify e1
  | Times (Const 1.0, e2) -> simplify e2
  | Times (Const a, Const b) -> Const (a * b)
  | Times (e1,Plus(e2,e3)) ->
    simplify (Plus(Times(e1,e2),Times(e1,e3)))
  | Times (e1,Minus(e2,e3)) ->
    simplify (Minus(Times(e1,e2),Times(e1,e3)))
  | Times (e1, e2) ->
    if e1 = e2 then simplify (Power (e1,2.0))
    else Times (simplify e1, simplify e2)
  | Sin e -> Sin (simplify e)
  | Cos e -> Cos (simplify e)
  | Log e -> Log (simplify e)
  | Exp e -> Exp (simplify e)
  | _ -> e
\end{lstlisting}

\end{footnotesize}
\end{frame}

\begin{frame}[fragile]
\begin{footnotesize}

  \head{Det tager flere steps før ``simplifyeren'' stabiliseres:}

  \vspace{1ex}
  \lstinline{toLaTeX(ddx ee)} $\leadsto$
  \[
  \frac{(\cos x\cdot 1)\cdot \cos x-\sin x\cdot ((0-\sin x)\cdot 1)}{(\cos x)^{2}}+(2\cdot (x+1)^{1})\cdot (1+0)
  \]

  \vspace{1ex}
  \lstinline{toLaTeX(simplify(ddx ee))} $\leadsto$
  \[
  \frac{\cos x\cdot \cos x-\sin x\cdot (0-\sin x)}{(\cos x)^{2}}+(2\cdot (x+1)+0)
  \]

  \vspace{1ex}
  \lstinline{toLaTeX(simplify(simplify(ddx ee)))} $\leadsto$
  \[
  \frac{(\cos x)^{2}-(0-(\sin x)^{2})}{(\cos x)^{2}}+(2\cdot x+2)
  \]

  \vspace{1ex}
  \lstinline{toLaTeX(simplify(simplify(simplify(ddx ee))))} $\leadsto$
  \[
  \frac{(\cos x)^{2}+(\sin x)^{2}}{(\cos x)^{2}}+(2\cdot x+2)
  \]

\end{footnotesize}
\end{frame}

\begin{frame}[fragile]
\begin{footnotesize}

  \head{Komplet Simplificering}

  \vspace{1ex}

  Med generisk lighed og rekursion kan vi let skrive en ``simplifyer''
  der kalder \lstinline{simplify} igen og igen indtil udtrykket ikke
  længere simplificeres:

\begin{lstlisting}[numbers=none,frame=none,mathescape]
let rec simplifyMax e =
  let se = simplify e
  in if se = e then e else simplifyMax se
\end{lstlisting}

\vspace{1ex}
\headsp{Komplet Simplificering i Aktion:}

\begin{lstlisting}[numbers=none,frame=none,mathescape]
toLaTeX(simplifyMax(ddx ee)) $\leadsto$
  "\\frac{(\\cos x)^{2}+(\\sin x)^{2}}{(\\cos x)^{2}}+(2\\cdot x+2)"
\end{lstlisting}

\vspace{1ex}
\headsp{\LaTeX{} Fortolkning:}

  \[
  \frac{(\cos x)^{2}+(\sin x)^{2}}{(\cos x)^{2}}+(2\cdot x+2)
  \]

\end{footnotesize}
\end{frame}

\subsection*{Konklusion}
\begin{frame}[fragile]
  \headsp{Konklusion}

  \vspace{3mm}
  \tableofcontents
\end{frame}

\end{document}
