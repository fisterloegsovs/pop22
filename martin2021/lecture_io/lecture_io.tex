\documentclass[rgb]{beamer}

\usepackage[english]{babel}
\usepackage[utf8]{inputenc}
\usepackage{xcolor}
\usepackage{listings}
\usepackage{adjustbox}
\usepackage{amsmath}
\usepackage{multirow}
\usepackage[linewidth=1pt]{mdframed}

% Graphics
\usepackage{graphicx}

\usepackage{tikz}
\usetikzlibrary{calc,shapes.multipart,chains,arrows}

% Font
\usepackage{paratype}
\setbeamerfont{frametitle}{family=\bf}

% Beamer theme settings
\usecolortheme{seagull}
\setbeamertemplate{itemize item}{\raisebox{0.8mm}{\rule{1.8mm}{1.2mm}}}
\usenavigationsymbolstemplate{} % no navigation buttons

\usepackage{listings}

% Define Language
\lstdefinelanguage{fsharp}
{
  % list of keywords
  morekeywords={
    and,
    do,
    else,
    exception,
    for,
    fun,
    function,
    if,
    in,
    let,
    match,
    module,
    mutable,
    open,
    of,
    rec,
    then,
    try,
    type,
    unsafe,
    use,
    val,
    when,
    while,
    with,
  },
  sensitive=true, % keywords are not case-sensitive
  morecomment=[l]{//}, % l is for line comment
%  otherkeywords={>,<,=,<=,>=,!,*,/,-,+,|,&,||,&&,==,=>},
  morestring=[b]" % defines that strings are enclosed in double quotes
}

% Define Colors
\usepackage{color}
\definecolor{eclipseBlue}{RGB}{42,0.0,255}
\definecolor{eclipseGreen}{RGB}{63,127,95}
\definecolor{eclipsePurple}{RGB}{127,0,85}

\newcommand{\fop}[1]{\mbox{\ttfamily\color{eclipseBlue}#1}}
\newcommand{\fw}[1]{\mbox{\ttfamily\bfseries\color{eclipsePurple}#1}}

% Set Language
\lstset{
  language={fsharp},
  basicstyle=\ttfamily, % Global Code Style
  captionpos=b, % Position of the Caption (t for top, b for bottom)
  extendedchars=true, % Allows 256 instead of 128 ASCII characters
  tabsize=2, % number of spaces indented when discovering a tab
  columns=fixed, % make all characters equal width
  keepspaces=true, % does not ignore spaces to fit width, convert tabs to spaces
  showstringspaces=false, % lets spaces in strings appear as real spaces
  breaklines=true, % wrap lines if they don't fit
  frame=trbl, % draw a frame at the top, right, left and bottom of the listing
  frameround=tttt, % make the frame round at all four corners
  framesep=4pt, % quarter circle size of the round corners
  numbers=left, % show line numbers at the left
  numberstyle=\small\ttfamily, % style of the line numbers
  commentstyle=\slshape\bfseries\color{eclipseGreen}, % style of comments
  keywordstyle=\bfseries\color{eclipsePurple}, % style of keywords
  stringstyle=\color{eclipseBlue}, % style of strings
  emph=[1] {
    false,
    true,
    Set,
    Map,
    List,
    ImgUtil,
    Pegs,
    String,
    Array,
    Array2D
  },
  emphstyle=[1]{\color{eclipseBlue}},
  moredelim=**[is][\color{red}]{@@}{@@}
}

\newcommand{\theyear}{2020}
\newcommand{\sem}[1]{[\![#1]\!]}
\newcommand{\seme}[1]{\sem{#1}\varepsilon}
\newcommand{\semzero}[1]{\sem{#1}_0}

\newcommand{\emptymap}{\{\}}
\newcommand{\fracc}[2]{\begin{eqnarray} \frac{\begin{array}{c} #1
    \end{array}}{\begin{array}{c} #2 \end{array}} \end{eqnarray}}
\newcommand{\sembox}[1]{\hfill \normalfont \mbox{\fbox{\(#1\)}}}
\newcommand{\sempart}[2]{\subsubsection*{\rm\em #1 \sembox{#2}}}
\newcommand{\axiom}[1]{\begin{eqnarray} \begin{array}{c} #1 \end{array} \end{eqnarray}}
\newcommand{\fraccn}[2]{\refstepcounter{equation}\mbox{$\frac{\begin{array}{c} #1 \end{array}}{\begin{array}{c} #2 \end{array}}$}~(\arabic{equation})}
\newcommand{\fraccc}[2]{\mbox{$\frac{\begin{array}{c} #1 \end{array}}{\begin{array}{c} #2 \end{array}}$}}
\newcommand{\onepart}[1]{\noindent\hfill#1\hfill~\vspace{2mm}}
\newcommand{\twopart}[2]{\noindent\hfill#1\hfill#2\hfill~\vspace{2mm}}
\newcommand{\threepart}[3]{\noindent\hfill#1\hfill#2\hfill#3\hfill~\vspace{2mm}}
%\newcommand{\axiomm}[1]{\refstepcounter{equation}\mbox{$\begin{array}{c} #1 \end{array}$}~(\arabic{equation})}
\newcommand{\axiomm}[1]{$\begin{array}{c} #1 \end{array}$}
%\newcommand{\ar}[1]{\stackrel{#1}{\longrightarrow}}
\newcommand{\vd}{\vdash}
\newcommand{\Ran}{{\rm Ran}}
\newcommand{\Dom}{{\rm Dom}}
\newcommand{\kw}[1]{\texttt{#1}}
\newcommand{\id}[1]{\mbox{\it{#1}}}
\newcommand{\rarr}{\rightarrow}
\newcommand{\eval}{\rarr}
\newcommand{\evals}{\leadsto}
\newcommand{\larr}{\leftarrow}

\newcommand{\head}[1]{\vspace{3mm} \textbf{\normalsize #1}}
\newcommand{\headsp}[1]{\head{#1}\vspace{1ex}}
\newcommand{\size}{\ensuremath{\mathrm{size}}}
\renewcommand{\log}{\ensuremath{\mathrm{log}}}

\newcommand{\setallthemecolors}[1]{%
\setbeamercolor*{palette primary}{use=structure,fg=white,bg=#1}%
\setbeamercolor*{palette secondary}{use=structure,fg=white,bg=#1}%
\setbeamercolor*{palette tertiary}{use=structure,fg=white,bg=#1}}

\definecolor{black}{RGB}{0,0,0}
\definecolor{maroon}{RGB}{128,0,0}
\definecolor{olive}{RGB}{128,128,0}
\definecolor{green}{RGB}{0,128,0}
\definecolor{purple}{RGB}{128,0,128}
\definecolor{teal}{RGB}{0,128,128}
\definecolor{darkteal}{RGB}{0,92,92}
\definecolor{navy}{RGB}{0,0,128}
\definecolor{gray}{RGB}{128,128,128}
\definecolor{darkgray}{RGB}{60,60,60}
\definecolor{darkred}{RGB}{139,0,0}

%palette

% #173F5F (dark blue)
\definecolor{darkblue}{RGB}{23,63,95}
% #20639B (blue)
\definecolor{blue}{RGB}{32,99,155}
% #3CAEA3 (green)
\definecolor{magenta}{RGB}{60,174,163}
% #F6D55C (yellow)
\definecolor{yellow}{RGB}{246,213,92}
% #ED553B (red)
\definecolor{red}{RGB}{237,85,59}


\usecolortheme{whale}
\useoutertheme{infolines}
\useinnertheme{rectangles}

\newcommand{\popsettitle}[2]{%
\setallthemecolors{#1}%
\newcommand{\popemne}{#2}%
\title{Programmering og Problemløsning}%
\subtitle{#2}%
\author{Martin Elsman}%
\date{}%
\institute[DIKU]{Datalogisk Institut, Københavns Universitet (DIKU)}}

\newcommand{\popmaketitleframe}{%
  \frame{\titlepage%
   \vspace{-15mm}%
   \par\noindent\rule{\textwidth}{0.4pt}%

   \vspace{4mm}%
   \tableofcontents%
   \vspace{-4mm}%
   \par\noindent\rule{\textwidth}{0.4pt}%
  }%
  \section*{\popemne}%
}


\popsettitle{darkgray}{Input og Output}  % see ../util.tex for colors

\begin{document}

\popmaketitleframe

\renewcommand{\sp}{\vspace{1ex}}
\newcommand{\shead}[1]{\vspace{1ex}\head{#1}\vspace{1ex}}

\section{Konsol input og output}

\subsection{Konsollen}

\begin{frame}[fragile]
\begin{footnotesize}

  \shead{Konsollen}

  Med begrebet konsol henvises der til at der, når en process startes,
  etableres (åbnes) tre strømme som konteksten kan benytte til at kommunikere
  med processen (programmet):

  \vspace{1ex}
  \begin{tabular}{lll}
    \textbf{Standard output} & \texttt{stdout} & \texttt{System.Console} \\
    \textbf{Standard input} & \texttt{stdin} & \texttt{System.Console} \\
    \textbf{Standard error} & \texttt{stderr} & \texttt{System.Console.Error}
  \end{tabular}
  \vspace{1ex}

  Vi har tidligere set hvordan \lstinline{printf} (og
  venner) kan benyttes til at udskrive på \lstinline{stdout}.

  \shead{Funktioner i \lstinline{System.Console}}

  Udskrivning på \lstinline{stdout}:

\begin{lstlisting}[numbers=none,frame=none,mathescape]
  val Write     : string -> unit   // Write to the console
  val WriteLine : string -> unit   // Write a line
\end{lstlisting}

Indlæsning fra \lstinline{stdin}:

\begin{lstlisting}[numbers=none,frame=none,mathescape]
  val Read     : unit -> int       // Read ascii character
  val ReadLine : unit -> string    // Write a line
\end{lstlisting}

\textbf{Bemærk:} Funktionerne \lstinline{Read} og \lstinline{ReadLine}
``blokerer'' indtil data er tilgængelig.
\end{footnotesize}
\end{frame}

\subsection{Kommandolinieargumenter}

\begin{frame}[fragile]
\begin{footnotesize}

  \shead{Kommandolinieargumenter}

  Kommandolinieargumenter kan læses ved at etablere en hovedfunktion
  af typen \lstinline{string array -> int} og tilføje en
  \texttt{EntryPoint}-attribut til funktionen:

\begin{lstlisting}[numbers=none,frame=none,mathescape]
  [<EntryPoint>]
  let main (args: string array) : int =
    for a in args do printfn "%s" a
    0                                // status code "ok"
\end{lstlisting}

\shead{Eksempel kørsel --- og generering af stand-alone eksekverbar:}

\begin{lstlisting}[numbers=none,frame=none]
bash-3.2$ fsharpc --nologo main.fs
bash-3.2$ mkbundle --simple -o main main.exe
bash-3.2$ ./main hi there
hi
there
bash-3.2$
\end{lstlisting}

\end{footnotesize}
\end{frame}

\subsection*{Eksempler}

\begin{frame}[fragile]
\begin{footnotesize}

  \shead{Eksempel: Temperatur omregner (\texttt{temp.fs})}

\begin{lstlisting}[numbers=none,frame=none,mathescape]
open System
let fahrenheit (c:float) : unit =
  if c < -273.15 then failwith "input too small"
  else printfn "Fahrenheit: %f" (9.0/5.0*c + 32.0)

do Console.Write "Temperature in degrees Celcius: "
let s = Console.ReadLine()
do try fahrenheit(float(s)) with
     | Failure s -> Console.Error.WriteLine s
     | _ -> Console.Error.WriteLine "Expecting number"
\end{lstlisting}

\shead{Bemærk:}

\begin{itemize}
\item Vi benytter \lstinline{Console.Error.WriteLine} til udskrivning af fejlbeskeder.
\end{itemize}

\end{footnotesize}
\end{frame}

\begin{frame}[fragile]
\begin{footnotesize}

  \shead{Eksempel: Gentagne input (\texttt{numbers.fs})}

\begin{lstlisting}[numbers=none,frame=none,mathescape]
open System
let rec loop (a:float) : float =
  match Console.ReadLine() with
    | "" -> a
    | s -> loop (a+float(s))
do Console.WriteLine "Enter numbers (end with empty line):"
do try printfn "Sum: %f" (loop 0.0) with
     | _ -> Console.Error.WriteLine "Expecting numbers"
\end{lstlisting}

\shead{Bemærk:}

\begin{itemize}
\item Vi benytter en exception ``wild card handler'' til at fange fejl
  i input.
\end{itemize}

\end{footnotesize}
\end{frame}

\section{Læsning og skrivning af filer}

\subsection{Læsning og skrivning}
\begin{frame}[fragile]
\begin{footnotesize}

  \head{Læsning af filer}

  Operationer til læsning af UTF-8 filer er tilgængelige i modulet \texttt{System.IO.File}:

\begin{lstlisting}[numbers=none,frame=none,mathescape]
  type StreamReader = System.IO.StreamReader
     EndOfStream : bool
     Close       : unit -> unit
     ReadToEnd   : unit -> string  // incl. newlines
     ReadLine    : unit -> string  // excl. newlines
     Read        : unit -> int
  val OpenText : string -> StreamReader
\end{lstlisting}

\head{Eksempel: linier i en fil (\lstinline{lines.fs})}

\begin{lstlisting}[numbers=none,frame=none,mathescape]
[<EntryPoint>]
let main (args:string array) : int =
  let rec loop n (r:System.IO.StreamReader) =
    if r.EndOfStream then n
    else (ignore(r.ReadLine()); loop (n+1) r)
  in if Array.length args > 0 then
       (printfn "%d" (
          loop 0 (System.IO.File.OpenText args.[0])); 0)
     else (printfn "Expects file name as argument"; 1)
\end{lstlisting}

\end{footnotesize}
\end{frame}

\begin{frame}[fragile]
\begin{footnotesize}

  \shead{Skrivning af filer}

  Operationer til skrivning af UTF-8 filer er tilgængelige i modulet \texttt{System.IO.File}:

\begin{lstlisting}[numbers=none,frame=none,mathescape]
  type StreamWriter = System.IO.StreamWriter
     Close       : unit -> unit
     WriteLine   : string -> unit  // add newline
     Write       : string -> unit  // no newline

  val CreateText : string -> StreamWriter
  val AppendText : string -> StreamWriter
\end{lstlisting}

\shead{Eksempel: Fibonacci tal i en fil (\lstinline{fibfile.fs})}

\begin{lstlisting}[numbers=none,frame=none,mathescape]
let rec fib n = if n <= 2 then 1 else fib(n-1)+fib(n-2)
let rec loop n i (w:System.IO.StreamWriter) =
  if i > n then w.Close()
  else (w.WriteLine(string(fib i)); loop n (i+1) w)
let n = loop 10 1 (System.IO.File.CreateText "out.txt")
\end{lstlisting}

\end{footnotesize}
\end{frame}

\subsection{Håndtering af bytes}

\begin{frame}[fragile]

  \shead{Læsning og skrivning af bytes}
  \vspace{4mm}

  UTF-8 karakterer har variabel vidde (en-fire bytes) hvilket gør
  indexing i en UTF-8 streng lineær i størrelsen på strengen.

  \vspace{4mm}

  Almindelige karakterer i UTF-8 har størrelse 8-bit og i mange
  tilfælde vil der derfor ikke være forskel på filer i UTF-8 format og
  filer i simpelt ascii-format.

  \vspace{4mm} F\# giver mulighed for at arbejde med filer på
  byte-niveau (8-bits) ved brug af \lstinline{System.IO.FileStream}
  klassen.

\end{frame}

\section*{Konklusion}
\begin{frame}[fragile]
  \headsp{Konklusion}

  \vspace{3mm}
  \tableofcontents
\end{frame}

\end{document}
