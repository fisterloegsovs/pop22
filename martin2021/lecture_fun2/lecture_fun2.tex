\documentclass[rgb]{beamer}

\usepackage[english]{babel}
\usepackage[utf8]{inputenc}
\usepackage{xcolor}
\usepackage{listings}
\usepackage{adjustbox}
\usepackage{amsmath}
\usepackage{multirow}
\usepackage[linewidth=1pt]{mdframed}

% Graphics
\usepackage{graphicx}

\usepackage{tikz}
\usetikzlibrary{calc,shapes.multipart,chains,arrows}

% Font
\usepackage{paratype}
\setbeamerfont{frametitle}{family=\bf}

% Beamer theme settings
\usecolortheme{seagull}
\setbeamertemplate{itemize item}{\raisebox{0.8mm}{\rule{1.8mm}{1.2mm}}}
\usenavigationsymbolstemplate{} % no navigation buttons

\usepackage{listings}

% Define Language
\lstdefinelanguage{fsharp}
{
  % list of keywords
  morekeywords={
    and,
    do,
    else,
    exception,
    for,
    fun,
    function,
    if,
    in,
    let,
    match,
    module,
    mutable,
    open,
    of,
    rec,
    then,
    try,
    type,
    unsafe,
    use,
    val,
    when,
    while,
    with,
  },
  sensitive=true, % keywords are not case-sensitive
  morecomment=[l]{//}, % l is for line comment
%  otherkeywords={>,<,=,<=,>=,!,*,/,-,+,|,&,||,&&,==,=>},
  morestring=[b]" % defines that strings are enclosed in double quotes
}

% Define Colors
\usepackage{color}
\definecolor{eclipseBlue}{RGB}{42,0.0,255}
\definecolor{eclipseGreen}{RGB}{63,127,95}
\definecolor{eclipsePurple}{RGB}{127,0,85}

\newcommand{\fop}[1]{\mbox{\ttfamily\color{eclipseBlue}#1}}
\newcommand{\fw}[1]{\mbox{\ttfamily\bfseries\color{eclipsePurple}#1}}

% Set Language
\lstset{
  language={fsharp},
  basicstyle=\ttfamily, % Global Code Style
  captionpos=b, % Position of the Caption (t for top, b for bottom)
  extendedchars=true, % Allows 256 instead of 128 ASCII characters
  tabsize=2, % number of spaces indented when discovering a tab
  columns=fixed, % make all characters equal width
  keepspaces=true, % does not ignore spaces to fit width, convert tabs to spaces
  showstringspaces=false, % lets spaces in strings appear as real spaces
  breaklines=true, % wrap lines if they don't fit
  frame=trbl, % draw a frame at the top, right, left and bottom of the listing
  frameround=tttt, % make the frame round at all four corners
  framesep=4pt, % quarter circle size of the round corners
  numbers=left, % show line numbers at the left
  numberstyle=\small\ttfamily, % style of the line numbers
  commentstyle=\slshape\bfseries\color{eclipseGreen}, % style of comments
  keywordstyle=\bfseries\color{eclipsePurple}, % style of keywords
  stringstyle=\color{eclipseBlue}, % style of strings
  emph=[1] {
    false,
    true,
    Set,
    Map,
    List,
    ImgUtil,
    Pegs,
    String,
    Array,
    Array2D
  },
  emphstyle=[1]{\color{eclipseBlue}},
  moredelim=**[is][\color{red}]{@@}{@@}
}

\newcommand{\theyear}{2020}
\newcommand{\sem}[1]{[\![#1]\!]}
\newcommand{\seme}[1]{\sem{#1}\varepsilon}
\newcommand{\semzero}[1]{\sem{#1}_0}

\newcommand{\emptymap}{\{\}}
\newcommand{\fracc}[2]{\begin{eqnarray} \frac{\begin{array}{c} #1
    \end{array}}{\begin{array}{c} #2 \end{array}} \end{eqnarray}}
\newcommand{\sembox}[1]{\hfill \normalfont \mbox{\fbox{\(#1\)}}}
\newcommand{\sempart}[2]{\subsubsection*{\rm\em #1 \sembox{#2}}}
\newcommand{\axiom}[1]{\begin{eqnarray} \begin{array}{c} #1 \end{array} \end{eqnarray}}
\newcommand{\fraccn}[2]{\refstepcounter{equation}\mbox{$\frac{\begin{array}{c} #1 \end{array}}{\begin{array}{c} #2 \end{array}}$}~(\arabic{equation})}
\newcommand{\fraccc}[2]{\mbox{$\frac{\begin{array}{c} #1 \end{array}}{\begin{array}{c} #2 \end{array}}$}}
\newcommand{\onepart}[1]{\noindent\hfill#1\hfill~\vspace{2mm}}
\newcommand{\twopart}[2]{\noindent\hfill#1\hfill#2\hfill~\vspace{2mm}}
\newcommand{\threepart}[3]{\noindent\hfill#1\hfill#2\hfill#3\hfill~\vspace{2mm}}
%\newcommand{\axiomm}[1]{\refstepcounter{equation}\mbox{$\begin{array}{c} #1 \end{array}$}~(\arabic{equation})}
\newcommand{\axiomm}[1]{$\begin{array}{c} #1 \end{array}$}
%\newcommand{\ar}[1]{\stackrel{#1}{\longrightarrow}}
\newcommand{\vd}{\vdash}
\newcommand{\Ran}{{\rm Ran}}
\newcommand{\Dom}{{\rm Dom}}
\newcommand{\kw}[1]{\texttt{#1}}
\newcommand{\id}[1]{\mbox{\it{#1}}}
\newcommand{\rarr}{\rightarrow}
\newcommand{\eval}{\rarr}
\newcommand{\evals}{\leadsto}
\newcommand{\larr}{\leftarrow}

\newcommand{\head}[1]{\vspace{3mm} \textbf{\normalsize #1}}
\newcommand{\headsp}[1]{\head{#1}\vspace{1ex}}
\newcommand{\size}{\ensuremath{\mathrm{size}}}
\renewcommand{\log}{\ensuremath{\mathrm{log}}}

\newcommand{\setallthemecolors}[1]{%
\setbeamercolor*{palette primary}{use=structure,fg=white,bg=#1}%
\setbeamercolor*{palette secondary}{use=structure,fg=white,bg=#1}%
\setbeamercolor*{palette tertiary}{use=structure,fg=white,bg=#1}}

\definecolor{black}{RGB}{0,0,0}
\definecolor{maroon}{RGB}{128,0,0}
\definecolor{olive}{RGB}{128,128,0}
\definecolor{green}{RGB}{0,128,0}
\definecolor{purple}{RGB}{128,0,128}
\definecolor{teal}{RGB}{0,128,128}
\definecolor{darkteal}{RGB}{0,92,92}
\definecolor{navy}{RGB}{0,0,128}
\definecolor{gray}{RGB}{128,128,128}
\definecolor{darkgray}{RGB}{60,60,60}
\definecolor{darkred}{RGB}{139,0,0}

%palette

% #173F5F (dark blue)
\definecolor{darkblue}{RGB}{23,63,95}
% #20639B (blue)
\definecolor{blue}{RGB}{32,99,155}
% #3CAEA3 (green)
\definecolor{magenta}{RGB}{60,174,163}
% #F6D55C (yellow)
\definecolor{yellow}{RGB}{246,213,92}
% #ED553B (red)
\definecolor{red}{RGB}{237,85,59}


\usecolortheme{whale}
\useoutertheme{infolines}
\useinnertheme{rectangles}

\newcommand{\popsettitle}[2]{%
\setallthemecolors{#1}%
\newcommand{\popemne}{#2}%
\title{Programmering og Problemløsning}%
\subtitle{#2}%
\author{Martin Elsman}%
\date{}%
\institute[DIKU]{Datalogisk Institut, Københavns Universitet (DIKU)}}

\newcommand{\popmaketitleframe}{%
  \frame{\titlepage%
   \vspace{-15mm}%
   \par\noindent\rule{\textwidth}{0.4pt}%

   \vspace{4mm}%
   \tableofcontents%
   \vspace{-4mm}%
   \par\noindent\rule{\textwidth}{0.4pt}%
  }%
  \section*{\popemne}%
}


\popsettitle{darkred}{Højereordens funktioner (Del 2)}  % see ../util.tex for colors

\begin{document}

\popmaketitleframe

\renewcommand{\sp}{\vspace{1ex}}
\newcommand{\shead}[1]{\vspace{1ex}\head{#1}\vspace{1ex}}

\subsection{Funktioner der returnerer funktioner}

\begin{frame}[fragile]
\begin{footnotesize}
  \shead{Funktioner der returnerer funktioner}

  Det er ofte anvendeligt at kunne skrive funktioner der selv
  returnerer funktioner.

  \shead{Eksempel: Funktionssammensætning}

\begin{lstlisting}[numbers=none,frame=none,mathescape]
// compose f g = f $\circ$ g
let compose (f: 'a->'b) (g:'c->'a) : 'c->'b =
  fun x -> f(g x)
\end{lstlisting}

Denne funktion er direkte tilgængelig i F\# som infix-funktionen \lstinline{<<}:

\begin{lstlisting}[numbers=none,frame=none,mathescape]
> ((fun x->x+1) << (fun x->x*2)) 5;;
val it : int = 11
\end{lstlisting}

\shead{Bemærk:}
\begin{itemize}
\item Funktionssammensætning (\lstinline{f << g}) i F\# svarer til
  matematisk funktionssammensætning, som i $f \circ g$, hvor $(f \circ g)(x) = f(g(x))$.
\item Omvendt funktionssammensætning er i F\# defineret ved \lstinline{f >> g = g << f}.
\end{itemize}

\end{footnotesize}
\end{frame}

\subsection{Piping og funktionssammensætning}

\begin{frame}[fragile]
\begin{small}
  \head{Piping versus Funktionssammensætning}

  \shead{Pipe-operatorer}

\begin{lstlisting}[numbers=none,frame=none,mathescape]
val |> : 'a -> ('a->'b) -> 'b   // x |> g = g x
val <| : ('a->'b) -> 'a -> 'b   // g <| x = g x
\end{lstlisting}

\shead{Funktionssammensætning}

\begin{lstlisting}[numbers=none,frame=none,mathescape]
val >> : ('a->'b) -> ('b->'c) -> ('a->'c)
      // (g >> f)x = f(g x)

val << : ('a->'b) -> ('c->'a) -> ('c->'b)
      // (f << g)x = f(g x) = (f $\circ$ g)(x)
\end{lstlisting}

\textbf{Bemærk:} Piping virker på værdier, hvor funktionssammensætning
benyttes til at definere nye funktioner på bagrund af andre funktioner.

\end{small}
\end{frame}

\subsection{Currying}

\begin{frame}[fragile]
\begin{footnotesize}
  \shead{Currying}

  Currying henviser til følgende indsigt:
  \begin{enumerate}
  \item En funktion \lstinline{f:'a*'b->'c} der tager et par som argument kan omskrives til en funktion \lstinline{g:'a->'b->'c} der tager to argumenter.
  \item En funktion \lstinline{g:'a->'b->'c} der tager to argumenter kan omskrives til en funktion \lstinline{f:'a*'b->'c} der tager et par som argument (purity antaget).
  \end{enumerate}

  Omskrivningerne kan realiseres med følgende to funktioner:

\begin{lstlisting}[numbers=none,frame=none,mathescape]
let curry (f:'a*'b->'c) : 'a->'b->'c =
  fun a -> fun b -> f(a,b)

let uncurry (f:'a->'b->'c) : 'a*'b->'c =
  fun (a,b) -> f a b
\end{lstlisting}

\shead{Eksempel:}
\begin{lstlisting}[numbers=none,frame=none,mathescape]
> List.map (uncurry (+)) [(2,5);(8,1);(7,6)];;
val it : int list = [7; 9; 13]
\end{lstlisting}

\end{footnotesize}
\end{frame}

\subsection{Delvist anvendte funktioner}

\begin{frame}[fragile]
\begin{footnotesize}
  \shead{Delvist anvendte funktioner}

  Det kan ofte være anvendeligt at anvende en funktion delvist for derved at skabe en ny funktion der passer i en sammenhæng.

  \shead{Eksempel --- pretty printing med ``point-free'' notation:}

\begin{lstlisting}[numbers=none,frame=none,mathescape]
let rec padl n s = if String.length s > n then s
                   else padl (n-1) (" " + s)
let pp n = String.concat "\n"
           << List.map (String.concat " "
                        << List.map (padl n << string))
do printfn "%s" (pp 3 [[1;2;5];[12;3;25];[7;32;1]])
\end{lstlisting}

\head{Kørsel:}

\begin{lstlisting}[numbers=none,frame=none,mathescape]
  1   2   5    // (<<)   : ('a->'b) -> ('c->'a) -> ('c->'b)
 12   3  25    // padl   : int -> string -> string
  7  32   1    // concat : string -> string list -> string
               // map    : ('a -> 'b) -> 'a list -> 'b list
               // string : int -> string
\end{lstlisting}

\end{footnotesize}
\end{frame}


\subsection*{Konklusion}
\begin{frame}[fragile]
  \headsp{Konklusion}

  \vspace{3mm}
  \tableofcontents
\end{frame}

\end{document}
