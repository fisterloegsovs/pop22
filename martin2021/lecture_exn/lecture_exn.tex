\documentclass[rgb]{beamer}

\usepackage[english]{babel}
\usepackage[utf8]{inputenc}
\usepackage{xcolor}
\usepackage{listings}
\usepackage{adjustbox}
\usepackage{amsmath}
\usepackage{multirow}
\usepackage[linewidth=1pt]{mdframed}

% Graphics
\usepackage{graphicx}

\usepackage{tikz}
\usetikzlibrary{calc,shapes.multipart,chains,arrows}

% Font
\usepackage{paratype}
\setbeamerfont{frametitle}{family=\bf}

% Beamer theme settings
\usecolortheme{seagull}
\setbeamertemplate{itemize item}{\raisebox{0.8mm}{\rule{1.8mm}{1.2mm}}}
\usenavigationsymbolstemplate{} % no navigation buttons

\usepackage{listings}

% Define Language
\lstdefinelanguage{fsharp}
{
  % list of keywords
  morekeywords={
    and,
    do,
    else,
    exception,
    for,
    fun,
    function,
    if,
    in,
    let,
    match,
    module,
    mutable,
    open,
    of,
    rec,
    then,
    try,
    type,
    unsafe,
    use,
    val,
    when,
    while,
    with,
  },
  sensitive=true, % keywords are not case-sensitive
  morecomment=[l]{//}, % l is for line comment
%  otherkeywords={>,<,=,<=,>=,!,*,/,-,+,|,&,||,&&,==,=>},
  morestring=[b]" % defines that strings are enclosed in double quotes
}

% Define Colors
\usepackage{color}
\definecolor{eclipseBlue}{RGB}{42,0.0,255}
\definecolor{eclipseGreen}{RGB}{63,127,95}
\definecolor{eclipsePurple}{RGB}{127,0,85}

\newcommand{\fop}[1]{\mbox{\ttfamily\color{eclipseBlue}#1}}
\newcommand{\fw}[1]{\mbox{\ttfamily\bfseries\color{eclipsePurple}#1}}

% Set Language
\lstset{
  language={fsharp},
  basicstyle=\ttfamily, % Global Code Style
  captionpos=b, % Position of the Caption (t for top, b for bottom)
  extendedchars=true, % Allows 256 instead of 128 ASCII characters
  tabsize=2, % number of spaces indented when discovering a tab
  columns=fixed, % make all characters equal width
  keepspaces=true, % does not ignore spaces to fit width, convert tabs to spaces
  showstringspaces=false, % lets spaces in strings appear as real spaces
  breaklines=true, % wrap lines if they don't fit
  frame=trbl, % draw a frame at the top, right, left and bottom of the listing
  frameround=tttt, % make the frame round at all four corners
  framesep=4pt, % quarter circle size of the round corners
  numbers=left, % show line numbers at the left
  numberstyle=\small\ttfamily, % style of the line numbers
  commentstyle=\slshape\bfseries\color{eclipseGreen}, % style of comments
  keywordstyle=\bfseries\color{eclipsePurple}, % style of keywords
  stringstyle=\color{eclipseBlue}, % style of strings
  emph=[1] {
    false,
    true,
    Set,
    Map,
    List,
    ImgUtil,
    Pegs,
    String,
    Array,
    Array2D
  },
  emphstyle=[1]{\color{eclipseBlue}},
  moredelim=**[is][\color{red}]{@@}{@@}
}

\newcommand{\theyear}{2020}
\newcommand{\sem}[1]{[\![#1]\!]}
\newcommand{\seme}[1]{\sem{#1}\varepsilon}
\newcommand{\semzero}[1]{\sem{#1}_0}

\newcommand{\emptymap}{\{\}}
\newcommand{\fracc}[2]{\begin{eqnarray} \frac{\begin{array}{c} #1
    \end{array}}{\begin{array}{c} #2 \end{array}} \end{eqnarray}}
\newcommand{\sembox}[1]{\hfill \normalfont \mbox{\fbox{\(#1\)}}}
\newcommand{\sempart}[2]{\subsubsection*{\rm\em #1 \sembox{#2}}}
\newcommand{\axiom}[1]{\begin{eqnarray} \begin{array}{c} #1 \end{array} \end{eqnarray}}
\newcommand{\fraccn}[2]{\refstepcounter{equation}\mbox{$\frac{\begin{array}{c} #1 \end{array}}{\begin{array}{c} #2 \end{array}}$}~(\arabic{equation})}
\newcommand{\fraccc}[2]{\mbox{$\frac{\begin{array}{c} #1 \end{array}}{\begin{array}{c} #2 \end{array}}$}}
\newcommand{\onepart}[1]{\noindent\hfill#1\hfill~\vspace{2mm}}
\newcommand{\twopart}[2]{\noindent\hfill#1\hfill#2\hfill~\vspace{2mm}}
\newcommand{\threepart}[3]{\noindent\hfill#1\hfill#2\hfill#3\hfill~\vspace{2mm}}
%\newcommand{\axiomm}[1]{\refstepcounter{equation}\mbox{$\begin{array}{c} #1 \end{array}$}~(\arabic{equation})}
\newcommand{\axiomm}[1]{$\begin{array}{c} #1 \end{array}$}
%\newcommand{\ar}[1]{\stackrel{#1}{\longrightarrow}}
\newcommand{\vd}{\vdash}
\newcommand{\Ran}{{\rm Ran}}
\newcommand{\Dom}{{\rm Dom}}
\newcommand{\kw}[1]{\texttt{#1}}
\newcommand{\id}[1]{\mbox{\it{#1}}}
\newcommand{\rarr}{\rightarrow}
\newcommand{\eval}{\rarr}
\newcommand{\evals}{\leadsto}
\newcommand{\larr}{\leftarrow}

\newcommand{\head}[1]{\vspace{3mm} \textbf{\normalsize #1}}
\newcommand{\headsp}[1]{\head{#1}\vspace{1ex}}
\newcommand{\size}{\ensuremath{\mathrm{size}}}
\renewcommand{\log}{\ensuremath{\mathrm{log}}}

\newcommand{\setallthemecolors}[1]{%
\setbeamercolor*{palette primary}{use=structure,fg=white,bg=#1}%
\setbeamercolor*{palette secondary}{use=structure,fg=white,bg=#1}%
\setbeamercolor*{palette tertiary}{use=structure,fg=white,bg=#1}}

\definecolor{black}{RGB}{0,0,0}
\definecolor{maroon}{RGB}{128,0,0}
\definecolor{olive}{RGB}{128,128,0}
\definecolor{green}{RGB}{0,128,0}
\definecolor{purple}{RGB}{128,0,128}
\definecolor{teal}{RGB}{0,128,128}
\definecolor{darkteal}{RGB}{0,92,92}
\definecolor{navy}{RGB}{0,0,128}
\definecolor{gray}{RGB}{128,128,128}
\definecolor{darkgray}{RGB}{60,60,60}
\definecolor{darkred}{RGB}{139,0,0}

%palette

% #173F5F (dark blue)
\definecolor{darkblue}{RGB}{23,63,95}
% #20639B (blue)
\definecolor{blue}{RGB}{32,99,155}
% #3CAEA3 (green)
\definecolor{magenta}{RGB}{60,174,163}
% #F6D55C (yellow)
\definecolor{yellow}{RGB}{246,213,92}
% #ED553B (red)
\definecolor{red}{RGB}{237,85,59}


\usecolortheme{whale}
\useoutertheme{infolines}
\useinnertheme{rectangles}

\newcommand{\popsettitle}[2]{%
\setallthemecolors{#1}%
\newcommand{\popemne}{#2}%
\title{Programmering og Problemløsning}%
\subtitle{#2}%
\author{Martin Elsman}%
\date{}%
\institute[DIKU]{Datalogisk Institut, Københavns Universitet (DIKU)}}

\newcommand{\popmaketitleframe}{%
  \frame{\titlepage%
   \vspace{-15mm}%
   \par\noindent\rule{\textwidth}{0.4pt}%

   \vspace{4mm}%
   \tableofcontents%
   \vspace{-4mm}%
   \par\noindent\rule{\textwidth}{0.4pt}%
  }%
  \section*{\popemne}%
}


\popsettitle{darkgray}{Exceptions (Undtagelser)}  % see ../util.tex for colors

\begin{document}

\popmaketitleframe

\renewcommand{\sp}{\vspace{1ex}}
\newcommand{\shead}[1]{\vspace{1ex}\head{#1}\vspace{1ex}}

%%%%%%%%%%%%%%%%%%%%%%%%%%%%%%%%%%%%%%%%%%%%%%%%
\subsection{Introduktion}
%%%%%%%%%%%%%%%%%%%%%%%%%%%%%%%%%%%%%%%%%%%%%%%%

\begin{frame}[fragile]
\begin{footnotesize}

  \head{Exceptions og alternativer}

  \vspace{1ex}

  \begin{minipage}{.65\textwidth}
  \begin{enumerate}
  \item \textbf{Exceptions (undtagelser):}

    \begin{itemize}
    \item Exception-værdier.

    \item Rejsning af exceptions (\lstinline{raise}/throw)

    \item Håndtering af exceptions (\lstinline{try}-\lstinline{with})

    \item Indbyggede exceptions.

    \item Nogle nyttige hjælpefunktioner.
    \end{itemize}

  \item \textbf{Alternativer til exceptions:}

    \begin{itemize}
    \item Fejlhåndtering med \lstinline{option} typer.
    \item Beskedbærende fejlmonade.
    \end{itemize}
  \end{enumerate}
  \end{minipage}~\begin{minipage}{.3\textwidth}
  \includegraphics[width=1.2\textwidth]{../images/exception.png}
  \end{minipage}

\end{footnotesize}
\end{frame}
\subsection{Exception-værdier}

\begin{frame}[fragile]
\begin{footnotesize}

  \shead{Exception-værdier}

  Exceptions er værdier af den indbyggede \emph{udvidbare} type
  \lstinline{exn}.

  \vspace{2ex}

  Exception-værdier kan være konstante værdier, der ikke bærer
  argumenter, og exception-værdier der bærer argumenter.

  \vspace{2ex}

  Nye \emph{exception-konstruktører} kan erklæres med
  \lstinline{exception}-konstruktionen:

  \begin{lstlisting}[numbers=none,frame=none,mathescape]
    exception MyError
    exception MyArgExn of int
    let e1 : exn = MyError
    let e2 : exn = MyArgExn 5
  \end{lstlisting}

  \textbf{Exception-værdier tillader lighed:}

  \begin{lstlisting}[numbers=none,frame=none,mathescape]
    let isMyError = e1 = e2                  // $\leadsto$ false
  \end{lstlisting}

  ... og kan benyttes i matches:

  \begin{lstlisting}[numbers=none,frame=none,mathescape]
    let isMyArgExn =
      match e2 with MyArgExn _ -> "yes"      // $\leadsto$ "yes"
                  | _ -> "no"
  \end{lstlisting}


\end{footnotesize}
\end{frame}

\subsection{Rejsning og håndtering af exceptions}

\begin{frame}[fragile]
\begin{footnotesize}

  \shead{Rejsning af Exceptions}

  Exceptions kan benyttes til at afbryde det normale kontrol-flow (deraf navnet exception).

  \vspace{2ex}

  Konstruktionen der benyttes til at ``\emph{rejse en exception}'' er
  operationen \lstinline{raise}:

\begin{lstlisting}[numbers=none,frame=none,mathescape]
  val raise : exn -> 'a
\end{lstlisting}

\textbf{Bemærk} at operationen kan instantieres til at returnere en værdi af
vilkårlig type, hvilket skyldes at operationen aldrig returnerer, men
derimod sender en besked (en exception) ``op i kaldstakken'' om at
beregningen blev afbrudt.

\vspace{2ex}

Det er muligt at \emph{håndtere} rejste exceptions på et højere niveau
ved at benytte en \lstinline{try}-\lstinline{with} konstruktion.

\vspace{2ex}

Her er hvad der sker hvis en rejst exception ikke håndteres:
\begin{lstlisting}[numbers=none,frame=none,mathescape]
  - exception MyArgExn of int;;
  - "hello " + raise (MyArgExn 42)
  FSI_0002+MyArgExn: Exception ... was thrown...
  Stopped due to error
\end{lstlisting}

Den rejste exception blev ``håndteret'' af \lstinline{fsharpi}'s \emph{top-level handler}.
\end{footnotesize}
\end{frame}

\begin{frame}[fragile]
\begin{footnotesize}

  \shead{Håndtering af rejste exceptions}

  Rejste exceptions kan håndteres på et højere niveau i programmet ved
  at benytte en \lstinline{try}-\lstinline{with} konstruktion.

  \shead{Eksempel (\texttt{exn.fs}):}

\begin{lstlisting}[numbers=none,frame=none,mathescape]
  exception MyExnArg of int
  let f () = if false then 8 else raise (MyExnArg 5)
  let y = try f () with MyExnArg x -> x
  do printfn "%d" y
\end{lstlisting}

\head{Kørsel:}
\vspace{1ex}

\begin{lstlisting}[numbers=none,frame=none]
  bash-3.2$ fsharpc --nologo exn.fs && mono exn.exe
  5
\end{lstlisting}

\head{Bemærk:}

\vspace{1ex}
Argumentet til den rejste exception blev trukket ud af
exception-værdien ved brug af \emph{exception pattern matching}.
\end{footnotesize}
\end{frame}

\subsection{Indbyggede exceptions og hjælpefunktioner}

\begin{frame}[fragile]
\begin{footnotesize}

  \shead{Indbyggede Exceptions og Hjælpefunktioner}

  For at matche de indbyggede Mono exceptions, kan det være nødvendigt at
  benytte \emph{dynamic type matching}, som benytter sig af følgende
  syntax:

\begin{lstlisting}[numbers=none,frame=none]
  let mydiv a b : int option =
    try Some (a / b) with
      :? System.DivideByZeroException -> None
\end{lstlisting}

\head{Nogle hjælpefunktioner:}
\vspace{1ex}

\begin{lstlisting}[numbers=none,frame=none]
  val failwith   : string -> 'a
  val invalidArg : string -> string -> 'a
\end{lstlisting}

\head{Brug af funktionen \lstinline{invalidArg}:}

\vspace{1ex}
\begin{lstlisting}[numbers=none,frame=none,mathescape]
  let toFahrenheit c =
    if c < -273.15 then invalidArg "c" "below absolute zero"
    else 9.0/5.0*float(c)+32.0
\end{lstlisting}

\end{footnotesize}
\end{frame}

\subsection{Alternativer til exceptions}

\begin{frame}[fragile]
\begin{footnotesize}
  \shead{Fejlhåndtering med \lstinline{option} typer}

  Funktionen \lstinline{mydiv} benytter værdien \lstinline{None} til
  at indikere en fejl.

  \vspace{1ex}

  Option-typer kan således bruges til at indkode exceptionel opførsel.

  \vspace{1ex}

  Funktionen \lstinline{Option.bind} kan bruges til at styre
  sammensætningen af beregninger:

\begin{lstlisting}[numbers=none,frame=none,mathescape]
  val bind : ('a -> 'b option) -> 'a option -> 'b option
\end{lstlisting}

  \head{Eksempel}

  \vspace{1ex}

\begin{lstlisting}[numbers=none,frame=none,mathescape]
  > let (>>=) x y = Option.bind y x
  > mydiv 8 3 >>= (fun x -> Some(float(x)+1.0));;
  val it : float option = Some 3.0
\end{lstlisting}

\head{Bemærk:}

\begin{itemize}
\item Funktionen \lstinline{Option.bind} er et simpelt eksempel på
  bindingsoperatoren i en såkaldt \emph{monad}, en
  abstraktionsmekanisme der giver mulighed for at sammensætte
  effektfulde beregninger på en sund måde.
\item Monads ligger blandt andet til grund for sammensætning af
  effektfulde beregninger i Haskell.
\end{itemize}

\end{footnotesize}
\end{frame}

\subsection{Beskedbærende fejlmonade}

\begin{frame}[fragile]
\begin{footnotesize}
  \shead{Beskedbærende fejlmonade}

  Det er muligt at udvidde teknikken med funktionalitet der bærer
  information om fejlen.


\begin{lstlisting}[numbers=none,frame=none,mathescape]
  type 'a result = Ok of 'a | Error of string
  val (>>=) : 'a result -> ('a -> 'b result) -> 'b result
\end{lstlisting}

  \head{Eksempel}


\begin{lstlisting}[numbers=none,frame=none,mathescape]
  let mydiv a b : int result =
    try Ok (a / b) with
      :? System.DivideByZeroException -> Error "div"

  let (>>=) a f =
    match a with Ok v -> f v
               | Error s -> Error s

  do printfn "%A" (mydiv 8 3 >>= (fun x -> Ok(float(x)+1.0)))
\end{lstlisting}

\head{Bemærk:}

\begin{itemize}
\item Funktionaliteten kan også udviddes til at understøtte tilfælde
  hvor det er muligt at opsamle og rapportere multiple fejl.
\end{itemize}

\end{footnotesize}
\end{frame}

\subsection*{Konklusion}
\begin{frame}[fragile]
  \headsp{Konklusion}

  \vspace{3mm}
  \tableofcontents
\end{frame}

\end{document}
