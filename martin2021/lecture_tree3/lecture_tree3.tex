\documentclass[rgb]{beamer}

\usepackage[english]{babel}
\usepackage[utf8]{inputenc}
\usepackage{xcolor}
\usepackage{listings}
\usepackage{adjustbox}
\usepackage{amsmath}
\usepackage{multirow}
\usepackage[linewidth=1pt]{mdframed}

% Graphics
\usepackage{graphicx}

\usepackage{tikz}
\usetikzlibrary{calc,shapes.multipart,chains,arrows}

% Font
\usepackage{paratype}
\setbeamerfont{frametitle}{family=\bf}

% Beamer theme settings
\usecolortheme{seagull}
\setbeamertemplate{itemize item}{\raisebox{0.8mm}{\rule{1.8mm}{1.2mm}}}
\usenavigationsymbolstemplate{} % no navigation buttons

\usepackage{listings}

% Define Language
\lstdefinelanguage{fsharp}
{
  % list of keywords
  morekeywords={
    and,
    do,
    else,
    exception,
    for,
    fun,
    function,
    if,
    in,
    let,
    match,
    module,
    mutable,
    open,
    of,
    rec,
    then,
    try,
    type,
    unsafe,
    use,
    val,
    when,
    while,
    with,
  },
  sensitive=true, % keywords are not case-sensitive
  morecomment=[l]{//}, % l is for line comment
%  otherkeywords={>,<,=,<=,>=,!,*,/,-,+,|,&,||,&&,==,=>},
  morestring=[b]" % defines that strings are enclosed in double quotes
}

% Define Colors
\usepackage{color}
\definecolor{eclipseBlue}{RGB}{42,0.0,255}
\definecolor{eclipseGreen}{RGB}{63,127,95}
\definecolor{eclipsePurple}{RGB}{127,0,85}

\newcommand{\fop}[1]{\mbox{\ttfamily\color{eclipseBlue}#1}}
\newcommand{\fw}[1]{\mbox{\ttfamily\bfseries\color{eclipsePurple}#1}}

% Set Language
\lstset{
  language={fsharp},
  basicstyle=\ttfamily, % Global Code Style
  captionpos=b, % Position of the Caption (t for top, b for bottom)
  extendedchars=true, % Allows 256 instead of 128 ASCII characters
  tabsize=2, % number of spaces indented when discovering a tab
  columns=fixed, % make all characters equal width
  keepspaces=true, % does not ignore spaces to fit width, convert tabs to spaces
  showstringspaces=false, % lets spaces in strings appear as real spaces
  breaklines=true, % wrap lines if they don't fit
  frame=trbl, % draw a frame at the top, right, left and bottom of the listing
  frameround=tttt, % make the frame round at all four corners
  framesep=4pt, % quarter circle size of the round corners
  numbers=left, % show line numbers at the left
  numberstyle=\small\ttfamily, % style of the line numbers
  commentstyle=\slshape\bfseries\color{eclipseGreen}, % style of comments
  keywordstyle=\bfseries\color{eclipsePurple}, % style of keywords
  stringstyle=\color{eclipseBlue}, % style of strings
  emph=[1] {
    false,
    true,
    Set,
    Map,
    List,
    ImgUtil,
    Pegs,
    String,
    Array,
    Array2D
  },
  emphstyle=[1]{\color{eclipseBlue}},
  moredelim=**[is][\color{red}]{@@}{@@}
}

\newcommand{\theyear}{2020}
\newcommand{\sem}[1]{[\![#1]\!]}
\newcommand{\seme}[1]{\sem{#1}\varepsilon}
\newcommand{\semzero}[1]{\sem{#1}_0}

\newcommand{\emptymap}{\{\}}
\newcommand{\fracc}[2]{\begin{eqnarray} \frac{\begin{array}{c} #1
    \end{array}}{\begin{array}{c} #2 \end{array}} \end{eqnarray}}
\newcommand{\sembox}[1]{\hfill \normalfont \mbox{\fbox{\(#1\)}}}
\newcommand{\sempart}[2]{\subsubsection*{\rm\em #1 \sembox{#2}}}
\newcommand{\axiom}[1]{\begin{eqnarray} \begin{array}{c} #1 \end{array} \end{eqnarray}}
\newcommand{\fraccn}[2]{\refstepcounter{equation}\mbox{$\frac{\begin{array}{c} #1 \end{array}}{\begin{array}{c} #2 \end{array}}$}~(\arabic{equation})}
\newcommand{\fraccc}[2]{\mbox{$\frac{\begin{array}{c} #1 \end{array}}{\begin{array}{c} #2 \end{array}}$}}
\newcommand{\onepart}[1]{\noindent\hfill#1\hfill~\vspace{2mm}}
\newcommand{\twopart}[2]{\noindent\hfill#1\hfill#2\hfill~\vspace{2mm}}
\newcommand{\threepart}[3]{\noindent\hfill#1\hfill#2\hfill#3\hfill~\vspace{2mm}}
%\newcommand{\axiomm}[1]{\refstepcounter{equation}\mbox{$\begin{array}{c} #1 \end{array}$}~(\arabic{equation})}
\newcommand{\axiomm}[1]{$\begin{array}{c} #1 \end{array}$}
%\newcommand{\ar}[1]{\stackrel{#1}{\longrightarrow}}
\newcommand{\vd}{\vdash}
\newcommand{\Ran}{{\rm Ran}}
\newcommand{\Dom}{{\rm Dom}}
\newcommand{\kw}[1]{\texttt{#1}}
\newcommand{\id}[1]{\mbox{\it{#1}}}
\newcommand{\rarr}{\rightarrow}
\newcommand{\eval}{\rarr}
\newcommand{\evals}{\leadsto}
\newcommand{\larr}{\leftarrow}

\newcommand{\head}[1]{\vspace{3mm} \textbf{\normalsize #1}}
\newcommand{\headsp}[1]{\head{#1}\vspace{1ex}}
\newcommand{\size}{\ensuremath{\mathrm{size}}}
\renewcommand{\log}{\ensuremath{\mathrm{log}}}

\newcommand{\setallthemecolors}[1]{%
\setbeamercolor*{palette primary}{use=structure,fg=white,bg=#1}%
\setbeamercolor*{palette secondary}{use=structure,fg=white,bg=#1}%
\setbeamercolor*{palette tertiary}{use=structure,fg=white,bg=#1}}

\definecolor{black}{RGB}{0,0,0}
\definecolor{maroon}{RGB}{128,0,0}
\definecolor{olive}{RGB}{128,128,0}
\definecolor{green}{RGB}{0,128,0}
\definecolor{purple}{RGB}{128,0,128}
\definecolor{teal}{RGB}{0,128,128}
\definecolor{darkteal}{RGB}{0,92,92}
\definecolor{navy}{RGB}{0,0,128}
\definecolor{gray}{RGB}{128,128,128}
\definecolor{darkgray}{RGB}{60,60,60}
\definecolor{darkred}{RGB}{139,0,0}

%palette

% #173F5F (dark blue)
\definecolor{darkblue}{RGB}{23,63,95}
% #20639B (blue)
\definecolor{blue}{RGB}{32,99,155}
% #3CAEA3 (green)
\definecolor{magenta}{RGB}{60,174,163}
% #F6D55C (yellow)
\definecolor{yellow}{RGB}{246,213,92}
% #ED553B (red)
\definecolor{red}{RGB}{237,85,59}


\usecolortheme{whale}
\useoutertheme{infolines}
\useinnertheme{rectangles}

\newcommand{\popsettitle}[2]{%
\setallthemecolors{#1}%
\newcommand{\popemne}{#2}%
\title{Programmering og Problemløsning}%
\subtitle{#2}%
\author{Martin Elsman}%
\date{}%
\institute[DIKU]{Datalogisk Institut, Københavns Universitet (DIKU)}}

\newcommand{\popmaketitleframe}{%
  \frame{\titlepage%
   \vspace{-15mm}%
   \par\noindent\rule{\textwidth}{0.4pt}%

   \vspace{4mm}%
   \tableofcontents%
   \vspace{-4mm}%
   \par\noindent\rule{\textwidth}{0.4pt}%
  }%
  \section*{\popemne}%
}


\popsettitle{teal}{Træstrukturer (Del 3)}  % see ../util.tex for colors

\begin{document}

\popmaketitleframe

\newcommand{\mytree}[1]{
    \begin{tikzpicture}[domain=0:80,scale=#1]
    %\draw[very thin,color=gray] (0,0) grid (80,80);
    \draw[->] (17.5,25) -- (12.5,15);
    \draw[->] (22.5,25) -- (27.5,15);
    \draw[->] (27.5,45) -- (22.5,35);
    \draw[->] (32.5,45) -- (37.5,35);
    \draw[->] (67.5,45) -- (62.5,35);
    \draw[->] (72.5,45) -- (77.5,35);
    \draw[->] (45,65) -- (35,55);
    \draw[->] (55,65) -- (65,55);
    \node at (50,70) {1};
    \node at (30,50) {2};
    \node at (70,50) {3};
    \node at (20,30) {4};
    \node at (40,30) {5};
    \node at (60,30) {6};
    \node at (80,30) {7};
    \node at (10,10) {8};
    \node at (30,10) {9};
    \end{tikzpicture}
}

\newcommand{\myinordertree}[1]{
    \begin{tikzpicture}[domain=0:80,scale=#1]
    %\draw[very thin,color=gray] (0,0) grid (80,80);
    \draw[->] (17.5,25) -- (12.5,15);
    \draw[->] (22.5,25) -- (27.5,15);
    \draw[->] (27.5,45) -- (22.5,35);
    \draw[->] (32.5,45) -- (37.5,35);
    \draw[->] (67.5,45) -- (62.5,35);
    \draw[->] (72.5,45) -- (77.5,35);
    \draw[->] (45,65) -- (35,55);
    \draw[->] (55,65) -- (65,55);
    \node at (50,70) {6};
    \node at (30,50) {4};
    \node at (70,50) {8};
    \node at (20,30) {2};
    \node at (40,30) {5};
    \node at (60,30) {7};
    \node at (80,30) {9};
    \node at (10,10) {1};
    \node at (30,10) {3};
    \end{tikzpicture}
}

%%%%%%%%%%%%%%%%%%%%%%%%%%%%%%%%%%%%%%%%%%%%%%%%
%\subsection{Introduktion}
%%%%%%%%%%%%%%%%%%%%%%%%%%%%%%%%%%%%%%%%%%%%%%%%

\subsection{Træterminologi}
\begin{frame}[fragile]
\begin{footnotesize}

  \head{Træterminologi}
  \vspace{1ex}

  \begin{minipage}{0.6\textwidth}
    \begin{itemize}
    \item Et træ består af \textbf{knuder} forbundet med \textbf{ordnede kanter}.
    \item En knude har højst en indgående kant (forælder).
    \item En knude kan have 0 eller flere udgående kanter (børn).
    \item En knude uden børn kaldes et \textbf{blad}, og en knude uden forældre kaldes en \textbf{rod}.
    \item Normalt tegnes træer med forældre ovenover børn.
    \item Et \textbf{binært træ} har præcis en rod og hver knude har højst to børn.
      \end{itemize}
  \end{minipage}
  \begin{minipage}{0.35\textwidth}
    \mytree{0.06}
    \end{minipage}

\end{footnotesize}
\end{frame}

\subsection{Trægennemløb}

\begin{frame}[fragile]
\begin{footnotesize}

  \head{Gennemløb af træer.}
  \vspace{1ex}

  Et \emph{gennemløb} (eng. \emph{traversal}) af et træ er et besøg af alle
  knuderne i træet.

  \vspace{1ex}

\head{Forskellige slags gennemløb:}

\begin{itemize}
\item \textbf{Dybde-først gennemløb}: besøg alle knuderne i venstre
  undertræ af en knude før knuderne i højre undertræ.  Der
  er tre undertyper af dybde-først gennemløb:

  \begin{enumerate}
  \item \emph{Præordens gennemløb}: knude før børn
  \item \emph{Postordens gennemløb}: knude efter børn
  \item \emph{Inordens gennemløb}: knude mellem børn
  \end{enumerate}
\item \textbf{Bredde-først gennemløb}: besøg knuder i rækkefølge efter
  afstand til roden, og knuder med samme afstand fra venstre mod højre.
\end{itemize}

\end{footnotesize}
\end{frame}

\begin{frame}[fragile]
\begin{footnotesize}

  \head{Eksempler på trægennemløb}

  \vspace{2ex}

  \begin{minipage}{0.55\textwidth}
\begin{tabular}{ll}
  \textbf{Præordens gennemløb}: & 1 2 4 8 9 5 3 6 7\\
  (knude før børn) \\[2ex]
  \textbf{Postordens gennemløb}: & 8 9 4 5 2 6 7 3 1 \\
  (knude efter børn) \\[2ex]
  \textbf{Inordens gennemløb}: & 8 4 9 2 5 1 6 3 7 \\
  (knude mellem børn) \\[2ex]
  \textbf{Bredde-først gennemløb}: & 1 2 3 4 5 6 7 8 9
\end{tabular}
  \end{minipage}
  \begin{minipage}{0.4\textwidth}
    \mytree{0.07}
    \end{minipage}

\end{footnotesize}
\end{frame}

\renewcommand{\sp}{\vspace{1ex}}
\newcommand{\shead}[1]{\vspace{1ex}\head{#1}\vspace{1ex}}

\begin{frame}[fragile]
\begin{footnotesize}

  \head{Implementation af præordens gennemløb}
  \sp

  Vi arbejder med følgende træstruktur:
  \sp
\begin{lstlisting}[numbers=none,frame=none,mathescape]
type 'a t = L | T of 'a t * 'a * 'a t
let E e = T(L,e,L)
\end{lstlisting}

\shead{Den simple version --- bruger \texttt{@}}
\begin{lstlisting}[numbers=none,frame=none,mathescape]
let rec preorder (t: 'a t) : 'a list =
  match t with     // visit node before children
    | L -> []
    | T(l,e,r) -> [e] @ preorder l @ preorder r
\end{lstlisting}

\shead{Effektiv version --- uden brug af \texttt{@}}

\begin{lstlisting}[numbers=none,frame=none,mathescape]
let rec preorder_acc (acc:'a list) (t: 'a t) : 'a list =
  match t with     // node before children
    | L -> acc
    | T(l,e,r) -> e :: preorder_acc (preorder_acc acc r) l
\end{lstlisting}

\end{footnotesize}
\end{frame}

\begin{frame}[fragile]
\begin{footnotesize}

  \head{Implementation af postordens gennemløb}
  \sp

  Vi arbejder med følgende træstruktur:
  \sp
\begin{lstlisting}[numbers=none,frame=none,mathescape]
type 'a t = L | T of 'a t * 'a * 'a t
let E e = T(L,e,L)
\end{lstlisting}

\shead{Den simple version --- bruger \texttt{@}}
\begin{lstlisting}[numbers=none,frame=none,mathescape]
let rec postorder (t: 'a t) : 'a list =
  match t with     // visit node after children
    | L -> []
    | T(l,e,r) -> postorder l @ postorder r @ [e]
\end{lstlisting}

\shead{Effektiv version --- uden brug af \texttt{@}}

\begin{lstlisting}[numbers=none,frame=none,mathescape]
let rec postorder_acc (acc:'a list) (t: 'a t) : 'a list =
  match t with     // node after children
    | L -> acc
    | T(l,e,r) -> postorder_acc (postorder_acc (e::acc) r) l
\end{lstlisting}

\end{footnotesize}
\end{frame}

\begin{frame}[fragile]
\begin{footnotesize}

  \head{Implementation af inordens gennemløb}
  \sp

  Vi arbejder med følgende træstruktur:
  \sp
\begin{lstlisting}[numbers=none,frame=none,mathescape]
type 'a t = L | T of 'a t * 'a * 'a t
let E e = T(L,e,L)
\end{lstlisting}

\shead{Den simple version --- bruger \texttt{@}}
\begin{lstlisting}[numbers=none,frame=none,mathescape]
let rec inorder (t: 'a t) : 'a list =
  match t with     // visit node between children
    | L -> []
    | T(l,e,r) -> inorder l @ [e] @ inorder r
\end{lstlisting}

\shead{Effektiv version --- uden brug af \texttt{@}}

\begin{lstlisting}[numbers=none,frame=none,mathescape]
let rec inorder_acc (acc:'a list) (t: 'a t) : 'a list =
  match t with     // visit node between children
    | L -> acc
    | T(l,e,r) -> inorder_acc (e :: inorder_acc acc r) l
\end{lstlisting}

\end{footnotesize}
\end{frame}

\begin{frame}[fragile]
\begin{footnotesize}

\shead{Simpel bredde-først implementation --- bruger \texttt{@}}

  \begin{minipage}{0.55\textwidth}
\begin{lstlisting}[numbers=none,frame=none,mathescape]
let breathfirst t =
  let rec bF ts =
    match ts with
      | [] -> []
      | L :: ts -> bF ts
      | T (l,a,r) :: ts ->
         a :: bF (ts @ [l; r])
  bF [t]
\end{lstlisting}
  \end{minipage}
  \begin{minipage}{0.4\textwidth}
    \mytree{0.05}
    \end{minipage}

\shead{Bemærk}
\begin{itemize}
\item Hjælpefunktionen \lstinline{bF} laver et bredde-først gennemløb af en liste af træer.
\item Listen fungerer som en kø: Vi tager ud fra fronten og sætter ind bagest.
\end{itemize}

\shead{Spørgsmål}
\begin{itemize}
\item Nogle gode ideer til hvordan vi kan undgå brug af \texttt{@}?
\end{itemize}

\end{footnotesize}
\end{frame}

\begin{frame}[fragile]
\begin{footnotesize}

  \shead{Løsningen er naturligvis at benytte vores effektive kø-modul!}

\begin{lstlisting}[numbers=none,frame=none,mathescape]
module Queue // content of queue.fsi

type 'a queue                        // FIFO
val empty : unit -> 'a queue
val insert  : 'a queue -> 'a -> 'a queue
val remove  : 'a queue -> ('a * 'a queue) option
\end{lstlisting}

\shead{Den nye effektive implementation:}

\begin{lstlisting}[numbers=none,frame=none,mathescape]
let breathfirst_good (t:'a t) : 'a list =
  let rec bF (q:'a t Queue.queue) : 'a list =
    match Queue.remove q with
      | None -> []
      | Some(L,q) -> bF q
      | Some(T (l,a,r), q) ->
          a :: bF (Queue.insert (Queue.insert q l) r)
  bF (Queue.insert (Queue.empty()) t)
\end{lstlisting}

\end{footnotesize}
\end{frame}

%% \begin{frame}[fragile]
%% \begin{footnotesize}

%% \shead{Bredde-først gennemløb af generelle træer}

%% Gennemløb kan generaliseres (pånær inorder-gennemløb) til ikke-binære træer.

%% \shead{Eksempel på bredde-først gennemløb af et generelt træ}

%% \begin{lstlisting}[numbers=none,frame=none,mathescape]
%% type 'a tg = Lg | Tg of 'a * 'a tg list
%% let breathfirst_gen (t:'a tg) : 'a list =
%%   let rec bF (q:'a tg list Queue.queue) : 'a list =
%%     match Queue.remove q with
%%       | None -> []
%%       | Some(gts,q) -> bFs q gts
%%   and bFs (q:'a tg list Queue.queue) tgs : 'a list =
%%     match tgs with
%%       | [] -> bF q
%%       | Lg::rest -> bFs q rest
%%       | Tg (a,tgs)::rest -> a :: bFs (Queue.insert q tgs) rest
%%   bF (Queue.insert (Queue.empty()) [t])
%% \end{lstlisting}

%% \head{Bemærk}
%% \begin{itemize}
%% \item Hjælpefunktionen \lstinline{ins} benyttes til at indsætte en liste af elementer i køen.
%% \end{itemize}

%% \end{footnotesize}
%% \end{frame}

\subsection*{Konklusion}
\begin{frame}[fragile]
  \headsp{Konklusion}

  \vspace{3mm}
  \tableofcontents
\end{frame}

\end{document}
