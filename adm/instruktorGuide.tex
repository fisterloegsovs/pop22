\documentclass[a4paper]{article}

\usepackage{cmap}
\usepackage[utf8]{inputenc}
\usepackage[danish]{babel}
\usepackage [T1]{fontenc}
\usepackage[margin=2.5cm]{geometry}
\usepackage{hyperref}

\title{Instruktorguide\\Programmering og problemløsning}
\author{Jon Sporring}
\date{\today}

\begin{document}
\maketitle

\section{Forord}
Formålet med denne tekst er at beskrive de væsentlige elementer i
opgaven som instruktor på Programmering og Problemløsning.

\section{Ramme}
Kurset gives som en blanding af forelæsninger og øvelser. Der er en
kursusansvarlig (pt.\ Jon Sporring), undervisere (inkl.\ den
kursusansvarlige) og en række instruktorer. På engelsk kaldes disse
roller nogle gange for Course Responsible, Teachers and Teaching
Assistants. Til de nyoptagne studerende på kurset er der også
tilknyttet et antal mentorer.

Kursets formål, omfang, indhold og eksamensbeskrivelsen kan læses på
\url{https://kurser.ku.dk/}. På \url{https://absalon.ku.dk/} findes
kursets Content-Management-System (CMS), som fungerer som vores
daglige læringsplatform. Vi gør også brug af \url{https://slack.com}
til intern og kommunikation og \url{https://github.com/diku-dk} til
deling af filer. På alle systemer vil der være sider/sites som
oprettes til indeværende årgang. Den kursusansvarlige har adgang til
tidligere årganges information.

Kurset består af et antal forelæsninger givet in-real-life eller
uddelt som videoer, øvelser som enten er face-to-face eller online, og
det er opdelt i et antal perioder, som for det meste svarer til 1 uges
arbejde. Til hver periode høre en arbejdsseddel, som beskriver, hvad
læringsformålene er for den kommende periode, trænings- og
afleveringsopgaver, og anden information som er vigtigt for
perioden. Kursusansvarlig, undervisere og instruktorer mødes ugentligt
til instruktormødet for at koordinere aktivitererne omkring kurset.

Kurset lægger vægt på at give de studerende erfaring med at løse
(simple) problemer ved at skrive et program, og derfor er der
ca. dobbelt så megen tid afsat til øvelser i forhold til
forelæsninger.

\section{Ansvarsfordeling}
Det overordnede ansvar er fordelt som følger:
\begin{itemize}
\item Den kursusansvarlige har det overordnede ansvar og står for den
  generelle koordinering af kurset
\item Underviserne står for at afvikle forelæsningerne, forberedelse
  af arbejdssedlerne, og koordineringen af kursets aktiviteter i de
  perioder, hvor den enkelte underviser underviser
\item Hver instruktor er ansvarlig for et hold af studerende på ca. 25
  studerende, og hvert hold af nyoptagne studerende har også
  tilknyttet en mentor. Instruktorernes opgave er at fokusere på de
  studerendes læring og faglige aktiviteter, mens mentorernes opgave
  er at organisere sociale aktiviter. Hver instruktor skal varetage en
  række opgaver, hvor de væsentligeste er:
  \begin{itemize}
  \item afholdelse af øvelserne for deres hold,
  \item rettelse af afleveringsopgaverne,
  \item deltagelse i ugentlige koordineringsmøde
  \item samarbejde med den tilknyttede mentor om de studerendes
    fremgang og velvære
  \item korrekturlæse arbejdssedler og prøveprogrammere
    afleveringsopgaver ca\. 2 gange på kurset
  \item deltagelse i studiecaf{\'e} ca.\ 3 gange på kurset
  \item holde øje med og interagere med de studerende på
    Absalonforummet efter tur ca.\ 3 gange på kurset
  \item deltagelse ca.\ 1 gang i løbet af kurset som moderator til spørgetimen.
  \item i mindre omfang afholdelse af ekstra øvelser og enkeltvejlendning af
    studerende efter behov
  \end{itemize}
\end{itemize}
Til dette er der afsat timer til koordinator og undervisere, som
aftales med deres respektive sektionsleder, og 260 timer (2 blokke af
130 i 2020 tal) til instruktorerne. Hvis der er brug for yderligere
resurser, skal dette aftales med kursuskoordinator, som derefter skal
forhandle med instituttet.

\section{Faste procedurer}
På grund af Corona virussen må vi kun være halvt så mange personer i
lokalerne som normalt, og da vi mere end 450 studerende fordelt på 18
hold betyder det, at næsten alle forelæsninger og øvelser må afholdes
som en blanding af fysisk fremmøde og online undervisning.

For at sikre effektive arbejdsgange og ensartet høj kvalitet vil vi i væsentligt omfang følge en række faste procedurer:
\begin{description}
\item[Ugens organisering i Coronatider] Kurset følger Skema A (tirsdag
  formiddag, torsdag formiddag og eftermiddag. Arbejdet for en typisk
  uge vil være
  \begin{description}
  \item[Inden tirsdag] forventes
    det at de studerende har set ugens udleverede vidoer.
  \item[Tirsdag kl.\ 9.15-12.00] afholdes øvleser, hvor de studerende
    skal fokusere på simple øvelser med fokus på at lære ugens
    grundlæggende syntaks og programmeringselementer. Øvelsesgangen
    afsluttes med at hvert hold finder {\'e}t spørgsmål og en stiller
    til spørgetimen.
  \item[Torsdag kl.\ 9.15-11.00] afholdes en forelæsning og en
    spørgetime af underviserne. {\'E}n instruktor deltager og hjælper
    underviseren med at sortere og udvælge spørgsmål under
    spørgetimen.
  \item[Torsdag kl.\ 11.15-12.00 og kl.\ 13.15-16.00] afholdes
    øvelser, hvor der primært arbejdes med afleveringsopgaven.
  \item[Fredag kl.\ 14.00-16.00] afholdes der studiecaf{\'e}, hvor
    de studerende kan arbejde med alle emner, men hvor der vil være
    2-3 PoP-instruktorer tilstede for at hjælpe med relevante
    problemer.
  \item[Lørdag kl.\ 22.00] afleverer de studerende deres afleveringsopgave.
  \end{description}

\item[En øvelsesgang] De studerende er fordelt efter erfaring og der
  vil være behov for en hvis fleksibilitet i
  undervisningstilgangen. Som minimum skal hver øvelsesgang afholdes
  efter følgende skabelon:
  \begin{enumerate}
  \item Øvelsesgangen startes f.eks.\ ved at instruktoren siger
    velkommen eller på anden vis påkalder sig opmærksomhed.
  \item Tirsdag morgen (hvor relevant) læses arbejdssedlen sammen med
    de studerende, og indholdet diskuteres for at afklare de
    væsentligeste spørgsmål og udfordringer og for at sikre at alle er
    klar til at gå igang med periodens opgaver.
  \item Tirsdag og torsdag sættes de studerende til at arbejde på
    opgaverne. De må gerne tale sammen og arbejde i grupper, men for
    individuelle opgaver er det vigtigt at der tales om, at man gerne
    må diskutere generelle programmeringselementer, men ikke om den
    specifikke afleveringsopgaver.
  \item Tirsdag og torsdag Instruktor opretter en spørgeliste, hvor de studerende skriver
    sig på. Når instruktoren har tid, kontakter instruktoren de
    studerende efter tur. Der ligges fokus på at hjælpe de studerende
    med selv at løse deres problemer. F.eks.\ ved at
    \begin{enumerate}
    \item bede dem beskrive problemet, og hvad de har forsøg at gøre,
    \item hjælpe dem med at finde relevant materiale i pensum,
    \item stille udforskende spørgsmål, 
    \item hjælpe dem med at diskutere problemet i deres gruppe eller
      dem de arbejder godt sammen med
    \item lave hypoteser og afprøve dem programmeringsmæssigt
    \item osv.
    \end{enumerate}
  \item Tirsdag inden afslutning findes et spørgsmål til spørgetimen i
    fællesskab samt en person til at stille spørgsmålet.
  \item Tirsdag og torsdag inden afslutning afrundes øvelsesgangen og
    der sikres, at de studerende har værktøj til og mod på at arbejde
    videre på egen hånd og særligt at de er godt klædt på til at løse
    afleveringsopgaven når relevant.
  \end{enumerate}
  
\item[Et Opgaveretning] [Afventer diskussion af rubrikker med Ken]

\item[Et Instruktormøde] Formålet med et instruktormøde er dele
  erfaring fra den forløbne uge og forberede det kommende
  arbejde. En typisk dagsorden er at,
  \begin{enumerate}
  \item Gennemgå undervisningsugen der gik inkl.\ udfordringer med rettearbejdet
  \item Gennemgå den kommende arbejdsseddel og periodens pædagogiske
    fokus
  \item Diskutere undervisningsugen der kommer
  \item Gennemgå studerende, som er udfordret med primær fokus på at
    sikre, at de studerende får afleveret og bestået deres opgaver
  \item Fordel/bekræft vagter til studiecaf{\'e}n og Absalon
  \item Aflev{\'e}r korrekturlæsning på den efterfølgende
    arbejdsseddel og fordel korrekturlæser til arbejdssedlen 2 uger
    frem i tiden.
  \end{enumerate}
\end{description}

\end{document}
