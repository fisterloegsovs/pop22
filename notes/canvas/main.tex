\documentclass{article}
\usepackage[utf8]{inputenc}

\title{imgutil}
\author{Martin Elsman og Jon Sporring}
\date{August 2022}

\begin{document}

\maketitle

\section{Introduction}
I det følgende skal vi benytte os af biblioteket \texttt{ImgUtil}, som
beskrevet i forelæsningerne. Biblioteket \texttt{ImgUtil} gør det
muligt at tegne punkter og linier på et canvas, at eksportere et
canvas til en billedfil (en PNG-fil), samt at vise et canvas på
skærmen i en simpel F\# applikation. Biblioteket (nærmere bestemt F\#
modulet \texttt{ImgUtil}) er gjort tilgængeligt via en F\# DLL kaldet
\texttt{img\_util.dll}. Koden for biblioteket og dokumentation for
hvordan DLL'en bygges og benyttes er tilgængelig via github på
\url{https://github.com/diku-dk/img-util-fs}.

\end{document}
