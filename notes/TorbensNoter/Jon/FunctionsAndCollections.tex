\documentclass[a4paper]{article}

\usepackage{cmap}
\usepackage[utf8x]{inputenc}
\usepackage{latexsym}
\usepackage[english]{babel}
\usepackage{graphicx}
\usepackage{hyperref}
\usepackage[all]{hypcap}
\usepackage{enumerate}
\usepackage[margin=2.5cm]{geometry}

\begin{document}
\title{Functions as Values and Collections (rev.\ 1.2)}

\author{Torben Mogensen}
\date{\today}

\maketitle

\noindent
These notes are supplementary to chapters 2 and 5 in Hansen \&
Rischel.

\section{Functional Values and Types}

In F\#, a function is a value that can be used in the same way as
numbers, strings, booleans and so on: You can pass functions as
arguments to other functions, return functions as results, build lists
of functions, and so on.  Just about the only things you can not do
with a function that you can do with numbers, strings, and other
values are:

\begin{itemize}
\item You can not compare functions.  If you define a function
  \texttt{f} and afterwards write \texttt{f=f}, you get an error
  saying that equality is not defined on function types.  Similarly
  for inequality (\texttt{<}, \texttt{>}, and so on).
\item You can not print a function to see its definition.  If you
  write \texttt{printfn "\%A" f}, where \texttt{f} is a function, F\#
  will write something like \texttt{<fun:it@2>}, and it will even be
  different strings if you print the function several times.  For
  example, \texttt{printfn "\%A \%A" f f} can print
  \texttt{<fun:it@8-6> <fun:it@8-7>} or something similar.
\end{itemize}

\noindent
To illustrate functions as values, let us define

\begin{verbatim}
let f x = x + 1
let g y = y * 2
let h z = z * z + 2 * z + 3
let fl = [f; g; h]
\end{verbatim}

\noindent
\texttt{fsharpi} will show the types of these definitions:

\begin{verbatim}
val f : x:int -> int
val g : y:int -> int
val h : z:int -> int
val fl : (int -> int) list = [<fun:fl@5>; <fun:fl@5-1>; <fun:fl@5-2>]
\end{verbatim}

\noindent
and for \texttt{fl} it will even print its value, though all you can
see of the value is that it is a list of three functions.  You can't
even see that if they are the same function three times or three
different functions.

The types \texttt{x:int -> int}, \texttt{y:int -> int}, \texttt{z:int
  -> int}, and \texttt{int -> int} are the same.  The name of the
parameter shown in the types for \texttt{f}, \texttt{g}, and
\texttt{h} is not actually a part of the type, which is just
\texttt{int -> int}, as shown in the list type.

We can pick an element from the list and apply it, for example,
\texttt{fl.[2]~5}, which will return the value 38.

When you define a function of two arguments, for example,

\begin{verbatim}
let plus x y = x + y
\end{verbatim}

\noindent
it has a type with two arrows, in this case \texttt{int -> int -> int}
(with some naming or arguments that we ignore).  This type is
equivalent to \texttt{int -> (int -> int)}, which says that when you
give \texttt{plus} one integer argument, it will return a function of
type \texttt{int -> int}.  This can later be applied (once or several
times) to another integer argument to give an integer result, for
example,

\begin{verbatim}
let plus5 = plus 5
(plus5 3) * (plus5 7)
\end{verbatim}

\noindent
which will give the value 96.

A function can also accept functions as arguments, so we can define

\begin{verbatim}
let applyTwice f x = f (f x)
\end{verbatim}

\noindent
which defines a function \texttt{applyTwice} that takes a function
\texttt{f} and applies it twice to an argument \texttt{x}.  For
example, \texttt{applyTwice plus5 3} will return 13.

\section{Anonymous Functions}

When you make a definition

\begin{verbatim}
let plus x y = x + y
\end{verbatim}

\noindent
it is actually an abbreviation for


\begin{verbatim}
let plus = fun x y -> x + y
\end{verbatim}

\noindent
which you can read as ``let \texttt{plus} be a function that takes two
arguments, \texttt{x} and  \texttt{y}, and returns \texttt{x+y}''.
This also works with recursive functions, so


\begin{verbatim}
let rec gcd a b = if b = 0 then a else gcd b (a % b)
\end{verbatim}

\noindent
is an abbreviation for

\begin{verbatim}
let rec gcd = fun a b -> if b = 0 then a else gcd b (a % b)
\end{verbatim}

\noindent
You do not need to give names to functions defined using \texttt{fun},
so you can, for example, write


\begin{verbatim}
[(fun x -> x + 1); (fun y -> y * y); (fun z -> 2 * z + 3)]
\end{verbatim}

\noindent
to define a list of three functions.  Note that the parentheses are
needed to avoid F\#{} treating the semicolons as part of the function
instead of separating these.  This is because seimicolon has higher
operator precedence than \texttt{->}.  If in doubt, always enclose
anonymous function definitions in parentheses.

Anonymous function definitions are useful if you want to pass a small
function as argument to another function.  For example, with
\texttt{applyTwice} as defined above, we can make the call
\texttt{applyTwice~(fun~x~->~x*x)~7} to get 2401.

You can also make anonymous pattern-matching functions.  When
you write a definition like


\begin{verbatim}
let isEmpty = function
  | [] -> true
  | n :: ns -> false
\end{verbatim}

\noindent
this first defines an anonymous function (using the keyword
\texttt{function} followed by a list of rules) and then binds the name
\texttt{isEmpty} to that function.  Note that a function defined with
\texttt{function} can use pattern matching with multiple rules, but it
can take only one argument.  In contrast, a function defined with
\texttt{fun} can take multiple arguments, but only use one rule.

An infix operator written in parentheses is a function of two
arguments.  For example: \texttt{(+) : int -> int -> int}, so you can
make a function like \texttt{plus5} above by simply writing
\texttt{((+) 5)}, which has type \texttt{int -> int}.  Note that
\texttt{..} can not be made into a function this way.  You have to
write \texttt{(fun x y -> x :: y)} or similar.

\section{Selected Functions from the \texttt{List} Library}

In addition to simple functions such as

\vspace{1ex}

\begin{tabular}{l}
\texttt{List.length : 'a list -> int} \\
\texttt{List.head : 'a list -> 'a} \\
\texttt{List.isEmpty : 'a list -> bool}
\end{tabular}

\vspace{1ex}

\noindent
the \texttt{List} library also defines a number
of useful functions that take functional arguments in addition to list
arguments.  By using these library functions, you can avoid having to
define recursive functions that use pattern matching.  Here are some
examples:

\begin{itemize}
\item \texttt{List.map : ('a -> 'b) -> 'a list -> 'b list}

takes a function $f$ and a list $xs$ and constructs a new list by
applying $f$ to all elements of $xs$.  For example,

\begin{verbatim}
List.map (fun x -> x*x) [1..5]
\end{verbatim}

\noindent
will produce the result \texttt{[1; 4; 9; 16; 25]}.

\item \texttt{List.map2 : ('a -> 'b -> 'c) -> 'a list -> 'b list- > 'c list}

takes a function $f$ and two lists $xs$ and $ys$ and constructs a new
list by applying $f$ to all pairs constructed by taking corresponding
elements of $xs$ and $ys$.  For example,

\begin{verbatim}
List.map2 (fun x y -> x * x + y) [1..5] [3..7]
\end{verbatim}

\noindent
will produce the result \texttt{[1*1+3; 2*2+4; 3*3+5; 4*4+6; 5*5+7]}
$\leadsto$ \texttt{[4; 8; 14; 24; 32]}.

If the two lists are not of the same length, an error message will be
given.


\item \texttt{List.exists : ('a -> bool) -> 'a list -> bool}

takes a predicate $p$ and a list $xs$ and checks if there is an
element in $xs$ for which $p$ returns \texttt{true}.  For example, if
you want to know if the value 7 is found in a list \texttt{xs}, you
can write 

\begin{verbatim}
List.exists (fun x -> x = 7) xs
\end{verbatim}


\item \texttt{List.forall : ('a -> bool) -> 'a list -> bool}

is similar to \texttt{List.exists}, but returns \texttt{true} only if
the predicate returns \texttt{true} for \emph{all} elements of the
list.

For example,


\begin{verbatim}
List.forall (fun x -> x = 7) xs
\end{verbatim}

returns \texttt{true} if all elements in \texttt{xs} are equal to 7.
If \texttt{xs} is empty, this is trivially true.

\item \texttt{List.find : ('a -> bool) -> 'a list -> 'a}

takes a predicate $p$ and a list $xs$ and returns the first
element in $xs$ for which $p$ returns \texttt{true}. If there are
none, an error is reported.

For example,

\begin{verbatim}
List.find (fun (x, y) -> y = "Emil")
          [(1, "Joachim"); (2, "Sune"); (3, "Emil"); (4, "Matthias")]
\end{verbatim}

will return \texttt{(3, "Emil")]}.

\item \texttt{List.tryFind : ('a -> bool) -> 'a list -> 'a option}

takes a predicate $p$ and a list $xs$ and, if there is an element in
$xs$ for wh which $p$ returns \texttt{true} returns \texttt{Some $x$},
where $x$ is the first element in $xs$ for which $p$ returns
\texttt{true}. If there are no such elements, it returns
\texttt{None}.  Option-typen er beskrevet i Hansen \& Rischel afsnit
3.11.

For example,

\begin{verbatim}
List.tryFind (fun (x, y) -> y = "Emil")
          [(1, "Joachim"); (2, "Sune"); (3, "Emil"); (4, "Matthias")]
\end{verbatim}

will return \texttt{Some (3, "Emil")]}, and 

\begin{verbatim}
List.tryFind (fun (x, y) -> y = "Hans")
          [(1, "Joachim"); (2, "Sune"); (3, "Emil"); (4, "Matthias")]
\end{verbatim}

will return \texttt{None}.

\item \texttt{List.filter : ('a -> bool) -> 'a list -> 'a list}

takes a predicate $p$ and a list $xs$ and returns all
elements in $xs$ for which $p$ returns \texttt{true}.

For example,

\begin{verbatim}
List.filter (fun (x, y) -> y < "Klaus")
          [(1, "Joachim"); (2, "Sune"); (3, "Emil"); (4, "Matthias")]
\end{verbatim}

will return \texttt{[(1, "Joachim"); (3, "Emil")]}.


\item \texttt{List.init : int -> (int -> 'a) -> 'a list}

is used to construct new lists.  The call \texttt{List.init\,$n$\,$f$}
will construct the list \texttt{[$f\,0$; $f\,1$;$\ldots$;$f\,(n{-}1)$]}.

Generally, \texttt{(List.init\,$n$\,$f$).[$i$]} will be $f\,i$ for
$0\leq i < n$.

For example,

\begin{verbatim}
List.Init 5 (fun x -> x * x)
\end{verbatim}

will return \texttt{[0; 1; 4; 9; 16]}.

\item \texttt{List.foldBack : ('a -> 'b -> 'b) -> 'a list -> 'b -> 'b}

A call \texttt{List.foldBack $f$ [$a_1$;\ldots; $a_n$] $b$} does the
following:

\begin{enumerate}[1.]
\item Views the list \texttt{[$a_1$; $a_2$;\ldots; $a_n$]} as the
  equivalent


  \texttt{$a_1$\,::\,($a_2$\,::\,\,\ldots  ($a_n$\,::\,\,[])  \ldots)}.

\item Replaces \texttt{$a_i$\,::} with $f~a_i$ and \texttt{[]} with
  $b$, yielding

\texttt{$f~a_1$ ($f~a_2$ \ldots  ($f~a_n~b$)  \ldots)}.


\item Calculates the result.

\end{enumerate}

For example,

\begin{verbatim}
let f x y = x * y
List.foldBack f [2; 3; 4] 1
\end{verbatim}

first sees \texttt{[2; 3; 4]} as

\texttt{2\,::\,(3\,::\,(4\,::\,[]))},

then rewrites this to 

\texttt{f 2 (f 3 (f 4 1))},

which is evaluates to \texttt{2 * 3 * 4 * 1} $\leadsto$ 24.

It is easier to understand the behavior if $f$ is an infix operator.
For example, 


\begin{verbatim}
List.foldBack (*) [2; 3; 4] 1
\end{verbatim}

sees the list as

\texttt{2\,::\,(3\,::\,(4\,::\,[]))},

then replaces \texttt{::} by \texttt{*} and \texttt{[]} by \texttt{1},
giving

\texttt{2 * (3 * (4 * 1))},

which evaluates to 24, as above.

So we can define the factorial function and the length function as

\begin{verbatim}
let fac n = List.foldBack (*) [2..n] 1
let length xs = List.foldBack (fun x y -> y + 1) xs 0
\end{verbatim}

The factorial function multiplies the elements in the list
\texttt{[2..n]}, using \texttt{1} as a starting point.  The length
function adds 1 for every element in the list, ignoring the actual
elements (by not using \texttt{x}).

We can define \texttt{foldBack} as a recursive function:


\begin{verbatim}
let rec foldback f xs b =
  match xs with
  | [] -> b
  | x :: xs -> f x (foldBack f xs b)
\end{verbatim}

It is easy to see that this precisely replaces \texttt{[]} by
\texttt{b} and every \texttt{x ::} by \texttt{f x}.


\item \texttt{List.fold : ('b -> 'a -> 'b) -> 'b -> 'a list -> 'b}

This function is similar to \texttt{List.foldBack}, but groups the
elements differently:

\texttt{List.fold $f$ $b$ [$a_1$; $a_2$;\ldots; $a_n$]} reduces to
$(f\ldots(f\,(f\,b~a_1)\,a_2)\ldots a_n)$.

If $f$ is an infix operator, this is easier to see.  For example,
\texttt{List.fold (+) 0 [1; 2; 3]} reduces to \texttt{(((0 + 1) + 2) +
  3)} $\leadsto$ 6.

We can define \texttt{fold} recursively using an accumulating
parameter \texttt{b}:

\begin{verbatim}
let rec fold f b xs =
  match xs with
  | [] -> b
  | x :: xs -> fold f (f b x) xs
\end{verbatim}

Note that, if $f$ is an associative and commutative infix operator
(like \texttt{+} or \texttt{*}), \texttt{List.fold $f$ $b$ $xs$} =
\texttt{List.foldBack $f$ $xs$ $b$}.  For example, \texttt{(((0 + 1) +
  2) + 3)} = \texttt{1 + (2 + (3 + 0))}.  So we can also define the
factorial function function by

\begin{verbatim}
let fac n = List.fold (*) 1 [2..n]
\end{verbatim}

\end{itemize}

\section{Arrays}

Arrays are similar to lists: Sequences of elements of the same type,
where the length of the sequence is unspecified by the type.  An array
with elements of type $a$ is written as \texttt{$a$ []}, for example
\texttt{int []} for an array of integers.

Some similarities between lists and arrays can be seen in the table
below:

\vspace{1ex}

\texttt{
\begin{tabular}{l@{\quad}l}
\textrm{List expressions} & \textrm{Array expressions} \\\hline
{}[1; 2; 3] : int list & [|1; 2; 3|] : int []\\
{}[] : 'a list & [||] : 'a []\\
List.head [1; 2; 3] $\leadsto$ 1 & Array.head [|1; 2; 3|] $\leadsto$ 1\\
List.tail [1; 2; 3] $\leadsto$ [2; 3] & Array.tail [|1; 2; 3|] $\leadsto$ [|2; 3|]\\
{}[1; 2; 3].[2] $\leadsto$ 3 & [|1; 2; 3|].[2] $\leadsto$ 3\\
{}[1; 2; 3].[1..2] $\leadsto$ [2; 3] & [|1; 2; 3|].[1..2] $\leadsto$ [|2; 3|]\\
List.length [1; 2; 3] $\leadsto$ 3& Array.length [|1; 2; 3|] $\leadsto$ 3\\
List.map f [1; 2] $\leadsto$  [f 1; f 2] &
 Array.map f [|1; 2|] $\leadsto$ [|f 1; f 2|]\\
List.filter odd [1; 2; 3] $\leadsto$  [1; 3] &
 Array.filter odd [|1; 2; 3|] $\leadsto$ [|1; 3|]\\
{}[1; 2] @ [3] $\leadsto$ [1; 2; 3] &
 Array.append [|1; 2|] [|3|] $\leadsto$ [|1; 2; 3|]\\
\\
\textrm{List patterns} & \textrm{Array patterns} \\\hline
{}[x; y; z] & [|x; y; z|]\\
{}[]  & [||]\\
\end{tabular}}

\vspace{1ex}

\noindent
Generally, almost all the library functions from the \texttt{List}
library (\texttt{init}, \texttt{exists}, \texttt{find},
\texttt{foldBack}, \ldots) are also found in the \texttt{Array}
library with equivalent behaviour.  The main differences between lists
and arrays are:

\begin{itemize}
\item The \texttt{::} operator and pattern is not defined for arrays.
\item Array elements are \emph{mutable}, so you can write, for example

\begin{verbatim}
let aa = [|1..5|]
aa.[2] <- 17
printfn "%A" aa
\end{verbatim}

\noindent
which will print \texttt{[|1; 2; 17; 4; 5|]}.

\item Indexing into an array $a$ to get an element using $a.[i]$ is faster
  than indexing into a list $l$ using $l.[i]$, unless $i<2$.

\item Appending two arrays has a cost proportional to their combined
  size, where appending two arrays has a cost proportional only to the
  first argument to \texttt{@}.

\item Using \texttt{::} to add an element to a list or split a list
  into head and tail, or using \texttt{List.head} and
  \texttt{List.tail} to split a list is \emph{very} cheap compared to
  using \texttt{Array.append} to add an element to a list and
  \texttt{Array.head} and \texttt{Array.tail} to split an array into
  head and tail.

\item There is a function \texttt{Array.create : int -> 'a -> 'a []}
  such that \texttt{Array.create $n$ $v$} creates an array of $n$
  elements all equal to $v$.

\end{itemize}

\noindent
You can say that lists are optimised for recursive functions using
\texttt{[]} and \texttt{::} patterns or \texttt{List.head} and
\texttt{List.tail}, while arrays are optimized for indexing using
\texttt{$a$.[$i$]}.  Library functions like \texttt{map},
\texttt{filter}, \texttt{fold} and \texttt{init} have roughly similar
cost for arrays and lists.

It is perfectly possible (and quite efficient) to work with arrays
without using assignment (the \texttt{<-} operator) by using
\texttt{Array.init}, \texttt{Array.fold}, and so on, but it is also
quite common to use loops and assignment when programming with arrays.
For example, we can make a function that prints all primes up to a
number $n$ this way, using the Sieve of Erastothenes
(\url{https://en.wikipedia.org/wiki/Sieve_of_Eratosthenes}):

\vspace{1ex}
\texttt{
  \begin{tabular}{ll}
\footnotesize{0:} &let primesUpto n =\\
\footnotesize{1:} &~~let prime = Array.create (n+1) true\\
\footnotesize{2:} &~~for p in 2 .. int (sqrt (float n)) do\\
\footnotesize{3:} &~~~~if prime.[p] then\\
\footnotesize{4:} &~~~~~~for i in (p*p) .. p .. n do\\
\footnotesize{5:} &~~~~~~~~prime.[i] <- false\\
\footnotesize{6:} &~~for i in 2 .. n do\\
\footnotesize{7:} &~~~~if prime.[i] then printf " \%d" i\\
  \end{tabular}
}

\noindent
\begin{description}
\item[Line 1] sets up an array with indices from 0 to $n$.  We will
  not use indices 0 or 1.  The array is initialised to the value
  \texttt{true} for all indices, indicating that, so far, we assume
  all numbers greater than 1 to be prime.
\item[Line 2] We loop for potential primes $p$ from 2 up to the square
  root of $n$.  We can stop there, because, by then, we have eliminated
  all prime factors up to $\sqrt{n}$, so the smallest possible
  non-prime we have not eliminated must be $(\sqrt{n}+1)^2$, which is
  greater than $n$.
\item[Line 3--5] If we have not already eliminated $p$, it is a prime.
  If so, we eliminate (by setting the array element to \texttt{false})
  all multiplas of $p$ up to $n$.  We can start from $p^2$ because we
  have already eliminated factors smaller than $p$.
\item[Line 6--7] We print all numbers that are not eliminated.  These
  are known to be primes.
\end{description}

\section{Two-dimensional Arrays}

Two-dimensional arrays are just that: Rows and columns of elements.
The type \texttt{$a$ [,]} denotes a two-dimensional array with
elements of type $a$.  The corresponding library is called
\texttt{Array2D} and has equivalents to many of the functions found in
the \texttt{Array} library, but slightly different since we work in
two dimensions.

For example, \texttt{Array2D.init 3 5 (fun x y -> 10*x+y)} creates the
two-dimensional array

\begin{verbatim}
[[0; 1; 2; 3; 4]
 [10; 11; 12; 13; 14]
 [20; 21; 22; 23; 24]]
\end{verbatim}

\noindent
Note that the notation looks like a list of lists, and typing in the
above in \texttt{fsharpi} will, indeed, produce something of type
\texttt{int~list~list}.  There is no notation for entering a constant
two-dimensional array, but you can use the function \texttt{array2D}
to convert a list of lists into a two-dimensional array.

Writing

\begin{verbatim}
[|[|0; 1; 2; 3; 4|]
  [|10; 11; 12; 13; 14|]
  [|20; 21; 22; 23; 24|]|]
\end{verbatim}

\noindent
produces an array of arrays, i.e., something of type
\texttt{int~[]~[]}.

The differences between a two-dimensional array and an array of arrays
are subtle:

\begin{itemize}
\item When indexing into a two-dimensional array $a2$, you use the notation
  \texttt{$a2$.[$i$,$j$]}, where you use \texttt{$aa$.[$i$][$j$]} to
  index into an array of arrays $aa$.
\item 
\item In a two-dimensional array, all the rows have the same length,
  where an array of arrays can have rows of different length.
\end{itemize}

\noindent
The following functions from \texttt{Array2D} are of interest:

\begin{itemize}

\item \texttt{array2D -> int list list -> -> 'a [,]} converts a list
  of lists to a two-dimensional array, provided the inner lists are of
  equal length.  The type is actually a bit more general, but it works
  at this type.

\item \texttt{Array2D.init : int -> int -> (int -> int -> 'a) -> 'a  [,]}

We have already seen this in the example above.  The call
\texttt{Array2D.init $i$ $j$ $f$} creates a two-dimensional array with
$i$ rows and $j$ columns where the element at position $(i,j)$ i
sgiven by $f(i,j)$.

\item \texttt{Array2D.length1 : 'a  [,] -> int}

Returns the number of rows in a two-dimensional array.

\item \texttt{Array2D.length2 : 'a  [,] -> int}

Returns the number of columns in a two-dimensional array.

\item \texttt{Array2D.map : ('a -> 'b) -> 'a [,] -> 'b [,]}

Works like \texttt{List.map} and \texttt{Array.map}, but on
two-dimensional arrays.

\item \texttt{$a$.[$i$,*]} finds the $i$'th row of $a$ (as a
  one-dimensional array).

\item \texttt{$a$.[*,$j$]} finds the $j$'th column of $a$ (as a
  one-dimensional array).


\end{itemize}

\section{Further reading}

The F\# language reference at Microsoft has a page about one- and
two-dimensional arrays that also explains how to take whole rows or
columns out of a two-dimensional arrays:

\url{https://docs.microsoft.com/en-us/dotnet/articles/fsharp/language-reference/arrays}

\end{document}

