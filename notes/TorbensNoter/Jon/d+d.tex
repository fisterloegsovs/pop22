\documentclass[a4paper]{article}

\usepackage[utf8x]{inputenc}
\usepackage{latexsym}
\usepackage[danish]{babel}
\usepackage{graphicx,color}
\usepackage{hyperref}
\usepackage[all]{hypcap}
\usepackage{enumerate}
\usepackage[margin=2.5cm]{geometry}

\begin{document}
\title{Suggestion for exercise based on D\&D}

\author{Torben Mogensen}
\date{\today}

\maketitle

\begin{abstract}
This exercise suggestion is about modelling a very simplified variant
of the role-playing game Dungeons \& Dragons\textsuperscript{\sf TM}.
\end{abstract}

\section{Characters}

A \emph{character} describes an adventurer.  A character is defined by
the following attributes:

\begin{itemize}
\item \texttt{body : int}, a read-only attribute that is a value
  between 3 and 18.
\item \texttt{mind : int}, a read-only attribute that is a value
  between 3 and 18.
\item \texttt{level : int}, a read-write attribute that is a non-negative
  integer.
\item \texttt{maxHitpoints : int}, a read-write attribute that is a
  non-negative integer.
\item \texttt{currentHitpoints : int}, a read-write attribute that is a
  non-negative integer less than or equal to \texttt{maxHitpoints}.
\end{itemize}

\noindent
Define a class \texttt{Character} with these attributes.  It should
include a constructor that takes the values of the first two
attributes.  The remaining three attributes should be set to 0.

The class should implement the following methods:

\begin{itemize}
\item \texttt{levelUp : unit} that increases the level by one and
  increases the maximum hit points by the value of the \texttt{body}
  attribute.  It also sets the current hit points to equal the maximum
  hit points.
\item \texttt{takeDamage : int -> unit}  that reduces the current hit
  points by the amount given, which must be a non-negative integer.
\item \texttt{healDamage : int -> unit} that increases the current hit
  points by the amount given, which must be a non-negative integer.
\item \texttt{isAlive : unit -> bool} that checks whether the current
  hit points is non-negative.
\end{itemize}

\section{Character Classes}

A \emph{character class} describes a special kind of adventurer, so it
is a subclass of \emph{character}.  You must implement the following
character classes:

\subsection{Fighter}

A fighter is a character that has the following extra attributes and
methods:

\begin{itemize}

\item A read-write attribute \texttt{weapon : Weapon} describing the
  weapon that the fighter uses, see Section~\ref{weapon}.

\item A read-write attribute \texttt{armour : Armour} describing the
  weapon that the fighter is wearing, see Section~\ref{armour}.

\item A method \texttt{hit : Character -> unit}, indicating that the
  fighter tries to hit another character.  If the fighter is dead,
  nothing happens.

The chance of hitting the opposing character is
(10×\texttt{level}+\texttt{mind})\%, where \texttt{level} and
\texttt{mind} are attributes of the fighter.

The amount of damage dealt (if successfully hitting the opponent) is
\texttt{level} + weapon damage, where \texttt{level} is the
\texttt{level} attribute of the fighter and weapon damage is found by
calling the \texttt{damage} method of the fighter's weapon.

\item An overridden method \texttt{takeDamage : int -> unit} that
  reduces the damage taken by the value found by calling the
  \texttt{protect} method of the figher's armour.

\end{itemize}

\noindent
A fighter starts with cloth armour and a knife (see
Sections~\ref{weapon} and~\ref{armour}, but must have at least 11 in
her \texttt{body} attribute.  Override the constructor method to
ensure this.

\subsection{Mage}

A \emph{mage} is a magician that employs spells instead of weapons.
It has the following additional attributes and methods:

\begin{itemize}
\item An attribute \texttt{currentMana : int} indicating the amount of
  mana points that the mage hsa available for spell casting.  It is a
  non-negative integer.

\item An attribute \texttt{maxMana : int} indicating the maximum
  amount of mana that points the mage can have.

\item A method \texttt{cast : Spell -> Character -> unit} that casts a
  spell (see Section~\ref{spell}) on the indicated character,
  subtracting the necessary amount from the mage's mana points.  If
  the mage has less than the indicated mana left, or if the mage is
  dead, the spell is not cast, and the mana is unmodified.

\item A method \texttt{rest : unit} that makes the mage rest,
  regaining his level in mana points, up to his maximum mana.  A dead
  mage can not rest.

\end{itemize}

\noindent
A mage starts (at level 0) with 0 mana points and must have at least
11 in \texttt{mind}.  Override the constructor method to ensure this.

At every level increase, the maximum mana of the mage is set to
\texttt{level×mind}.  Override the \texttt{levelUp} method to ensure
this.

\subsection{Your Choice}

Define a character class of your own (such as Priest or Rogue).  It
should use the \texttt{body} and \texttt{mind} attributes more or less
equally.

\section{Weapons, Armour and Spells}

These are the tools that characters can employ.

\subsection{Weapons}\label{weapon}

A \texttt{Weapon} has the following method:

\begin{itemize}
\item \texttt{damage : unit -> int} that returns the amount of damage
  that the weapons deals.
\end{itemize}

\noindent
You must implement at least the following instances of the
\texttt{Weapon} class:

\begin{itemize}
\item A \texttt{Knife} is a weapon that deals 1d8 points of damage,
  i.e, 1--8 points chosen randomly with equal probability.

\item A \texttt{Mace} is a weapon that deals 2d4 points of damage,
  i.e, 1--4 points chosen randomly with equal probability plus another
  1--4 points chosen randomly with equal probability.

\item One more weapon of your design.
\end{itemize}

\subsection{Armour}\label{armour}

A suit of \texttt{Armour} has the following method:

\begin{itemize}
\item \texttt{protect : unit -> int} that returns the amount of damage
  by which the armour reduces the damage taken when hit.
\end{itemize}

\noindent
You must implement at least the following instances of the
\texttt{Armour} class:

\begin{itemize}
\item \texttt{Cloth} is a type of armour that subtracts 2 points of
  damage.
\item \texttt{Chain} is a type of armour that subtracts 1d4 points of
  damage, i.e, 1--4 points chosen randomly with equal probability.

\item One more type of armour of your design.
\end{itemize}

\subsection{Spells}\label{spells}

A mage can employ spells.  A spell has the following attributes and
methods:

\begin{itemize}
\item An attribute \texttt{manaCost : integer} is the number of mana
  points it costs to cast the spell.
\item A method \texttt{effect : Character -> unit} is the effect that
  the spell has on the recipient character.
\end{itemize}

\noindent
You must implement the following spells:

\begin{itemize}
\item \texttt{MagicMissile} costs 1 mana point and deals 2 points of
  damage to the recipient.

\item \texttt{Heal} costs 2 mana points and heals 4 points of damage
  to the recipient.  The recipient can be the mage herself.

\item \texttt{Drain} costs 5 mana points and deals 1d6 damage to the
  recipient.  The damage dealt will be added as mana points to the
  mana pool of the caster.

\item And one spell of your own design, for example a spell that
  enchants the weapon of the recipient (who must be a fighter) to do
  more damage for a limited number of uses.
\end{itemize}

\section{Random character creation}

\begin{itemize}

\item Make a function \texttt{createCharacter : unit -> Character}
  that randomly creates a character by using the following steps:

\begin{enumerate}
\item roll 3d6, i.e, find a value equal to adding three random numbers
  equally distributed in the 1--6 range.
\item Set \texttt{body} to this value and \texttt{mind} to 21 minus
  \texttt{body}.
\item Choose an appropriate character class.
\end{enumerate}

\noindent
The character starts at level 0.

\item Make a function \texttt{createParty : int -> int -> Character
  list} that creates a party of $M$ level $L$ characters, using the
  \texttt{createCharacter} function and the \texttt{levelUp} method of
  each character.

\end{itemize}

\section{Simulate Battle}

Simulate a battle between two parties.  A battle consists of a number
of rounds in which all characters in both parties take an action
each.  The order in which characters act is determined randomly in
each round.  An action is using a method such as \texttt{hit},
\texttt{cast} or \texttt{rest} implemented by the character class.

The battle continues until all characters in one of the parties are
dead.

\end{document}

