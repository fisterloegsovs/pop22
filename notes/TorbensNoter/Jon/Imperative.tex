\documentclass[a4paper]{article}

\usepackage{cmap}
\usepackage[utf8x]{inputenc}
\usepackage{latexsym}
\usepackage[english]{babel}
\usepackage{graphicx}
\usepackage{hyperref}
\usepackage[all]{hypcap}
\usepackage{enumerate}
\usepackage[margin=2.5cm]{geometry}

\begin{document}
\title{Examples of Imperative Programming in F\#}

\author{Torben Mogensen}
\date{\today}

\maketitle

\noindent
These notes show some simple examples of imperative programming to
supplement the notes and textbook.

\section{Example 1: Using a global counter to count function calls}

We start from the doubly-recursive version of Fibonacci's function:

\begin{tabular}
  let rec fib n =
    if n < 2 then n else fib (n-1) + fib (n-2)
\end{tabular}

\noindent
We can call this to find, for example \texttt{fib 10 $\leadsto$ 55}.
But how many function calls does this version of \texttt{fib} use?

To figure this out, we use a mutable variable \texttt{counter} and
modify \texttt{fib} to increment this for every function call and,
finally, print the result out:

\begin{verbatim}
  let fib n =
    let mutable counter = 0
    let rec fib' n =
      counter <- counter + 1
      if n < 2 then n else fib' (n-1) + fib' (n-2)
    let result = fib' n
    printfn "fib used %d function calls" counter
    result
\end{verbatim}

\noindent
When we call \texttt{fib 10}, it will print ``\texttt{fib used 177
  function calls}'' before returning the value (which is still 55).
Note that the call to \texttt{fib'} needs to come before the print
statement, but it is the value of this call that must be returned.
Hence, we bind the value in \texttt{result}, so we can return it after
the print statement.

We can use such counters to get a rough estimate of the time used by a
function.  We can, for example, compare to the version of fib that
uses accumulating parameters:

\begin{verbatim}
  let fibA n =
    let mutable counter = 0
    let rec fib' n a b =
      counter <- counter + 1
      if n = 0 then a else fib' (n-1) b (a+b)
    let result = fib' n 0 1
    printfn "fibA used %d function calls" counter
    result
\end{verbatim}

\noindent
When calling \texttt{fibA 10}, it prints ``\texttt{fibA used 11
  function calls}'' before returning 55.  Since there is not
signifantly more work done per function call, the accumulating version
is much faster.


\section{Example 2: Imperative Functions on Arrays}

Array elements are mutable, so it can sometimes be useful to use
imperative programming when working with arrays, especially if using
library functions like \texttt{Array.init}, \texttt{Array.map},
\texttt{Array.filter}, and \texttt{Array.fold} are not applicable to
the situation.  For example, the Sieve of Erastothenes method for
finding primes (as shown in previous notes) is not easily written
using the library functions.

Imperative programming is also useful when working on arrays that are
so large that creating a new array of similar size using, say,
\texttt{Array.map}, can use more memory than available.  With
Imperative programming, we can re-use the original array and write the
result to this instead of creating a new array.  We can no longer use
the old array, but we have used less memory during the operation.  But
programming \emph{in-place} (without using extra space) can be more
difficult than when we are allowed to use extra space.

Let us consider a function that takes an array of numbers and moves
all the odd numbers to the beginning of the array and all the even
numbers to the end, but keeping the relative positions of the odd and
even numbers.  For example, the array \texttt{[|1;2;3;8;7;5;4|]}
should be replaced by \texttt{[|1;3;7;5;2;8;4|]}.  The function should
have the type

\begin{verbatim}
  oddEven : int [] -> int []
\end{verbatim}

\noindent
Doing this functionally using library functions is easy:

\begin{verbatim}
  let oddEven a =
    Array.append (Array.filter (fun n -> n%2 = 1) a)
                 (Array.filter (fun n -> n%2 = 0) a)
\end{verbatim}

\noindent
If we are allowed to create a new array, doing it imperatively is not
that difficult either (albeit slightly more verbose):


\begin{verbatim}
  let oddEven' a =
    let ll = Array.length a
    let newA = Array.create ll 0  // make new array
    let mutable j = 0
    for i = 0 to ll - 1 do // copy odd elements to new array
      if a.[i] % 2 = 1 then
        newA.[j] <- a.[i]
        j <- j + 1
    for i = 0 to ll - 1 do // copy even elements to new array
      if a.[i] % 2 = 0 then
        newA.[j] <- a.[i]
        j <- j + 1
    newA // return new array
\end{verbatim}

\noindent
Note how \texttt{j} keeps track of where the copied elements go.

Now, let us consider doing this in-place, i.e., without allocating a
new array, but leaving the result in the same array as the input.  
We want a function of the type

\begin{verbatim}
  oddEvenI : int [] -> unit
\end{verbatim}

\noindent
We can not from the type see that the input array is modified, but we
can see that the result type is \texttt{unit}, so the function does
not return a new array.

Since we are not allowed to create a new array, we must work by
re-arranging elements in the old array, which we can do by swapping
elements.  One possible strategy is:

\begin{enumerate}[1.]
\item Start from position $p=0$.
\item Locate the first \emph{even} element at some position $i\geq p$, if any.
\item If there are none, we are done, as the array starting from $p$
  is empty or all elements in the array are odd.
\item Locate the first \emph{odd} element at some position $j>i$, if
  any.
\item If there is no such element, we are done, because there are no
  odd elements after the first even element.
\item We now have even elements in positions $i$ to $j-1$ and an odd
  element in position $j$.  We can get the odd element at position $j$
  in place by putting it in position $i$, but then we must slide the
  even element in position $i$ to position $i+1$, and so on, until the
  even element in position $j-1$ is put into position $j$.
\item Set $p$ to $p+1$ and repeat from step 2.
\end{enumerate}

\noindent
We can illustrate this on the array \texttt{a = [|1;2;3;8;7;5;4|]}:

\begin{enumerate}[i.]
\item We start by $p=0$ and find $i\geq p$ such that \texttt{a.[i]} is even.
This makes $i=1$, since \texttt{a.[1] = 2}.

\item We now find $j>i$ such that \texttt{a.[j]} is odd.  This makes $j=2$,
as \texttt{a.[2] = 3}.

\item We now put 3 into place 1 and 2 into place 2, making \texttt{a =
  [|1;3;2;8;7;5;4|]}.

\item We set $p$ to $i+1=2$, set $i$ to the position of the next even
element, so $i=2$, since \texttt{a.[2] = 2}.

\item We now find $j>i$ such that \texttt{a.[j]} is odd.  This makes $j=4$,
as \texttt{a.[4] = 7}.

\item We now put 7 into \texttt{a.[2]} and slide the elements in position 2
to $4-1$ one position up, making \texttt{a = [|1;3;7;2;8;5;4|]}.

\item Repeating this once more brings 5 into place, so we get \texttt{a =
  [|1;3;7;5;2;8;4|]}.

\item Since there are no odd elements after the 2, we are now done.

\end{enumerate}

\noindent
We can code this strategy in F\#{} like this:

\begin{verbatim}
  let oddEvenI a =
    let ll = Array.length a
    let mutable p = 0
    while p < ll do
      let mutable i = p
      while i < ll && a.[i] % 2 = 1 do // find even element
        i <- i + 1
      if i < ll then  // if not done, then
        let mutable j = i + 1
        while j < ll && a.[j] % 2 = 0 do  // find odd element after that
          j <- j + 1
        if j < ll then // if not done, then
           // odd number at j is put into place and even elements slide up
          let mutable elem = a.[j]
          while i <= j do
            let tmp = a.[i]
            a.[i] <- elem
            elem <- tmp
            i <- i + 1
      p <- p + 1  // and repeat with larger p
\end{verbatim}

\noindent
This is not only very verbose, it is also less efficient: Where the
previous two versions use time linear in the size of the array, this
version can in the worst case need quadratic time: Consider an array
with even numbers in position 0 to $n-1$ and odd numbers in position
$n$ to $2n-1$.  Each iteration will slide all $n$ even numbers one
position up, and $n$ iterations are needed, so the total time used is
proportional to $n^2$.

But it does use less memory.

Besides being examples of imperative programming, the lesson in this
is that doing array operations completely in-place can seriously
complicate programs and even make them run slower.  But when operation
on \emph{very} large arrays, it may be neccesary.

\end{document}

