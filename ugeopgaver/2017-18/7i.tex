\documentclass[a4paper]{article}

\usepackage{cmap}
\usepackage[utf8x]{inputenc}
\usepackage{latexsym}
\usepackage[danish]{babel}
\usepackage{graphicx}
\usepackage{hyperref}
\usepackage[all]{hypcap}
\usepackage{enumerate}
\usepackage[margin=2.5cm]{geometry}

\begin{document}
\title{Programmering og Problemløsning\\
Datalogisk Institut, Københavns Universitet\\
Uge(r)seddel 4 -- gruppeopgave (rev. 1.1)}

\author{Torben Mogensen}
\date{Deadline 5.\ oktober}

\maketitle

\noindent
I denne periode skal I arbejde i grupper.  Formålet er at arbejde med
tupler, lister, mønstergenkendelse og afprøvning.

Opgaverne i denne uge er delt i øve- og afleveringsopgaver.  Vi
bruger forkortelsen ``HR'' for Hansen \& Rischels bog ``Functional
Programming Using F\#''.

NB! Der skal for alle opgaver laves black-box testing som beskrevet i
afsnit 10.2 i Jon Sporrings F\#{} noter.  Dette gælder også afleveringsopgaverne.

\subsubsection*{Øveopgaverne er:}

\begin{description}
\item[4ø.1.] HR: 4.1, 4.7, 4.8.

\item[4ø.2.] En tabel kan repræsenteres som en liste af lister, hvor alle
  listerne er lige lange.

Listen \texttt{[[1; 2; 3]; [4; 5; 6]]} repræsenterer for eksempel tabellen

\[\left [\begin{array}{rrr}
1 & 2 & 3 \\
4 & 5 & 6
\end{array}
\right ]\]

\begin{enumerate}[i]
\item Lav en funktion \texttt{isTable : 'a list list -> bool}, der
  givet en liste af lister afgør, om det er en lovlig ikke-tom tabel,
  altså om alle listerne har ens længde, og at der er mindst en liste
  med mindst et element.  Vink: Se \texttt{TuplesAndLists.pdf}.

\item Lav en funktion \texttt{firstColumn : 'a list list -> 'a list},
  der tager en liste af lister og returnerer listen af førsteelementer
  i de indre lister.  F.eks.\ skal \texttt{firstColumn [[1; 2; 3]; [4;
        5; 6]]} returnere listen \texttt{[1; 4]}.

  Hvis en eller flere af listerne er tomme, er \texttt{firstColumn}
  udefineret.  Derfor er det i orden, hvis  \texttt{fsharpi} giver
  advarsel om ``incomplete pattern matches'', og en tilsvarende
  fejlmeddelelse, hvis \texttt{firstColumn} kaldes med et
  argument, hvor en af listerne er tom.


\item Lav en funktion \texttt{dropFirstColumn : 'a list list -> 'a
  list list}, der tager en liste af lister og returnerer en liste af
  lister, hvor førsteelementerne i de indre lister er fjernet.
  F.eks.\ skal \texttt{dropFirstColumn [[1; 2; 3]; [4; 5; 6]]}
  returnere \texttt{[[2; 3]; [5; 6]]}.

  Hvis en eller flere af listerne er tomme, er \texttt{dropFirstColumn}
  udefineret.  Derfor er det i orden, hvis  \texttt{fsharpi} giver
  advarsel om ``incomplete pattern matches'', og en tilsvarende
  fejlmeddelelse, hvis \texttt{dropFirstColumn} kaldes med et
  argument, hvor en af listerne er tom.

\item Lav en funktion \texttt{transpose : 'a list list -> 'a list
  list}, der \emph{transponerer} en tabel.  Transponering er spejling over
  diagonalen, så den transponerede tabel til den herover viste tabel er

\[\left [\begin{array}{rr}
1 & 4 \\
2 & 5 \\
3 & 6
\end{array}
\right ]\]

\noindent
Kaldet \texttt{transpose [[1; 2; 3]; [4; 5; 6]]} skal altså returnere \texttt{[[1; 4]; [2; 5]; [3; 6]]}.

Det kan antages, at argumentet til \texttt{transpose} er en lovlig
tabel, så advarsler om ufuldstændige mønstre er acceptable -- hvis
funktionen eller virker.

Bemærk, at \texttt{transpose (transpose t) = t}, hvis \texttt{t} er en
tabel.

Vink: Brug funktionerne \texttt{firstColumn} og \texttt{dropFirstColumn}.

\end{enumerate}

\end{description}

\subsubsection*{Afleveringsopgaven løses i grupper og er:}


\begin{description}
\item[4g.1.] HR: 4.11 og 4.15.

\item[4g.2.] Lav en funktion \texttt{removeDuplicates : 'a list -> 'a list
  when 'a : equality}, som fjerner duplikater i en liste.  For
  eksempel skal kaldet \texttt{removeDuplicates [1; 2; 1; 3: 2]}
  give resultatet \texttt{[1; 2; 3]}.  Bemærk, at den første
  forekomst af et givet elememt bevares, mens de øvrige forekomster
  fjernes.  De bevarede elementer bevarer deres indbyrdes position.

\end{description}

\noindent
Afleveringsopgaven skal afleveres som en f\#{} fil med løsningerne for
alle delopgaverne, og som kan køres med fsharpi.  Indsæt delopgavernes
nummer som kommentarer over løsningerne til disse.  Husk at inkludere
afprøvning (black-box testing) for alle funktioner, og dokumenter
funktionerne med kommentarer som beskrevet i kapitel 7 af Jon
Sporrings F\#{} noter.  Filen skal navngives efter følgende
konvention: \texttt{H$n$-\emph{fornavn$1$}.\emph{efternavn$1$}-\ldots-\emph{fornavn$m$}.\emph{efternavn$m$}-4g.fsx}, hvor
$n$ er et tocifret holdnummer.  Eksempel:
\texttt{H07-Anders.And-Faetter.Guf-Georg.Gearloes-4g.fsx}.

\vspace{1ex}

\hfill God fornøjelse

\section*{Ugens nød \#1}

Vi vil i udvalgte uger stille særligt udfordrende og sjove opgaver,
som interesserede kan løse.  Det er helt frivilligt at lave disse
opgaver, som vi kalder ``Ugens nød'', men der vil blive givet en
mindre præmie til den bedste løsning, der afleveres i Absalon.

Ugens nød i denne uge omhandler en variant af spillet "`Minestryger"'
(\url{http://da.wikipedia.org/wiki/Minestryger_%28spil%29}).

I vores variant er det for alle felter på forhånd kendt, hvor mange
bomber, der i alt er i de otte nabofelter samt feltet selv.  Det
gælder både felter med og felter uden bomber.  Det er samme variant,
der blev brugt i dagsløbet i campusdagene.

Et eksempel på et sådant spil er

\begin{verbatim}
        221
        232
        232
\end{verbatim}
  
svarende til bombeplaceringen

\begin{verbatim}
        000
        110
        001
\end{verbatim}

\noindent
hvor tomme felter er angivet med \texttt{0} og bomber med \texttt{1}.

Der er ikke altid en entydig løsning.  For eksempel vil spillet

\begin{verbatim}
        11
        11
\end{verbatim}
 
\noindent
have fire løsninger:

\vspace{1ex}

\begin{tabular}{@{\quad\quad\quad}l@{\quad\quad}l@{\quad\quad}l@{\quad\quad}l}
\texttt{01} & \texttt{10} & \texttt{00} & \texttt{00}\\
\texttt{00} & \texttt{00} & \texttt{01} & \texttt{10}
\end{tabular}
\vspace{1ex}


\noindent
Der er heller ikke altid løsninger, idet for eksempel

\begin{verbatim}
        10
        00
\end{verbatim}
 
\noindent
ikke har en løsning.

Vi repræsenterer et spil som en tabel af heltal, som alle ligger
mellem 0 og 9.  Løsninger repræsenteres som tabeller af heltal, der
alle har værdi 0 eller 1.

\begin{enumerate}[Nød {1}.1]

\item funktion \texttt{minelaegger : int list list -> int list list},
  som givet en \emph{løsning} returnerer et spil.  Det kan antages, at
  input er en liste af lister af tal, der alle er enten 0 eller 1.

For eksempel skal kaldet
\texttt{minelaegger [[0; 0; 0]; [1; 1; 0]; [0; 0; 1]]}
returnere spillet \texttt{[[2; 2; 1]; [2; 3; 2]; [2; 3; 2]]}.


\item Lav en funktion \texttt{minestryger : int list list -> int list list
  list}, som givet et spil finder alle løsninger.  Det kan antages, at input er en liste af lister af tal mellem 0 og 9.

For eksempel skal kaldet
\texttt{minestryger [[1; 1]; [1; 1]];;}
returnere listen

\begin{verbatim}
  [[[0; 1]; [0; 0]];
   [[1; 0]; [0; 0]];
   [[0; 0]; [0; 1]];
   [[0; 0]; [1; 0]]]
\end{verbatim}

\noindent
eller en permutation af denne liste, mens 
\texttt{minestryger [[1; 0]; [0; 0]]}
skal returnere den tomme liste.


Det er ikke vigtigt, om løsningerne findes hurtigt, men hvis flere
korrekte programmer indleveres, gives præmien til det hurtigste.

\end{enumerate}

\noindent
Der skal uploades både en \LaTeX-fil, der beskriver fremgangsmåden,
samt en fsx fil, der indeholder definitionerne af de to funktioner.
Navngivningen af filerne er ikke vigtig.  Der er oprettet en særlig
opgave til ugens nød.  Opgaven er individuel.

Endnu et par eksempelspil til afprøvning er angivet herunder.

\begin{verbatim}
        22222      33333      44444      23232      23321
        22222      33333      44444      24453      34421
        22222      33333      44444      45543      45532
        22222      33333      44444      34443      23321
        22222      33333      44444      33222      12221
\end{verbatim}

\end{document}

