\documentclass[a4paper,12pt]{article}

\usepackage[margin=2cm]{geometry}
\usepackage[T1]{fontenc}
\usepackage[utf8]{inputenc}
\usepackage[danish]{babel}
\usepackage{listings}
\usepackage{graphicx}
\graphicspath{{figures/}}
\setlength{\parindent}{0cm}
\setlength{\parskip}{1em}
\usepackage{hyperref}

\newcommand{\sbl}{\textsc{SimpleJack }}
\newcommand{\sblp}{\textsc{SimpleJack. }}

\title{Programmering og Problemløsning\\Datalogisk Institut,
  Københavns Universitet\\Uge(r)seddel 12 - individuel opgave}
\author{Martin Petersen, Jon Sporring og Christina Lioma}
\date{Deadline  19.\ januar}

\begin{document}
\maketitle

Forelæsningerne vil afslutte kurset. Vi vil behandle opsamlingsemner
bla.\ polimorfi og evaluering

Til øvelserne på alm.\ skema forventer vi at I arbejder efter nedenstående plan.
\begin{description}
\item[Mandag-Tirsdag 11-12/1:]~\\[0cm] 
  Der arbejdes med 11g.
\item[Fredag 15/1:]~\\[0cm] 
  Der arbejdes med 12i.
\item[Mandag-tirsdag 18/1-19/1:]~\\[0cm] 
  Der arbejdes med 12i.
\end{description}

Afleveringsopgaven er:
\begin{description}
\item[12.1] Du skal implementere en udvidelse til \sbl som indeholder en omstrukturering af nogle af klasserne, samt indførelse af en række nye strategier. Du skal simulere nogle \sbl spil hvor du afprøver forskellige strategier, for at afgøre
  hvilken strategi som lader til at være den bedste.

  Implementér super-klassen \texttt{Player}, og klasserne \texttt{Dealer}, \texttt{Human} og \texttt{AI} som nedarver fra \texttt{Player.} \texttt{Player} skal indeholde attributter og metoder som implementerer den fælles funktionalitet
  som alle tre typer "spillere" har, f.eks. en metode som vælger "Hit" eller "Stand".

  Implementér super-klassen \texttt{Strategy}, samt en klasse for hver af følgende strategier, som alle nedarver fra \texttt{Strategy}
\begin{enumerate}
\item Vælg altid "Hit", medmindre summen af egne kort kan være 15 eller over, ellers vælg "Stand".
\item Vælg altid "Hit", medmindre summen af egne kort kan være 17 eller over, ellers vælg "Stand".
\item Vælg altid "Hit", medmindre summen af egne kort kan være 19 eller over, ellers vælg "Stand".
\item Vælg tilfældigt mellem "Hit" og "Stand". Hvis "Hit" er valgt, trækkes et kort og der vælges igen tilfældigt mellem
"Hit" os "Stand" osv.
\item Følg strategi 2. hvis ét af egne kort er et Es, ellers følg strategi 1.
\end{enumerate}
Simulér 3000 spil \sbl med 5 \texttt{AI} spillere som følger de 5 ovenstående strategier.  Dealer skal følge strategi 2.
Konkludér hvilken af strategierne som lader til at være bedst.

Du skal også
\begin{itemize}
\item Opdatere dit UML-diagram
\item Lave Unittests
\item Kommentere ny kode jævnfør kommentarstandarden for F\#
\end{itemize}
 
Afleveringsopgaven skal afleveres som både LaTeX, den genererede PDF, samt en fsx tekstfil med løsningen, som kan oversættes med fsharpc, og hvis resultat kan køres med mono. Det hele skal samles i en zip fil efter sædvanlig navnekonvention:
\begin{quote}
  \lstinline|<instructor's-initial>_<firstname.lastname>_<exercise-number>.zip|
\end{quote}
I zip filen skal en delopgave navngives ved opgavenummer, således at filen for opgave 12.1 hedder \lstinline|opg12_1.fsx|.
\end{description}

\flushright God fornøjelse.
\end{document}

%%% Local Variables:
%%% mode: latex
%%% TeX-master: t
%%% End:
