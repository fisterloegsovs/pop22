\sbl er en forsimplet udgave af kortspillet Blackjack. I \sbl\ spiller man ikke om penge/jetoner men blot om sejr/tab mellem en spiller og dealer.  Reglerne for \sbl er som følger:

Spillet består af en dealer, 1-5 spillere samt et normalt kortspil (uden jokere). Ved spillets start får dealer qog hver spiller tildelt 2 tilfældige kort fra bunken som placeres med billedsiden opad foran spilleren, så alle kan se dem. I \sbl\ spilles der med åbne kort dvs. alle trukne kort til hver en tid er synlige for alle spillere.  Kortene har værdi som følger:
\begin{enumerate}
\item Billedkort (knægt, dame og konge) har værdien 10
\item Es kan antage enten værdien 1 eller 11
\item Resten af kortene har den påtrykte værdi
\end{enumerate}
For hver spiller gælder spillet om at ende med en korthånd hvis sum af værdier er højere en dealers sum af værdier,  uden at summen overstiger 21, i hvilket tilfælde spilleren er "bust". Spillerne får nu en tur hver, hvor de skal udføre en af følgende handlinger:
\begin{enumerate}
\item "Stand": Spilleren/dealeren vælger ikke at modtage kort og turen går videre.
\item "Hit": Spilleren/dealeren vælger at modtage kort fra bunken et ad gangen indtil han/hun vælger at stoppe og turen går videre.
\end{enumerate}
Det er dealers tur til sidst efter alle andre spillere har haft deres tur. Når dealer har haft sin tur afsluttes spillet. Ved spillets afslutning afgøres udfaldet på følgende måde: En spiller vinder hvis ingen af følgende tilfælde gør sig gældende:
\begin{enumerate}
\item Spilleren er "bust"
\item Summen af spillerens kort-værdier er lavere end, eller lig med dealers sum af kort-værdier
\item Både spilleren og dealer har SimpleJack (SimpleJack er et Es og et billedkort)
\end{enumerate}
Bemærk at flere spillere altså godt kan vinde på en gang. Et spil \sbl\ er mellem en spiller og dealer, 
så med 5 spillere ved bordet, er det altså 5 separate spil som spilles.
