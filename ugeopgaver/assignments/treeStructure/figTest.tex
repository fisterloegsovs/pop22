\label{figTest} Lav en figur \texttt{figTest : figure}, der består af en rød cirkel
  med centrum i (50,50) og radius 45, samt en blå rektangel med
  hjørnerne (40,40) og (90,110), som illustreret i tegningen nedenfor (hvor vi dog har brugt skravering i stedet for udfyldende farver.)
  \begin{center}
    \begin{minipage}{.23\textwidth}
      \begin{tikzpicture}[domain=0:12,scale=0.25]
        \draw[very thin,color=gray] (0,0) grid (10,12);
        \draw[->] (0,12) node[left] {$0$} -- (11,12) node[right] {$x$};
        \draw[->] (0,12) node[above] {$0$} -- (0,-1) node[below] {$y$};
        \draw[-] (5,12) node[above] {$50$} -- (5,0);
        \draw[-] (10,12) node[above] {$100$} -- (10,0);
        \draw[-] (0,7) node[left] {$50$} -- (10,7);
        \draw[-] (0,2) node[left] {$100$} -- (10,2);
        \draw[pattern=north west lines, pattern color=red] (5,7) circle (4.5);
        \draw[pattern=north east lines, pattern color=blue] (4,1) rectangle (9,8);
      \end{tikzpicture}
    \end{minipage}
  \end{center}
