Temaet for ugernes opgaver er at programmere et Sudoku-spil. \emph{Sudoku} er et puslespil, som er blevet opfundet uafh{\ae}ngigt flere gange; den tidligste ``{\ae}gte'' version af sudoku synes at kunne spores tilbage til det franske dagblad \emph{Le Si{\`e}cle} i 1892.

Vi betragter her kun den basale variant, som spilles p{\aa} en matrix af 81 sm{\aa} felter, arrangeret i 9 r{\ae}kker og 9 s{\o}jler.  Matricen er desuden inddelt i 9 ``bokse'' eller ``regioner'', hver med 3 gange 3 felter.

Nogle af felterne er udfyldt p{\aa} forh{\aa}nd, og puslespillet g{\aa}r ud p{\aa} at udfylde de resterende felter p{\aa} en s{\aa}dan m{\aa}de, at hver af de 9 r{\ae}kker, hver af de 9 s{\o}jler og hver af de 9 regioner kommer til at indeholde en permutation af symbolerne fra et forelagt alfabet af st{\o}rrelse 9; vi v{\ae}lger her (som man plejer at se det) alfabetet best{\aa}ende af cifrene fra 1 til 9.

Her er en lovlig starttilstand for et spil sudoku:
\begin{sudoku}
|5|3| | |7| | | | |.
|6| | |1|9|5| | | |.
| |9|8| | | | |6| |.
|8| | | |6| | | |3|.
|4| | |8| |3| | |1|.
|7| | | |2| | | |6|.
| |6| | | | |2|8| |.
| | | |4|1|9| | |5|.
| | | | |8| | |7|9|.
\end{sudoku}

Og f{\o}lgende er en vindende tilstand (en ``l{\o}sning'') af ovenst{\aa}ende:
\begin{sudoku}
|5|3|4|6|7|8|9|1|2|.
|6|7|2|1|9|5|3|4|8|.
|1|9|8|3|4|2|5|6|7|.
|8|5|9|7|6|1|4|2|3|.
|4|2|6|8|5|3|7|9|1|.
|7|1|3|9|2|4|8|5|6|.
|9|6|1|5|3|7|2|8|4|.
|2|8|7|4|1|9|6|3|5|.
|3|4|5|2|8|6|1|7|9|.
\end{sudoku}

\subsection*{Nummerering og filformat}
Lad os nummerere r{\ae}kkerne $r = 0, 1, \ldots, 8$ ovenfra og ned og s{\o}jlerne $s=0, 1, \ldots, 8$ fra venstre mod h{\o}jre.  Ogs{\aa} regionerne vil vi nummerere $q=0, 1, \ldots, 8$, i ``normal l{\ae}seretning'' (for vestlige sprog).  Sammenh{\ae}ngen mellem r{\ae}kkenummer $r$, s{\o}jlenummer $s$ og regionsnummer $q$ kunne s{\aa} udtrykkes i f{\o}lgende formel ved brug af heltalsoperationer: 
\begin{itemize}
\item Feltet i r{\ae}kke nummer $r$ og s{\o}jle nummer $s$ vil ligge i region nummer
  \begin{equation}
    q = \texttt{$r$ / 3 * 3 + $s$ / 3}
    \label{eq:1}
  \end{equation}
\item Region nummer $q$ best{\aa}r af felterne
  \begin{equation}
    (r,s)=\left(q\texttt{ / 3 * 3 + }m, q\texttt{ \% 3 * 3 + }n\right);\; m,n\in\{0,1,2\}.
    \label{eq:2}
  \end{equation}
\end{itemize}
En spiltilstand kan gemmes i en fil, ved at antage, at indholdet \emph{altid} ser ud som f{\o}lger:
\begin{itemize}
\item Der er mindst 90 tegn i filen (der kan v{\ae}re flere, men vi er kun interesseret i de 90 f{\o}rste)
\item De f{\o}rste 90 tegn i filen er delt op i 9 grupper, som repræsenterer indholdet af række 0,\dots, 8 i nævnte rækkefølge. Hver gruppe best{\aa}r af 10 tegn: F{\o}rst 9 tegn, som er et blandt \texttt{'1'}, \ldots, \texttt{'9'}, \texttt{'*'}, og til sidst strengen \texttt{"$\backslash$n"}.
\end{itemize}
F.eks.\ er nedenst{\aa}ende indholdet i en fil, som indeholder starttilstanden for ovenst{\aa}ende sudoku:
\begin{quote}
  {\scriptsize \verb|53**7****\n6**195***\n*98****6*\n8***6***3\n4**8*3**1\n7***2***6\n*6****28*\n***419**5\n****8**79\n|}
\end{quote}
Hvis vi fortolker strengen \texttt{"$\backslash$n"} som ``ny linje'', bliver ovenst{\aa}ende lettere at l{\ae}se:
\begin{quote}
\begin{verbatim}
53**7****
6**195***
*98****6*
8***6***3
4**8*3**1
7***2***6
*6****28*
***419**5
****8**79
\end{verbatim}
\end{quote}
