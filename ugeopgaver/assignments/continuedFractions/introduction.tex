I denne opgave skal I regne med kædebrøker (continued fractions)\footnote{\url{https://en.wikipedia.org/wiki/Continued_fraction}}. Kædebrøker er lister af heltal, som repræsenterer reelle tal. Listen er endelig for rationelle og uendelig for irrationelle tal. 

En kædebrøk skrives som:
$x = [q_0; q_1, q_2, \ldots]$, hvilket svarer til tallet,
\begin{equation}
  x = q_0 + \frac{1}{q_1 + \frac{1}{q_2 + \dots}}.
\end{equation}
F.eks.\ listen $[3;4, 12, 4]$ evaluerer til
\begin{align}
  x &= 3 + \frac{1}{4 + \frac{1}{12 + \frac{1}{4}}}
  \\&=  3 + \frac{1}{4 + \frac{1}{12.25}}
  \\&=  3 + \frac{1}{4.081632653}
  \\&=  3.245.
\end{align}
Omvendt, for et givet tal $x$ kan kædebrøken $[q_0; q_1, q_2, \ldots]$ udregnes ved følgende algoritme: For $x_0 = x$ og $i \geq 0$ udregn $q_i = \lfloor x_i \rfloor$, $r_i = x_i - q_i$ og $x_{i+1} = 1/r_i$ indtil $r_i = 0$. Nedenfor ses en udregning for tallet $x=3.245$:
\begin{center}
  \begin{tabular}{|l|l|l|l|l|}
    \hline
    $i$ & $x_i$ & $q_i$ & $r_i$ & $1/r_i$\\
    \hline
    0 & 3.245 & 3 & 0.245 & 4.081632653\ldots\\
    1 & 4.081632653\ldots & 4 & 0.081632653 & 12.25\\
    2 & 12.25 & 12 & 0.25 & 4\\
    3 & 4 & 4 & 0 & -\\
    \hline
  \end{tabular}
\end{center}
Resultatet aflæses i anden søjle: $[3; 4, 12, 4]$.

Kædebrøker af heltalsbrøkker $t/n$ er særligt effektive at udregne vha.\ Euclids algoritme for største fællesnævner. Algoritmen i \ref{gcd} regner rekursivt på relationen mellem heltalsdivision og rest: Hvis $a = t \text{ div } n$ er heltalsdivision mellem $t$ og $n$, og $b = t \text{ \% } n$ er resten efter heltalsdivision, så er $t = a n + b$. Man kan nu forestille sig $t$, $n$ og $b$ som en sekvens af heltal $r_i$, hvor
\begin{align}
  r_{i-2} &= q_i r_{i-1} + r_i,
  \\r_i &= r_{i-2}\text{ \% }r_{-1}\quad\text{(rest ved heltalsdivision)},
  \\q_i &= r_{i-2}\text{ div }r_{i-1}\quad\text{(heltalsdivision)},
\end{align}
Hvis man starter sekvensen med $r_{-2} = t$ og $r_{-1}=n$ og beregninger resten iterativt indtil $r_{i-1}=0$, så vil største fællesnævner mellem $t$ og $n$ være lig $r_{i-2}$, og $[q_0; q_1,\ldots,q_j]$ vil være $t/n$ som kædebrøk. Til beregning af største fællesnævner regner algoritmen i \ref{gcd} udelukkende på $r_i$ som transformationen $(r_{i-2}, r_{i-1}) \rightarrow (r_{i-1}, r_i)  = (r_{i-1}, r_{i-2}\text{ \% }r_{-1})$ indtil $(r_{i-2}, r_{i-1}) = (r_{i-2},0)$. Tilføjer man beregning af $q_i$ i rekursionen, har man samtidigt beregnet brøkken som kædebrøk. Nedenfor ses en udregning for brøken 649/200:
\begin{center}
  \begin{tabular}{|r|r|r|r|r|}
    \hline
    $i$ & $r_{i-2}$ & $r_{i-1}$ & $r_i = r_{i-2}\text{ \% }r_{-1} $ & $q_i = r_{i-2} \text{ div } r_{i-1}$\\
    \hline
    0 & 649 & 200 & 49 & 3 \\
    1 & 200 & 49 & 4 & 4\\
    2 & 49 & 4 & 1  & 12\\
    3 & 4 & 1 & 0  & 4\\
    4 & 1 & 0 & -  & -\\
    \hline
  \end{tabular}
\end{center}
Kædebrøkken aflæses som højre søjle: $[3; 4, 12, 4]$.

Omvendt, approksimationen af en kædebrøk som heltalsbrøken $\frac{t_i}{n_i}, i \geq 0$ fås ved
\begin{align}
  t_i &= q_it_{i-1}+t_{i-2},
  \\n_i &= q_in_{i-1}+n_{i-2},
  \\t_{-2} &= n_{-1} = 0,
  \\t_{-1} &= n_{-2} =1,
\end{align}
Alle approximationerne for $[3; 4, 12, 4]$ er givet ved,
\begin{align}
  \frac{t_0}{n_0} &= \frac{q_0t_{-1} + t_{-2}}{q_0n_{-1}+n_{-2}} = \frac{3\cdot 1+0}{3\cdot 0 + 1} = \frac{3}{1} = 3,
  \\\frac{t_1}{n_1} &= \frac{q_1t_0 + t_{-1}}{q_1n_0+n_{-1}} = \frac{4\cdot 3 + 1}{4\cdot 1+0} = \frac{13}{4} = 3.25,
  \\\frac{t_2}{n_2} &= \frac{q_2t_1 + t_{0}}{q_2n_1+n_{0}} = \frac{12\cdot 13 + 3}{12\cdot 4 + 1} = \frac{159}{49} = 3.244897959\ldots,
  \\\frac{t_3}{n_3} &= \frac{q_3t_2 + t_{1}}{q_3n_2+n_{1}} = \frac{4\cdot 159 + 13}{4\cdot 49+4} = \frac{649}{200} = 3.245.
\end{align}
Bemærk at approximationen nærmer sig 3.245 når $i$ vokser.
