I html-standarden angives links med \lstinline!<a></a>! tags, f.eks. kunne et link til Googles hjemmeside skrives som \lstinline!<a href="https://google.com">Tryk her for Google</a>!. Der skal laves et program
  \begin{quote}
    \mbox{\lstinline!countLinks : url:string -> int!}
  \end{quote}
  som henter internetsiden angivet med argument \lstinline!url! og som tæller, hvor mange links der er på siden ved at tælle antallet af \lstinline!<a! delstrenge.

  Bemærk: Langt de fleste internetsider kræver et gyldigt certifikat for at dit program kan læse siden, og som udgangspunkt har mono ingen certifikater installeret. For at installere et nyttigt sæt certifikater kan du bruge \lstinline[language=console]{mozroots}, som er en del af Mono pakken. På Linux/MacOS gør følgende fra Konsollen:
  \begin{quote}
    \lstinline[language=console]{mozroots --import --sync}
  \end{quote}
  På Windows gør du følgende (på samme linje)
  \begin{quote}
    \lstinline[language=console]{mono "C:\Program Files (x86)\Mono\lib\mono\4.5\mozroots.exe" --import --sync}
  \end{quote}
  Ret evt.\ stien, hvis din installation af mozroots ligger et andetsted. Derefter kan du læse de fleste sider uden at blive afvist.

  Til besvarelsen skal der laves en kort afprøvning, og en kort beskrivelse af løsningen med argumenter for større valg, der er foretaget, for at nå til den givne løsning.
