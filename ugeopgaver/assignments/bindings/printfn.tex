Indtast følgende program i en tekstfil, og oversæt og kør programmet 
  \begin{codeNOutput}[label=linear]{: Værdibindinger.}
\begin{lstlisting}
let a = 3
let b = 4
let x = 5
printfn "%A * %A + %A = %A" a x b (a * x + b)
\end{lstlisting}
\end{codeNOutput}
Forklar hvad parentesen i kaldet af \lstinline!printfn! funktionen gør godt for. Tilføj en linje i programmet, som udregner udtrykket $ax+b$ og binder resultatet til \lstinline!y!, og modificer kaldet til \lstinline!printfn! så det benytter denne nye binding. Er det stadig nødvendigt at bruge parentes?
