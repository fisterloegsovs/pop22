Fakultetsfunktionen kan skrives som,
  \begin{equation}
    n! = \prod_{i=1}^n i = 1\cdot 2\cdot \ldots \cdot n
  \end{equation}
  \begin{enumerate}
  \item Skriv en funktion
    \begin{quote}
      \mbox{\lstinline!fac : n:int -> int!}
    \end{quote}
    som benytter en \lstinline!while! løkke, en tællevariablen og en lokal variable til at beregne fakultetsfunktionen.
    % \item Lav en variant
    %   \begin{quote}
    %     \mbox{\lstinline!recFac : n:int -> int!}
    %   \end{quote}
    %   som benytter rekursion og ingen variable til at beregne fakultetsfunktionen.
    % \item Afprøv begge funktioner ved at lave et program, som laver en tabel med 3 kolonner \lstinline!n!, \mbox{\lstinline!fac n!} og \mbox{\lstinline!recFac n!}, og sikr dig at de 2 funktioner regner rigtigt.
    % \item Hvad er det største $n$, som disse funktioner kan beregne fakultetsfunktionen for, og hvad er begrænsningen?
  \item Skriv et program, som beder brugeren indtaste et tal \lstinline!n!, læser det fra tastaturet, og derefter udskriver resultatet af \lstinline!fac n!.
  \item Hvad er det største $n$, som funktionen kan beregne fakultetsfunktionen for, og hvad er begrænsningen? Lav en ny version, 
    \begin{quote}
      \mbox{\lstinline!fac : n:int -> int64!}
    \end{quote}
    som benytter \lstinline{int64} istedet for \lstinline{int} til at beregne fakultetsfunktionen. Hvad er nu det største $n$, som funktionen kan beregne fakultetsfunktionen for?
  \end{enumerate}
