\documentclass[a4paper]{article}

\usepackage[utf8x]{inputenc}
\usepackage{latexsym}
\usepackage[danish]{babel}
\usepackage{graphicx}
\usepackage{hyperref}
\usepackage[all]{hypcap}
\usepackage{enumerate}
\usepackage[margin=2.5cm]{geometry}

\begin{document}
\title{Programmering og Problemløsning\\
Datalogisk Institut, Københavns Universitet\\
Uge(r)seddel 5 -- individuel opgave}

\author{Torben Mogensen}
\date{Deadline 6. oktober}

\maketitle

\noindent
I denne periode skal I arbejde individuelt.  Formålet er at arbejde
med lister, mængder (sets) og afbildninger (maps), og specielt med
biblioteksfunktioner, der arbejder på disse typer.

Opgaverne i denne uge er delt i øve- og afleveringsopgaver.

\subsubsection*{Øveopgaverne er:}

\begin{description}
\item[Ø5.1.] HR: 5.1, 5.2.

\item[Ø5.2.] Definer en funktion \texttt{map} ækvivalent til
  \texttt{List.map} ved brug af \texttt{List.fold} eller
  \texttt{List.foldBack}.

\item[Ø5.3.] Givet funktionen \texttt{flip : ('a -> 'b - 'c) -> ('b -> 'a
  - 'c)} defineret ved

\texttt{let flip f x y = f y x}

lav definition af en funktion \texttt{fold} ækvivalent med
\texttt{List.fold} ved at bruge funktionerne \texttt{List.rev} og
\texttt{List.foldBack}.

\item[Ø5.4.] Brug \texttt{flip} fra opgave \textbf{5.3} til at definere
  en funktion \texttt{foldBack} ækvivalent til \texttt{List.foldBack}
  ved at bruge funktionerne \texttt{List.rev} og \texttt{List.fold}.

\item[Ø5.5.]  Definer en funktion \texttt{ofList} ækvivalent til
  \texttt{Set.ofList} ved at bruge et udvalg af funktionerne
  \texttt{Set.add}, \texttt{List.fold} og \texttt{List.foldBack}.

\item[Ø5.6.] En snedig programmør definerer en sorteringsfunktion med
  definitionen

\texttt{ssort xs = Set.toList (Set.ofList xs)}

For eksempel giver \texttt{ssort [4; 3; 7; 2]} resultatet \texttt{[2;
    3; 4; 7]}.

Diskuter, om programmøren faktisk er så snedig, som han tror.

\item[Ø5.7.] Undersøg, hvad der sker, hvis man bruger \texttt{Map.add}
  til at tilføje et nøgle/værdipar med en nøgle, der allerede findes i
  afbildningen.  Diskuter fordele og ulemper ved denne opførsel.

\end{description}

\subsubsection*{Afleveringsopgaven er:}


\begin{description}

\item[A5.1.] Brug funktionerne fra Tabel 5.1 i HR (side 94) til at
  definere en funktion \texttt{concat : 'a list list -> 'a list}, der
  sammensætter en liste af lister til en enkelt liste.

F.eks.\ skal \texttt{concat [[2]; [6; 4]; [1]]} give resultatet
\texttt{[2; 6; 4; 1]}.

\item[A5.2.] Brug funktionerne fra Tabel 5.1 i HR (side 94) til at
  definere en funktion \texttt{gennemsnit : float list -> float
    option}, der finder gennemsnittet af en liste af kommatal, såfremt
  dette er veldefineret, og \texttt{None}, hvis ikke.  Forsøg at lave
  din løsning, så den kun laver et gennemløb af listen.

\item[A5.3.] Definer en funktion \texttt{toList} ækvivalent til
  \texttt{Set.toList} ved at bruge \texttt{Set.fold} eller
  \texttt{Set.foldBack}.

\end{description}

\noindent
Afleveringsopgaven skal afleveres som både \LaTeX, den genererede PDF,
samt en fsx fil med løsningen for hver delopgave, navngivet efter
opgaven (f.eks.\ \texttt{A5-1.fsx}), som kan oversættes med
fsharpc, og hvis resultat kan køres med mono.  Det hele samles i en
zip-fil med sædvanlig navnekonvention (se tidligere ugesedler).


\vspace{1ex}

\hfill God fornøjelse

\section*{Ugens nød 2}

Vi vil i udvalgte uger stille særligt udfordrende og sjove opgaver,
som interesserede kan løse.  Det er helt frivilligt at lave disse
opgaver, som vi kalder ``Ugens nød'', men der vil blive givet en
mindre præmie til den bedste løsning, der afleveres i Absalon.

Denne uges opgave omhandler afbildninger (maps).

Der er stor lighed mellem funktioner og \emph{maps} som defineret i
afsnit 5.3 i HR.  Vi definerer derfor typen

\vspace{1ex}

\texttt{type myMap<'a,'b> = ('a -> 'b) * Set<'a>}

\vspace{1ex}

\noindent
hvor funktionen definerer afbildningens funktion, og mængden bruges
til at beskrive domænet for afbildningen.

\begin{description}

\item[Nød 2.1.] Definer for \texttt{myMap<'a,'b>} alle de funktioner
  fra Table 5.4 i HR, der er defineret for \texttt{Map<'a,'b>}.  Mere
  præcist skal alle forekomster af \texttt{Map<'a,'b>} i
  typesignaturerne erstattes af \texttt{myMap<'a,'b>}, og funktionerne
  skal gøre det, der er beskrevet i tabellen.

\end{description}

\noindent
Der skal uploades både en \LaTeX-fil, der beskriver fremgangsmåden,
samt en fsx fil, der indeholder definitionerne.  Navngivningen af
filerne er ikke vigtig.


\end{document}

