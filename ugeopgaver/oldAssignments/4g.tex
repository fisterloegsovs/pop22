\documentclass[a4paper]{article}

\usepackage[utf8x]{inputenc}
\usepackage{latexsym}
\usepackage[danish]{babel}
\usepackage{graphicx}
\usepackage{hyperref}
\usepackage[all]{hypcap}
\usepackage{enumerate}
\usepackage[margin=2.5cm]{geometry}

\begin{document}
\title{Programmering og Problemløsning\\
Datalogisk Institut, Københavns Universitet\\
Uge(r)seddel 4 -- gruppeopgave}

\author{Torben Mogensen}
\date{Deadline 29. september}

\maketitle

\noindent
I denne periode skal I arbejde i grupper.  Formålet er at arbejde med lister.

Opgaverne i denne uge er delt i øve- og afleveringsopgaver.

\subsubsection*{Øveopgaverne er:}

\begin{description}
\item[4.1.] HR: 4.1, 4.7, 4.8, 4.14, 4.16.

\item[4.2.] En tabel kan repræsenteres som en liste af lister, hvor alle
  listerne er lige lange.

Listen \texttt{[[1; 2; 3]; [4; 5; 6]]} repræsenterer for eksempel tabellen

\[\left [\begin{array}{rrr}
1 & 2 & 3 \\
4 & 5 & 6
\end{array}
\right ]\]

\begin{enumerate}[i]
\item Lav en funktion \texttt{isTable : 'a list list -> bool}, der
  givet en liste af lister afgør, om det er en lovlig ikke-tom tabel,
  altså om alle listerne har ens længde, og at der er mindst en liste
  med mindst et element.

\item Lav en funktion \texttt{transpose : 'a list list -> 'a list
  list}, der \emph{transponerer} en tabel.  Transponering er spejling over
  diagonalen, så den transponerede tabel til den herover viste tabel er

\[\left [\begin{array}{rr}
1 & 4 \\
2 & 5 \\
3 & 6
\end{array}
\right ]\]

\noindent
Kaldet \texttt{transpose [[1; 2; 3]; [4; 5; 6]]} skal altså returnere \texttt{[[1; 4]; [2; 5]; [3; 6]]}.

Bemærk, at \texttt{transpose (transpose t) = t}, hvis \texttt{t} er en
tabel.  Hvis argumentet til \texttt{transpose} ikke er en lovlig
tabel, skal en passende fejl rapporteres.

\end{enumerate}

\end{description}

\subsubsection*{Afleveringsopgaven er:}


\begin{description}
\item[4.3.] HR: 4.4, 4.12 og 4.15.

\item[4.4.] Lav en funktion \texttt{removeDuplicates : 'a list -> 'a list
  when 'a : equality}, som fjerner duplikater i en liste.  For
  eksempel skal kaldet \texttt{removeDuplicates [1; 2; 1; 3: 2]}
  give resultatet \texttt{[1; 2; 3]}.  Bemærk, at den første
  forekomst af et givet elememt bevares, mens de øvrige forekomster
  fjernes.  De bevarede elementer bevarer deres indbyrdes position.

\textbf{Vink:} I kan bruge \texttt{isMember} funktionen fra side 79 i HR.

\end{description}

\noindent
Afleveringsopgaven skal aflevres som både \LaTeX, den genererede PDF,
samt en fsx tekstfil med løsningen for hver delopgave, som kan
oversættes med fsharpc, og hvis resultat kan køres med mono.  Det hele
samles i en zip-fil med navnekonventionen

\[
{<}\textrm{instructor's initials}{>}\underline{~}{<}\textrm{firstname.lastname}_1{>}\underline{~}\ldots\underline{~}{<}\textrm{firstname.lastname}_n{>}\underline{~}{<}\textrm{exercise-number}{>}.\textrm{zip}
\]

\noindent
I zip-filen skal en delopgave navngives med opgavenummer, således at filen for opgave 4.3 hedder opg4\underline{~}3.fsx osv.

Aflever kun en gang per gruppe, og brug Absalon's
gruppeafleveringsfunktion.

\vspace{1ex}

\hfill God fornøjelse

\section*{Ugens nød 1}

Vi vil i udvalgte uger stille særligt udfordrende og sjove opgaver,
som interesserede kan løse.  Det er helt frivilligt at lave disse
opgaver, som vi kalder ``Ugens nød'', men der vil blive givet en
mindre præmie til den bedste løsning, der afleveres i Absalon.

Ugens nød i denne uge omhandler en variant af spillet "`Minestryger"'
(\url{http://da.wikipedia.org/wiki/Minestryger_%28spil%29}).

I vores variant er det for alle felter på forhånd kendt, hvor mange
bomber, der i alt er i de otte nabofelter samt feltet selv.  Det
gælder både felter med og felter uden bomber.

Et eksempel på et sådant spil er

\begin{verbatim}
        221
        232
        232
\end{verbatim}
  
svarende til bombeplaceringen

\begin{verbatim}
        000
        110
        001
\end{verbatim}

\noindent
hvor tomme felter er angivet med \texttt{0} og bomber med \texttt{1}.

Der er ikke altid en entydig løsning.  For eksempel vil spillet

\begin{verbatim}
        11
        11
\end{verbatim}
 
have fire løsninger, blandt andet:

\vspace{1ex}

\noindent
\begin{tabular}{@{\texttt{~~~~~~~~}}lcl}
\texttt{01} & og & \texttt{10} \\
\texttt{00} &  & \texttt{00}
\end{tabular}
\vspace{1ex}

Der er heller ikke altid løsninger, idet for eksempel

\begin{verbatim}
        10
        00
\end{verbatim}
 
ikke har en løsning.

Vi repræsenterer et spil som en tabel af heltal, som alle ligger
mellem 0 og 9.  Løsninger repræsenteres som tabeller af heltal, der
alle har værdi 0 eller 1.

\begin{enumerate}[Nød {1}.1]

\item funktion \texttt{minelaegger : int list list -> int list list},
  som givet en \emph{løsning} returnerer et spil.  Det kan antages, at
  input er en liste af lister af tal, der alle er enten 0 eller 1.

For eksempel skal kaldet
\texttt{minelaegger [[0; 0; 0]; [1; 1; 0]; [0; 0; 1]]}
returnere spillet \texttt{[[2; 2; 1]; [2; 3; 2]; [2; 3; 2]]}.


\item Lav en funktion \texttt{minestryger : int list list -> int list list
  option}, som givet et spil finder en løsning, hvis en sådan findes,
og returnerer \texttt{None}, hvis der ingen løsninger er.  Det kan antages, at input er en liste af lister af tal mellem 0 og 9.

For eksempel skal kaldet
\texttt{minestryger [[2; 2; 1]; [2; 3; 2]; [2; 3; 2]]}
returnere løsningen
\texttt{Some [[0; 0; 0]; [1; 1; 0]; [0; 0; 1]]},
mens 
\texttt{minestryger [[1; 0]; [0; 0]]}
returnerer \texttt{None}, og
\texttt{minestryger [[1; 1]; [1; 1]]}
returnerer en af de fire mulige løsninger, f.eks.
\texttt{Some [[0; 0]; [1; 0]]}


Det er ikke vigtigt, om løsningen findes hurtigt, men hvis flere
korrekte løsninger indleveres, gives præmien til den hurtigste.

\end{enumerate}

\noindent
Der skal uploades både en \LaTeX-fil, der beskriver fremgangsmåden,
samt en fsx fil, der indeholder definitionerne af de to funktioner.
Navngivningen af filerne er ikke vigtig.

Endnu et par eksempelspil til afprøvning er angivet herunder.

\begin{verbatim}
        22222      33333      44444      23232      23321
        22222      33333      44444      24453      34421
        22222      33333      44444      45543      45532
        22222      33333      44444      34443      23321
        22222      33333      44444      33222      12221
\end{verbatim}

\end{document}

