\documentclass[a4paper,12pt]{article}

\usepackage[margin=2cm]{geometry}
\usepackage[T1]{fontenc}
\usepackage[utf8]{inputenc}
\usepackage[danish]{babel}
\usepackage{listings}
\usepackage{graphicx}
\graphicspath{{figures/}}
\setlength{\parindent}{0cm}
\setlength{\parskip}{1em}
\usepackage{hyperref}

\title{Programmering og Problemløsning\\Datalogisk Institut,
  Københavns Universitet\\Uge(r)seddel 9 - individuel opgave}
\author{Jon Sporring og Torben Mogensen}
\date{Deadline  8.\ december}

\begin{document}
\maketitle

I denne periode skal I arbejde individuelt. Formålet er at arbejde med:
\begin{itemize}
\item mut\'{e}rbare variable
\item sekventiel eksekvering
\item while løkker
\item arrays
\end{itemize}

Opgaverne for denne uge er delt i øve- og afleveringsopgaver. 

Til øvelserne på alm.\ skema forventer vi at I arbejder efter nedenstående plan.
\begin{description}
\item[Mandag-Tirsdag 23-24/11:]~\\[0cm]
  Der arbejdes med 8g, og følgende opgaver:
  \begin{description}
  \item[9.0] Lav en rekursiv funktion med argmentet $n$, som adderer
    tallene 1 til $n$. Lav derefter en imperativ version af det samme
    program uden brug af rekursion men med 1 eller flere mut\'{e}rbar
    variable og \lstinline|while| nøgleordet.
  \item[9.1] HR: 8.1, 8.2
  \item[9.2] Lav en rekursiv funktion som udregner værdien af et
    $n$'te ordens polynomium. Funktionen skal tage 2 argumenter: en
    liste af koefficienter $[a_0, a_1, \dots, a_{n-1}]$
    samt evaluaeringspunktet $x$, og polynomiets orden angives ved
    listens længde. Hint: et $n$'te ordens polynomium kan skrives
    rekursivt som:
    \begin{equation}
      f(x) = a_0+ a_1 x + a_2x^2+\dots a_{n-1}x^{n-1} = a_0 +
      x\left(a_1 + x\left(a_2 + x\left(\dots\right)\right)\right)
    \end{equation}
    Lav derefter en ikke-rekursiv funktion, som udregner værdien af
    den samme polynomium. Benyt de tidligere udviklede unit-test som
    demonstration for at de 2 funktioner regner rigtigt.
  \end{description}
\item[Fredag 27/11:]~\\[-5mm]
  \begin{description}
  \item[9.3] HR: 8.5, 8.6
  \end{description}
\item[Mandag-tirsdag 30/11-1/12:]~\\[-5mm]
   \begin{description}
  \item[9.4] Programmer spillet Game of life
    (\url{https://en.wikipedia.org/wiki/Conway%27s_Game_of_Life}),
    uden rekursive funktioner og ved hjælp af en eller flere
    \lstinline|array| variable. Reglerne er angivet på hjemmesiden
    under 'Rules', og eksempler på udviklingsmønstre ser I under
    'Examples of patterns'. Bemærk at for at opnå de viste mønstre er
    det essentielt at ændringerne skal ske simultant på hele brættet
    på \'{e}n gang. Output skal vises som tekst i terminalen.
  \end{description}
\item[Fredag 4/12:]~\\[-5mm]
  \begin{description}
  \item[9.5] Udvid jeres program Game of Life, så outputet vises i et
    vindue ved brug af Windows forms.
  \end{description}
\end{description}

Afleveringsopgaven er:
\begin{description}
\item[9.6] Lav en oversættelse af \lstinline|fsharpc|'s dokumentationsfiler fra \lstinline|xml| til \LaTeX. Med oversættelsesargumentet \lstinline|-doc:| produceres en \lstinline|xml| fil, hvilket desværre hverken er særlig køn eller læsevenlig. Den følger derimod en fast struktur, f.eks.:
\begin{lstlisting}
<?xml version="1.0" encoding="utf-8"?>
<doc>
<assembly><name>Vector</name></assembly>
<members>
<member name="T:Vector.Vector">
<summary>

 A demonstration of defining a module from H &amp; R, Functional
 Programming Using F#. Note: Bad style, better use augmented types.

 How to compile:
 &lt;code&gt;
 fsharpc --doc:Vector.xml -a Vector.fsi Vector.fs
 fsharpc --doc:testVector.xml -r Vector.dll testVector.fsx
 &lt;/code&gt;

 Author: Jon Sporring.
 Date: 2015/10/27
 A 2 dimensional vector type, whose elements are floats.
</summary>
</member>
<member name="M:Vector.coord(Vector.Vector)">
<summary>
 Get coordinates
</summary>
</member>
<member name="M:Vector.make(System.Double,System.Double)">
<summary>
 Make vector
</summary>
</members>
</doc>
\end{lstlisting}
I biblioteket \lstinline|Fsharp.Data|, som kan hentes her:
\begin{quote}
  \url{https://github.com/fsharp/FSharp.Data/zipball/release}
\end{quote}
findes en \lstinline|xml|-parser, som kan parse filer af ovennævnte type, pånær at først linje skal fjernes.

I skal skrive et program, som tager dokumentationsfiler, og som
minimum producerer \LaTeX filer, hvor "`\lstinline|<members>|"' og
"`\lstinline|</members>|"' oversættes til
"`\lstinline|\begin{description}|"' og
  "`\lstinline|\end{description}|"', og "`\lstinline|<member>|"'
"`\lstinline|</member>|"' skal oversættest til de tilhørende
"`\lstinline|\item[name]~\\ summary|"', hvor "`\lstinline|name|"' og
"`\lstinline|summary|"' erstattest med indholdet af tilsvarende felter
i "`\lstinline|xml|"' filen.
\end{description}
Afleveringsopgaven skal afleveres som både LaTeX, den genererede PDF,
samt en fsx tekstfil med løsningen, som kan oversættes med fsharpc, og
hvis resultat kan køres med mono. Det hele skal samles i en zip fil
efter sædvanlig navnekonvention:
\begin{quote}
  \lstinline|<instructor's-initial>_<firstname.lastname>_<exercise-number>.zip|
\end{quote}
I zip filen skal en delopgave navngives ved opgavenummer, således at
filen for opgave 9.6 hedder \lstinline|opg9_6.fsx|, osv..

\flushright God fornøjelse.
\end{document}

%%% Local Variables:
%%% mode: latex
%%% TeX-master: t
%%% End:
