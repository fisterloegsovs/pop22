\documentclass[a4paper,12pt]{article}

% Character set
\usepackage{cmap}
\usepackage[utf8]{inputenc}
\usepackage[T1]{fontenc} % ensure that all the characters in characterSets.tex prints
\usepackage{upquote} % \textcent
\usepackage{pifont} % add \ding, http://ctan.org/pkg/pifont

% A background text to prevent wide distribution
\usepackage{draftwatermark}
\SetWatermarkText{DRAFT}
\SetWatermarkScale{6}
\SetWatermarkLightness{.95}

% Page setup
\usepackage[top=25mm,bottom=20mm,inner=20mm,outer=40mm,marginparsep=3mm,marginparwidth=35mm]{geometry}
\renewcommand{\floatpagefraction}{.8}%

% paragraph indentation is stupid
\setlength\parindent{0pt}
\setlength{\parskip}{1em}

% Globally defined colors
\usepackage[table,x11names]{xcolor}
\definecolor{alternateKeywordsColor}{rgb}{0.13,1,0.13}
\definecolor{keywordsColor}{rgb}{0.13,0.13,1}
%\definecolor{commentsColor}{rgb}{0,0.5,0}
\definecolor{commentsColor}{rgb}{0,0.5,0}
%\definecolor{stringsColor}{rgb}{0.9,0,0}
\definecolor{stringsColor}{rgb}{0,0,0.5}
\definecolor{light-gray}{gray}{0.95}
\definecolor{codeLineHighlight}{named}{SlateGray1}
%\definecolor{codeLineHighlight}{rgb}{0.975,0.975,0.975}
\definecolor{syntaxColor}{rgb}{0,.45,0}

\definecolor{headerRowColor}{rgb}{0.85,0.85,0.85}
\definecolor{oddRowColor}{rgb}{0.95,0.95,0.95}
\definecolor{evenRowColor}{rgb}{1,1,1}

% add check- and crossmarks, http://ctan.org/pkg/pifont
\newcommand{\cmark}{{\color{green}\ding{51}}}%
\newcommand{\xmark}{{\color{red}\ding{55}}}%

% Extra math stuff
\usepackage{amsmath,amssymb}

% Typeset chess
\usepackage{skak}

% Figures
\usepackage{graphicx}
\graphicspath{{figures/}}

% clickable url
\usepackage{url}

% figures
\usepackage{subfigure}

% Clickable table of content
\usepackage[pdfpagelabels]{hyperref}
%\usepackage{multirow}
\usepackage{makecell}

% Include label name in ref
\usepackage[noabbrev,capitalize]{cleveref}
\newcommand{\creflastconjunction}{, and\nobreakspace~}
\Crefformat{tcb@cnt@codeNOutput}{Listing~#2#1#3}
\crefformat{tcb@cnt@codeNOutput}{Listing~#2#1#3}
\crefrangeformat{tcb@cnt@codeNOutput}{Listing~#3#1#4\nobreakdash--#5#2#6}
\Crefrangeformat{tcb@cnt@codeNOutput}{Listing~#3#1#4\nobreakdash--#5#2#6}
\crefmultiformat{tcb@cnt@codeNOutput}{Listing~#2#1#3}{ and~#2#1#3}{, #2#1#3}{\creflastconjunction#2#1#3}
\Crefmultiformat{tcb@cnt@codeNOutput}{Listing~#2#1#3}{ and~#2#1#3}{, #2#1#3}{\creflastconjunction#2#1#3}
\crefrangeformat{table}{Table~#3#1#4\nobreakdash--#5#2#6}
\Crefrangeformat{table}{Table~#3#1#4\nobreakdash--#5#2#6}
\crefrangeformat{part}{Part~#3#1#4\nobreakdash--#5#2#6}
\Crefrangeformat{part}{Part~#3#1#4\nobreakdash--#5#2#6}

% paragraphs in tables
\usepackage{tabularx}

% formatting lists
\usepackage{enumitem}
%\setlist[description]{leftmargin=0pt,labelindent=0pt,itemindent=0pt}
%\setlist[description]{itemindent=-\leftmargin}

% latex comment environment
\usepackage{comment}

% UML
\usepackage{pgf-umlcd}
\renewcommand{\umltextcolor}{black} 
\renewcommand{\umlfillcolor}{black!5!white}
\renewcommand{\umldrawcolor}{teal}

% List of indices
\usepackage{xstring}
\usepackage{makeidx}
\usepackage{marginfix} % fixes marginpar location problem in 2 -page mode.
\newcommand{\idxs}[1]{\marginpar{$\cdot$~\parbox[t]{\linewidth}{\raggedright \expandarg\IfSubStr{#1}{@}{\StrBehind{#1}{@}}{#1}}}\index{#1}} % The parbox is too wide, since the line also includes cdot-space
\newcommand{\idxss}[1]{\index{#1}}
% Define a new command idx with an optional parameter, which if given is the key to the index
\makeatletter
\def\idx{\@ifnextchar[{\@with}{\@without}}
\def\@with[#1]#2{\emph{#2}\idxs{#1}}
\def\@without#1{\emph{#1}\idxs{#1}}
\makeatother
%\newcommand{\idx}[1]{\emph{#1}\idxs{#1}}
\newcommand{\keyword}[1]{{\lstinline[language=fsharp]|#1|}}
\newcommand{\lexeme}[1]{\mbox{``\lstinline[language=fsharp]|#1|''}}
\makeindex

% display tree like structures
\usepackage{qtree}

% We frame all listings and problems
\usepackage{tcolorbox}
\tcbuselibrary{listings}
\tcbuselibrary{raster}
\tcbset{%
  colframe=teal, %PaleGreen1!45!black,
  %coltitle=black,
  fonttitle=\bfseries, 
  leftrule=3mm,
  sharp corners=downhill,
  colback=black!5!white,
  left=1mm,
  top=1mm,
  right=1mm,
  bottom=1mm,
  middle=1mm,
  arc=2mm,
}
\newtcolorbox[auto counter]{problem}[1][]{%
  title=\textbf{Problem~\thetcbcounter},
  colframe=DeepSkyBlue1, %green!30!blue,
  #1}
\newcommand{\src}{src}
\newtcolorbox[auto counter]{codeNOutput}[2][]{%
  title=\textbf{Listing~\thetcbcounter#2},
  #1}

%% lstlisting stuff
\usepackage{listings} 
\def\lstfs#1{\mbox{\lstinline{{#1}}}}
% Get counters from references for firstnumber references in lstinputlisting
\usepackage{refcount}
\newcounter{lstFrom}
\newcounter{lstTo}
% Example: 
% \setcounterref{lstFrom}{dynamicScopeTracing:a1}
% \setcounterref{lstTo}{dynamicScopeTracing:a2}
% \lstinputlisting[firstline=\thelstFrom,lastline=\thelstTo,escapechar=|]{\src/dynamicScopeTracing.fsx}
\usepackage{lstlinebgrd}
\makeatletter
%The following sets the box compatible with tcolorbox setup
\def\lst@linebgrdcolor{\color{black!5!white}}
\def\lst@linebgrdsep{1em}
\def\lst@linebackgroundwidth{1em}
\def\lst@linebackgroundhighlight{\color{codeLineHighlight}}
\renewcommand{\lst@linebgrd}{%
  \ifx\lst@linebgrdcolor\empty
  \else
    \rlap{
       \lst@basicstyle\color{black!5!white} % tcolorbox background color
       \lst@linebgrdcolor{
          \kern-\dimexpr\lst@linebgrdsep\relax
          \lst@linebgrdcmd{\lst@linebgrdwidth}{\lst@linebgrdheight}{\lst@linebgrddepth}
       }
    }
  \fi
}
% Highlight a range of lines with green. Use \getrefnumber{label} for refs
\newcommand{\highlightRange}[2]{\ifnum\value{lstnumber}>\numexpr#1-1\ifnum\value{lstnumber}<\numexpr1+#2\lst@linebackgroundhighlight\fi\fi}
% \highlight conflicts with skak. Just rewriting, wonder what breaks in skak
\renewcommand{\highlight}[1]{\ifnum\value{lstnumber}=#1\lst@linebackgroundhighlight\fi}

% To use verbatimwrite to write listing to file, e.g., in conjunction with ebnfs
\usepackage{moreverb} 

\lstdefinelanguage{fsharp}{%
  keywords={abstract, and, as, assert, base, begin, class, default, delegate, do, done, downcast, downto, elif, else, end, exception, extern, false, finally, for, fun, function, global, if, in, inherit, inline, interface, internal, lazy, let, match, member, module, mutable, namespace, new, null, of, open, or, override, private, public, rec, return, sig, static, struct, then, to, true, try, type, upcast, use, val, void, when, while, with, yield},
  morekeywords={atomic, break, checked, component, const, constraint, constructor, continue, eager, fixed, fori, functor, include, measure, method, mixin, object, parallel, params, process, protected, pure, recursive, sealed, tailcall, trait, virtual, volatile},
  otherkeywords={ let!, return!, do!, yield!, use!},
  keywordstyle=\color{keywordsColor},
  % sensitive=true,
  basicstyle=\ttfamily\lst@ifdisplaystyle\small\fi, % make font small for listings but not for lstinline
  breaklines=true,
  breakatwhitespace=true
  showstringspaces=false,
  morecomment=[l][\color{commentsColor}]{///},
  morecomment=[l][\color{commentsColor}]{//},
  morecomment=[n][\color{commentsColor}]{(*}{*)},
  morecomment=[is][\color{white}]{(*//}{*)},
  morestring=[b]",
  literate={`}{\`}1,
  stringstyle=\color{stringsColor},
  showspaces=true,
  numbers=left,
  numbersep=6pt,
  numberstyle=\scriptsize\color{white},
  % aboveskip=0pt, 
  % belowskip=0pt,
  % resetmargins=true,
  % captionpos=b,
  backgroundcolor=\color{black!5!white},
}


\lstdefinelanguage{syntax}{%
  classoffset=0,
  keywords={abstract, and, as, assert, base, begin, class, default, delegate, do, done, downcast, downto, elif, else, end, exception, extern, false, finally, for, fun, function, global, if, in, inherit, inline, interface, internal, lazy, let, match, member, module, mutable, namespace, new, null, of, open, or, override, private, public, rec, return, sig, static, struct, then, to, true, try, type, upcast, use, val, void, when, while, with, yield, atomic, break, checked, component, const, constraint, constructor, continue, eager, fixed, fori, functor, include, measure, method, mixin, object, parallel, params, process, protected, pure, recursive, sealed, tailcall, trait, virtual, volatile, let!, return!, do!, yield!, use!},
  keywordstyle=\color{keywordsColor},
  % classoffset=1,
  % morekeywords={ident, expr, arg, format-string},
  % keywordstyle=\color{syntaxColor},
  % classoffset=0,
  otherkeywords={},
  basicstyle=\ttfamily\lst@ifdisplaystyle\small\fi, % make font small for listings but not for lstinline
  breaklines=true,
  breakatwhitespace=true
  showstringspaces=false,
  classoffset=0,
  morecomment=[l][\color{commentsColor}]{////},
  literate={`}{\`}1 {\{*}{{{\color{syntaxColor}\{}}}1 {*\}}{{{\color{syntaxColor}\}}}}1 {[*}{{{\color{syntaxColor}[}}}1  {*]}{{{\color{syntaxColor}]}}}1 {|*}{{{\color{syntaxColor}|}}}1, % {etc*}{{{\color{syntaxColor}...}}}3,
  moredelim  = **[is][\processmydelims]{<*}{*>}, % delete delimiters, typeset keywords. Don't know how to avoid the last...
  showspaces=true,
  numbers=left,
  numbersep=6pt,
  numberstyle=\scriptsize\color{white},
  backgroundcolor=\color{black!5!white},
}
%Tweek of deliminter and literate: https://tex.stackexchange.com/questions/203263/listings-package-custom-language-delimiter-match-left-side
\newcommand\processmydelimsend{}
\newcommand\processmydelims{%
  \renewcommand\processmydelimsend{\textcolor{syntaxColor}{>}\egroup}%
  \bgroup\color{syntaxColor}<\aftergroup\processmydelimsend%
}
% \makeatletter
% \newcommand\processhash{%
%   \ifnum\lst@mode=\lst@Pmode%
%     \bfseries%
%   \fi
%   \#%
% }
% \makeatother


\lstdefinelanguage{ebnf}{%
  keywords={},
  morekeywords={},
  otherkeywords={},
  keywordstyle=\color{keywordsColor},
  % sensitive=true,
  basicstyle=\fontfamily{pcr}\selectfont\lst@ifdisplaystyle\small\fi, 
  breaklines=true,
  breakatwhitespace=true
  morecomment=[s][\color{commentsColor}]{(*}{*)},
  morestring=[b]",
  morestring=[b]',
  alsoletter={\\},
  showstringspaces=false,
  % stringstyle=\color{stringsColor},
  % aboveskip=0pt, 
  % belowskip=0pt,
  % resetmargins=true,
  % captionpos=b,
  % backgroundcolor=\color{blue!10!white},
}
\lstdefinelanguage{console}{%
  keywords={},
  morekeywords={},
  otherkeywords={},
  basicstyle=\ttfamily\lst@ifdisplaystyle\small\fi, 
  breaklines=true,
  showstringspaces=false,
  % aboveskip=0pt,
  % belowskip=0pt,
  % resetmargins=true,
  % captionpos=b,
  % backgroundcolor=\color{green!10!white},
}
%\lstset{language=fsharp, frame=single}
\lstset{language=fsharp,showlines=false}
\makeatletter
\def\lst@visiblespace{ }
\makeatother

% input .fsx and .out listings from \src and display as code and result in same figure
% #1 = optional further arguments for lstinputlisting
% #2 = filename without suffix, and label
% #3 = caption
\newtcbinputlisting[use counter from=codeNOutput]{\fs}[3][]{%
  listing file={src/#2.fsx},
  listing and comment,
  listing options={language=fsharp,escapechar=§,#1},
  title=\textbf{Listing \thetcbcounter} {#2.fsx:\\#3},
  label={#2},
  comment={\lstinputlisting[language=console]{\src/#2.out}}
}

% dispaly fsharp code \src
% #1 = optional further arguments for lstinputlisting
% #2 = filename
% #3 = label
% #4 = caption
\newtcbinputlisting[use counter from=codeNOutput]{\fsharp}[4][]{%
  listing file={\src/#2},
  listing only,
  listing options={language=fsharp,escapechar=§,#1},
  title=\textbf{Listing \thetcbcounter} {#2:\\#4},
  label={#3},
}

% dispaly console file \src
% #1 = optional further arguments for lstinputlisting
% #2 = filename
% #3 = label
% #4 = caption
\newtcbinputlisting[use counter from=codeNOutput]{\console}[4][]{%
  listing file={\src/#2},
  listing only,
  listing options={language=console,escapechar=§,#1},
  title=\textbf{Listing \thetcbcounter} {#2:\\#4},
  label={#3},
}

\newtcbinputlisting[use counter from=codeNOutput]{\fsCode}[4]{%
  listing file={src/#1.fsx},
  listing only,
  listing options={language=fsharp,escapechar=§,#4},
  title=\textbf{Listing \thetcbcounter} {#1.fsx:\\#3},
  label={#2},
}

% dispaly ebnf file, no label
% #1 = optional further arguments for lstinputlisting
% #2 = filename
% #3 = caption
\newtcbinputlisting[use counter from=codeNOutput]{\ebnf}[3][]{%
  listing file={#2},
  listing only,
  colframe=black!50!white,
  listing options={language=ebnf,escapechar=§,#1},
  title=\textbf{Listing \thetcbcounter} {#3},
}

% dispaly syntax file, no label
% #1 = optional further arguments for lstinputlisting
% #2 = filename without suffix, and label
% #3 = caption
\newtcbinputlisting[use counter from=codeNOutput]{\syntax}[3][]{%
  listing file={#2},
  listing only,
  colframe=black!50!white,
  listing options={language=syntax,escapechar=§,#1},
  title=\textbf{Listing \thetcbcounter} {#3},
  label={#2}
}

\newtcbinputlisting[use counter from=codeNOutput]{\fsSignature}[4]{%
  listing file={src/#1.fsi},
  listing only,
  listing options={language=fsharp,escapechar=§,#4},
  title=\textbf{Listing \thetcbcounter} {#1.fsi:\\#3},
  label={#2},
}
\newtcbinputlisting[use counter from=codeNOutput]{\fsImplementation}[4]{%
  listing file={src/#1.fs},
  listing only,
  listing options={language=fsharp,escapechar=§,#4},
  title=\textbf{Listing \thetcbcounter} {#1.fs:\\#3},
  label={#2},
}

% dispaly output file .out from \src
% #1 = optional further arguments for lstinputlisting
% #2 = filename without suffix, and label
% #3 = caption
\newtcbinputlisting[use counter from=codeNOutput]{\fsOutput}[3][]{%
  listing file={src/#2.out},
  listing only,
  listing options={language=console,escapechar=§,#1},
  title=\textbf{Listing \thetcbcounter}: {#3},
  label={#2},
}

% dispaly output file .out from \src as an element in tabularx
% #1 = optional further arguments for lstinputlisting
% #2 = filename without suffix, and label
% #3 = caption
\newtcbinputlisting[use counter from=codeNOutput]{\fsOutputTabx}[3][]{%
  listing file={src/#2.out},
  listing only,
  width=\hsize,
  box align=top,
  listing options={language=console,escapechar=§,aboveskip=0pt,belowskip=0pt,emptylines=0,#1},
  title=\textbf{Listing \thetcbcounter}: {#3},
  label={#2},
}

\newcommand{\filename}[1]{\lstinline[language=console]{#1}}

% highlighted text snippets
\newcommand{\advice}[1]{\marginpar{Advice}{\textbf{#1}}}
\newcommand{\advanced}[1]{\marginpar{Advanced material}\textbf{#1}}

% sometimes we need to include hash sign as arguments
\begingroup\catcode`\#=12
\newcommand\hashchar{}%check that is doesn't exist
\gdef\hashchar{#}
\endgroup

% Scratch out math, used in test.tex
\usepackage{cancel}
%\newcommand{\bcancel}[1]{#1}

% Draw arrows between elements
\usepackage{tikz}
%\usepackage{sphack} % make overlays invisible where stated in text
\usetikzlibrary{arrows,shapes,calc,decorations.pathreplacing}
\newcommand{\tikzmark}[1]{\tikz[overlay,remember picture] \node (#1) {};}
\newcommand*{\DrawArrow}[3][]{%
  % #1 = draw options
  % #2 = left point
  % #3 = right point
  \begin{tikzpicture}[overlay,remember picture]
    %\draw [-latex, #1,ultra thick,red] ($(#2)+(0.1em,0.5ex)$) to ($(#3)+(0,0.5ex)$);
    \draw [-latex, #1,ultra thick,red] ($(#2) -(0,0.5ex)$) to ($(#3)+(0,2ex)$);
  \end{tikzpicture}%
}%
\newcommand*{\AddNote}[4]{%
  \begin{tikzpicture}[overlay, remember picture]
    \draw [decoration={brace,amplitude=0.5em},decorate,ultra thick,red]
    ($(#3)!([yshift=1.5ex]#1)!($(#3)-(0,1)$)$) -- ($(#3)!(#2)!($(#3)-(0,1)$)$)
    node [align=left, text width=0cm, pos=0.5, anchor=west, xshift=.2cm] {#4};
  \end{tikzpicture}
}%
\newcommand{\FrameArea}[2]{%
  % #1 = top left point
  % #2 = bottom right point
  % The overlay is drawn in the margin in order not to screw with
  % horizontal spacing.
  %\ifvmode\vspace*{-1.2em}\else\fi%
  \begin{tikzpicture}[overlay,remember picture]%
    \draw[red,rounded corners] ([shift={(-2pt,1.9ex)}] #1)  rectangle  ([shift={(2pt,-.9ex)}] #2);%
  \end{tikzpicture}\noindent % I don't know why this command shift to the right, but this seems to fix it.
}%

% One can write to a file during compilation with the following
% low-level code.
%  \newwrite\tempfile
%  \immediate\openout\tempfile=list.tex
%  \immediate\write\tempfile{Text to write to file}
%  \immediate\closeout\tempfile

\usepackage{xspace}
\newcommand{\monoVersion}{5.2.0\xspace}
\newcommand{\fsharpVersion}{4.1\xspace}


% Notes to self
\newcommand{\jon}[1]{\footnote{Jon: \textbf{#1}}}
%\renewcommand{\jon}[1]{}
\newcommand{\mael}[1]{\footnote{Mael: \textbf{#1}}}
%\renewcommand{\mael}[1]{}
\newcommand{\spec}[1]{\footnote{Spec: \textbf{#1}}}
\renewcommand{\spec}[1]{}

%%% Local Variables:
%%% TeX-master: "fsharpNotes"
%%% End:

\newcommand{\sbl}{Simple Jack}

\title{Programmering og Problemløsning\\Datalogisk Institut,
  Københavns Universitet\\Arbejdsseddel 10 - gruppeopgave}
\author{Jon Sporring and Christina Lioma}
\date{5.\ december -- 2. januar.\\Afleveringsfrist: onsdag d. 2. januar kl. 22:00}

\begin{document}
\maketitle

I denne periode skal I arbejde i grupper. Regler for individuelle afleveringsopgaver er beskrevet i "`Noter, links, software m.m."'$\rightarrow$"`Generel information om opgaver"'. Formålet er at arbejde med:
\begin{itemize}
\item Classes
\item Objects
\item Methods
\item Attributes
\item Nedarvning
\end{itemize}

Opgaverne for denne uge er delt i øve- og afleveringsopgaver. 

\section*{Øveopgave(r)}
\begin{enumerate}[label=10ø.\arabic*,start=0]
\item Implement a class \lstinline{Counter}. The class must have 3 methods:
\begin{itemize}
\item The constructor must make a counter field whose value initially is 0,
\item \lstinline{get} which returns the present value of the counter field, and
\item \lstinline{incr} which increases the counter field by 1.
\end{itemize}
Write a white-box test class that tests \lstinline{Counter}.

\item Implementér en klasse \texttt{Car} med følgende egenskaber. En bil har en specifik benzin effektivitet (målt i km/liter) og en bestemt mængde benzin i tanken. Benzin effektiviteten for en bil er specificeret med konstruktoren ved oprettelse af et \texttt{Car} objekt. Den indledende mængde benzin er 0 liter. Implementer følgende metoder til \texttt{Car} klassen:
  \begin{itemize}
  \item \texttt{addGas}: Tilføjer en specificeret mængde benzin til bilen.
  \item \texttt{gasLeft}: Returnerer den nuværende mængde benzin i bilen.
  \item \texttt{drive}: Bilen køres en specificeret distance, og bruger tilsvarende benzin. Hvis der ikke er nok benzin på tanken til at køre hele distancen kastes en undtagelse.
  \end{itemize}
  Lav også en klasse \texttt{CarTest} som tester alle metoder i \texttt{Car}.

\item Implement a class \lstinline{Moth}, which represents a moth that is attracted to light. The moth and the light live in a 2-dimensional coordinate system with axes $(x,y)$, and the light is placed at $(0,0)$. The moth must have a field for its position in a 2-dimensional coordinate system of floats. Objects of the \lstinline{Moth} class must have the following methods:
\begin{itemize}
\item The constructor must accept the initial coordinates of the moth.
\item \lstinline{moveToLight} which moves the moth in a straight line from its position halfway to the position of the light.
\item \lstinline{getPosition} which returns the moth's initial position.
\end{itemize}
Make a white-box test class and test the \lstinline{Moth} class.

\item I en ikke-så-fjern fremtid bliver droner massivt brugt til levering af varer købt på nettet.  Drone-trafikken er blevet så voldsom i dit område, at du er blevet bedt om at skrive et program som kan afgøre om droner flyver ind i hinanden. Antag at alle droner flyver i samme højde og at 2 droner rammer hinanden hvis der på et givent tidspunkt (kun hele minutter) er mindre end 5 meter imellem dem.  Droner flyver med forskellig hastighed (meter/minut) og i forskellige retninger. En drone flyver altid i en lige linje mod sin destination, og når destinationen er nået, lander dronen og kan ikke længere kollidere med andre droner.  Ved oprettelse af et \texttt{Drone} objekt specificeres start positionen, destinationen og hastigheden.  Implementér klassen \texttt{Drone} så den som minimum har attributterne og metoderne:
    \begin{itemize}
    \item \texttt{position} (attribut) : Angiver dronens position i (x, y) koordinater.
    \item \texttt{speed} (attribut) : Angiver distancen som dronen flyver for hvert minut.
    \item \texttt{destination} (attribut) : Angiver positionen for dronens destination i (x, y) koordinater.
    \item \texttt{fly} (metode) : Beregner dronens nye position efter et minuts flyvning.
    \item \texttt{isFinished} (metode) : Afgør om dronen har nået sin destination eller ej.
    \end{itemize}
    og klassen \texttt{Airspace} så den som minimum har attributterne og metoderne:
    \begin{itemize}
    \item \texttt{drones} (attribut) : En samling droner i luftrummet.
    \item \texttt{droneDist} (metode) : Beregner afstanden mellem to droner.
    \item \texttt{flyDrones} (metode) : Lader et minut passere og opdaterer dronernes positioner tilsvarende.
    \item \texttt{addDrone} (metode) : Tilføjer en ny drone til luftrummet.
    \item \texttt{willCollide} (metode) : Afgør om der sker en eller flere kollisioner indenfor et specificeret tidsinterval givet
      i hele minutter.
    \end{itemize}
    Test alle metoder i begge klasser. Opret en samling \texttt{Drone} objekter som du ved ikke vil medføre kollisioner, samt en anden samling som du ved vil medføre kollisioner og test om din \texttt{willCollide} metode virker korrekt.

\item Write a class \lstinline{Car} that has the following data properties:
  \begin{itemize}
  \item \lstinline{yearOfModel}: The car's year model.
  \item \lstinline{make}: The make of the car.
  \item \lstinline{speed}: The car's current speed.
  \end{itemize}
  The \lstinline{Car} class should have a constructor that accepts the car's year model and make as arguments. Set the car's initial speed to 0.  The \texttt{Car} class should have the following methods:
  \begin{itemize}
  \item \lstinline{accelerate}: The \lstinline{accelerate} method should add 5 to the \texttt{speed} attribute each time it is called.
  \item \lstinline{brake}: The \lstinline{brake} method should subtract 5 from the \texttt{speed} attribute each time it is called.
  \item \lstinline{getSpeed}: The \lstinline{getSpeed} method should return the current speed.
  \end{itemize}
  Design a program that instantiates a \lstinline{Car} object, and then calls the \lstinline{accelerate} method five times. After each call to the \lstinline{accelerate} method, get the current speed of the car and display it. Then call the \lstinline{brake} method five times. After each call to the \lstinline{brake} method, get the current speed of the car and display it.
  
  Extend class \lstinline{Car} with the attributes \v{addGas, gasLeft} from exercise \ref{ex:car}, and modify methods \lstinline{accelerate, break} so that the amount of gas left is reduced when the car accelerates or breaks. Call \lstinline{accelerate, brake} five times, as above, and after each call display both the current speed and the current amount of gas left.
  
  Test all methods. Create an object instance that you know will not run out of gas, and another object instance that you know will run out of gas and test that your \lstinline{accelerate, brake} methods work properly.
  
\item \label{inheritance:customer}
Write a \texttt{Person} class with data properties for a person's name, address, and telephone number. Next, write a class named \texttt{Customer} that is a subclass of the \texttt{Person} class. The \texttt{Customer} class should have a data property for a unique customer number and a Boolean data property indicating whether the customer wishes to be on a mailing list. Write a small program, which makes an instance of the \texttt{Customer} class.

\item Draw an UML class diagram for the following structure:

A \texttt{Employee} class that keeps data properties for the following pieces of information: 
\begin{itemize}
\item Employee name
\item Employee number
\end{itemize}

A subclass \texttt{ProductionWorker} that is a subclass of the \texttt{Employee} class. The \texttt{ProductionWorker} class should keep data properties for the following information:
\begin{itemize}
\item Shift number (an integer, such as 1 or 2)
\item Hourly pay rate
\end{itemize}

A class \texttt{Factory} which has one or more instances of \texttt{ProductionWorker} objects.

\item Cheetahs, antelopes and wildebeests are among the world's fastest mammals. This exercise asks you to simulate a race between them. You are not asked to simulate their movement on some plane, but only some of the conditions that affect their speed when running a certain distance. 
  
  Your base class is called \texttt{Animal} and has these properties: 
  \begin{itemize}
  \item The amount of food needed daily (measured in kilograms)
  \item The weight of the animal (measured in kilograms)
  \item The maximum speed of the animal (measured in kilometres per hour)
  \item The current speed of the animal (measured in kilometres per hour)
  \end{itemize}
  The \texttt{Animal} class should have a primary constructor that takes two arguments: the animal's weight and the animal's maximum speed. The \texttt{Animal} class should also have an additional constructor that takes as input only the animal's maximum speed and generates the animal's weight randomly within the range of 70 - 300 kg. The \texttt{Animal} class should have two methods:
  \begin{itemize}
  \item The first method should set the current speed of the animal proportionately to its food intake and maximum speed as follows: if the animal eats 100\% of the amount of food it needs daily, the animal's current speed should be its maximum speed; if the animal eats 50\% of the amount of food it needs daily, the animal's current speed should be 50\% of its maximum speed, and so on.
  \item The second method should set the amount of food needed daily proportionately to the animal's weight as follows: the animal should eat half its own weight in food every day (if the animal weighs 50 kg, it should eat 25kg of food daily).
  \end{itemize}
  
  
  Create a subclass \texttt{Carnivore} that inherits everything from class \texttt{Animal}, and modifies the second method as follows: the animal should eat 8\% of its own weight in food every day.
  
  Create a subclass \texttt{Herbivore} that inherits everything from class \texttt{Animal}, and modifies the second method as follows: the animal should eat 40\% of its own weight in food every day.
  
  Create an instance of \texttt{Carnivore} called \texttt{cheetah} and two instances of \texttt{Herbivore} called \texttt{antelope, wildebeest}. Set their weight and maximum speed to:
  \begin{itemize}
  \item cheetah: 50kg, 114km/hour
  \item antelope: 50kg, 95km/hour
  \item wildebeest: 200kg, 80km/hour
  \end{itemize}
  
  Generate a random percentage between 1 - 100\% (inclusive) separately for each instance. This random percentage represents the amount of food the animal eats with respect to the amount of food it needs daily. E.g., if you generate the random percentage 50\% for the antelope, this means that the antelope will eat 50\% of the amount it should have eaten (as decided by the second method). 
  
  For each instance, display the random percentage you generated, how much food each animal consumed, how much food it should have consumed, and how long it took for the animal to cover 10km. 
  Repeat this 3 times (generating different random percentages each time), and declare winner the animal that was fastest on average all three times. If there is a draw, repeat and recompute until there is a clear winner.
  
  Write a white-box test of your classes. %You should include a UML class diagram, comment your code and describe in max. 2 pages (excluding the UML class diagram) what your program does and how you have tested the methods. 
  
  \textbf{Optional extra:} repeat the race without passing as input argument the weight of each animal (i.e. letting the additional constructor generate a different random weight for each instance).

\end{enumerate}

\section*{Afleveringsopgave(r)}
Denne opgave omhandler undtagelser (exceptions), option typer og Stirlings formel. Stirlings formel er en approximation til fakultetsfunktionen via $$\ln n! \simeq n \ln n - n.$$
\begin{enumerate}[label=10g.\arabic*,start=0]
\item \documentclass{article}
\usepackage[school,simplified]{pgf-umlcd}
\usepackage[margin=2.5cm]{geometry}
\usepackage[utf8]{inputenc}
\usepackage[T1]{fontenc} % ensure that all the characters in characterSets.tex prints

\begin{document}
\section*{Classes}
\begin{center}
  \begin{tikzpicture}
    \begin{class}[text width=5cm]{animal}{0,0}
    \end{class}
    \begin{class}[text width=5cm]{moose}{-3,-2}
    \end{class}
    \begin{class}[text width=5cm]{wolf}{-3,2}
    \end{class}
    \begin{class}[text width=5cm]{environment}{-6,0}
    \end{class}
  \end{tikzpicture}
\end{center}

\section*{Inheritance}
\begin{center}
  \begin{tikzpicture}
    \begin{class}[text width=5cm]{animal}{0,0}
    \end{class}
    \begin{class}[text width=5cm]{moose}{-3,-2}
      \inherit{animal}
    \end{class}
    \begin{class}[text width=5cm]{wolf}{-3,2}
      \inherit{animal}
    \end{class}
    \begin{class}[text width=5cm]{environment}{-6,0}
    \end{class}
  \end{tikzpicture}
\end{center}
\vspace*{1cm}

\begin{center}
  \begin{tikzpicture}
    \begin{class}[text width=5cm]{animal}{0,0}
    \end{class}
    \begin{class}[text width=5cm]{moose}{-3,-2}
      \inherit{animal}
    \end{class}
    \begin{class}[text width=5cm]{wolf}{3,-2}
      \inherit{animal}
    \end{class}
    \begin{class}[text width=5cm]{environment}{0,-4}
    \end{class}
  \end{tikzpicture}
\end{center}

\section*{Associations}
\begin{description}
\item[Association]~\\
  Host 'kender til' gæst
\item[Aggregation]~\\
  Host 'har kopi af' gæst
\item[Composition]~\\
  Host 'opretter afhængig og har kopi af' gæst. Når host slettes slettes gæst ligeså.
\end{description}

\begin{center}
  \begin{tikzpicture}
    \begin{class}[text width=5cm]{animal}{0,0}
    \end{class}
    \begin{class}[text width=5cm]{moose}{-3,-2}
      \inherit{animal}
    \end{class}
    \begin{class}[text width=5cm]{wolf}{3,-2}
      \inherit{animal}
    \end{class}
    \begin{class}[text width=5cm]{environment}{0,-4}
    \end{class}
    \composition{environment}{}{}{moose}
    \composition{environment}{}{}{wolf}
  \end{tikzpicture}
\end{center}


\section*{Relations}
\begin{center}
  \fbox{
    \begin{tikzpicture}
      % \draw[umlcd style dashed line] (0,4) --(8,4);
      % \draw[splitline] (0,3) --(8,3);
      \node [left] at (-2,5) {Compose:}; \node (composeLeft) [left] at
      (0,5) {Owner}; \node (composeRight) [right] at (8,5)
      {Dependent};
      \composition{composeLeft}{depObj}{1..*}{composeRight};
      
      \node [left] at (-2,4) {Aggregate:}; \node (aggregateLeft)
      [left] at (0,4) {Host}; \node (aggregateRight) [right] at (8,4)
      {Guest};
      \aggregation{aggregateLeft}{guestObj}{4}{aggregateRight};
      
      \node [left] at (-2,3) {\parbox[r]{2.5cm}{\raggedleft
          Unidirectional associate:}}; \node (uniAssociateLeft) [left]
      at (0,3) {Host}; \node (uniAssociateRight) [right] at (8,3)
      {Guest};
      \unidirectionalAssociation{uniAssociateLeft}{guestObj}{0..*}{uniAssociateRight};

      \node [left] at (-2,2) {Associate:}; \node (associateLeft)
      [left] at (0,2) {HostA}; \node (associateRight) [right] at (8,2)
      {HostB};
      \association{associateLeft}{objectsAinB}{0..1}{associateRight}{objectsBinA}{0..*}
      ;
      
      \node [left] at (-2,1) {Implement:}; \node (interfaceDerived)
      [left] at (0,1) {Derived}; \node (interfaceBase) [right] at
      (8,1) {Interface}; \draw[umlcd style implement line]
      (interfaceBase) -- (interfaceDerived);
      
      \node [left] at (-2,0) {Inherit:}; \node (inheritDerived) [left]
      at (0,0) {Derived}; \node (inheritBase) [right] at (8,0) {Base};
      \draw[umlcd style inherit line] (inheritBase) --
      (inheritDerived);
    \end{tikzpicture}
  }
\end{center}

\section*{Properties and methods}
\begin{center}
  \begin{tikzpicture}
    \begin{class}[text width=10cm]{animal}{0,0}
      \attribute{symbol : symbol}
      \attribute{position : position option with set}
      \attribute{reproduction : int}
      \operation{new (symb * repLen : symbol * int) : animal}
      \operation{ToString  unit : string}
      \operation{resetReproduction  unit : unit}
      \operation{updateReproduction  unit : unit}
    \end{class}
  \end{tikzpicture}
\end{center}
\vspace*{1cm}

\begin{center}
  \begin{tikzpicture}
    \begin{class}[text width=10cm]{animal}{0,0}
      \attribute{symbol : symbol}
      \attribute{position : position option with set}
      \attribute{reproduction : int}
      \operation{new (symb * repLen : symbol * int) : animal}
      \operation{ToString  unit : string}
      \operation{resetReproduction  unit : unit}
      \operation{updateReproduction  unit : unit}
    \end{class}
    \begin{class}[text width=7cm]{moose}{-4,-5}
      \inherit{animal}
      \attribute{\dots}
      \operation{\dots}
      \operation{new (repLen : int) : moose}
      \operation{tick  unit : moose option}
    \end{class}
    \begin{class}[text width=7cm]{wolf}{4,-5}
      \inherit{animal}
      \attribute{\dots}
      \attribute{hunger : int}
      \operation{\dots}
      \operation{new (repLen * hungLen : int * int) : wolf}
      \operation{resetHunger  unit : unit}
      \operation{updateHunger  unit : unit}
      \operation{tick  unit : wolf option}
    \end{class}
    \begin{class}[text width=\textwidth]{environment}{0,-10}
      \attribute{board : board}
      \attribute{count : int}
      \attribute{size : int}
      \operation{new (boardWidth * NMooses * mooseRepLen * NWolves * wolvesRepLen * wolvesHungLen * verbose\newline\hspace*{1cm} : int * int * int * int * int * int * bool) : environment}
      \operation{resetHunger  unit : unit}
      \operation{updateHunger  unit : unit}
      \operation{tick  unit : moose option}
    \end{class}
    \composition{environment}{}{}{moose}
    \composition{environment}{}{}{wolf}
  \end{tikzpicture}
\end{center}

\section*{Krig}
\noindent Navneord:
\begin{center}
  \begin{tikzpicture}
    \begin{class}[text width=5cm]{spiller}{0,0}
    \end{class}
    \begin{class}[text width=5cm]{kort}{-3,-2}
    \end{class}
    \begin{class}[text width=5cm]{bunke}{3,-2}
    \end{class}
    \begin{class}[text width=5cm]{billedside}{-3,-4}
    \end{class}
    \begin{class}[text width=5cm]{vaerdi}{3,-4}
    \end{class}
  \end{tikzpicture}
\end{center}
\vspace*{1cm}
\noindent Associationer:
\begin{center}
  \begin{tikzpicture}
    \begin{class}[text width=5cm]{spiller}{0,0}
    \end{class}
    \begin{class}[text width=5cm]{bunke}{0,-2}
    \end{class}
    \begin{class}[text width=5cm]{kort}{0,-4}
    \end{class}
   \begin{class}[text width=5cm]{vaerdi}{0,-6}
    \end{class}
    \composition{spiller}{}{}{bunke}
    \composition{bunke}{}{}{kort}
    \composition{kort}{}{}{vaerdi}
  \end{tikzpicture}
\end{center}
\vspace*{1cm}
\noindent Properties:
\begin{center}
  \begin{tikzpicture}
    \begin{class}[text width=5cm]{spiller}{0,0}
      \attribute{stak : bunke}
    \end{class}
    \begin{class}[text width=5cm]{bunke}{0,-2}
      \attribute{lst : kort}
    \end{class}
    \begin{class}[text width=5cm]{kort}{0,-4}
      \attribute{v : værdi}
    \end{class}
    \begin{class}[text width=5cm]{vaerdi}{0,-6}
    \end{class}
    \composition{spiller}{}{}{bunke}
    \composition{bunke}{}{}{kort}
    \composition{kort}{}{}{vaerdi}
  \end{tikzpicture}
\end{center}
\vspace*{1cm}
\noindent Methods:
\begin{center}
  \begin{tikzpicture}
    \begin{class}[text width=5cm]{spiller}{0,0}
      \attribute{stak : bunke}
      \operation{afgivKort unit : kort}
      \operation{modtagKort kort : unit}
    \end{class}
    \begin{class}[text width=5cm]{bunke}{0,-3}
      \attribute{lst : kort}
      \operation{afgivKort kort : unit}
      \operation{modtagKort kort : unit}
    \end{class}
    \begin{class}[text width=5cm]{kort}{0,-6}
      \attribute{v : værdi}
      \operation{sammenlign kort : bool}
    \end{class}
    \begin{class}[text width=5cm]{vaerdi}{0,-9}
    \end{class}
    \composition{spiller}{}{}{bunke}
    \composition{bunke}{}{}{kort}
    \composition{kort}{}{}{vaerdi}
  \end{tikzpicture}
\end{center}

\noindent Correction:
\begin{center}
  \begin{tikzpicture}
    \begin{class}[text width=5cm]{spiller}{0,0}
      \attribute{stak : bunke}
      \operation{afgivKort unit : kort}
      \operation{modtagKort kort : unit}
    \end{class}
    \begin{class}[text width=5cm]{bunke}{-3,-3}
      \attribute{lst : kort}
      \operation{afgivKort kort : unit}
      \operation{modtagKort kort : unit}
    \end{class}
    \begin{class}[text width=5cm]{kort}{3,-3}
      \attribute{v : værdi}
      \operation{sammenlign kort : bool}
    \end{class}
    \begin{class}[text width=5cm]{vaerdi}{3,-6}
    \end{class}
    \composition{spiller}{}{}{bunke}
    \aggregation{bunke}{}{}{kort}
    \composition{kort}{}{}{vaerdi}
    \aggregation{spiller}{}{}{kort};
  \end{tikzpicture}
\end{center}

\noindent Correction2:
\begin{center}
  \begin{tikzpicture}
    \begin{class}[text width=10cm]{bord}{3,4}
      \attribute{deltagere : spiller []}
      \operation{uddelKort (spiller1 * spiller2 : spiller * spiller) : unit}
      \operation{spilOmgang (spiller1 * spiller2 : spiller * spiller) : bool}
    \end{class}
    \begin{class}[text width=5cm]{spiller}{0,0}
      \attribute{stak : bunke}
      \operation{afgivKort unit : kort}
      \operation{modtagKort kort : unit}
    \end{class}
    \begin{class}[text width=5cm]{bunke}{-3,-3}
      \attribute{lst : kort}
      \operation{afgivKort kort : unit}
      \operation{modtagKort kort : unit}
    \end{class}
    \begin{class}[text width=5cm]{kort}{3,-3}
      \attribute{v : værdi}
      \operation{sammenlign kort : bool}
    \end{class}
    \begin{class}[text width=5cm]{vaerdi}{3,-6}
    \end{class}
    \composition{bord}{}{}{spiller}
    \composition{spiller}{}{}{bunke}
    \aggregation{bunke}{}{}{kort}
    \composition{kort}{}{}{vaerdi}
    \aggregation{spiller}{}{}{kort}
    \aggregation{bord}{}{}{kort}
  \end{tikzpicture}
\end{center}

\end{document}

\end{enumerate}
 
Afleveringsopgaven skal afleveres som et antal fsx tekstfiler navngivet efter opgaven, som f.eks. \lstinline!10g0.fsx!. Tekstfilerne skal kunne oversættes med fsharpc, og resultatet skal kunne køres med mono. Funktioner skal dokumenteres ifølge dokumentationsstandarden, og udover selve programteksten skal der vedlægges en kort rapport i pdf format kaldet \lstinline{10g.pdf}, som indholder de dele af besvarelsen, som ikke naturligt vil være i en programtekst. Det hele skal samles i en zip fil og uploades på Absalon.

\flushright God fornøjelse.
\end{document}

%%% Local Variables:
%%% mode: latex
%%% TeX-master: t
%%% End:


\subsection*{Krav til opgavebesvarelsen}
You must make a program consisting of one or more F\# files, that extends \chess\ as described above, and you must write a small report. The hand-in must consists of the report on pdf-format, one or more fsharp source files, and a single compiled \lstinline[language=console]{exe} file, which can be run using \lstinline[language=console]{mono} command. The hand-in must also give the list of console commands used to compile the program. Besides the requirements described in the previous section, the program must be documented using the F\# documentation standard, and the program must be tested with both a black- and white-box testing. The report must be written in either English or Danish, typeset using \LaTeX, and as its main parts include the sections Introduction (Introduktion), Problem analysis and design (Problemanalyse og design), Program description (Programbeskrivelse), Testing (Afprøvning), and Discussion and Conclusion (Diskussion og Konklusion) as shown in \texttt{rapport.tex}. As appendix the report must be include a User Guide (Brugervejledning). The report excluding frontpage and appendix should be no larger than 10 pages.
