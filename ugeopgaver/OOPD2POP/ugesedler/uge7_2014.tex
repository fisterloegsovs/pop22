\documentclass[a4paper,12pt]{article}

\title{Introduktion til programmering, ugeseddel 7}
\author{Version 1.1}% Torben Mogensen
\date{17.\ oktober 2014}

\usepackage[top=2cm,left=25mm]{geometry}
\usepackage[T1]{fontenc}
\usepackage[utf8]{inputenc}
\usepackage[danish]{babel}
\usepackage{cmap}
%\usepackage{microtype}
\hyphenpenalty=750

\usepackage{amsmath}
%\usepackage{lmodern}
%\usepackage{mathptmx}
%\usepackage{libertine}
%\usepackage[scaled=0.90]{inconsolata}
%\usepackage[scaled=0.83]{beramono}

\usepackage{enumerate}
\usepackage{graphics,graphpap}
\usepackage{listing}
\usepackage{fancyvrb}
\usepackage{graphics,tikz}
\usepackage{listings}
\usepackage{framed}
\usepackage{upquote}

\lstset{
    language=ML,
    keywordstyle=\bfseries,
    showstringspaces=false,
    basicstyle=\footnotesize\ttfamily,
    %numberstyle=\footnotesize,
    %numbers=left,
    %stepnumber=1,
    numbersep=10pt,
    tabsize=2,
    breaklines=false,
    aboveskip={0.7\baselineskip},
    columns=fixed,
    escapechar=@
%    extendedchars=true
% frame=single,
% backgroundcolor=\color{lbcolor},
}

\usepackage{todonotes}
\usepackage{hyperref}
\hypersetup{pdftitle={Introduktion til programmering, ugeseddel 7},
            pdfsubject={},
            pdfauthor={},
            pdfkeywords={rekursive datatyper, gensidig rekursion},
            pdfborder={0 0 0}}

\setlength{\parskip}{1ex}
\setlength{\parindent}{0pt}
\setlength{\parfillskip}{30pt plus 1 fil}

\makeatletter
\newcommand\footnoteref[1]{\protected@xdef\@thefnmark{\ref{#1}}\@footnotemark}
\makeatother

\begin{document}
\maketitle{} Den syvende undervisningsuge handler om \textit{moduler}.

\section{Plan for ugen}
\label{sec:pensum-og-plan}

\subsubsection*{Mandag}
Strukturer og signaturer, separat oversættelse.

\textit{Pensum:} HR: 11.1-11.4. Moscow ML Owner's Manual afsnit 1 og
5-7 (findes i Absalon under ``Undervisningsmateriale'').

\subsubsection*{Tirsdag}
Strukturer, signaturer og funktorer.

\textit{Pensum:} HR: 11.5-11.7.

\subsubsection*{Fredag}
Repetition af ugens pensum. Se ovenfor.

\definecolor{darkred}{rgb}{0.5,0,0}

\vspace{5ex}

\textcolor{darkred}{\textbf{Bemærk:} Da eksamen ligger i kursusuge 8,
  er det ikke muligt at genaflevere ugeopgave 7.  Desuden er
  afleveringsfristen for den individuelle opgave allerede lørdag aften
  i stedet for søndag aften.}

\newpage
\section{Mandagsopgaver}
\label{sec:mandagsopgaver}

\textit{Emner:} Strukturer og signaturer.

\vspace{1ex}

\noindent
Vi har i tidligere uger (uge 4 og 5) brugt to forskellige typer til at
repræsentere farver.  Vi vil nu generalisere dette med SML's
modulsystem.

Et farvemodul skal definere følgende elementer:

\begin{enumerate}[1.]
\item en type \lstinline{colour}, der bruges til at repræsentere
  farveværdier.

\item Værdier \lstinline{black : colour}, \lstinline{white : colour},
  \lstinline{red : colour}, \lstinline{green : colour},
  \lstinline{blue : colour}, \lstinline{cyan : colour}, \lstinline{magenta : colour},
  \lstinline{yellow : colour}, der definerer primærfarverne.

\item En funktion \lstinline{complement : colour -> colour}, der
  returnerer \emph{komplementærfarven} til en farve.  Se
  \url{http://en.wikipedia.org/wiki/Complementary_colors#The_RGB_model}
  for en forklaring af komplementærfarver.

\item En funktion \lstinline{toRGB : colour -> int * int * int}, der kan
  konvertere en farve til et tripel af heltal i intervallet 0-255, der
  repræsenterer farven som en RGB-værdi (ligesom i
  \lstinline{InstagraML}).

\item En funktion \lstinline{fromRGB : int * int * int -> colour}, der
  kan konvertere et RGB-tripel (tre heltal i intervallet 0-255) til en
  farve.
\end{enumerate}

Mandagsopgaverne omhandler farvemoduler af denne slags.  Alle
signaturer og strukturer skrives i en enkelt \texttt{.sml} fil (altså
som en toplevel-mode compilation unit, som beskrevet i Moscow ML
Owner's Manual afsnit 5 og 7).

\begin{enumerate}[{7}M1]

\item Brug beskrivelsen herover af farvemoduler til at skrive en
  signatur for et farvemodul.  Signaturen skal have navnet
  \lstinline{Colour}.

\item Skriv en struktur \lstinline{RGB : Colour}, der implementerer
  farver som RGB-tripler, således at \lstinline{toRGB} bliver
  identitetsfunktionen.  Bemærk, at vi bruger gennemsigtig
  (\emph{transparent}) signaturmatching.

  Farvekomplementering sker ved at hvert RGB-komponent komplementeres.
  Dette gøres ved at trække komponentens værdi fra 255.  For eksempel
  er\newline \lstinline{(255, 155, 55) = complement (0, 100, 200)}.

\item Skriv en struktur \lstinline{PrimaryColours :> Colours} (bemærk
  uigennemsigtig (\emph{opaque}) signaturmatching), hvor farver er
  repræsenteret med den datatype, der blev brugt i uge5-opgaven:

\begin{lstlisting}
datatype primaryColours = Black | Red | Green | Blue
                   | White | Cyan | Magenta | Yellow
\end{lstlisting}

Der skal gælde følgende:

\begin{itemize}
\item Funktionen \lstinline{toRGB} skal implementeres som beskrevet i opgave 5G1.

\item Funktionen \lstinline{fromRGB} skal runde RGB-værdier mellem 0
  og 127 (begge inklusive) til 0 og værdier mellem 128 og 255 til 255,
  således at hvert RGB-komponent har værdier, der enten er 0 eller
  255, og derefter returnere den primærfarve, der med
  \lstinline{toRGB} afbildes til dette RGB-tripel.  Med andre ord,
  ``rundes'' en farve af til nærmeste primærfarve.  \textbf{Vink:} Lav
  en hjælpefunktion \lstinline{round : int -> int}, der afrunder et
  tal til enten 0 eller 255 som beskrevet herover.

\item \lstinline{complement} følger de almindelige regler for
  komplementærfarver.  For eksempel er \lstinline{Green = complement Magenta}.

\end{itemize}

\item Prøv at skrive \lstinline{PrimaryColours.red = PrimaryColours.green}.
Forklar resultatet.

\vspace{2ex}

\hrule

\vspace{1ex}

Vi ønsker nu at definere udvidede moduler for farver med to ekstra operationer:

\begin{enumerate}[1.]
\item En funktion \lstinline{+++ : colour * colour -> colour}, der kan
  lægge to farveværdier sammen.

\item En funktion \lstinline{*** : real * colour -> colour}, der kan
  skalere en farveværdi.

\end{enumerate}

\item Definer en signatur \lstinline{ExtendedColour}, der udvider
  \lstinline{Colour}.  Brug \lstinline{include Colour} til dette.

\item Definer en struktur \lstinline{ExtendedRGB : ExtendedColour},
  der udvider \lstinline{RGB} med de to operationer.  Brug
  \lstinline{open RGB} til dette.

  Sørg for at RGB-værdier ligger i intervallet 0-255 selv efter
  addition og skalering af farver.  Definer til dette en
  hjælpefunktion \lstinline{constrain : int -> int}, der ``skubber''
  tal ind i intervallet 0-255.

\item Definer en struktur \lstinline{ExtendedPrimaryColours : ExtendedColour},
  der udvider\newline \lstinline{PrimaryColours} med de to operationer.

\textbf{Vink:} Det kan være en fordel at konvertere farverne til RGB,
behandle RGB-værdierne, og konvertere resultatet tilbage til en
primærfarve.  Du kan godt kalde funktioner fra \lstinline{ExtendedRGB}
til at gøre dette.

\end{enumerate}

\newpage
\section{Tirsdagsopgaver}
\label{sec:tirsdagsopgaver}
\textit{Emner:} Strukturer og signaturer, separat oversættelse.

Det forventes, at du inden øvelserne tirsdag har forberedt dig på
opgaverne ved at løse så mange som muligt på egen hånd.

\vspace{1ex}

Til forskel fra mandagsopgaverne bruger vi nu \emph{structure mode
  compilation units} som beskrevet i Moscow ML Owner's Manual afsnit 5
og 6.

\begin{enumerate}[{7}T1]
\item Skriv signaturen \lstinline{Colour} fra mandagsøvelserne i en
  fil \texttt{Colour.sig}.

\item Fra en kommandolinje (brug evt.\ ``Shell Command'' fra menuen
  ``Tools'' i Emacs) oversæt denne med kommanden \texttt{mosmlc -c
  Colour.sig}.

\item Lav en fil \texttt{Colour.sml}, der indeholder en struktur
  \lstinline{Colour :> Colour}, der implementere signaturen
  \lstinline{Colour} som RGB-tripler (dvs.\ at den bortset fra navnet
  og uigennemsigtigheden er identisk med strukturen \lstinline{RGB}).

\item Fra en kommandolinje (brug evt.\ ``Shell Command'' fra menuen
  ``Tools'' i Emacs) oversæt denne med kommanden \texttt{mosmlc -c
  Colour.sml}.

\item Start mosml med \texttt{mosml -P full}, men uden at give nogen
  fil eller buffer som inddata (brug ``Start SML repl'' fra SML menuen
  i Emacs).  Skriv derefter kommandoen \texttt{load "Colour";} i
  mosml-kommandolinjen.  Skriv derefter \texttt{Colour.red;} og
  observer svaret.

\item Skriv følgende kodelinjer:

\begin{Verbatim}
structure Hello =
struct
  val _ = TextIO.output (TextIO.stdOut, "Hello World!\n");
end
\end{Verbatim}

  i en fil \texttt{Hello.sml} og oversæt denne med kommandoen\newline
  \texttt{mosmlc Hello.sml -o Hello}.  Hvis du bruger Windows, så brug i
  stedet\newline \texttt{mosmlc Hello.sml -o Hello.exe}.

\item Kald programmet \texttt{Hello} fra en kommandolinje.  Hvis du er
  bruger Linux eller MacOS, så skriv \texttt{./Hello} på kommandolinjen,
  hvis du er i Windows, skriv \texttt{Hello}.  \textbf{NB!} Sørg for
  at din kommandolinje bliver afviklet i det filkatalog, hvor filen
  \texttt{Hello} (eller \texttt{Hello.exe}) findes.


\end{enumerate}

\newpage
\section{Fredagsopgaver}
\label{sec:fredagsopgaver}
\textit{Emne:} Funktorer.

Det forventes, at du inden øvelserne fredag har forberedt dig på
opgaverne ved at løse så mange som muligt på egen hånd.

\vspace{1ex}

Fredagsopgaverne bygger videre på mandagsopgaverne og inddrager flere
elementer fra uge5-opgaverne.

Alle funktorer, signaturer og strukturer skrives i en enkelt
\texttt{.sml} fil (altså som en toplevel-mode compilation unit, som
beskrevet i Moscow ML Owner's Manual afsnit 5 og 7).  Som udgangspunkt
bruges en fil med alle signaturer og strukturer fra mandagsopgaverne.


\begin{enumerate}[{7}F1]
\item Skriv en funktor \lstinline{Figure}, der er parameteriseret med
  en struktur \lstinline{C}, der matcher signaturen \lstinline{Colour},
  og som implementerer følgende elementer:

\begin{enumerate}[1.]

\item En type \lstinline{colour}, der er identisk med den farvetype, der
  er defineret i parameteren \lstinline{C}.

\item En funktion \lstinline{toRGB}, der er identisk med den funktion, der
  er defineret i parameteren \lstinline{C}.

\item En type \lstinline{figure}, der ligner den, der er beskrevet i
  opgavetemaet fra uge 5, pånær at den i stedet for at være polymorf i
  farven bruger farvetypen \lstinline{colour}.

\item En funktion \lstinline{colourOf : figure -> point -> colour option},
  svarende til den funktion, der blev implementeret i opgave 5G3.

\item En funktion\newline
\lstinline{toInstagraML : figure * point * int * int * real -> InstagraML.image}\newline
svarende til den funktion, der blev implementeret i opgave 5G6.

Bemærk, at denne funktion selv skal konvertere fra \texttt{colour} til RGB.
\end{enumerate}

\item Definer en struktur \lstinline{RGBFigure} ved at anvende
  funktoren \lstinline{Figure} på strukturen \lstinline{RGB}.

\item Definer en struktur \lstinline{PrimaryFigure} ved at anvende
  funktoren \lstinline{Figure} på strukturen \lstinline{PrimaryColours}.

\item Skriv en funktor \lstinline{ExtendedFigure}, der er
  parameteriseret over en struktur\newline \lstinline{C : ExtendedColour}.
  Den skal implementere de samme typer og funktioner som \lstinline{Figure},
  pånær at datatypen \lstinline{figure} implementerer en ekstra konstruktor\newline
  \lstinline{Blend of real * figure * figure}.

  Ideen er, at \lstinline{Blend}\,$(b,f,g)$ blander farverne fra de to
  figurer $f$ og $g$ i forholdet $b$, hvor $0<b<1$ efter følgende regler:

  \begin{enumerate}[1.]
  \item Hvis et punkt ligger i den ene figur, men ikke i dem begge,
    har det farven fra denne figur.
  \item Hvis et punkt ligger i begge figurer, farven fra $f$ er $c_f$
    og farven fra $g$ er $c_g$, gives farven
    \texttt{+++(***($b$,\,$c_f$),\,***($1{-}b$,\,$c_g$))}.
  \end{enumerate}

\item Definer en struktur \lstinline{ExtendedRGBFigure} ved at anvende
  funktoren \lstinline{ExtendedFigure} på strukturen \lstinline{ExtendedRGB}.

\item Definer en struktur \lstinline{ExtendedPrimaryFigure} ved at anvende
  funktoren \lstinline{Figure} på strukturen \lstinline{ExtendedPrimaryColours}.

\end{enumerate}


\newpage
\section{Opgavetema: }\label{tema}

\newcommand{\forekomster}{\textit{forekomster}}

Ugens opgavetema er \emph{multimængder}.
En \emph{multimængde} er en variant af mængder, hvor antallet af
forekomster af et element har betydning.  Derfor kan man ikke blot
spørge, \emph{om} et element findes i multimængden, men også
\emph{hvor mange gange} elementet findes. $\forekomster(m,~x)$ vil være
antallet af forekomster af $x$ i multimængden $m$.  Dette vil altid
være større end eller lig med 0.

Foreningsmængde, fællesmængde og mængdedifferens for multimængder
defineres således:

\begin{itemize}
\item Foreningsmængden $m_1 \uplus m_2$ af to multimængder $m_1$ og
  $m_2$ er en multimængde, der indeholder elementerne fra $m_1$ og
  $m_2$ lige så mange gange, som de forekommer i $m_1$ og $m_2$
  tilsammen.  For eksempel er

\[\begin{array}{rcl}
\{1,2,2,3\} \uplus \{1,1,4\} &=& \{1,1,1,2,2,3,4\}
\end{array}\]

\item Fællesmængden $m_1 \cap m_2$ af to multimængder $m_1$ og $m_2$
  er en multimængde, der indeholder elementerne, der findes i både
  $m_1$ og $m_2$ lige så mange gange, som de forekommer i den af $m_1$
  og $m_2$, hvor de forekommer færrest gange.  For eksempel er

\[\begin{array}{rcl}
\{1,2,2,3\} \cap \{1,1,4\} &=& \{1\}
\end{array}\]

\item Differensmængden $m_1 \setminus m_2$ af to multimængder $m_1$ og
  $m_2$ er en multimængde, der indeholder elementerne, der findes i
  $m_1$ flere gange end i $m_2$.  Antallet af forekomster i
  differensmængden er da differensen mellem antal forekomster i $m_1$
  og $m_2$.  For eksempel er

\[\begin{array}{rcl}
\{1,2,2,3\} \setminus \{1,1,4\} &=& \{2,2,3\}
\end{array}\]

\end{itemize}

Dette kan udtrykkes med formlerne

\[\begin{array}{lcl}
\forekomster(m_1 \uplus m_2,~x) &=&
 \forekomster(m_1,~x)+\forekomster(m_2,~x)\\
\forekomster(m_1 \cap m_2,~x) &=&
 \min(\forekomster(m_1,~x),~\forekomster(m_2,~x))\\
\forekomster(m_1 \setminus m_2,~x) &=&
 \max(0,~\forekomster(m_1,~x)-\forekomster(m_2,~x))
\end{array}\]

\noindent
$\max$-operationen i ligningen for multimængdedifferens er for at
undgå et negativt antal forekomster.

Se evt.\ også \url{http://en.wikipedia.org/wiki/Multiset} og\newline
\url{http://theory.stanford.edu/~arbrad/pivc/sets.pdf}.  Bemærk dog,
at sidstnævnte bruger $-$ i stedet for $\setminus$ til at angive
multimængdedifferens.

En mulig signatur for multimængder er:
\vspace{1ex}

\begin{Verbatim}
signature MSET =
sig
  type 'a mset  (* typen af multimængder med elementer af type 'a *)
  val multiplicity : ''a mset * ''a -> int (* antal forekomster af element *)
  val empty : 'a mset  (* den tomme multimængde *)
  val singleton : ''a -> ''a mset  (* laver multimængde med et element *)
  val union : ''a mset * ''a mset -> ''a mset (* foreningsmængde *)
  val intersect : ''a mset * ''a mset -> ''a mset (* fællesmængde *)
  val minus : ''a mset * ''a mset -> ''a mset (* mængdedifferens *)
end
\end{Verbatim}

\newpage
\section{Gruppeaflevering}
\label{sec:gruppeaflevering}
Gruppeafleveringen obligatorisk.  Alle delspørgsmål skal besvares.
Opgaven afleveres i Absalon.  Der afleveres en fil pr.\ gruppe, men
den skal angive alle deltageres fulde navne i kommentarlinjer øverst i
filen. Filens navn skal være af formen
\texttt{7G-\textit{initialer}.sml}, hvor initialer er erstattet af
gruppemedlemmernes initialer. Hvis f.eks.\ Bill~Gates, Linus~Torvalds,
Steve~Jobs og Gabe~Logan~Newell afleverer en opgave sammen, skal filen
hedde \texttt{7G-BG-LT-SJ-GLN.sml}. Brug gruppeafleveringsfunktionen i
Absalon.

Gruppeopgaven giver op til 2 point, som tæller til de 20 point, der
kræves for eksamensdeltagelse.

\textcolor{darkred}{\textbf{NB!} Der er af tidshensyn \emph{ikke}
  mulighed for genaflevering af ugeopgave 7.}

\vspace{1ex}

Alle signaturer og strukturer skrives i den SML-fil, der afleveres.
Denne fil skal kunne åbnes med \texttt{mosml -P full} uden fejl.  Der
bruges altså \emph{toplevel mode compilation units}.

\begin{enumerate}[{7G}1]

\item Kopier ovenstående signatur til SML-filen.

\item Skriv en struktur \texttt{Mset : MSET}, som implementerer en
  multimængde som en uordnet liste af elementer, hvor antallet af
  forekomster af et element i en liste angiver antallet af forekomster
  af elementet i multimængden.

\item Definer følgende multimængder:

\begin{lstlisting}
val one = Mset.singleton 1
val two = Mset.singleton 2
val three = Mset.singleton 3
val four = Mset.singleton 4
val m1223 = Mset.union (one,
                        Mset.union(two,
                                   Mset.union (two, three)))
val m114 = Mset.union (one,
                       Mset.union(one,
                                  Mset.union (four, Mset.empty)))
\end{lstlisting}

Ser de ud som forventet?

\item Definer følgende multimængder:

\begin{lstlisting}
val forening = Mset.union (m1223, m114)
val faelles = Mset.intersect (m1223, m114)
val minus = Mset.minus (m1223, m114)
\end{lstlisting}

Sammenlign resultaterne med eksemplerne i opgavetemaet.

\item Definer følgende værdier:

\begin{lstlisting}
val forekomster1 = Mset.multiplicity (forening, 1)
val forekomster2 = Mset.multiplicity (faelles, 1)
val forekomster3 = Mset.multiplicity (minus, 1)
\end{lstlisting}

Er resultaterne som forventet?
\vspace{1ex}

\hrule

\newpage

Vi laver nu en signatur:

\begin{Verbatim}
signature EMSET =
sig
  include MSET
  val twice : ''a mset -> ''a mset
end
\end{Verbatim}

der udvider \texttt{MSET} med en funktion \texttt{twice}, der
fordobler forekomsten af alle elementer i en multimængde.

\item Skriv en funktor \texttt{UMset(mm : MSET) : EMSET}, der tager en
  multimængdestruktur og laver en udvidet multimængdestruktur.

\item Anvend \texttt{UMset} på strukturen \texttt{Mset} for at få en
  struktur \texttt{Eset}, der implementerer \texttt{EMSET}.

\end{enumerate}


\newpage
\section{Individuel aflevering}
\label{sec:indiv-aflev}
Den individuelle opgave er obligatorisk.  Alle delspørgsmål skal
besvares.  Opgaven afleveres i Absalon som en fil med navnet
\texttt{7I-\textit{navn}.sml}, hvor \texttt{\textit{navn}} er
erstatttet med dit navn. Hvis du fx hedder Anders~A.~And, skal
filnavnet være \texttt{7I-Anders-A-And.sml}. Skriv også dit fulde navn
som en kommentar i starten af filen.

Den individielle opgave giver op til 3 point, som tæller til de 20
point, der kræves for eksamensdeltagelse.

\textcolor{darkred}{\textbf{NB!} Af tidshensyn er afleveringsfristen
  for den individuelle opgave allerede \emph{lørdag aften} i stedet
  for søndag aften.  Endvidere er der \emph{ikke} mulighed for
  genaflevering af ugeopgave 7.}

\vspace{1ex}

Alle signaturer og strukturer skrives i den SML-fil, der afleveres.
Denne fil skal kunne åbnes med \texttt{mosml -P full} uden fejl.  Der
bruges altså \emph{toplevel mode compilation units}.

\begin{enumerate}[{7I}1]

\item Kopier signaturen \texttt{MSET} fra opgavetemaet til din SML-fil.

\item Skriv en struktur \texttt{MsetF :> MSET}, som implementerer en
  multimængde som en funktion fra elementer til antallet af
  forekomster af disse.

  Din struktur skal blandt andet altså indeholde linjerne

\begin{Verbatim}
  type 'a mset = 'a -> int

  fun multiplicity (m, x) = m x
      (* anvend multimængden på elementet *)

  fun singleton x = fn y => if y = x then 1 else 0
      (* x har 1 forekomst, alle andre elementer har 0 *)

  fun union (m1, m2) = fn x => m1 x + m2 x
      (* bruger formlen fra beskrivelsen i opgavetemaet *)
\end{Verbatim}

\textbf{Vink:} Brug formlerne vist i opgavetemaet for antal
forekomster af $x$ i $m_1 \cap m_2$ og $m_1 \setminus m_2$ til at
definere \texttt{intersect} og \texttt{minus} på samme måde som
\texttt{union} er defineret herover.

\end{enumerate}


\newpage
\section{Ugens nød}
\label{sec:ugens-nod}
De fleste kender nok spillet "`Minestryger"'
(\url{http://da.wikipedia.org/wiki/Minestryger_%28spil%29}).

  Vi laver nu en variant af dette spil, hvor det for alle felter på
  forhånd er kendt, hvor mange bomber, der i alt er i de otte
  nabofelter samt feltet selv.  Det g<E6>lder både felter med og felter
  uden bomber.

  Et eksempel på et sådant spil er

\begin{verbatim}
221
232
232
\end{verbatim}
  
  Opgaven går nu ud på at finde ud af, i hvilke felter bomberne er.
  Svaret skal gives som en tekst, hvor tomme felter angives med
  \texttt{.} og bomber med \texttt{B}.  Eksempelvis skal det
  ovenstående inddata give følgende uddata:

\begin{verbatim}
...
BB.
..B
\end{verbatim}
  
Der er ikke altid en entydig løsning.  For eksempel vil inddataet


\begin{verbatim}
11
11
\end{verbatim}
 
have fire løsninger, blandt andet:

\vspace{1ex}

\noindent
\begin{tabular}{lcl}
\texttt{.B} & og & \texttt{B.} \\
\texttt{..} &  & \texttt{..}
\end{tabular}
\vspace{1ex}

Der er heller ikke altid løsninger, idet for eksempel

\begin{verbatim}
10
00
\end{verbatim}
 
ikke har en løsning.

\begin{enumerate}[{7}N1]

\item Lav i SML en funktion \texttt{minestryger : string -> string ->
    unit}, sådan at\newline \texttt{minestryger $in$ $out$} læser et
  spil fra filen $in$ og skriver en løsning i filen $out$.  Hvis der
  er flere mulige løsninger, er det ligegyldigt, hvilken af dem, der
  skrives til $out$.  Hvis ingen løsning findes, kastes undtagelsen
  \texttt{Uloeselig}.  Et spil består af $m$ linjers tekst med hver
  $n$ cifre fra 0 til 9.  Hvis inddatafilen ikke har dette format,
  kastes undtagelsen \texttt{Domain}.

  Et par eksempelspil er angivet herunder:

\begin{verbatim}
4421332
4543543
3433432
2434321
2312110
\end{verbatim}

\begin{verbatim}
233322234320
233333335541
012333335652
112244324763
334455445763
554234444653
554233333543
332012221322
111013442211
001234442100
111233332100
111232111100
\end{verbatim}

\begin{verbatim}
22122235444654334453
33133458654776546685
23144557754665457674
24343456644666556685
14454565657765567785
36554565668754556685
36454677766443556675
36454555654222333354
24344655553123433465
23345743354213332465
33224643365424443465
33113542244323444465
44102331344334445464
34223554444455545575
35432345666766565565
36553456667987653355
47553455656655564455
58553455644334442244
46432466755223442355
34310134544223320133
\end{verbatim}

I bedømmelsen af løsninger, gives programmet større og større opgaver
og stoppes når den samlede køretid overstiger et minut.  Det program,
der har løst flest opgaver inden da, vinder.  Bemærk: Bedømmerens PC
kører måske ikke helt så hurtigt som din.

\end{enumerate}


\end{document}

%%% Local Variables:
%%% mode: latex
%%% TeX-master: t
%%% End:
