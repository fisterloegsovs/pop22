\documentclass[a4paper,12pt]{article}

\usepackage[margin=2.5cm]{geometry}
\usepackage[T1]{fontenc}
\usepackage[utf8]{inputenc}
\usepackage[danish]{babel}
\usepackage{listings}
\usepackage{graphicx}
\graphicspath{{figures/}}
\setlength{\parindent}{0cm}
\setlength{\parskip}{1em}

\title{Programmering og Problemløsning\\Datalogisk Institut,
  Københavns Universitet\\Uge(r)seddel 2 - individuel opgave}
\author{Jon Sporring}
\date{7.\ - 15.\ september}

\begin{document}
\maketitle

I denne periode skal I arbejde individuelt. Formålet er at:
\begin{itemize}
\item lære forskellen mellem et oversat og et fortolket program
\item I laver jeres første fsharp program
\item I stifter bekendtskab med bindinger, simple typer, funktioner,
  rekursion, 2-tupler (par), operatorer, strenge, virkefelt (scope) og closures.
\end{itemize}

Opgaverne for denne uge er delt i øve- og afleveringsopgaver. For de individuelle afleveringsopgaver skal I være særlig opmærksomme på, at jeres løsning ikke må udarbejdes i samarbejde med andre. Den præcise formulering for kravene desangående finder I under punktet "`Opgaver$\rightarrow$Generel information"'.

Øve-opgaverne er:
\begin{description}
\item[2.0] HR: 1.1, 1.2, 1.4, 1.6, 2.1, 2.2, 2.12
\end{description}

Afleveringsopgaven er:
\begin{description}
\item[2.1] HR: 1.5, 1.8
\item[2.2] HR 2.8, 2.9, 2.10
\end{description}
Afleveringsopgaven skal afleveres som både LaTeX, den genererede PDF, samt en fsx tekstfil med løsningen for hver delopgave, som kan oversættes med fsharpc og hvis resultat kan køres med mono. Det hele skal samles i en zip fil efter sædvanlig navnekonvention:
\begin{quote}
  \lstinline|<instructor's-initial>_<firstname.lastname>_<exercise-number>.fsx|
\end{quote}
I zip filen skal en delopgave navngives ved opgavenummer således at filen for opgave 1.5 hedder \lstinline|opg1_5.fsx|.

\flushright God fornøjelse.
\end{document}

%%% Local Variables:
%%% mode: latex
%%% TeX-master: t
%%% End:

