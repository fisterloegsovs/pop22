\documentclass[a4paper,12pt]{article}

\usepackage[margin=2.5cm]{geometry}
\usepackage[T1]{fontenc}
\usepackage[utf8]{inputenc}
\usepackage[danish]{babel}
\usepackage{listings}
\usepackage{graphicx}
\graphicspath{{figures/}}
\setlength{\parindent}{0cm}
\setlength{\parskip}{1em}

\title{Programmering og Problemløsning\\Datalogisk Institut,
  Københavns Universitet\\Uge(r)seddel 3 - individuel opgave}
\author{Jon Sporring}
\date{Deadline  22.\ september}

\begin{document}
\maketitle

I denne periode skal I arbejde individuelt. Formålet er at arbejde med:
\begin{itemize}
\item tupler og records
\item mærkater (tags) og enumerede typer
\item selvdefinerede operatorer
\item fejlhåndtering og undtagelser
\end{itemize}

Opgaverne for denne uge er delt i øve- og afleveringsopgaver. 

Øve-opgaverne er:
\begin{description}
\item[3.0] HR: 3.1, 3.2, 3.5, 3.6, 3.7 
\end{description}

Afleveringsopgaven er:
\begin{description}
\item[3.1] HR  3.4
\item[3.2] Skriv en funktion, der tager 2 \lstinline|StraightLine| og
  returnerer skæringspunktet mellem dem, hvis det eksisterer og i
  modsat fald skal fejlsituationen håndteres passende.
\end{description}
Afleveringsopgaven skal afleveres som både LaTeX, den genererede PDF, samt en fsx tekstfil med løsningen for hver delopgave, som kan oversættes med fsharpc og hvis resultat kan køres med mono. Det hele skal samles i en zip fil efter sædvanlig navnekonvention:
\begin{quote}
  \lstinline|<instructor's-initial>_<firstname.lastname>_<exercise-number>.fsx|
\end{quote}
I zip filen skal en delopgave navngives ved opgavenummer således at filen for opgave 1.5 hedder \lstinline|opg1_5.fsx|.

\flushright God fornøjelse.
\end{document}

%%% Local Variables:
%%% mode: latex
%%% TeX-master: t
%%% End:
